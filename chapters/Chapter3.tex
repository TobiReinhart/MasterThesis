\dictum{
In this chapter, we investigate two concrete examples of possible gravitational fields each one motivated by a particular matter theory that exploits the thereby provided spacetime geometry as background. Concretely we will study the well-known case of gravity being described by a metric tensor field but also deviate from this by considering the case of spacetime geometry being provided by a so-called area metric. For each case, we will then apply the previously developed framework to not only qualitatively examine possible gravitational dynamics but also explicitly construct the most general meaningful perturbative theory of gravity.
}
\section{Perturbative Metric Gravity around Minkowski Spacetime}
As a first example, we consider the case where the gravitational field is provided by a symmetric $(0,2)$ tensor field, the metric tensor $g_{ab}$. The field bundle is hence given by the bundle of such tensors and as before labeled as $F_{\text{GR}} := S^0_2M \subset T^0_2M$. For some of the following considerations, we might more appropriately restrict $F_{\text{GR}}$ to the subbundle of $(0,2)$ tensors that are non-degenerate in the sense that the induced map between tangent and cotangent bundle that is obtained by partially applying the metric to a vector field defines an isomorphism. Furthermore, we additionally might restrict future considerations to those non-degenerate metrics that have signature $(-,+,+,+)$. As both of these restrictions define open conditions, this does not cause any problems.

Note that the case of gravity being described by a metric tensor field not only is the standard example and hence excellent for testing our framework, from the point of view of Constructive gravity it is also unique as the metric tensor is the "smallest" structure --- in the sense that the corresponding field bundle has the least dimensional fibers --- that allows for the support of a non trivial matter theory.

In the simplest possible case the matter field is described by a scalar field, i.e., a spacetime function $\phi \in C^{\infty}(M,\mathbb{R})$ which corresponds to the trivial matter field bundle $M \times \mathbb{R}$. We are again restricting to matter theories that are described by first-derivative-order Lagrangians and hence generate second-derivative-order EOM. In the simplest case, the matter EOM are linear. This corresponds to a quadratic matter Lagrangian. Denoting adapted coordinates on $J^1(M \times \mathbb{R})$ by $(x^m,\phi,\phi_m)$ the most general, such matter Lagrangian is given as:
\begin{align}\label{KGL}
    \mathcal{L}_{KG} = \frac{1}{2} \left ( B^{ab} \phi_a \phi_b - m^2 \phi^2\right )\omega \mathrm{d}^4x.
\end{align}
This is the so called Lagrangian \textit{\textbf{Klein-Gordon}} Lagrangian. It is important to observe that this really described the most general quadratic, first-order Lagrangian for a scalar field. 
In particular note that specifying such a Lagrangian forces us to prescribe $10$ spacetime functions $B^{ab}$ that tell us how the product $\phi_a\cdot \phi_b$ has to be contracted and furthermore one scalar density of weight one $\omega$ that ensures that the Lagrangian really defines a volume form valued bundle map and thus provides us with a well-defined action functional and EOM.
These additional ingredients are precisely provided by a metric tensor field $g_{ab}$. We then simply take $B^{ab}$ as the corresponding inverse metric $(g^{-1})^{ab}$. Note that in the case of non-degenerate metrics mapping a metric to its inverse is a smooth bundle isomorphism. Furthermore, one can in fact show that up to an overall constant there is exactly one scalar density of weight $1$ that can be obtained from the metric, namely $\omega = \sqrt{ \vert \operatorname{det}(g) \vert }$. 
Thus we see that the simplest possible non-trivial Lagrangian one can write down for a scalar field forces us to additionally provide exactly the amount of structure that is provided by a metric tensor field. If this particular metric, i.e., its the values of $g_{ab}$ at the various spacetime points, is not specified by hand, we need further equations that allow us to determine it.  

To that end, we are going to apply the previously developed perturbative framework to construct a perturbative expansion of a diffeomorphism invariant gravitational Lagrangian $\mathcal{L}_{\text{GR}} = L_{\text{GR}} \mathrm{d}^4x$ that is causally compatible with the Klein-Gordon Lagrangian (\ref{KGL}) and generates second-derivative-order EOM. 
We have already remarked that for the field bundle $F_{\text{GR}}$ there exist precisely 14 functionally independent curvature invariants that each solve the homogeneous invariance equation. Due to Lovelock (\cite{Lovelock1969}, \cite{doi:10.1063/1.1665613} and also \cite{doi:10.1063/1.1666069}) we know that out of the 14 independent gravitational Lagrangians that can be constructed from these up to contributions that vanish once we compute the EOM and equate to zero the Einstein-Hilbert Lagrangian is the only one that generates second-derivative-order EOM. 
Hence we expect from our perturbative construction recipe to precisely recover a perturbative expansion of the Einstein-Hilbert Lagrangian from first principles. Note that thereby, we get an excellent test for our framework. 


We start by introducing the necessary structure on $F_{\text{GR}}$. We have already seen that a pair of intertwiners for this field bundle can simply be obtained by the matrices  (\ref{interIMet}) and (\ref{interJMet}). These are, of course exactly the same intertwiners as the ones we use for describing second spacetime derivatives. We thus obtain coordinates on $J^2F_{\text{GR}}$ as $(x^m,v_A,v_{Ap},v_{AI})$ where now both $A$ and $I$ run from 0 to 9. For a metric tensor, i.e., a section $g \in \Gamma(F_{\text{GR}}))$ we further get the relations:
\begin{align}
    g_{ab} = I^A _{ab} g_A \ \ \text{and} \ \ g_A = J^{ab}_{A} g_{ab},
\end{align}
and similar for the inverse metric $g^{-1} \in \Gamma(F_{\text{GR}}^{\ast})$.
Following along the lines described in algorithm \ref{Algo2} we choose $J^2F_{\text{GR}} \ni p_0 \equiv (x_0^m,\eta_A,0,0) =: (N_{AI})$ where $\eta_A = J^{ab}_A \eta_{ab}$ as $\eta$-induced expansion point. Furthermore, we choose to construct the perturbative gravitational Lagrangian up to order $k=3$. 
Next, we compute the coordinate expression of the Lie derivative of a metric tensor field to obtain the vertical coefficients $C_{An}^{Bm}$. We get:
\begin{align}
    \mathcal{L}_{\xi} g_{ab} = \partial_m g_{ab} \cdot \xi^m + \left (-2 \delta_n^{(c\vert} \delta_{(a}^m \delta_{b)}^{\vert d)} \right ) g_{cd} \partial_m \xi^n.
\end{align}
Inserting $g_{cd} = I^B_{cd} g_B$ further contracting the whole expression with $J^{ab}_A$ to convert the free $ab$ indices to a $A$ index we find:
\begin{align}
    \mathcal{L}_{\xi} g_A = \partial_m g_A \xi^m + \left (-2 I^B_{nb}J^{mb}_{A} \right )g_B \partial_m \xi ^n. 
\end{align}
Here we also used that the two intertwiners $I^A_{ab}$ and $J_A^{ab}$ are symmetric in their spacetime indices. Thus we can simply read off from the last expression:
\begin{align}
    C_{An}^{Bm} = -2 I^B_{nb}J_A^{mb}.
\end{align}
As before we define $(H_{AI}) := (v_{AI}) - (N_{AI})$ to obtain the most general expansion of the Lagrangian that generates second-derivative-order EOM by:
\begin{align}\label{LGR}
    L_{\text{GR,per}} =  a_0 + a^A H_A + a^{AI}H_{AI} + a^{AB} H_{A}H_{B} + a^{ApBq} H_{Ap}H_{Bq} + a^{ABI} H_{A} H_{BI} \\
    + a^{ABC} H_a H_B H_C + a^{ABpCq} H_{A}H_{Bp}H_{Cq} +
    + a^{ABCI} H_A H_B H_{CI} 
    + \mathcal{O}(4).
\end{align}
We again restrict attention to those expansion coefficients with an even number of spacetime indices, as any coefficients with an odd number of such cannot be Lorentz invariant and thus are forbidden by the equivariance equations.

According to algorithm \ref{Algo2}, the next step now consists of inserting the most general Lorentz invariant tensors with appropriate index structure for the expansion coefficients. This is done entirely in terms of the developed computer algebra. The explicit expressions can be found in the appendix (\ref{LorentzGR1}).
Nevertheless providing an illustrative example we consider the expansion coefficient $a^{AB} = I^{A}_{ab}I^{B}_{cd}a^{abcd}$. We clearly see the symmetries in the pairs $(ab)$ and $(cd)$ and the additional symmetry under the exchange $(ab) \leftrightarrow (cd)$ as the coefficient only appears contracted against $H_AH_B$ in $L_{\text{GR,per}}$. Writing down the most general Lorentz invariant tensor with these symmetries we readily find that we cannot obtain non zero contributions from $\epsilon^{abcd}$ to any tensor that features these symmetries. The inverse Minkowski metric a priory allows for $3$ different expressions with $4$ upper case indices:
\begin{itemize}
    \item[(i)] $\eta^{ab} \eta^{cd}$ 
    \item[(ii)] $\eta^{ac} \eta^{bd}$ 
    \item[(iii)] $\eta^{ad} \eta^{bc}$.
\end{itemize}
Enforcing the symmetries on these $3$ expressions, we find that (i) already features the desired symmetries and (ii) and (iii) both provide the symmetry once we symmetrize in $(ab)$. These two terms then are precisely the same. We thus find that:
\begin{align}\label{ansatzExample}
    a^{AB} = I^{A}_{ab}I^{B}_{cd} \left ( 8\mu_3 \cdot \eta^{ab}\eta^{cd} + 8\mu_4 \cdot \eta^{c(a} \eta^{b)d}   \right ).
\end{align}
Here the factor $8$ is a result of the factor less symmetrization that we implement in our computer program, i.e., when symmetrizing the expression w.r.t. $(ab),(cd)$ and $(ab) \leftrightarrow (cd) $ we do not divide by $2$ each time and hence get an overall factor of $2^3=8$. This approach is equivalent to the symmetrization including these factors as we are free to absorb such factors by redefining the constants. It furthermore provides the advantage that we can then deal with integer factors throughout the whole computation. Details are explained in section (\ref{LorentzGen}).
Thus we have obtained an expression that features $2$ constants for $a^{AB}$.

Note that computing the remaining expansion coefficients works precisely along the same lines as the above with the only difference being that the expressions will become increasingly involved and thus are best obtained by using computer algebra.
Doing this the dimensions, i.e., the number of arbitrary constants in the expansion coefficients are displayed in table \ref{GRExp}.
\begin{table}
\centering 
\begin{tabular}{lll}\toprule
    expansion coefficient & dimension & constants   \\ \midrule
    $a_0$ & 1 & $\{\mu_1\}$ \\
    $a^A$ & 1 & $\{\mu_2\}$ \\
    $a^{AI}$ & 2 & $\{\nu_1, \nu_2\}$ \\
    $a^{AB}$ & 2 & $\{\mu_3, \mu_4 \} $ \\
    $a^{ApBq}$ & 6 & $\{\nu_3,...,\nu_8\}$ \\
    $a^{ABI}$ & 5 & $\{ \nu_9,...,\nu_{13} \}$ \\
    $a^{ABC}$ & 3 & $\{ \mu_5,...\mu_7 \}$\\
    $a^{ABpCq}$ & 21 & $\{\nu_{14},...,\nu_{34} \}$ \\
    $a^{ABCI}$ & 13 & $\{ \nu_{35},...,\nu_{47}\}$\\ \bottomrule
\end{tabular}
\caption{Dimensions of the Lorentz Invariant Expansion Coefficients for $\mathcal{L}_{\text{GR,per}}$.}\label{GRExp}
\end{table}
Here we separated the appearing constants in those that appear in front of terms including derivatives of the metric tensor field and those that appear in terms that do not include such. We adopt some terminology more present in QFT related subjects and refer to the former expressions as \textbf{\textit{kinetic terms}} ans the latter as \textit{\textbf{mass terms}}. The constants are then called \textit{\textbf{kinetic parameters}} $(\nu_1,...)$ and \textit{\textbf{masses}} $(\mu_1,...)$ respectively.   

According to algorithm \ref{Algo2} the next step consists of inserting these expansion coefficients and the previously computed expression for the constant tensor $C_{An}^{Bm}$ into the perturbative equivariance equations (\ref{order1}), (\ref{order2}) and (\ref{order3}). These are then linear equations for the unknown constants $\{ \mu_1,...,\mu_7\}$ and $\{\nu_1,...,\nu_{47}\}$. The solution we get from doing this by using our computer program is displayed in equation (\ref{GRSol}) in the appendix. Note that the expansion (\ref{LGR}) in generall features $15.906$ undetermined constants in the $9$ expansion coefficients. The restriction to those that define the components of Lorentz invariant tensor that is demanded by the perturbative equivariance equations reduces these to $54$ constants. Solving the perturbative equivariance equations then further decreases the number of undetermined parameters to $2$. Also, observe that in the solution, there is no contribution from terms that involve $\epsilon^{abcd}$. In fact, the constant parameters in front of expansion coefficients that contain the Levi-Civita symbol are the only ones that the equivariance equations set to zero.  

We continue with construction algorithm \ref{Algo2} by computing the perturbative expansion of the gravitational principal polynomial. Inserting the solution for the expansion coefficients (\ref{GRSol}) into (\ref{LGR}) now yields the most general diffeomorphism invariant perturbative Lagrangian. Normally we would now proceed by plugging the solved expansion coefficients into the perturbative EOM (\ref{EOMPert}) and then compute the corresponding expansion of the principal symbol and polynomial as outlined in the last chapter. It is, however, worth noting that for the simple case of the gravitational field being provided by a metric tensor it is actually not necessary to follow along with these general steps. We can, in fact, save ourselves much work by taking a closer look at equation (\ref{polyEqn}) that the principal polynomial has to satisfy as a consequence of the diffeomorphism equivariance. For the metric tensor field, this equation takes the particularly simple form:
\begin{align}\label{metricPoly}
    0 = \partial^A \mathcal{P}_{\text{GR}}^{abcd} C_{An}^{Bm} v_B - 4\mathcal{P}_{\text{GR}}^{(abc\vert m} \delta_n^{\vert d)} + 2 \mathcal{P}_{\text{GR}}^{abcd} \delta^m_n.
\end{align}
We now expand $\mathcal{P}_{\text{GR}}^{abcd}$ around $(N_{A}) = (x_0^m, \eta_A)$ and construct a power series solution to (\ref{metricPoly}). From before we know that given the expansion of the Lagrangian up to third order we only need to expand the polynomial up to first order and hence also the totally symmetric expression $\mathcal{P}_{\text{GR}}^{abcd}$ is expanded up to linear order. We get:
\begin{align}
    \mathcal{P}_{\text{GR}}^{abcd} = (P_0)_{\text{GR}}^{abcd} + (P_1)_{\text{GR}}^{abcdA} H_A + \mathcal{O}(2).
\end{align}
Evaluating equation (\ref{metricPoly}) and its first prolongation at the expansion point $\eta_A$ and contracting it against the Lorentz generators yields additional zero and first-order equations that state that the expansion constants must be the components of constant Lorentz invariant tensors. One readily finds that for the constant-order expansion coefficient, there exists exactly one such expression, namely:
\begin{align}
   (P_0)_{\text{GR}}^{abcd} = a \cdot \eta^{(ab} \eta^{cd)}. 
\end{align}
For the linear-order coefficients we obtain the following two possible expression:
\begin{align}
    (P_1)_{\text{GR}}^{abcdA} = I^A_{ef} \left (b \cdot \eta^{(ab} \eta^{cd)}  \eta^{ef} + c \cdot \eta^{(ab} \eta^{c \vert e} \eta^{f \vert d)} \right ).
\end{align}
Upon inserting the thereby obtained expansion in (\ref{metricPoly}) and evaluating at the expansion point $(N_A)$, we get:
\begin{align}
    b = a \ \ \text{and} \ \ c = -2a.
\end{align}
Inserting this into $\mathcal{P}_{\text{GR}}^{abcd}$ and contracting again against $k_a\cdot ...\cdot k_d$ in total we find the following expression for the most general linear-order expansion of the principal polynomial that is consistent with the diffeomorphism equivariance of $L_{\text{GR,per}}$:
\begin{align}
\begin{aligned}
    \mathcal{P}_{\text{GR}} &= a \cdot  \eta(k) \cdot \left [\eta(k) + H \cdot \eta(k)  -
    2 \cdot  \eta(k) H(k) \right ] + \mathcal{O}(2)\\
    &= a \cdot (1 + H) \cdot (\eta(k) - H(k))^2 + \mathcal{O}(2).
\end{aligned}
\end{align}
where  $H:=\eta^{ab} I^A_{ab} H_{A}$ and $H(k) := \eta^{ap} \eta^{bq} I_{pq}^A H_A k_a k_b$ and $\eta(k) := \eta^{ab}k_a k_b$. 
\begin{comment}
Taking a section $g \in \Gamma(F_{\text{GR}})$ and the corresponding inverse metric $g^{-1} \in \Gamma(F_{\text{GR}}^{\ast})$ with $(g^{-1})^{ab}=:g^{ab}$ and the usual relation $g^{ab}g_{bc}= \delta^a_c$ and evaluating the thus obtained most general expression for the principal polynomial we find
\begin{align}
    \mathcal{P}_{\text{GR,per}} (g_{ab}) = -a \cdot \operatorname{det}(g) g^{ab} g^{cd} k_ak_bk_ck_d + \mathcal{O}(2).
\end{align}
%check notation v vs g
\end{comment}
Note in particular that the vanishing set of this principal polynomial is entirely independent of the choice of the remaining parameter $a$. Thus the causal structure of the perturbative metric gravity EOM is already uniquely fixed by requiring diffeomorphism invariance.

Furthermore, one readily finds that the principal polynomial of the Klein-Gordon equation yields the following linear-order expansion:
\begin{align}
    \mathcal{P}_{KG} = (1 + H/2) \cdot (\eta(k) - H(k)) + \mathcal{O}(2) .
\end{align}
To relate the perturbative vanishing sets of $\mathcal{P}_{\text{GR,per}}$ and $\mathcal{P}_{KG}$ note that we can change the first factor of the two polynomials $(1+H)$ and $(1+H/2)$ respectively at wish by multiplying with the perturbative expansion of the following non-vanishing scalar density of weight $2n$:
\begin{align}
    f_{dens}(n) := \vert \operatorname{det}\left (I^A_{pq}v_A \right )\vert ^n = 1 + nH + \mathcal{O}(2).
\end{align}
This obviously does not alter the vanishing set described by either one of the polynomials. Thus multiplying the gravitational polynomial by $f_{dens}(-1)$ and dividing by the prefactor $a$ and multiplying the matter polynomial by $f_{dens}(-1/2)$ we find:
\begin{align}
    \mathcal{P}_{\text{GR,per}} = (\eta(k)-H(k))^2 = \mathcal{P}_{KG}^2.
\end{align}
Therefore the matter and gravitational polynomial $\mathcal{P}_{KG}$ and $\mathcal{P}_{\text{GR,per}}$ define the same vanishing set in linear order.
Hence the two theories are causal compatible, and our construction procedure is finished. We are left with a perturbative Lagrangian that contains $2$ undetermined constants describing a $2$ parameter family of perturbative metric gravity theories.  

As mentioned earlier in this section Lovelock has shown that the only diffeomorphism invariant metric theory of gravity with second-order EOM is provided by the \textit{\textbf{Einstein-Hilbert}} Lagrangian (\cite{Lovelock1969}, \cite{doi:10.1063/1.1665613}, \cite{doi:10.1063/1.1666069}) that reads:
\begin{align}
    L_{EH} := \kappa \sqrt{\vert \operatorname{det} \left ( I^A_{pq}v_A \right ) \vert }  \left( R + \Lambda \right ),
\end{align}
where the two constant parameters, $\kappa$, and $\Lambda$ are called gravitational constant and cosmological constant respectively. Here $R$ is the so-called Ricci scalar. As we have mentioned earlier in this thesis, one usually first constructs the Riemann tensor, the curvature  $2$-form of the unique metric compatible connection on the frame bundle over $M$. From this one obtains the Ricci tensor by computing its trace. Finally one contracts the Ricci-tensor with the inverse metric to end up with the Ricci scalar $R$. From the point of view of classical field theory, all the deeper connections to Riemannian geometry that are present in the above procedure are, however, actually not necessary. The only relevant fact one really needs is that the Ricci scalar defines a function on $J^2F_{\text{GR}}$. To that end recall that any non-degenerate metric defines a smooth bundle isomorphism:
\begin{align}\label{music}
\begin{aligned}
\flat : TM &\longrightarrow T^{\ast}M\\
X &\longmapsto g(X,-) .
\end{aligned}
\end{align}
Through this construction we therefore also obtain an isomorphism\footnote{At least if we restrict to the subbundles of non-degenerate, symmetric, $(0,2)$ tensors.} between the field bundle $F_{\text{GR}}$ and its dual $F_{\text{GR}}^{\ast}$ that with a slight abuse of notation will also be called $\flat$. 
Using the dual coordinates $(x^m,v^A)$ on $F_{\text{GR}}^{\ast}$ we define $v^A_{\flat} := v^A \circ \flat$. Thus applying $v^A_{\flat}$ on any metric exactly yields the components of its inverse. Using this we get the following explicit expression for $R$ understood as a map on $J^2F$ as:
\begin{align}
R = v_{\flat}^A J_A^{ab} \left ( D_p \Gamma^p_{ba} - D_b \Gamma^p_{pa} + \Gamma^p_{pq} \Gamma^q_{ba} - \Gamma^p_{bq} \Gamma^q_{pq} \right ),
\end{align}
where we also defined the Christoffel symbols that are given by the following functions on $J^1F_{\text{GR}}$:
\begin{align}
\Gamma^a_{bc} = \frac{1}{2} v_{\flat}^A J_A^{ap} \left ( I^B_{bq}\delta^p_b + I^B_{cq}\delta^p_c - I^B_{bc}\delta^p_q  \right ) v_{Bq}.
\end{align}
To compare the computed gravitational Lagrangian with the Einstein-Hilbert Lagrangian expand $L_{EH}$ to third-order around the chosen eta-induced expansion point $(N_{AI})$. The expansion of the determinant can be obtained from (\ref{detExp}). Doing so, we find that:
\begin{align}
   \sqrt{\vert \operatorname{det} \left ( I^A_{pq}v_A \right ) \vert } = 1 + H/2 +1/8 \cdot H^2 - 1/4 \cdot I^A_{ab}I^B_{cd} \eta^{ac} \eta^{bd} H_A H_B + \mathcal{O}(3).   
\end{align}
In order to compute the expansion of $v^A_{\flat}$ one uses that $-1/2 \cdot v_{\flat}^AC_{An}^{Bm}v_B = v_{\flat}^A J_A^{nb}I^B_{bm}v_B = \delta^m_n$. Deriving this equation w.r.t. $v_B$ and performing some simplifications one thus finds the usual expression:
\begin{align}
    \partial^Bv_{\flat}^A = 1/4 \cdot C_{Cn}^{Am}C_{Dm}^{Bn}v_{\flat}^C v_{\flat}^{D}.
\end{align}
Using this one obtains the following perturbative expansion for $v_{\flat}^A$:
\begin{align}
    v_{\flat}^A = \eta^A - \eta^{ap}\eta^{bq} I^B_{ab} I^A_{pq} \cdot H_B + \eta^{ac}\eta^{bd}\eta^{ef} I^B_{ab} I^C_{ce} I^A_{df} \cdot H_BH_C + \mathcal{O}(3).  
\end{align}
With these results computing, this expansion is straight forward yet quite laborious.
Starting with the linear-order of the Einstein-Hilbert Lagrangian, we find:
\begin{align}
        L_{EH} = \kappa \cdot (\eta^{ap}\eta^{bq} - \eta^{ab}\eta^{pq}) I^{A}_{ab}I^{I}_{pq} H_{AI} + \kappa \Lambda \cdot (1 + 1/2 \cdot \eta^{ab} I_{ab}^A H_A) + \mathcal{O}(2).
\end{align}
On the other hand inserting the explicit Lorentz invariant expressions for the expansion coefficients (\ref{LorentzGR1}) with the appropriate multiplicities from the factor free symmetrizations and the computed solution to the perturbative equivariance equations (\ref{GRSol}) into $\mathcal{L}_{\text{GR,per}}$, we find:
\begin{align}
    \mathcal{L}_{\text{GR,per}} = \mu_1(1 + 1/2 \cdot \eta^{ab} I_{ab}^A H_A) - 8 \nu_1 \left(\eta^{ap}\eta^{bq} - \eta^{ab}\eta^{pq} \right )I^{A}_{ab}I^{I}_{pq} H_{AI} + \mathcal{O}(2).
\end{align}
Thus we see that the obtained expansion for $\mathcal{L}_{\text{GR,per}}$ up to linear order precisely yields the Einstein-Hilbert Lagrangian with gravitational constant $\kappa = -8 \nu_1$ and cosmological constant $\Lambda = -1/8 \cdot \frac{\mu_1}{\nu_1}$. Carefully proceeding with the comparison one then finds that also in second and third order the computed perturbative Lagrangian exactly coincides with the expansion of $\mathcal{L}_{EH}$, once we identify the constants as described above. 
Concluding, we constructed a $3$rd-order perturbative expansion of the most general metric gravity Lagrangian that is compatible with a Klein-Gordon scalar field. By employing the perturbative construction algorithm (\ref{Algo2}) and the developed computed algebra (see chapter \ref{computerAlg}) this procedure was really straight forward. 
Further, we saw that, except for the restriction to EOM with principal symbol depending only on the metric, not its derivatives, for this particular case the requirement of causal compatibility did not yield further conditions on top of the imposed diffeomorphism invariance on the metric theory of gravity.
Already the requirement of diffeomorphism invariant dynamics sufficed to recover the perturbative expansion of the Einstein-Hilbert Lagrangian.

\section{Perturbative Area Metric Gravity around a Flat Minkowski Induced Spacetime}
Whereas the first example theory we constructed mainly served the purpose of testing our framework the perturbative theory of gravity that we are going to construct now really is intended to provide a viable alternative to GR in describing the geometry of our spacetime.  
To that end, we are going to construct a perturbative theory of gravity that is compatible with the most general linear theory of Electrodynamics. 

We start by providing some results from the premetric treatment of Electrodynamics that can be found in \cite{1999PhLB..458..466O}, \cite{1999gr.qc....11096H}, (\cite{hehl2003foundations}, \cite{2006physics..10221H}, \cite{2004PhRvD..70j5022L} and also \cite{Hehl2005}). The basic idea of this premetric approach lies in describing all fields that are involved in the treatment of Electrodynamics without referring to an additional spacetime geometry. This is most easily illustrated using the calculus of exterior forms. One starts by introducing the 4-dimensional electric charge current density $J$. As one wishes to integrate $J$ over 3 dimensional submanifold $\Omega$ of the spacetime $M$ to obtain the total electric charge current --- the total electrical current flowing through the boundary of $\Omega$ and the electric charge inside --- that is confined therein the current density is most naturally described as a 3-form $J \in \Gamma(\Lambda^3M)$. Demanding charge conservation in the sense that the integral of $J$ over any closed 3-dimensional submanifold of $M$ vanishes, by using the theorem of stokes on arrives at the requirement $dJ =0$. By de Rham's first theorem we then find that $J$ is necessarily given as exterior derivative of some 2-form $H \in \Gamma(\Lambda^2M)$:
\begin{align}
    J = d H.
\end{align}
From considerations involving the Lorentz force one similarly obtains that the magnetic analog to $J$ the field strength $F$ which plays the role of a 4-dimensional magnetic flux current density should be described by a 2 form $F \in \Gamma(\Lambda^2M)$. Its integral over a 2 dimensional submanifold $C \subset M$ describes the total magnetic flux current in terms of the total magnetic flux flowing through $C$ and the total magnetic flux current\footnote{Note that by this interpretation rewriting $F$ in terms of a 3+1 split of spacetime and the usual $E$ and $B$ field the magnetic flux is as usual given by $B$ whereas $E$ is now interpreted as the corresponding magnetic flux current (see \cite{2006physics..10221H}).} flowing through the boundary of $C$. The corresponding conversation law deduced in similar fashion as before then reads $dF =0$. Thus we conclude using again de Rham's first theorem the existence of a 1 form $A \in \Gamma(\Lambda^1M)$ that satisfies:
\begin{align}
    F = d A.
\end{align}
Finally one assumes that the two 2-forms $F$ and $H$ are related via a local and linear spacetime relation. Writing the 2 forms in terms of some chart induced basis $F = F_{ab} = \mathrm{d}x^a \otimes \mathrm{d}x^b$ with $F_{ab} = - F_{ba}$ and similar for $H$ specifying such a relation requires an additional $(0,4)$ tensor field $G$ that obeys the following symmetries:
\begin{align}\label{areaSym}
    G_{abcd} = -G_{bacd} = G_{cdab}.
\end{align}
Note that such a tensor field at each spacetime point $p\in M$ defines a symmetric inner product on the space of $2$-forms $\Lambda^2_pM$. Consequently we will call such a tensor field an \textbf{\textit{area metric}} tensor field and denote the bundle of these area metric tensors by $F_{\text{Area}}$. Due to (\ref{areaSym}) one need $21$ independent component functions to uniquely specify such an area metric. Thus $F_{\text{Area}}$ has $21$ dimensional fibers.   In the following we restrict the bundle of such area metric tensor fields $G$ by requiring that the inner product they define is non degenerate for all $p$ in the sense that $G$ provides us with a bundle isomorphism: 
\begin{align}
\flat_{\text{Area}} : \Lambda^2M \longrightarrow (\Lambda^2M)^{\ast} = A^2_0M,
\end{align}
that is defined exactly the same way as in the metric case (\ref{music}). Here $A^2_0M$ denotes the bundle of antisymmetric $(2,0)$ tensors. 

The linear relation between $H$ and $F$ can now be obtained by first using the isomorphism provided by the area metric to map the 2-form $F$ to an antisymmetric $(2,0)$ vector field and then applying the Hodge star operator\footnote{Details regarding the construction and properties of the Hodge start operator can be found in \cite{Abraham:1988:MTA:50877}, and also in \cite{nlab:Hodge}.} to again end up with a 2 form. In coordinates we therefore get:
\begin{align}
    H_{ab} = 1/2 \omega_G \epsilon_{abcd} G^{cdef} F_{ef},
\end{align}
Note that in absence of a (pseudo)-Riemanian metric the Hodge star operator requires the choice of a volume form (or equivalently a scalar density of weigh 1) $\omega_G$. One possible such choice is for instance given by $\omega_G = \epsilon^{abcd}G_{abcd}$ which obviously only works if this expression is nowhere vanishing.

From considering the appropriate dimensions of $F$ and $H$ on finds that the only viable Lagrangian constructed from $F$ and $H$ is given by $F \wedge H$. Expressing this in terms of the constitutive relation between $H$ and $F$ we finally arrive at the Lagrangian of so-called \textbf{\textit{General Linear Electrodynamics}} (in short GLED):
\begin{align}
    \mathcal{L}_{GLED} =  G^{abcd}F_{ab}F_{cd}\omega_G\mathrm{d}^4x.
\end{align}
Note that this really defines the most general quadratic Lagrangian which thus generates linear EOM that one can specify in terms of the electromagnetic field strength 2-form $F$. In particular this Lagrangian contains the standard Maxwell electrodynamics on a metric background for the special case $G^{abcd} = 2 g^{c^[a}g^{b]d}$ and $\omega_{G}=\sqrt{\vert -\operatorname{det}(g) \vert}$.

Just as it was the case for the Klein-Gordon Lagrangian for the Lagrangian of GLED there exist now two possibilities: either one specifies the area metric tensor field by hand, or --- and this is the route that we want to take --- one supplements the GLED matter theory with a suitable theory of area metric gravity that then allows one to determine the area metric by solving the corresponding area metric gravity EOM. 
In the following, we are going to construct the perturbative expansion of such a theory of are metric gravity by following our construction recipe algorithm \ref{Algo2}.

To that end we quickly want to translate $\mathcal{L}_{GLED}$ into the language of jet bundles. Note that the electromagnetic field strength can be expressed in terms of the 1-form $A \in \Gamma(\Lambda^1(M))$ as:
\begin{align}
F = F_{ab} \cdot  \mathrm{d}x^a \otimes \mathrm{d}x^b = 2 \partial_{[a} A_{b]} \cdot \mathrm{d}x^a \otimes \mathrm{d}x^b.
\end{align}
Thus $F$ can be understood as a function on the first-order jet bundle $J^1(\Lambda^1M)$. We again use adapted coordinate $(x^m,A_a,A_{am})$ on this bundle. Together with the area metric field bundle $F_{\text{Area}}$ and the bundle isomorphism $\flat_{\text{Area}}$ we thus see that the GLED Lagrangian describes a bundle map:
\begin{align}
\mathcal{L}_{GLED} : F_{\text{Area}} \oplus J^1(\Lambda^1M) \rightarrow \Lambda^4M,
\end{align}
where now also $\omega_G$ is understood as a bundle map on $\Lambda^1M$.  

We start the derivation of perturbative area metric gravity by constructing the pair of intertwiners for the area metric field bundle $F_{\text{Area}} \subset T^0_4M$. Recall from chapter 1 that given coordinates $(x^m)$ on $M$ one can obtain a pair of intertwiners for such a subbundle of $T^0_4M$ by dividing the set of the $4^4$ fiber coordinates of $T^0_4M$ $v_{abcd}=\frac{\partial}{\partial x^a} \otimes ... \otimes \frac{\partial}{\partial x^d}$ in equivalence classes defined by the symmetries (\ref{areaSym}) sorting them according to some order relation and then constructing the pair of intertwiners as described in (\ref{defI}) and (\ref{defJ}). We will use the following ordered set of $21$ equivalence classes $[v]$ that are specified in terms of their representative $v$ as:
\begin{align}
\begin{aligned}
    \bigl [ [v_{0101}], [v_{0102}], [v_{0103}], [v_{0112}], [v_{0113}], [v_{0123}], [v_{0202}], [v_{0203}], [v_{0212}], [v_{0213}], [v_{0223}],\\
     [v_{0303}], [v_{0313}], [v_{0323}], [v_{1212}], [v_{1203}], [v_{1213}], [v_{1223}], [v_{1313}], [v_{1323}], [v_{2323}]  \bigr ].
\end{aligned}
\end{align}
We label the equivalence classes by indices $A$ that now run from $0$ to $20$.
There are now two possibilities for the individual classes: either we have an equivalence class with representative of the form $v_{ijij}$, i.e., the equivalence class consists of coordinate functions that only have two distinct index values $i$ and $j$. Such a class then consists of the following $4$ elements:
\begin{align}
    [v_{ijij}] = \{ v_{ijij}, v_{jiji}, -v_{ijji}, -v_{jiij} \}.
\end{align}
Thus the multiplicity $\sigma$ is 4 for these classes.
The other possibility is given when the representative of the equivalence class takes the general form $v_{ijkl}$ where $i<j$ or $(i=j) \land (k<l)$. The equivalence class consists then of 8 elements:
\begin{align}
    [v_{ijkl}] = \{v_{ijkl},v_{jikl}, v_{klij}, v_{lkji}, -v_{jikl}, -v_{ijlk}, -v_{lkij}, v_{klji} \}.
\end{align}
Such classes have multiplicity 8. The intertwiners $I^A_{abcd}$ and $J_A^{abcd}$ can now be constructed according to (\ref{defI}) and (\ref{defJ}). The explicit non-vanishing components of these intertwiners are displayed in the appendix (\ref{AreaI}) and (\ref{AreaJ}). Using these we define the new adapted coordinate functions on $F_{\text{Area}}$ and its dual $F_{\text{Area}}^{\ast}$:
\begin{align}
    \begin{aligned}
    v_A = J_A^{abcd}v_{abcd} \\
    v^A = I^A_{abcd}v^{abcd}.
    \end{aligned}
\end{align}
As usual we are going to label the corresponding adapted coordinates on $J^2F_{\text{Area}}$ by $(x^m,v_A,v_{Ap},v_{AI})$. 
Note that from the fact that the fiber dimension of $F_{\text{Area}}$ equals $21$ whereas $F_{\text{GR}}$ only has fiber dimension $10$ one readily sees that the area metric indeed compromises a richer structure than the metric. It is, however, not a priory clear if this is also reflected in the corresponding theory of gravity. Recall that the metric allowed for $14$ functionally independent curvature scalars on $J^2F_{\text{GR}}$ and hence also for $14$ functionally independent diffeomorphism invariant Lagrangians. With theorem (\ref{GeneralSol}) at hand, we can now immediately compute the corresponding quantity for the area metric. Doing so we find that there exist $179$ functionally independent diffeomorphism invariant Lagrangians that can be constructed from the area metric and its first two derivatives. Thus, in general, we expect the perturbative theory of area metric gravity to contain more undetermined parameters than the gravitational and cosmological constant that we obtained for the metric case.


Next we are going to chose the $\eta$-induced expansion point $(N_{AI}) = (x_0^m, N_A, 0,0)$ to be provided by:
\begin{align}
N_A := 2 \eta_{c[a} \eta_{b]d} - \epsilon_{abcd}.
\end{align}
Not that assuming that $\omega_G(N_A) = const \neq 0$ inserting this into $\mathcal{L}_{GLED}$ one really obtains the Lagrangian of standard Maxwell Electrodynamics on a flat Minkowskian background:
\begin{align}
    \mathcal{L}_{GLED}\bigl((N_{AI})\bigr ) = 2 \omega_G(N_A) \eta^{ac}\eta^{bd}F_{ab}F_{cd} \mathrm{d}^4x.
\end{align}
Thus expanding the to-be-constructed Lagrangian of area metric gravity around $(N_{AI})$ serves the purpose of interpreting the result as perturbative theory around a flat spacetime. Note that we included the totally antisymmetric $\epsilon_{abcd}$ in the definition of $N_A$ to ensure that  the possible density choice $\omega_G = \epsilon^{abcd}G_{abcd}$ does not vanish when evaluated at the flat $\eta$-induced background. 

We have already stated that we want to construct the perturbative are metric gravity theory up to order $k=3$. Following algorithm \ref{Algo2}, we proceed by computing the coordinate expression of the Lie derivative of $G_{abcd}$ to read of the vertical coefficient tensor $C_{An}^{Bm}$:
\begin{align}\label{LieArea}
    \mathcal{L}_{\xi} G_{abcd} = \partial_m G_{abcd} \cdot \xi^n + \left (-4 \delta_n^{[e\vert} \delta_{[\mathnode{a}}^m \delta_{b]}^{\vert \mathnode{f} ]} \delta_{[c}^{[g} \delta_{\mathnode{d}]}^{\mathnode{h}]} \right ) \cdot G_{efgh} \partial_m \xi^n.
\end{align}
\begin{tikzpicture}[overlay]
   \path [>=stealth, <->, shorten <= 3pt, shorten >=3pt]
     (a) edge [bend left=-80] (d);
    \path [>=stealth, <->, shorten <= 2pt, shorten >=2pt]
     (f) edge [bend left=80] (h);
\end{tikzpicture}

Here the additional arrows denote the symmetrization w.r.t. the interchange of the antisymmetric pairs $[ab] \leftrightarrow [cd]$ and $[ef] \leftrightarrow [gh]$ respectively.

Similar to before we now insert several intertwiners $I^A_{abcd}$ and $J_A^{abcd}$ to rewrite the second term in (\ref{LieArea}) in terms of abstract area metric indices $A$. Here we can use the fact that the two intertwiners already feature the full set of area metric symmetries.  Doing so we simply can read of the constant tensor as:
\begin{align}\label{areaGotayMInter}
    C_{An}^{Bm} = -4 I^B_{nbcd} J_A^{mbcd}.
\end{align}
The most general power series expansion of $\mathcal{L}_{\text{Area,per}} = L_{\text{Area,per}}\mathrm{d}^4x$ that generates second-derivative-order EOM is again given by:
\begin{align}\label{LArea}
    L_{\text{Area,per}} =  a_0 + a^A H_A + a^{AI}H_{AI} + a^{AB} H_{A}H_{B} + a^{ApBq} H_{Ap}H_{Bq} + a^{ABI} H_{A} H_{BI} \\
    + a^{ABC} H_a H_B H_C + a^{ABpCq} H_{A}H_{Bp}H_{Cq} +
    + a^{ABCI} H_A H_B H_{CI} 
    + \mathcal{O}(4),
\end{align}
where as before we introduced the coordinate deviation from the expansion point: 
\begin{align}
(H_{AI}) = (x^m,H_A,H_{Ap},H_{AI}) = (v_{AI}) - (N_{AI}).
\end{align}
Note that although we use the same symbols as in $L_{\text{GR,per}}$ to denote $L_{\text{Area,per}}$ this Lagrangian defines a fundamentally different object. In particular the $21$ dimensional fibers of  $F_{\text{Area}}$ result in the fact that there will be more freedom in constructing the expansion coefficients of (\ref{LArea}). Again the perturbative equivariance equations demand that these expansion coefficients define the components of constant Lorentz invariant tensors. The corresponding expressions that we obtained by means of our computer program are displayed in the appendix in table \ref{LorentzArea}. The dimensions and parameters of the individual expansion coefficients are displayed in table \ref{AreaExp}.
\begin{table}
\centering 
\begin{tabular}{lll} \toprule
    expansion coefficient & dimension & constants   \\ \midrule
    $a_0$ & 1 & $\{\mu_1\}$ \\
    $a^A$ & 2 & $\{\mu_2,\mu_3\}$ \\
    $a^{AI}$ & 3 & $\{\nu_1,..., \nu_3\}$ \\
    $a^{AB}$ & 6 & $\{\mu_4,..., \mu_9 \} $ \\
    $a^{ApBq}$ & 15 & $\{\nu_4,...,\nu_{18}\}$ \\
    $a^{ABI}$ & 16 & $\{ \nu_{19},...,\nu_{34} \}$ \\
    $a^{ABC}$ & 15 & $\{ \mu_{10},...\mu_{24} \}$\\
    $a^{ABpCq}$ & 110 & $\{\nu_{35},...,\nu_{144} \}$ \\
    $a^{ABCI}$ & 72 & $\{ \nu_{145},...,\nu_{216}\}$ \\ \bottomrule
\end{tabular}
\caption{Dimensions of the Lorentz Invariant Expansion Coefficients for $\mathcal{L}_{\text{Area,per}}$.}\label{AreaExp}
\end{table}
In total, we thus get from solving this first implication of the perturbative equivariance equations, the Lorentz invariance of the expansion coefficients, $240$ yet undetermined constant parameters. These can be divided into $24$ masses and $216$ kinetic parameters.

It is worth noting that the perturbative equivariance equations (\ref{order1}), (\ref{order2}) and (\ref{order3}) decouple into the two subsystems containing only the mass term and kinetic term coefficients, respectively. Once we insert the obtained Lorentz invariant expression, the former yields a linear system that is compromised of $14$ independent equations for the $24$ masses. The latter contains $174$ independent equations for the $216$ kinetic parameters. Thus in total, we get $188$ independent linear equations that therefore reduce the number of undetermined parameters that are contained in $\mathcal{L}_{\text{Area,per}}$ from $240$ to $52$. We get a total of $10$ masses and $42$ kinetic parameters. The solution to the linear system in terms of the $52$ parameters (\ref{AreaParas}) is displayed in the equations table \ref{AreaSol}. Although the perturbative Lagrangian that we thus obtain is clearly more complicated then the expansion of the Einstein-Hilbert action and in particular now contains $52$ undetermined parameters compared to $2$ gravitational constants that contribute to $\mathcal{L}_{\text{GR,per}}$ we can now see the real strength of the required diffeomorphism invariance that is encoded in the perturbative equivariance equations (\ref{order1}), (\ref{order2}) and (\ref{order3}). To that end, it is vital to observe that a priori the expansion (\ref{LArea}) allows for a total of $133694$ undetermined parameters. It is remarkable that solely the requirement of diffeomorphism invariance reduces this number to the $52$ parameters that are left in the obtained solution. 

Following algorithm \ref{Algo2} we now compute the principal polynomials of the EOM generated from $\mathcal{L}_{GLED}$ and $\mathcal{L}_{\text{Area,per}}$ to deduce whether the two theories are already causally compatible or this second requirement yields further conditions for the $52$ gravitational constants in the perturbative area metric gravity Lagrangian. This time we start by computing the GLED polynomial. It was first computed by Rubilar (see \cite{2009JPhA...42U5402I}) in the context of premetric Electrodynamics and is given by the following expression with corresponding first-order expansion: 
\begin{align} \label{GLEDPoly}
\begin{aligned}
    \mathcal{P}_{GLED} &= -\frac{1}{24}\omega_G^2\epsilon_{mnpq}\epsilon_{rstu}J_A^{mnra}J_B^{bpsc}J_C^{dqtu} v_{\flat_{\text{Area}}}^A v_{\flat_{\text{Area}}}^B v_{\flat_{\text{Area}}}^C k_ak_bk_ck_d \\
                &= \bigl[ 1 -  A \eta(H)- \frac{1}{2} A \epsilon(H) + \frac{1}{12} \epsilon(H) \bigr] \eta(k)^2 - H(k)\eta(k) + \mathcal{O}(2)\\
                &= \bigl\{  \bigl[ 1 - \frac{1}{2} A \eta(H) - \frac{1}{4} A \epsilon(H) +  \frac{1}{24} \epsilon(H) \bigr] \eta(k) - \frac{1}{2} H(k)       \bigr\}^2 + \mathcal{O}(2).
\end{aligned}
\end{align}
Here we defined $v_{\flat_{\text{Area}}}^A = v^A \circ \flat_{\text{Area}}$ and used $v_{\flat_{\text{Area}}}^A J_A^{abcd} I^B_{cdef}v_B = 4 \delta^{[a}_e \delta^{b]}_f$ to expand $v^A_{\flat_{\text{Area}}}$ around $N_A$. Further, we defined:
\begin{align}
\begin{aligned}
\eta(H) &:= \eta^{ac}\eta^{bd} I^A_{abcd} H_A, \\ 
\epsilon(H) &:=\epsilon^{abcd}I^A_{abcd}H_{A},\\ H(k) &:=\eta^{ac}\eta^{bp}\eta^{cq} I^A_{abcd}H_Ak_pk_q, \\ \eta(k)&:=\eta^{pq}k_pk_q.
\end{aligned}
\end{align}
Besides, $A$ denotes a constant that depends on the specific choice of density $\omega_G$ in the GLED Lagrangian.  

We now compute the principal polynomial of the constructed diffeomorphism invariant perturbative theory of area metric gravity. To that end, we start by inserting the calculated solution of the expansion coefficients in to (\ref{EOMPert}) to obtain the explicit expression of the perturbative EOM. From this, we can simply read off the principal symbol. Following further along the lines described in the last chapter, we find the following expression for the area metric gravity polynomial:
\begin{align} \label{AreaPoly}
\begin{aligned}
    \mathcal{P}_{\text{Area}} &= \bigl[1 - \frac{1}{2} \tilde{C}\eta(H) - \frac{1}{4} \tilde{C} \epsilon(H) + \frac{7}{12} \epsilon(H) \bigr] \eta(k)^{13} - \frac{13}{2}H(k) \eta(k)^{12} + \mathcal{O}(2) \\
    &=\bigl\{  \bigl[ 1 - \frac{1}{2} C \eta(H) - \frac{1}{4} C \epsilon(H) +  \frac{7}{12\cdot13} \epsilon(H) \bigr] \eta(k) - \frac{1}{2} H(k)       \bigr\}^{13} + \mathcal{O}(2).
\end{aligned}
\end{align}
Here $C$ is a constant the depends on the parameters that are left in (\ref{LArea}) and $\Tilde{C} = 13 C$.
There is a slight caveat in the computation of this expression for the area metric gravity polynomial. 
To explain this further first note that since there still appear $52$ constant parameters in the area metric gravity Lagrangian in a rather complicated fashion, it is virtually impossible to derive the principal polynomial without employing efficient computer algebra. Once we have selected a $17 \times 17$ submatrix of the $21 \times 21$ principal symbol matrix we have to compute the two expressions, $D_0$ and $D_{1}^C$ in (\ref{polyMatrices}) in order to find the constant and linear order of the expansion of the principal polynomial. Note that the involved matrices no longer depend on $H_A$. To compute $D_0$ and $D_1^A$ we now need the determinant and the inverse of the constant order $T_0$ of the chosen $17 \times 17$ symbol submatrix. These two steps precisely represent the bottleneck of the computation of the polynomial. Although $T_0$ does not depend on $H_A$, it still includes some of the $52$ parameters, namely precisely those that are left in $a^{ApBq}$ and $a^{ABI}$ after solving the equivariance equations and contribute to the principal symbol. We find that these still consist of $7$ parameters. Although this might not sound much computing, the determinant or the inverse of a $17 \times 17$ matrix with $7$ symbolic parameters that appear in rather complicated fashion turns out to be surprisingly hard. We tried to tackle this problem by using a fraction free implementation of Gaussian elimination in the computer algebra system Maple\footnote{We expect that one might achieve even superior performance in computing the determinant by using techniques from multivariate polynomial interpolation as it is for instance suggested in \cite{Qin2018}, \cite{MARCO2004749} and also \cite{articleDet}} \cite{Maple}.

Nevertheless, with the available resources, the program did not terminate in adequate time. Thus the only way we nevertheless were able to obtain an expression for the polynomial was by evaluating the $7$ parameters that appear in $T_0$ at randomly generated integer values and then only computing the principal polynomial with the remaining parameters from the two third-order coefficients $a^{ABpCq}$ and $a^{ABCI}$. Taking a closer look at (\ref{polyMatrices}) these parameters only appear in traces, not in determinants or inverses. We then proceeded by performing this computation to obtain the principal polynomial several times, each time taking newly obtained random integers for the $7$ parameters in $T_0$. Doing so, we observed that they only contribute an overall factor to the principal polynomial (\ref{AreaPoly}). We have to admit that this was not done often enough to rigorously interpolate the precise way the prefactor depends on these $7$ parameters. This is however not necessary as we are only interested in the vanishing set of the area metric gravity polynomial. 

Note that, as we are only interested in the vanishing set of the principal polynomial, we can multiply a given polynomial with a non-vanishing scalar density of arbitrary weight. The expansion to linear order of such a density admits the general form:
\begin{align}\label{dens}
\omega = 1+ b_1 \cdot (\eta(H) + \frac{1}{2} \epsilon(H)) + \frac{b_2}{12}\epsilon(H) + \mathcal{O}(2),
\end{align}
for arbitrary constants $b_1$ and $b_2$.
Multiplying  (\ref{GLEDPoly}) with such a density we see that perturbatively the GLED polynomial describes the same vanishing set as:
\begin{align} \label{GLEDPoly2}
\begin{aligned}
    \widetilde{\mathcal{P}}_{GLED}(k) = \omega \cdot \mathcal{P}_{GLED} = 
    \bigl\{  \bigl[ 1 - \frac{1}{2} (A-b_1) \eta(H) - \frac{1}{4} (A-b_1) \epsilon(H)) + \\ \frac{1+b_2}{24} \epsilon(H) \bigr] \eta(k)
    -\frac{1}{2} H(k)       \bigr\}^2 + \mathcal{O}(2).
\end{aligned}
\end{align}
Comparing this expression with the previously obtained area metric gravity polynomial (\ref{AreaPoly}), we find that multiplying the GLED polynomial with a density with constants $b_1 = A -C$ and $b_2 = \frac{1}{13}$, the two polynomials are products of the same factor and therefore, in particular, describe the same vanishing set in $\mathcal{O}(2)$. Thus also for the case of area metric gravity requiring causal compatibility with the GLED mater theory in addition to diffeomorphism invariance at least perturbatively yields no further conditions for the parameters of $\mathcal{L}_{\text{Area,per}}$.

Finally, we wish to investigate if this was, in fact, a coincidence of the chosen GLED matter theory or a general feature of perturbative area metric gravity independent on the specific matter theory. To this end, we take a closer look at the consequences of the required diffeomorphism invariance on the causal structure of the corresponding area metric gravity EOM by studying general solutions to (\ref{polyEqn}). This equation now takes the following form:
\begin{align}\label{AreaPolyEqn}
    0 = \partial^A \mathcal{P}_{\text{Area}}^{p_1...p_{26}} C_{An}^{Bm} v_B - 26\mathcal{P}_{\text{Area}}^{(p_1...p_{25}\vert m} \delta_n^{\vert p_{26})} + 13 \mathcal{P}_{\text{Area}}^{p_1...p_{26}} \delta^m_n.
\end{align}
Proceeding as before for the metric in computing the general power series solution up to linear order in $H_A$ for this equation and contracting it against $k_{p_1} \cdot k_{p_{26}}$ we get the most general linear-order expansion of $\mathcal{P}_{\text{Area}}$ that is consistent with the required diffeomorphism invariance:
\begin{align}
\begin{aligned}
    \mathcal{P}_{\text{Area}} &= a \cdot \bigl \{  \eta(k)^{13} + \tilde{b} \cdot \eta(H) \cdot \eta(k) - \frac{13}{2} \cdot  H(k) \cdot \eta(k)^{12} + (\frac{\tilde{b}}{2}+\frac{14}{24}) \cdot \epsilon(H) \eta(k)^{13}  \bigr \} + \mathcal{O}(2)\\
    &=a \cdot \bigl\{  \bigl[ 1 + b \eta(H) + \frac{b}{2} \epsilon(H) +  \frac{7}{12\cdot13} \epsilon(H) \bigr] \eta(k) - \frac{1}{2} H(k)       \bigr\}^{13} + \mathcal{O}(2),
\end{aligned}
\end{align}
with constants $a$, $b$ and $\tilde{b} = 13b$. It is clear that the overall constant $a$ does not influence the vanishing set. Additionally we can multiply this expression with with the general form of a non-vanishing scalar density of arbitrary weight (\ref{dens}) that in linear order allows for $2$ arbitrary parameters $b_1$ and $b_2$. Doing so we find:
\begin{multline}
    \tilde{\mathcal{P}}_{\text{Area}} = \bigl\{  \bigl[ 1 + (b+ b_1) \eta(H) + \frac{b+b_1}{2} \epsilon(H) +  (\frac{7}{12\cdot13}+b_2) \epsilon(H) \bigr] \eta(k) - \frac{1}{2} H(k)       \bigr\}^{13} \\
    + \mathcal{O}(2).
\end{multline}
This polynomial now obviously perturbatively describes the same vanishing set. However, now we see that we can, in fact, specify the remaining constant $b$ at wish just by multiplying with an appropriate density and hence in particular without changing the vanishing set. 
Thus the fact that the requirement of causal compatibility between are metric gravity and GLED let to no further condition was no coincidence but a general feature of the perturbative expansion of area metric gravity
We conclude that for the case of the third-order perturbative area metric gravity Lagrangian already the required diffeomorphism invariance determines the vanishing set of the principal polynomial and thereby the causal structure if the theory uniquely. 

Hence the construction procedure is completed. The most general meaningful third-order expansion of the Lagrangian of area metric gravity thus involves $52$ undetermined parameters. Its explicit expression can be obtained by inserting the relations between the parameters displayed in table \ref{AreaSol} together with the computed Lorentz invariant expansion coefficients in the third-order expansion (\ref{LArea}). It is of great importance to be aware of the meaning of this result. This really compromises the most general perturbative, diffeomorphism invariant, second-derivative-order theory of gravity that is compatible with a linear theory of Electrodynamics. As such, it is a particularly exciting candidate to test against nature, for instance, by predicting gravitational wave emission\footnote{Such predictions will be discussed in \cite{NilsPHD}.} of some known binary system and then comparing this prediction with experiments. It is essential to note that such a prediction of the emission of gravitational waves really requires the third order of the gravitational Lagrangian. Although often stated differently, this is already true for standard GR (see for instance \cite{1984grra.book.....S} chapter 4.5).

Summing up the results that we developed in this chapter, not only did our perturbative Constructive gravity framework pass its first test in successfully recovering the third-order expansion of the Einstein-Hilbert Lagrangian, with the perturbative expansion of area metric gravity we also have developed the most general perturbative theory of gravity that is consistent with linear Electrodynamics. Although the expansion of such a theory to second order in the Lagrangian has already been computed in \cite{2017arXiv170803870S}, the expansion to third order that we derived here for the first time provides access to the computation of gravitational wave in such a theory of gravity and thereby allows us to concretely test the area metric against measurements. For precisely that reason, the computation of the emission of gravitational waves should compromise the main focus of future developments in this area of research. 