Summing up, in this thesis, we have in-depth surveyed the central problem of Constructive Gravity: we have provided a reasoned, well-structured method for the construction of new theories of gravity. We have translated the crucial requirements of diffeomorphism invariant dynamics and causal compatibility between matter and gravitational field equations into rigorous mathematics. Diffeomorphism invariance was encoded as an equivalent system of linear, first-order partial differential equations for the gravitational Lagrangian. The causal compatibility requirement was further included by enforcing equality of the vanishing sets of the matter and gravitational principal polynomials.
This finally allowed us to arrive at the first main result that we presented in this thesis: the construction recipe (algorithm \ref{Algo1}) that takes an arbitrary matter theory which is supported by a tensorial background geometry as input data and then provides a step-by-step manual for the construction of the most general diffeomorphism invariant, compatible theory of gravity. 

Although, this achievement is unarguably of great importance, it suffers from the fact that for nearly all relevant examples it is strikingly hard to obtain general solutions to the 
PDE system that encodes diffeomorphism invariance. 
This fact forced us to take a slight detour into the field of formal PDE theory, to collect information and tools how one might nevertheless obtain solutions to such a PDE system. 
With the detection of gravitational waves as a possible test for alternative theories of gravity in mind, we decided to take the route of developing perturbative solution techniques\footnote{As remarked in the main part of the thesis, it would certainly be possible to apply symmetry reduction techniques for obtaining a solution to this PDE. This could then, for instance, yield a cosmological approximation of the specific alternative gravity theory. A first step in this direction is described in \cite{ToBePublished}. Without a doubt, this constitutes an exciting area of future research that might build upon the foundation laid by this thesis.} to such PDEs. These can be computed through power series expansions of the Lagrangian. By proving involution of the equivariance equations, we further ensured that all information that is contained in this PDE system is also included in the construction of perturbative solutions. This in the end justifies the perturbative approach. 
Finally, we have also cast the requirement of causal compatibility into an equivalent perturbative form. The perturbative approach to Constructive Gravity culminated in the formulation of yet another construction manual (see algorithm \ref{Algo2}), which represents the perturbative equivalent to algorithm \ref{Algo1}. Providing a given matter theory with tensorial background geometry, an expansion point for this geometry and a chosen expansion order as input data, in algorithm \ref{Algo2} we outlined in detail which steps one has to undertake in order to derive the most general, compatible, diffeomorphism invariant, perturbative theory of gravity. The all-important observation now lies in the fact that this construction algorithm almost only involves linear algebra. Although the resulting problems represent a technical challenge, as once considering real generalizations to GR, the dimensions of the resulting linear systems become quite sizeable, using modern computer technique they are certainly solvable for many concrete examples. 

We further developed a computer algebra library, written in the programming language Haskell, that is explicitly designed to deal with problems that arise in this context, but moreover, may also suit as a general-purpose tensor algebra computer program. 
In this thesis, we explained the main ideas that underlie the incorporated data structures and implemented algorithms. Additionally, we also provided a short step-by-step user guide. 

Finally, in this thesis, we also considered two real-world applications of the developed perturbative framework. In particular, the two examples, traditional metric gravity and the more general area metric gravity, are not to be confused with simple toy theories, but should rather be seen as real candidates for the description of gravitational interactions. In both examples, we explicitly followed the steps required by algorithm \ref{Algo2} in constructing a perturbative expansion of the relevant Lagrangians up to third order. In the metric case, by interpreting the two arising constants as gravitational and cosmological constant, we successfully recovered the third-order expansion of the Einstein-Hilbert-Lagrangian.
Thus, the perturbative framework passed its first test.
In the case of area metric gravity there obviously does not exist any established reference theory to which we can compare the computed result. In this case, we obtained a theory that contained $52$ undetermined gravitational parameters. 
With such a third-order expansion of the area metric gravity Lagrangian, it is now possible to not only calculate the propagation, but most importantly, also the emission of gravitational waves in the context of this theory of gravity. Thus, in principle, it is straight forward how one can compare area metric gravity against real observations. This indeed is precisely where I personally hope that future research will build upon. 
As we did not include any ad hoc assumptions in the construction of the area metric gravity Lagrangian any failed comparison to experiment will not only rule out a particular, specific candidate theory but a whole family of such. In other words, comparing the constructed area metric Lagrangian against observations does not only test a particular modified theory of gravity but the whole idea of an area metric as describing gravitational interactions. 
On the other hand, observations that do not entirely contradict the calculations that one can compute in the context of the developed theory of area metric gravity will yield essential knowledge regarding possible boundaries for the $52$ yet undetermined parameters that are contained therein. 

Along the same lines that we followed in constructing perturbative metric and area metric gravity theories, one can readily compute further alternative gravity theories.
Thus,I also hope that the framework that was presented in this thesis will help in developing additional alternative theories of gravity. At least in the perturbative setting, with the developed computer algebra program it is mostly a matter of days to obtain gravitational Lagrangians for any given tensorial geometry. The real challenge thus is to find worthy candidates of spacetime geometries. 

Finally, I hope that the Haskell library that was developed in the context of this thesis will be further supplemented and improved. Also in future, I  plan to actively maintain and enhance it, personally. One particular thing that we already have in mind is providing bindings to a symbolic simplifier such as \cite{SymPy}, thus further allowing for symbolic tensor values. This would then really complete sparse-tensor to a universal tensor algebra library.

With this, I would like to end my thesis. I sincerely hope that the presented results contribute towards a more structured treatment of modifying gravity, eliminated some of the included guesswork and thus in the sense of Rovelli advance science, at least to some extent.
