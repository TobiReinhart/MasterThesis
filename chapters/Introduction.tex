\setlength{\epigraphwidth}{0.6\textwidth}
\epigraph{"...theoretical physics has not done great in the last decades. Why? Well, one of the reasons, I think, is that it got trapped in a wrong philosophy: the idea that you can make progress by guessing new theory and disregarding the qualitative content of previous theories. This is the physics of the "why not?" Why not studying this theory, or the other? Why not another dimension, another field, another universe? Science has never advanced in this manner in the past. Science does not advance by guessing."}{\textit{Carlo Rovelli}}

Over 100 years ago in 1915, Einstein first published his theory of General Relativity \cite{1915SPAW.......844E}, and still today it constitutes a widely accepted description of the gravitational interaction. 
At least in parts, this is undoubtedly caused by the remarkable agreement of General Relativity with experiment. 
From early predictions such as Einstein's explanation of the observed perihelion precession of Mercury \cite{Einst2}, and the deflection of light caused by the sun \cite{Will2006}, to more recent observations including the exceptional detection of gravitational waves \cite{2016PhRvL.116f1102A}, Einstein's General Relativity has survived almost all tests that were thrown at it.

However, there also exist observations that cannot be described by General Relativity completely satisfactory. In 1998, discoveries by the Supernova Cosmology Project \cite{1999ApJ...517..565P} and the High-Z Supernova Search Team \cite{1998AJ....116.1009R} revealed that the expansion of the universe is accelerating over time. For such an accelerated expansion of the universe to be consistent with General Relativity, one needs to subject Einstein's theory to modifications. In order to account for it, one either appends an additional constant, the cosmological constant, or an additional scalar field to the theory. The entirety of these additional quantities that are added to General Relativity to incorporate the accelerated expansion of the universe is called \textit{\textbf{dark energy}}.

Furthermore, observations of galaxy rotation curves from the 1960s and 1970s \cite{1970ApJ...160..811F}, \cite{1970ApJ...159..379R} and \cite{1980ApJ...238..471R} thoroughly contradict the predictions obtained from General Relativity. If one nevertheless wants to explain them in the context of Einstein's theory of gravity, the observations require most galaxies to contain a large amount of un-observable \textit{\textbf{dark matter}}, in addition to their visible matter. Thus even further, auxiliary adjustments to General Relativity are necessary. 

The most prominent General Relativity based cosmological model that includes both dark energy and dark matter is the so-called cosmological standard model or Lambda-CDM model. Based on this model, measurements of the cosmic microwave background can be used to determine to which extent the energy density in the observable universe is composed of standard model matter and to which extent additional sources of dark matter and dark energy contribute. The Planck mission \cite{Planck13_1}, \cite{Planck13_2}, \cite{Planck15} and \cite{Planck18}, a recent such measurement series, revealed that standard model matter only compromises $4.82\pm0.05\%$ of the total energy contained in the observable universe, whereas dark matter accounts for $ 25.8\pm0.4\%$ and dark energy even for $ 69 \pm 1 \%$. Thus dark energy and dark matter combined contribute an astonishing total of nearly $95 \%$ to the total energy contained in the present observable universe.
Justifiably it seems inappropriate that two adjustments that ad hoc are added to an existing theory to render it consistent with otherwise contradicting observations in the end account for the description of nearly $95\%$ of the theories "content".  

Regarding this observation it is no surprise that there exist countless proposed alternatives\footnote{There exist theories that describe the gravitational field as a scalar field \cite{Scalar1} \cite{Scalar2}, theories that use a combination of scalar and tensor field \cite{ST1}, \cite{ST2}, \cite{ST3}, theories that employ scalar, vector and tensor field \cite{SVT1}, \cite{SVT2}, theories that use the usual metric but different EOM \cite{fR1}, \cite{fR2} and even theories that describe the gravitational field by two metric tensors \cite{BIM1} and \cite{BIM2} to provide a few examples.} to General Relativity, many of which aim to explain phenomena like dark energy and dark matter without ad hoc adjusting the theory by hand. 
Despite this large output in this research area of alternative theories of gravity, there has been no real breakthrough in the last couple of years. 
The deviating from General Relativity into unknown terrain is too often performed in the fashion that Rovelli appropriately criticizes in the quotation above. Authors follow a specific route not because observations or fundamental principles urge them to do so but simply because yet an additional field or yet another structure to describe gravity to them seems more appealing than the present one. In short, central aspects of such alternative theories of gravity are often postulated without further reasoning. One could even say these theories are guessed. As Rovelli strikingly notes, this is not how science has progressed in the past, and certainly not how science will progress in the future. Such criticism is even more appropriate keeping in mind the complexity of the endeavor of constructing alternative theories of gravity, where the possibilities of new proposals really seem infinite.
With the numerous alternatives that have already been proposed and countless many yet undiscovered, possible further ones it is particularly surprising how one might be confident enough to think there exists even the slightest chance of directly guessing the correct one. 

It seems more promising to tackle such a complicated quest with a more reasoned plan: Only those properties, which constitute the minimal requirement that any meaningful theory of gravity must admit, are taken as fundamental principles. The thus obtained principles  provide a well-founded baseline for the construction of alternative theories of gravity. From there, any further step towards the final theory should be deduced as a logical consequence of the fundamental principles. In particular, once the baseline is set, such an approach completely eliminates any guesswork from the development of alternative theories of gravity. 
This concept precisely summarizes the ideology that lies at the heart of the research area \textit{\textbf{Constructive Gravity}}, and thus the goal that we will pursue in this thesis.
We are going to formulate specific fundamental requirements that any meaningful theory of gravity must satisfy. These requirements are then cast into a modern mathematical language such that they can be used to guide the formulation of new theories of gravity. This will, in fact, be achieved to the extent that we will formulate an algorithm that incorporates the precise steps necessary to obtain a valid theory of gravity from a given choice of input data. 

The fundamental requirements that we will adopt throughout this thesis are:
\begin{itemize}
    \item[\textbf{\textit{(i)}}] \textbf{\textit{Gravity is described by a tensor field on a 4-d manifold called spacetime.}}
    \item[\textbf{\textit{(ii)}}] \textbf{\textit{The dynamical laws that govern gravity are invariant under spacetime diffeomorphisms.}}
    \item[\textbf{\textit{(iii)}}] \textbf{\textit{Provided spacetime is additionally inhabited by matter fields their dynamics is causally compatible with the gravitational dynamics.}}
\end{itemize}

The beginning of chapter \ref{chapter1} will be used to lay the foundation of the remaining thesis. We will provide the necessary mathematics for a precise formulation of classical field theories. Further, we will investigate the consequences of the required diffeomorphism invariance of the gravitational dynamics, more precisely we will deduce from it an equivalent set of linear, first-order partial differential equations that the gravitational Lagrangian has to satisfy. The second half of chapter \ref{chapter1} investigates the associated canonical formulation of classical field theories, paying particular attention to how diffeomorphism invariance enters such an apparently non-covariant setting. The connection to gravitational physics will be picked up again at the end of chapter \ref{chapter1} where we will see how the three different ways of how General Relativity historically was rediscovered from first principles by Lovelock \cite{doi:10.1063/1.1665613} Hojman, Kuchař, and Teitelboim \cite{HOJMAN197688} and by Deser \cite{1970GReGr...1....9D} can all be understood as consequences of the fundamental requirement of diffeomorphism invariant gravitational dynamics.

The second chapter is dedicated to the connection between the central problem of Constructive Gravity, the construction of meaningful, new theories of gravity and the formal theory of partial differential equations. Not only will we use tools and techniques that we borrow from formal theory to analyze the PDE system that encodes diffeomorphism invariance which ultimately will reveal how at least perturbatively one might always obtain solutions to it, we will also utilize formal PDE theory to formulate the remaining fundamental requirement incorporated in our framework: the causal compatibility between a given matter theory and the to-be-constructed gravitational dynamics. The second chapter will finally culminate with a concrete, algorithmic manual for the (perturbative) construction of gravity theories.  

In the third chapter, we are going to apply the perturbative construction recipe that we developed in chapter 2 to two explicit examples, namely perturbative metric gravity and the more general case of perturbative Area Metric Gravity. It is of particular importance to note that both these theories represent in no way simple toy-models for a test of the developed framework. Both theories are unique in the sense that they constitute the most general option for describing the gravitational field that is consistent with a specific class of matter theories. More precisely the metric is the most general geometric background that might support a classical scalar field with linear, second-derivative-order equations of motion, i.e., a Klein-Gordon field, whereas the Area Metric represents the most general geometric background for a linear second-derivative-order theory of electrodynamics described by a co-vector field. Thus both chosen examples are extraordinarily interesting to study.

In the fourth and last chapter, we outline the basic functionality of the Haskell library that we developed specifically to deal with problems that arise in the context of perturbative Constructive Gravity. We provide short explanations of the main algorithms involved therein and also supplement the relevant portions of source code. Furthermore, we describe in short, by considering a particular example, how exactly the computer program can be used to tackle the relevant Constructive Gravity problems.

Finally, we conclude this thesis by outlining potential future research interests.

\section*{Notation}

Before we start with the main part, we quickly illustrate the particularities of the chosen notation. Throughout this thesis, even if not stated explicitly, all objects are considered to be smooth, i.e., we work with smooth maps defined on smooth manifolds. 
Furthermore, as the relevant constructions presented in this work are entirely local, we denote maps between manifolds without being too rigorous with the description of the appropriate domain and co-domain, respectively. Hence we might, for instance, write $x^a : M
\rightarrow \mathbb{R}$ for a chart map on a manifold $M$ where we should more appropriately write $x^a : M \supset U  \rightarrow x(U) \subset \mathbb{R} $. 

Lower case Latin indices $a,b,c,...$ are used as spacetime indices and hence run from 0 to 3. Spatial indices between 1 and 3 are denoted by lower case Greek letters $\alpha,\beta,\gamma,...$. In addition to that, we will use further types of indices to describe coordinate functions on additional bundles that we are going to construct over the spacetime manifold $M$. Upper case Latin letters from the middle of the alphabet, such as $I_k,J_k,L_k,...$ will be used to describe derivative indices of order $k$ and hence run from $0$ to $\binom{b+k-1}{n-1}-1$ where $n$ is the dimension of the base space w.r.t. which the derivatives are taken. As we will work extensively over second-order jet bundles, we will sometimes drop the labels that specify the order of such indices and hence use $I,J,K,...$ to solely describe such second-order derivative indices whenever this does not confuse the reader.
There will also be situations with two different types of derivative indices appearing in the same equation, describing higher derivatives w.r.t. the coordinates of the spacetime manifold and w.r.t. coordinates of an additional bundle. We will then label the derivative coordinates of the latter kind by letters $\tilde{I}_k, \tilde{J}_k,...$ to allow for distinction.
Upper case Latin letters from the beginning of the alphabet $A,B,C,D,...$ are used to label fiber coordinate functions on the space of fields. Whenever we are dealing with a second field that belongs to a different space, we label the coordinates by $\tilde{A}, \tilde{B},...$ . The respective range obviously depends on the specific field at hand.

In this work, we will deal with bundles of smooth manifolds to quite some extent. When we introduce a new bundle, we will denote it as a triple $(F,\pi_F,M)$, consisting of the total space $F$, the bundle projection $\pi_F$, and the base space $M$. Priorly introduced bundles might also be denoted simply by their total space. Morphisms between bundles $(F_1,\pi_1,M_1)$ and $(F_2,\pi_2,M_2)$ are denoted as pairs of maps $(f,h)$, with $f$ being the map between total spaces that covers $h$. Bundle morphisms that cover the identity or such morphisms that are constructed from a given map between base spaces will be referred to simply by the map on the total space.

Given a vector space $V$, a linear map $\phi$, a vector bundle $(E,\pi,N)$, or any other object with a notion of duality, a superscripted $\ast$ on the relevant object will denote the corresponding dual object, i.e., the dual vector space $V^{\ast}$, dual linear map $\phi^{\ast}$, dual vector bundle $(E^{\ast}, \pi^{\ast},N)$. Given a smooth map between manifolds $f : M \rightarrow N$, we also denote its pushforward by $f_{\ast} : Tm \rightarrow TN$ and its pullback by $f^{\ast} : T^{\ast}N \rightarrow T^{\ast}M$.

We denote tensor products by $\otimes$ and the corresponding space of $(m,n)$ tensors over $M$ by $T^m_nM$. The space of n-forms over $M$ is denoted by $\Lambda^nM$. Given a bundle $(F,\pi_{F},M)$, we write $G \in \Gamma(F) $ for a section of this bundle. We denote direct sums of vector spaces or the Whitney sum of vector bundles by $\oplus$. 

The jet bundle of order $q$ over a given bundle $(F,\pi_F,M)$ will be denoted by $J^qF$ and any associated jet prolongation map, either in the sense of prolonging sections of $F$ to sections of $J^qF$, or in the sense of prolonging vector fields on $M$ to vector fields on $J^qF$, will be referred to as $j^q$. 

In general, we will label bundle maps that take values in the bundle of volume forms over some base manifold by letters in calligraphic font $\mathcal{L}, \mathcal{H}, \mathcal{f}$ and the corresponding representation in terms of the chart induced volume form by the corresponding standard font letter $\mathcal{L}=L \mathrm{d}^4x, \mathcal{H}= H \mathrm{d}^3x$. The only exception is the Lie algebra morphism $\mathcal{f}$ defined in (\ref{LieF}) in chapter \ref{chapter1}. Here the notation is used to reflect its respective co-domain, the field bundle $F$. 

As usual, $\left[ \  ,   \  \right]$ denotes the commutator of vector fields and $\left \{  \  ,   \   \right \}$ the Poisson bracket of phase space functions/functionals. \\

\newpage

\section*{Remarks}

The results presented in this thesis were, to a large extent, obtained in close collaboration with Nils Alex\footnote{Department Physik, Friedrich-Alexander Universität Erlangen-Nürnberg.}. 
A more in-depth treatment of some of the topics presented in the following, as well as further, exciting aspects of Constructive Gravity that build on the foundation created by this thesis will be provided in:
\begin{itemize}
    \item \fullcite{NilsPHD}
\end{itemize}
Parts of the results that are presented in this thesis have been published in:
\begin{itemize}
    \item \fullcite{TobiR}
    \item \fullcite{ToBePublished}
\end{itemize}
The developed software is available at
\begin{itemize}
    \item \fullcite{sparse-tensor}
\end{itemize}