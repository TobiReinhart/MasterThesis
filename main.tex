\documentclass[a4paper,12pt, DIV=14, BCOR=5mm, twoside, headsepline, numbers=noenddot]{scrbook}

\usepackage{mathtools}
\usepackage{amsmath}
\usepackage{amssymb}
\usepackage{amsfonts}
\usepackage{amsthm}
\usepackage{verbatim}
\usepackage[cal=boondoxo]{mathalfa}
\usepackage{relsize}
\usepackage[boxruled, linesnumbered]{algorithm2e}
\usepackage{stmaryrd}
\usepackage{fancyvrb}
\usepackage{caption}
\usepackage{booktabs}
\usepackage{epigraph}
\usepackage{longtable}
\usepackage{array}  

\usepackage{blkarray, bigstrut} 


%mathmode columns
\newcolumntype{L}{>{$}l<{$}}
\newcolumntype{C}{>{$}c<{$}}
\newcolumntype{R}{>{$}r<{$}}

%more space between array rows
\renewcommand{\arraystretch}{1.3} 

%for code included
\usepackage{minted}
\usepackage[dvipsnames]{xcolor}
\definecolor{LG}{rgb}{0.863, 0.863, 0.863}

%for centered minted environment 
\usepackage{xpatch,letltxmacro}
\LetLtxMacro{\cminted}{\minted}
\let\endcminted\endminted
\xpretocmd{\cminted}{\RecustomVerbatimEnvironment{Verbatim}{BVerbatim}{}}{}{}
\usemintedstyle{vs}

%List of Illustrations
\providecommand\phantomsection{}
\makeatletter
\newcommand\listofillustrations{
    \chapter*{List of Illustrations}
    \phantomsection
    \section*{Figures}
    \phantomsection
    \@starttoc{lof}%
    \bigskip
    \section*{Tables}%
    \phantomsection
    \@starttoc{lot}%
    \bigskip
    \section*{Algorithms}%
    \phantomsection
    \@starttoc{loa}
    }
\makeatother
\renewcommand\listoflistingscaption{List of source code listings}

%Links
\usepackage{hyperref}
\hypersetup{
    colorlinks,
    citecolor=black,
    filecolor=black,
    linkcolor=black,
    urlcolor=black
}

%Diagrams
\usepackage{tikz}
\usetikzlibrary{shapes}
\tikzset{every picture/.style=remember picture}
\newcommand{\mathnode}[1]{%
   \mathord{\tikz[baseline=(#1.base), inner sep = 0pt]{\node (#1) {$\scriptstyle{#1}$};}}}
   
%Glossary
\usepackage[toc, sort = none, style = long]{glossaries-extra}
\loadglsentries{glossaries}

\usepackage[utf8]{inputenc}
\setcounter{MaxMatrixCols}{20}

\pagestyle{headings}

%Dictum
\renewcommand*{\dictumwidth}{0.65\textwidth}
\renewcommand*{\raggeddictum}{\centering}
\renewcommand*{\raggeddictumtext}{}
\setkomafont{dictumtext}{\normalfont\normalcolor\itshape\setlength{\parindent}{2em}\noindent}

\newcommand{\bigzero}{\mbox{\normalfont\Large\bfseries 0}}

%Theorems, etc.
\newtheorem{theorem}{Theorem}[section]
\newtheorem{definition}{Definition}[section]
\newtheorem*{remark}{Remark}
\newtheorem{example}{Example}[chapter]

%Bibliography
\usepackage[backend=biber, style=alphabetic, sorting=ynt]{biblatex}
\addbibresource{references.bib}

\AtEveryBibitem{%
  \ifentrytype{misc}{%
  }{%
    \clearfield{url}%
  }%
}

%Abstract
\newenvironment{abstract}{%
  \titlepage% X
  \null\vfil% X
  \begin{center}\normalfont\usekomafont{disposition}Abstract\end{center}%
}{%
  \par
  \vfil\null% X
  \endtitlepage% X
}
\usepackage{xkeyval}




%citation nasa ads macros
\def\apjl{\textit{ApJ}}                % Astrophysical Journal, Letters
\def\apj{\textit{ApJ}}                % Astrophysical Journal, 
\def\aj{\textit{AJ}}                % Astronomical Journal, 
\def\aap{\textit{A\&A}}              % Astronomy & Astrophysics
\def\prd{\textit{Phys.~Rev.~D}}        % Physical Review D
\def\prl{\textit{Phys.~Rev.~Lett.}}    % Physical Review Letters
\def\jcap{\textit{J. Cosmology Astropart. Phys.}}
                % Journal of Cosmology and Astroparticle Physics
                
%\interfootnotelinepenalty=100000
                
%Titlepage                
\title{Covariant Constructive Gravity}
\author{Tobias Reinhart }
\date{April 2019}

\begin{document}

\frontmatter

\begin{titlepage}
	\centering

	{\Huge\bfseries Covariant Constructive Gravity\par}
	\vspace{2cm}
	
	{\scshape\Large Masterthesis \par}
	\vspace{1cm}
	{ submitted by \par}
	
	
	
	{\bfseries Tobias Reinhart\par}
		{ 11. September 2019\par}
	\vspace{2cm}
	
	\includegraphics[width=0.5\textwidth]{fau-siegel.pdf}\par
	\vspace{1cm}Department of Physics\par
	Friedrich-Alexander-Universität Erlangen-Nürnberg\par
	Supervisor: Prof. Dr. Jörn Wilms , Dr. Frederic P. Schuller
	
	\vfill
\end{titlepage}

\begin{abstract}
In this thesis, we investigate the central question of Constructive Gravity: 
How one can derive gravitational dynamics from first principles instead of merely postulating them. 
Incorporating two fundamental requirements into our endeavor, diffeomorphism invariance of the gravitational theory and causal compatibility of the gravitational EOM with given matter field equations, we use modern mathematical tools such as the jet bundle apparatus and techniques from formal PDE theory to cast these two requirements into rigorous mathematical language. Doing so we end up with a system of linear first-order partial differential equations that fully encodes the required diffeomorphism invariance of any theory of gravity. We further include causal compatibility as a condition on the gravitational principal polynomial, a certain sub-part of the gravitational EOM. 
Proving that the PDE system is involutive and hence allows for a perturbative treatment, we develop a framework for the construction of perturbative theories of gravity, described in terms of a power series expansions of the gravitational Lagrangian that satisfy the two conditions. The two crucial requirements are then expressed by linear equations for the expansion coefficients of the power series which can readily be solved using a computer algebra program that we specifically designed for this purpose. We test the thus obtained framework treating two examples of particular interest, namely the construction of a perturbative metric theory of gravity --- which turns out to be equivalent to General relativity --- and the generalization to so-called Area Metric Gravity.
\end{abstract}

\begin{abstract}
In dieser Arbeit untersuchen wir die zentrale Frage des Forschungsgebietes Constructive Gravity: Wie kann Gravitationsdynamik aus grundlegenden Forderungen hergeleitet werden, anstatt diese zu postulieren.
Unter der Verwendung zweier solcher grundlegender Forderungen, der Diffeomorphismen Invarianz der Gravitations Theorie sowie deren kausaler Kompatibilität mit einer gegebenen Materie Theorie, nutzen wir moderne mathematische Techniken wie die Theorie der Jet Bündel und Aspekte aus der formellen Theorie partieller Differentialgleichungen, um die beiden Forderungen in rigorose Mathematik zu übersetzen. So leiten wir ein zur geforderten Diffeomorphismen Invarianz äquivalentes System aus linearen partiellen Differentialgleichungen erster Ordnung her. Die Forderung nach kausaler Kompatibilität kann schließlich als Bedingung an das Prinzipal Polynom der Gravitations Theorie, einem speziellen Bestandteil der Gravitations Bewegungsgleichung, formuliert werden. 
Weiterhin beweisen wir, dass das System aus partiellen Diffferentialgleichungen involutiv ist und daher eine störungstheoretische Betrachtung ermöglicht. So entwickeln wir ein Rahmenwerk für die Konstruktion von störungstheoretischen Reihenent- wicklungen von Gravitationstheorien, die die beiden fundamentalen Bedingungen erfüllen. Unter Verwendung der Reihenentwicklung offenbaren sich die beiden fundamentalen Bedingungen als lineare Gleichungen für die entsprechenden Entwicklungskoeffizienten. Diese können mithilfe eines speziell für diesen Zweck von uns entwickelten Computerprogramms gelöst werden.
Wir testen die von uns entwickelten Techniken an zwei besonders bedeutenden Beispielen, der Konstruktion einer störungstheoretischen, metrischen Gravitations Theorie --- welche sich letztendlich als äquivalent zur Allgemeinen Relativitätstheorie herausstellt ---  sowie der Verallgemeinerung hin zur sogenannten Area Metrischen Gravitation. 
\end{abstract}

\tableofcontents

\listofillustrations

\listoflistings

\chapter*{Acknowledgments}
Firstly, I would like to thank Frederic Schuller without whose inspiring ideas this thesis undoubtedly would not have been possible. Frederic without your distinctive way of motivating your students, I certainly would not have worked as hard on this thesis as I did.

Further, I would like to thank Jörn Wilms for accepting my thesis as an official supervisor and thereby allowed me to register it formally. 
I also have to thank Marcus Werner (Yukawa Institute for Theoretical Physics in Kyoto, Japan) who in collaboration with the Universe Journal provided financial support for my participation at the 15th Marcel Grossmann meeting in Rome. 
Besides, I would also like to mention that a large number of computer calculations that were carried out during this thesis were only possible due to the great IT-infrastructure that the physics department provides for their students. 


Moreover, I would like to thank my colleagues for helping me in numerous aspects and never becoming tired of answering my countless questions.
Amongst them, I would especially like to thank Nils Alex.
Not only did he provide a patient instructor in introducing me to the world of software engineering, 
but a large part of the results that are presented in this thesis was, in fact, developed in collaboration with him. 


I would also like to thank my family for tirelessly supporting my passion for physics and my friends for accepting my lack of time.
Especially I would like to thank my girlfriend Nici not only for her assistance in nearly every part of my life but also
for tolerating that during numerous occasions throughout the day my mind almost entirely revolves around physics. 
Special thanks are also due to our cat Lucky, who selflessly provided me accompany during the long periods of writing this thesis. 

Lastly, I certainly have to acknowledge anyone who will prove the stamina of not only reading but also finishing this rather long master's thesis. I appreciate it. 

\mainmatter

\chapter{Introduction: The Aim of Constructive Gravity}
\setlength{\epigraphwidth}{0.6\textwidth}
\epigraph{"...theoretical physics has not done great in the last decades. Why? Well, one of the reasons, I think, is that it got trapped in a wrong philosophy: the idea that you can make progress by guessing new theory and disregarding the qualitative content of previous theories. This is the physics of the "why not?" Why not studying this theory, or the other? Why not another dimension, another field, another universe? Science has never advanced in this manner in the past. Science does not advance by guessing."}{\textit{Carlo Rovelli}}

Over 100 years ago in 1915, Einstein first published his theory of General relativity \cite{1915SPAW.......844E}, and still today it constitutes a widely accepted description of gravitational interaction. However, it is still unsatisfactory in describing specific observations. 1998 discoveries by the Supernova Cosmology Project \cite{1999ApJ...517..565P} and the High-Z Supernova Search Team \cite{1998AJ....116.1009R} revealed that the expansion of the universe is accelerating over time. For such an accelerated expansion of the universe to be consistent with General Relativity, one needs to account for it by adding either a cosmological constant or an additional scalar field to the Lagrangian. The entire contribution of such additional quantities is then usually called \textit{\textbf{dark energy}}.
In addition to that, observations of galaxy rotation curves from the 1960s and 1970s \cite{1970ApJ...160..811F}, \cite{1970ApJ...159..379R} and \cite{1980ApJ...238..471R} suggest that to explain their observed rotation behavior most galaxies in addition to their visible matter need to contain a large amount of un-observable hence \textit{\textbf{dark matter}}.
If one wants to describe this phenomenon still in the context of General Relativity, this requires one to add further field content to the theory to also account for dark matter. 

The most prominent General relativity based cosmological model that includes both dark energy and dark matter is the so-called cosmological standard model or Lambda-CDM model. Based on this model, measurements of the cosmic microwave background can be used to determine the percentage of the energy density in the observable universe that is composed of standard model matter and to which extent the additional sources of dark matter and dark energy contribute. Recent measurement lead by the Planck mission \cite{Planck13_1}, \cite{Planck13_2}, \cite{Planck15} and \cite{Planck18} revealed that standard model matter only compromises $4.82\pm0.05\%$ of its total energy content, whereas dark matter accounts for $ 25.8\pm0.4\%$ and dark energy even for $ 69 \pm 1 \%$. Thus dark energy and dark matter combined contribute an astonishing total of nearly $95 \%$ to the total energy contained in the present observable universe.
Justifiably it seems inappropriate that two adjustments that ad hoc are added to an existing theory to render it consistent with otherwise contradicting observations in the end account for the description of nearly $95\%$ of the theories "content".  

Regarding this observation it is no surprise that there exist countless proposed alternatives\footnote{There exist theories that describe the gravitational field as a scalar field \cite{Scalar1} \cite{Scalar2}, theories that use a combination of scalar and tensor field \cite{ST1}, \cite{ST2}, \cite{ST3}, theories that employ scalar, vector and tensor field \cite{SVT1}, \cite{SVT2}, theories that use the usual metric but different EOM \cite{fR1}, \cite{fR2} and even theories that describe the gravitational field by two metric tensors \cite{BIM1} and \cite{BIM2} to provide a few examples.} to General relativity, many of which aim to explain phenomena like dark energy and dark matter without such an ad hoc adjusting the theory by hand. 
Despite this large output in this research area of alternative theories of gravity, there has been no real breakthrough in the last couple of years. 
The deviating from General Relativity into unknown terrain is too often performed in the fashion that Rovelli appropriately criticizes in the quotation above. Changes to General relativity are not performed because observations or even some underlying fundamental principle that one tries to incorporate in the theory of gravity urges one to do so but simply because yet an additional field or yet another structure to describe gravity might seem more appealing than the present one. In short central aspects of such alternative theories of gravity are often postulated, or one could also say guessed. As Rovelli strikingly notes, this is not how science has progressed in the past. This critic is even more appropriate for the complicated endeavor of constructing alternative theories of gravity. 
With the numerous alternatives that have already been proposed and countless many yet undiscovered, possible further ones it is in particular surprising how one might be confident enough to think he or she even has the slightest chance of directly guessing the correct one. It seems more promising to tackle such a complicated quest with a more reasoned plan, by clearly deciding which properties a meaningful alternative theory of gravity must admit in any case and then from there deducing consequences of these properties. 
This approach precisely constitutes a vital aspect of the ideology that lies at the heart of the research area \textit{\textbf{Constructive Gravity}}.

In this thesis, we will attempt the endeavor of supporting the process of constructing arbitrary alternative gravity theories with reasoned structure. 
We are going to formulate certain fundamental requirements that any meaningful theory of gravity must satisfy. These requirements are then cast into a modern mathematical language such that they can be used to guide the formulation of new theories of gravity. This will, in fact, be achieved to the extent that we will formulate a "recipe" that incorporates the precise steps necessary to obtain a valid theory of gravity from a given choice of input data. 

The fundamental requirements that we will adopt throughout this thesis are:
\begin{itemize}
    \item[(i)] \textbf{Gravity is described by a tensor field on a 4-d manifold called spacetime.}
    \item[(ii)] \textbf{The dynamical laws that govern gravity are invariant under spacetime diffeomorphisms.}
    \item[(iii)] \textbf{Provided spacetime is additionally inhabited by matter fields their dynamics is causally compatible with the gravitational dynamics.}
\end{itemize}

The beginning of chapter (\ref{chapter1}) will be used to lay the foundation of the remaining thesis. We will provide the necessary mathematics for a precise formulation of classical field theories. Further, we will investigate consequences of the required diffeomorphism invariance of the gravitational dynamics, more precisely we will deduce from it an equivalent set of linear, first-order partial differential equations that the gravitational Lagrangian has to satisfy. The second half of chapter (\ref{chapter1}) investigates the associated canonical formulation of classical field theories, paying particular attention to how diffeomorphism invariance enters such an apparently non-covariant setting. The connection to gravitational physics will be picked up again in the end of chapter (\ref{chapter1}) where we will see how the three different ways how General relativity historically was rediscovered from first principles by Lovelock \cite{doi:10.1063/1.1665613} Hojman, Kuchař, and Teitelboim \cite{HOJMAN197688} and by Deser \cite{1970GReGr...1....9D} can all be understood as consequences of the fundamental requirement of diffeomorphism invariant gravitational dynamics.

The second chapter is dedicated to the connection between the central problem of Constructive Gravity, the construction of meaningful, new theories of gravity and the formal theory of partial differential equations. Not only will we use tools and techniques that we borrow from formal theory to analyze the PDE system that encodes diffeomorphism invariance which ultimately will reveal how at least perturbatively one might always obtain solutions to it, we will also formal PDE theory to formulate the remaining fundamental requirement incorporated in our framework: the causal compatibility between a given matter theory and the to be constructed gravitational dynamics. The second chapter will finally culminate with the concrete, almost algorithmic manual for the perturbative construction of gravity theories.  

In the third chapter, we are going to apply the perturbative construction recipe that we developed in chapter 2 to two explicit examples, namely perturbative metric gravity and the more general case of perturbative Area Metric Gravity. It is of particular importance to note that both two theories represent in no way unreasoned guesses of alternative gravity theories. They are unique in the sense that they both constitute the most general option for describing the gravitational field that is consistent with a specific class of matter theories. More precisely the metric is the most general geometric background that might support a classical scalar field with linear, second-derivative-order equations of motion, i.e., a Klein-Gordon field, whereas the Area Metric represents the most general geometric background for a linear second-derivative-order theory of electrodynamics described by a co-vector field. Thus both chosen examples are extraordinarily interesting to study.

In the fourth and last chapter, we outline the basic functionality of the Haskell library that we developed specifically to deal with problems that arise in the context of perturbative Constructive Gravity. We provide short explanations of the main algorithms involved therein and also supplement the relevant portions of source code. Furthermore, we describe in short, by considering a particular example of how exactly the computer program can be used to tackle the relevant Constructive Gravity problems.

We then conclude this thesis by outlining potential future research interests.

\section*{Notation}

Before we start with the main part, we quickly illustrate the particularities of the chosen notation. Throughout this thesis, even if not stated explicitly, all objects are considered to be smooth, i.e., we work with smooth maps defined on smooth manifolds. 

Furthermore, as the relevant constructions presented in this work are entirely local, we denote maps between manifolds without being too rigorous with the description of the appropriate domain and co-domain, respectively. Hence we might, for instance, write $x^a : M
\rightarrow \mathbb{R}$ for a chart map on a manifold $M$ where we should more appropriately write $x^a : M \supset U : \rightarrow x(U) \subset \mathbb{R} $. 

Lower case Latin indices $a,b,c,...$ are used as spacetime indices and hence run from 0 to 3, spatial indices between 1 and 3 are denoted by lower case Greek letters $\alpha,\beta,\gamma,...$. In addition to that, we will use further types of indices to describe coordinate functions on additional bundles that we are going to construct over the spacetime manifold $M$. Upper case Latin letters from the middle of the alphabet $I_k,J_k,L_k,...$ will be used to describe derivative indices of order $k$ and hence run from $0$ to $\binom{b+k-1}{n-1}-1$ where $n$ is the dimension of the base space w.r.t. which the derivatives are taken. As we will work extensively over some second-order jet bundle, we will sometimes drop the labels that specify the order of such indices and hence use $I,J,K,...$ to solely describe such second-derivative-order indices whenever this does not confuse.

There will also be situations with two different types of derivative indices appearing in the same equation, describing higher derivatives w.r.t. the coordinates of the spacetime manifold and w.r.t. coordinates of some additional bundle. We will then label the derivative coordinates of the latter kind by calligraphic letters $\mathcal{I}_k, \mathcal{J}_k,...$ to allow for distinction.

Upper case Latin letters from the beginning of the alphabet $A,B,C,D,...$ are used to label fiber coordinate functions on the space of fields. Whenever we are dealing with a second field that belongs to a different space, we label the coordinates by $\tilde{A}, \tilde{B},...$ . The respective range obviously depends on the specific field at hand.

In this work, we will deal with bundles of smooth manifolds to quite some extent. When we introduce a new bundle, we will denote it as a triple $(F,\pi_F,M)$ consisting of the total space $F$ the bundle projection $\pi_F$ and the base space $M$. Priorly introduced bundles might also be simply denoted by their total space. Morphisms between bundles $(F_1,\pi_1,M_1)$ and $(F_2,\pi_2,M_2)$ are denoted as pairs of maps $(f,h)$, with $f$ being the map between total spaces that covers $h$. Bundle morphisms that cover the identity or such morphisms that are constructed from a given map between base spaces will be referred to simply by the map on the total space.

Given a vector space $V$, a linear map $\phi$, a vector bundle $(E,\pi,N)$ or any other object with a notion of duality, a superscripted $\ast$ on the relevant object will denote the corresponding dual object, i.e., the dual vector space $V^{\ast}$, dual linear map $\phi^{\ast}$, dual vector bundle $(E^{\ast}, \pi^{\ast},N)$. In particular, given a smooth map between manifolds $f : M \rightarrow N$, we denote its pushforward by $f_{\ast} : Tm \rightarrow TN$ and its pullback by $f^{\ast} : T^{\ast}N \rightarrow T^{\ast}M$.

We denote tensor products by $\otimes$ and the corresponding space of $(m,n)$ tensors over $M$ by $T^m_nM$. The space of n-forms over $M$ is denoted by $\Lambda^nM$. Given a bundle $(F,\pi_{F},M)$, we write $G \in \Gamma(F) $ for a section of this bundle. We denote direct sums of vector spaces or the Whitney sum of vector bundles by $\oplus$ respectively. 

The jet bundle of order q over a given bundle $(F,\pi_F,M)$ will be denoted by $J^qF$ and any associated jet prolongation map, either in the sense of prolonging sections of $F$ to sections of $J^qF$ or in the sense of prolonging vector fields on $M$ to vector fields on $J^qF$ will be referred to as $j^q$. 

In general, we will label bundle maps that take values in the bundle of volume forms over some base manifold letters in calligraphic font $\mathcal{L}, \mathcal{H}$ and the corresponding representation in terms of the chart induced volume form by the corresponding standard font letter $\mathcal{L}=L \mathrm{d}^4x, \mathcal{H}= H \mathrm{d}^3x$.

As usual, $\left[ \  ,   \  \right]$ denotes the commutator of vector fields and $\left \{  \  ,   \   \right \}$ the Poisson bracket of phase space functions/functionals. \\

\section*{Remarks}

The results presented in this thesis were to a large extend obtained in close collaboration with Nils Alex\footnote{Department Physik, Friedrich-Alexander Universität Erlangen-Nürnberg.}. 
A more in depth treatment of some of the topics presented in the following, as well as further, exciting aspects of Constructive Gravity that build on the foundation created by this thesis will be provided in:
\begin{itemize}
    \item \fullcite{NilsPHD}
\end{itemize}
Parts of the results that are presented in this thesis have been published in:
\begin{itemize}
    \item \fullcite{TobiR}
    \item \fullcite{ToBePublished}
\end{itemize}
The developed software is available at
\begin{itemize}
    \item \fullcite{sparse-tensor}
\end{itemize}

\chapter{Diffeomorphism Invariant Field Theory}\label{chapter1}
\dictum{
We provide the essential mathematical tools necessary for a precise formulation of classical Lagrangian field theory. Furthermore, we study the implications of the requirement that the Lagrangian be infinitesimally equivariant w.r.t. spacetime diffeomorphisms and derive from it an equivalent set of first-order partial differential equations. Lastly, we investigate consequences of the diffeomorphism invariance for the Hamiltonian formulation of classical field theory.
}

\section{Field Bundles, Jet Bundles, and Prolongations}
In order to develop a concise framework for classical Lagrangian field theory, it is insightful to recall the situation in standard classical mechanics. There, one encodes the possible configurations a given system might adopt at given time as finite-dimensional manifold, the so-called configuration space $Q$. The objects of interest are curves $\gamma : \mathbb{R} \rightarrow Q $ that represent the system's path in this configuration space, i.e., the particular system configuration dependent on some parameter time. Given such a curve, by adjoining its velocity $\dot{\gamma}$ at each parameter value, one can lift the curve to the tangent bundle $TQ$. The dynamics of the system can then be described by a Lagrangian function $L : TQ \rightarrow \mathbb{R}$. Composing $L$ with a lifted curve one obtains a real-valued function. As such, one can compute its integral. This procedure defines a local functional on the space of curves:
\begin{align}
S[\gamma] = \int \mathrm{d}\lambda L(\gamma (\lambda), \dot{\gamma} (\lambda)).  
\end{align}
Fixing start and end configuration and velocity of the system for all curves in consideration, the physical curves are those that extremize this action functional, or equivalently those that solve the corresponding Euler-Lagrange-Equations. 

Once the transition from standard particle mechanics to field theories is made, the system is no longer described by curves in some finite-dimensional configuration space representing the system's configuration at each parameter time, but by the values, a given field attains at different spacetime points. Hence, instead of curves, one deals with sections of some bundle $(F,\pi_F,M)$ over the 4-dimensional spacetime manifold $M$. The notion of adjoining velocities to a curve is replaced by adjoining derivatives to a section. The appropriate setting for this procedure is the \textit{\textbf{jet bundle}} over $F$. Just as the tangent bundle $TQ$ is obtained from a coordinate-independent description of derivatives of curves, the jet bundle over $F$ is constructed by a coordinate-independent description of derivatives of sections of $F$. As before, the Lagrangian is then simply a function on this jet bundle. 
In the following, we outline how these ideas can be cast into precise mathematical language. 

We follow \cite{1998physics...1019G} to some extent. Further information concerning the geometry of jet bundles can be found in \cite{saunders_1989} and \cite{seiler2009involution}. A more sophisticated treatment with the focus lying on the naturality\footnote{Loosely speaking naturality states that certain constructions do not require further structure or additional choices than what is already provided.} of certain constructions that we are going to meet in the next chapters can be found in \cite{kolar1993natural}. Many details regarding the involved standard differential geometry are concisely collected in \cite{doi:10.1142/3867}.

Let $M$ be an oriented 4-dimensional manifold that represents spacetime. A field that inhabits spacetime is described as a \textit{\textbf{section}} of a bundle $(F,\pi_F,M)$, i.e., a smooth map that satisfies:
\begin{align}
\begin{aligned}
G : M &\longrightarrow F\\
& \pi_F \circ G = \mathrm{id}_M.
\end{aligned}
\end{align}
In the following, we will refer to the bundle $F$ as $\textit{\textbf{field bundle}}$.  As we want to consider tensor fields, the field bundle is required to be given by a vector subbundle of some $T^m_n M$. We denote adapted coordinates\footnote{Recall that adapted coordinates $(x^m,v_A)$ on a fiber bundle $(F, \pi_F, M)$ are defined by the requirement that there exits a coordinate chart on $M$, $(y^m)$, s.t. $y^m = x^m \circ \pi_F$, and hence we denote the first $dim(M)$ coordinate functions of both charts by the same label. It should be clear from the context which coordinates are meant, those on $M$ or those on $F$.} on $F$ by $(x^m,v_A)$. Here the abstract index $A$ runs from $0$ to $n - 1$ with $n$ being the fiber dimension of $F$, i.e., $n := \mathrm{dim}(\pi_F^{-1}(p))$ for some $p \in M$. Note that as $F$ is assumed to be a vector bundle, we can always choose adapted coordinates s.t. the $v_A$ are linear on the fibers. Details regarding this remark can be found in \cite{saunders_1989}. In the following, we will only regard such adapted coordinates. The corresponding coordinate representation of sections is denoted by $G_A = v_A \circ G $. Summing up, writing $G_A$ we refer to the coordinate representation of a section, whereas $v_A$ denotes fiber coordinates on $F$.

Given a field bundle $(F, \pi_F, M)$, we can define the corresponding $\textit{\textbf{dual field bundle}}$ $(F^{\ast}, \pi_{F^{\ast}},M)$ as its vector bundle dual. We quickly recall the definition of the vector bundle dual and the dual of a linear map.
\begin{definition}[dual linear map] \label{dual}
Given vector spaces $V$ and $W$, and a linear map, i.e., a morphism of vector spaces $f : V \rightarrow W$, the dual linear map $f^{\ast} : W^{\ast} \rightarrow V^{\ast}$ is defined by the requirement: 
\begin{align}
    \forall v \in V, \forall \omega \in W^{\ast} : \omega (f(v)) = f^{\ast}(\omega) (v).
\end{align}
\end{definition}
Sometimes the dual of a linear map is also called its adjoint. With the definition of the vector space dual and the dual of a linear map, we can now lift these notions to the case of vector bundles and vector bundle morphisms.
\begin{definition} [vector bundle dual]
Given a vector bundle $(F, \pi_F,M)$. Its vector bundle dual is the vector bundle $(F^{\ast}, \pi_{F^{\ast}},M)$ over the same base space, with fibers $\pi_{F^{\ast}}^{-1}(p) = (\pi_F^{-1}(p))^{\ast} $ for all $p$ in $M$ and equipped with the obvious bundle projection.  
\end{definition}

In other words, one obtains the dual to a given vector bundle by implementing the dual construction fiber by fiber.
Given adapted coordinates on the field bundle $(x^m, v_A)$ with the coordinate functions $v_A$ being linear on the fibers, we can obtain \textbf{\textit{dual coordinates}} on the dual field bundle by $(x^m, ((v_A)^{\ast})^{-1})$, i.e., by taking the inverse dual of the fiber coordinates $v_A$ (see \cite{saunders_1989}). Note that we do not simply take the dual of the $v_A$ but, its inverse as the dual construction reverses the direction of linear maps\footnote{In the language of category theory: Duality defines a contravariant functor \cite{MacLane:205493}.}. In the following, we denote the fiber coordinates dual to $v_A$ by $v^A$. With these coordinates we get $ v_A(v^B) = \delta_A^B$. From now on given a chart on the field bundle, we always work with the corresponding dual chart on the dual field bundle.

Note that as the field bundle is required to be given by a vector subbundle of some $T^m_n M$, one usually works directly on $T^m_n M$ and uses chart induced basis fields $\mathrm{d}x^{a_1}\otimes ... \otimes \mathrm{d}x^{a_m} \otimes \frac{\partial}{\partial x^{b_1}} \otimes ... \otimes \frac{\partial}{\partial x^{b_n}}$ to complete a chart on $M$ to one on $T ^m _ n M$. In the following, we denote such coordinates on $T^m_n M$ as $(x^m, v^{a_1 ... a_m}_{b_1 ... b_n})$:
\begin{align}
    v^{a_1 ... a_m}_{b_1 ... b_n} = \mathrm{d}x^{a_1}\otimes ... \otimes \mathrm{d}x^{a_m} \otimes \frac{\partial}{\partial x^{b_1}} \otimes ... \otimes \frac{\partial}{\partial x^{b_n}}.
\end{align}
The coordinate representation of sections is then given as $G^{a_1 ... a_m}_{b_1 ... b_n} = v^{a_1 ... a_m}_{b_1 ... b_n} \circ G $. Although this is the standard way of representing tensor fields in coordinates, it has two significant disadvantages. Firstly the notation gets increasingly complicated once products and contractions of higher-ranked fields are involved. Secondly, if the field bundle is a true subbundle of $T^m_nM$ and hence has fibers with dimension less than $4^{(m+n)}$ the chart induced fields simply are too many functions to provide a valid chart on this subbundle. 
This is, for instance, the case considering the field bundle for a metric tensor. The total space of this particular bundle is the space of symmetric rank $(0,2)$ tensors $S^0_2M$. The typical fiber of the field bundle has dimension $10$. Nevertheless one usually uses the $16$ chart induced basis fields $ \frac{\partial}{\partial x^a}  \otimes \frac{\partial}{\partial x^b}$ to obtain coordinate expressions $g_{ab} = g(\frac{\partial}{\partial x^a},\frac{\partial}{\partial x^b})$ for a section of this field bundle. These component functions then necessarily satisfy $g_{ab} = g_{ba}$, which reduces the fiber dimension from $16$ to $10$. 

As we are going to deal with the bundle $(F, \pi_F, M)$ in a rather abstract setting, we do not follow this practice but stick to accurate coordinates $(x^m,v_A)$ on $F$. 
In order to relate the thus obtained intrinsic description of $F$ to the one that is provided by considering $F$ as being an embedded subbundle in $T^m_n M$ we construct vector bundle isomorphisms that allow us to identify the two perspectives. We quickly recall the definition of a \textit{\textbf{bundle morphism}}:
\begin{definition}[bundle morphism]
Given two bundles $(F_1, \pi_{F_1}, M)$ and $(F_2, \pi_{F_2}, N)$, a bundle morphism covering $h : M \rightarrow N$ is a smooth map $f : F_1 \rightarrow F_2$ such that the diagram in Figure \ref{BundleMorph} commutes.
\begin{figure}[hbt!]
\centering 
\begin{tikzpicture}
\node (M) at (0,0) {$M$};
\node (N) at (4,0) {$N$};
\node (F1) at (0,3) {$F_1$};
\node (F2) at (4,3) {$F_2$};
\draw [-latex] (M) -- node[pos=0.5, below] {$h$} (N);
\draw [-latex] (F1) -- node[pos=0.5, above] {$f$} (F2);
\draw [-latex] (F1) -- node[pos=0.5, left] {$\pi_{F_1}$} (M);
\draw [-latex] (F2) -- node[pos=0.5, right] {$\pi_{F_2}$} (N);
\end{tikzpicture}
\caption{Commutative Diagram: Bundle Morphisms.}\label{BundleMorph}
\end{figure}
\end{definition}
\begin{comment}
\begin{remark}
We normally denote a bundle morphism $f$ that covers $h$ as pair $(f,h)$. For the special case $M=N$ and $h=\mathrm{id}_M$ we shorten the notation to simply $f$. Also we might refer to a bundle morphism by the total space function $f$ if $f$ and $h$ are related via some specified construction, i.e., if $f$ can be obtained uniquely once $h$ is given. 
\end{remark}
\end{comment}
We often consider the case where the bundles carry additional structure on their fibers, for instance, in the case of $(F, \pi_F, M)$ being a vector bundle, an additional vector space structure. Then one is in particular interested in those bundle morphisms that preserve the given structure. To provide an example, a vector bundle morphism between vector bundles $(F_1, \pi_{F_1}, M)$ and $(F_2, \pi_{F_2}, N)$ is defined as a bundle morphism $(f,h)$ that restricts to a linear map on the fibers of $F_1$, i.e., $\forall p \in M$ the restriction $f \vert_{\pi_{F_1}^{-1}(p)}$ defines a linear map.
Similarly, one can define bundle morphisms that preserve arbitrary other structures.

With these definitions at hand, we can now specify the maps that allow for the transition between the description of the field bundle $F$ as abstract, independent vector bundle and the one obtained from considering it as being embedded in some tensor bundle over $M$.

\begin{definition}[intertwiner]\label{interDef}
Let $(F,\pi_F,M)$ be a vector bundle. We call a pair of vector bundle morphisms $(I, J)$:
\begin{align}
    \begin{aligned}
    I&: F \longrightarrow T^m_n M\\
    J&: T^m_n M \longrightarrow F 
    \end{aligned}
\end{align}
that cover $id_M$, and that additionally satisfy
$J \circ I = \mathrm{id}_F$ a pair of \textbf{\textit{intertwiners}} for the bundle $(F, \pi_F, M)$.
\end{definition}
The last requirement, namely the existence of a left inverse to $I$ ensures that: 
\begin{align}\tilde{I} : F \longrightarrow I(F) \subset T^m_nM
\end{align}
defines a vector bundle isomorphism and thereby allows us to relate the two distinct points of view.
We can think of the bundle morphism $I$ as embedding $F$ into $T^m_nM$ and $J$ as projecting general tensors from $T^m_nM$ to $F$. 

From a given pair of intertwiners $(I,J)$ for $F$ we can readily construct a pair of intertwiners for the dual vector bundle $F^{\ast}$. 
We first take the dual of the intertwiner $I$, which is $I^{\ast}$. As taking the dual reverses the direction one wants to proceeds by inverting $I^{\ast}$. Here this cannot be done as $I$ is not surjective, but only admits a left inverse. Hence, also its dual is not invertible. It is however well known (see for instance \cite{MacLane:205493}) that the existence of a left inverse for $I$ implies the existence of a right inverse for $I^{\ast}$. 
In the following, we denote this map simply by $(I^{\ast})^{-1}$ but keep in mind that it only provides a right inverse of $I^{\ast}$. Vice versa the existence of a right inverse of $J$ guarantees the existence of a left inverse $(J^{\ast})^{-1}$ of its dual. From the definition of the dual map, we further find that $(J^{\ast})^{-1} \circ (I^{\ast})^{-1} = id_{F^{\ast}}$.Thus, in total, given a pair of intertwiners for the field bundle $(I,J)$ we can obtain a pair of intertwiners for the dual field bundle via this construction, i.e., by $((I^{\ast})^{-1}, (J^{\ast})^{-1})$. 

Note that, as the two intertwiners are linear maps, they can be understood as certain tensors themselves: 
\begin{align}
\begin{aligned}
I &\in F^{\ast} \otimes T^m_n M\\
J &\in (T^m_nM)^{\ast} \otimes F \cong T^n_m M \otimes F.
\end{aligned}
\end{align}
We can now choose adapted coordinates $(x^m,v_A)$ on the field bundle, and the corresponding dual coordinates $(x^m, v^A)$ on the dual field bundle and chart induced coordinates $(x^m, v^{a_1 ... a_m}_{b_1 ... b_n})$ on $T^m_n M$ and the dual coordinates $(x^m, v^{b_1 ... b_n}_{a_1 ... a_m})$ on its dual $T^n_mM$. Using these charts, the maps $I$ and $J$ are represented as follows:
\begin{align} \label{interAbs}
    \begin{aligned}
    I &= I^{A a_1 ... a_m}_{b_1 ... b_n} \cdot v_A \otimes  v^{b_1 ... b_n}_{a_1 ... a_m}\\
    J &= J^{b_1 ... b_n}_{A a_1 ... a_m} \cdot v^A \otimes  v^{a_1 ... a_m}_{b_1 ... b_n}.
    \end{aligned}
\end{align}
Computing similar coordinate expressions for the pair of intertwiners of the dual field bundle $((I^{\ast})^{-1}, (J^{\ast})^{-1})$ one finds that:
\begin{align} \label{dualInterAbs}
    \begin{aligned}
         (I^{\ast})^{-1} &= J^{b_1 ... b_n}_{A a_1 ... a_m} \cdot v^A \otimes  v^{a_1 ... a_m}_{b_1 ... b_n}.\\
         (J^{\ast})^{-1} &= I^{A a_1 ... a_m}_{b_1 ... b_n} \cdot v_A \otimes  v^{b_1 ... b_n}_{a_1 ... a_m}.
    \end{aligned}
\end{align} 
Hence, although the constructed intertwiners for the dual field bundle define different maps, their components w.r.t. a given chart are already given by the intertwiners of the field bundle $(I,J)$ if one agrees on choosing the appropriate dual coordinates with a given choice of coordinates. Therefore in the following, we will drop the distinction of intertwiners for the field bundle and its dual.  This should not cause confusion, as when working with dual coordinates on the dual field bundle already the index position of the fiber coordinates $v_A$, and $v^{A}$ respectively contains the information regarding which intertwiner relates these coordinates to those one $T^m_n M$, and $T^n_mM$ respectively. In total, we find the following relations connecting the two ways of coordinatizing the field bundle and its dual: 
\begin{align} \label{interRel}
    \begin{aligned}
    & v^{a_1 ... a_m}_{b_1 ... b_n} & = & \ \ I^{A a_1 ... a_m}_{b_1 ... b_n} \cdot v_{A},\\  
    & v_A & = & \ \ J^{b_1 ... b_n}_{A a_1 ... a_m} \cdot v^{a_1 ... a_m}_{b_1 ... b_n}, \\
    & v^{b_1 ... b_n}_{a_1 ... a_m} & = & \ \  J^{b_1 ... b_n}_{A a_1 ... a_m} \cdot v^A, \\ 
    & v^A & = & \ \  I^{A a_1 ... a_m}_{b_1 ... b_n} \cdot v^{b_1 ... b_n}_{a_1 ... a_m}, \\
    & \delta^A _ B & = & \ \   I^{A a_1 ... a_m}_{b_1 ... b_n} \cdot J^{b_1 ... b_n}_{B a_1 ... a_m}.  
    \end{aligned}
\end{align}
In particular, we find that intertwiners defined in this way keep contractions between elements of $F$ and $F^{\ast}$, and all possible higher tensor products that can be obtained from these, invariant. In other words, it is equivalent if we compute expressions using the chart induced coordinates or, by means of the intertwiners $(I,J)$, using $(x^m, v_A)$ and the corresponding dual coordinates.

In practise one can specify the coordinate expressions $I^{A a_1 ... a_m}_{b_1 ... b_n}$ and $J^{b_1 ... b_n}_{A a_1 ... a_m}$ in order to define coordinates $v_A$ in terms of the chart induced coordinates $v^{a_1 ... a_m}_{b_1 ... b_n}$, i.e., one constructs a valid pair of intertwiners with the desired properties and then defines the fiber coordinates on $F$ by:
\begin{align}
    \begin{aligned}
    v_A &:= J^{b_1 ... b_n}_{A a_1 ... a_m} \cdot v^{a_1 ... a_m}_{b_1 ... b_n}\\
    v^A &:= I^{A a_1 ... a_m}_{b_1 ... b_n} \cdot  v^{b_1 ... b_n}_{a_1 ... a_m} .
    \end{aligned}
\end{align}
This approach can, for instance, be used if the field bundle is defined in terms of symmetries the tensor field has to satisfy. We use these symmetries to define an equivalence relation on the set of $4^{m+n}$ fiber coordinate functions $\left \{ v^{a_1 ... a_m}_{b_1, ..., b_n} \ \big \vert \  a_1,...,a_m,b_1...b_n \in \{0,...,3 \} \right \}$ on $T^m_nM$. We call two such fiber coordinates $v^{a_1 ... a_m}_{b_1 ... b_n}$ and $v^{c_1 ... c_m}_{d_1 ... d_n}$ equivalent if once they are restricted to $F \subset T^m_nM$ they only differ by a sign:
\begin{align}\label{equivCord}
v^{a_1 ... a_m}_{b_1 ... b_n} \cong v^{c_1 ... c_m}_{d_1 ... d_n} : \iff \exists \epsilon \in \{-1,+1 \} : v^{a_1 ... a_m}_{b_1 ... b_n} \big \vert _F = \epsilon \cdot  v^{c_1 ... c_m}_{d_1 ... d_n} \big \vert_F.
\end{align}
Removing all coordinate functions that restrict to identical zero on $F$ one then sorts the set of equivalence classes according to some ordering that can be specified at wish.  Note that the number of equivalence classes that we obtain in this way precisely corresponds to the fiber dimension of $F$. In the following, we label the $A$th equivalence classes starting from 0 by $[A]$. Next, we select a representative out of each equivalence class.  The selection of a representative $v^{a_1 ... a_m}_{b_1 ... b_n}$ of a given equivalence class $[A]$ allows us to split this equivalence class into the subset $[A]_+$ containing those fiber coordinates that restrict to $+1 \cdot v^{a_1 ... a_m}_{b_1 ... b_n}$  on $F$ and the remaining class $[A]_-$ with thus according to (\ref{equivCord}) necessarily restricts to $-1 \cdot v^{a_1 ... a_m}_{b_1 ... b_n}$ on $F$.
With each equivalence class $[A]$ we further associate its \textbf{\textit{multiplicity}} $\sigma(A) = \vert [A] \vert$, which is simply given by the number of its elements.
Now we can define:
\begin{align}\label{defI}
    I^{A a_1 ... a_m}_{b_1 ... b_n} = \begin{cases} 
        +1 \ \  &\text{if} \  v^{a_1 ... a_m}_{b_1 ... b_n} \in [A]_+ \\
        -1 \ \ &\text{if} \  v^{a_1 ... a_m}_{b_1 ... b_n} \in [A]_-  \\
        0 \ \   &\text{otherwise}. 
    \end{cases}
\end{align}
$J^{b_1 ... b_n}_{A a_1 ... a_m}$ is then defined such that the pair of intertwiners satisfies the last condition in (\ref{interRel}), i.e., $J$ is chosen as an left inverse of $I$. On readily finds that:
\begin{align}\label{defJ}
    J^{b_1 ... b_n}_{A a_1 ... a_m} = \begin{cases}  +\frac{1}{\sigma(A)} \ \ &\text{if} \  v^{a_1 ... a_m}_{b_1 ... b_n} \in [A]_+\\
    -\frac{1}{\sigma(A)} \ \  &\text{if} \  v^{a_1 ... a_m}_{b_1 ... b_n} \in [A]_- \\ 
    0   \ \ &\text{otherwise}.
    \end{cases}
\end{align}
Considering again the metric tensor as an example, we have $v_{ab} = v_{ba}$ as only symmetry of the $16$ coordinate functions on $T^0_2M$. We sort the equivalence classes as: 
\begin{align}
    \bigl[[v_{00}], [v_{01}], [v_{02}], [v_{03}], [v_{11}], [v_{12}], [v_{13}], [v_{22}], [v_{23}], [v_{33}]\bigr ].
\end{align}
Proceeding along the lines outlined above, we now obtain the intertwiner $I$ in terms of the components of the following matrix: 
\begin{equation}\label{interIMet}
I^A_{ab} = \begin{bmatrix}
                1 & \cdot & \cdot & \cdot & \cdot & \cdot & \cdot & \cdot & \cdot & \cdot \\
                \cdot & 1 & \cdot & \cdot & \cdot & \cdot & \cdot & \cdot & \cdot & \cdot \\
                \cdot & \cdot & 1 & \cdot & \cdot & \cdot & \cdot & \cdot & \cdot & \cdot \\
                \cdot & \cdot & \cdot & 1 & \cdot & \cdot & \cdot & \cdot & \cdot & \cdot \\
                \cdot & 1 & \cdot & \cdot & \cdot & \cdot & \cdot & \cdot & \cdot & \cdot \\
                \cdot & \cdot & \cdot & \cdot & 1 & \cdot & \cdot & \cdot & \cdot & \cdot \\
                \cdot & \cdot & \cdot & \cdot & \cdot & 1 & \cdot & \cdot & \cdot & \cdot \\
                \cdot & \cdot & \cdot & \cdot & \cdot & \cdot & 1 & \cdot & \cdot & \cdot \\
                \cdot & \cdot & 1 & \cdot & \cdot & \cdot & \cdot & \cdot & \cdot & \cdot  \\
                \cdot & \cdot & \cdot & \cdot & \cdot & 1 & \cdot & \cdot & \cdot & \cdot  \\
                \cdot & \cdot & \cdot & \cdot & \cdot & \cdot & \cdot & 1 & \cdot & \cdot\\
                \cdot & \cdot & \cdot & \cdot & \cdot & \cdot & \cdot & \cdot & 1 & \cdot \\
                \cdot & \cdot & \cdot & 1 & \cdot & \cdot & \cdot & \cdot & \cdot & \cdot \\
                \cdot & \cdot & \cdot & \cdot & \cdot & \cdot & 1 & \cdot & \cdot & \cdot \\
                \cdot & \cdot & \cdot & \cdot & \cdot & \cdot & \cdot & \cdot & 1 & \cdot \\
                \cdot & \cdot & \cdot & \cdot & \cdot & \cdot & \cdot & \cdot & \cdot & 1 
            \end{bmatrix}.
\end{equation}
Here the rows run over $(a,b)={(0,0),(0,1),(0,2),...,(3,3)}$ and the columns run from $0$ to $9$. Only non-vanishing components are displayed. The inverse intertwiner is then given as: 
\begin{equation}\label{interJMet}
J^{ab}_{A} = \begin{bmatrix} 
                1 & \cdot & \cdot & \cdot & \cdot & \cdot & \cdot & \cdot & \cdot & \cdot & \cdot & \cdot & \cdot & \cdot & \cdot & \cdot \\
                \cdot & \frac{1}{2} & \cdot & \cdot & \frac{1}{2} & \cdot & \cdot & \cdot & \cdot & \cdot & \cdot & \cdot & \cdot & \cdot & \cdot & \cdot  \\
                \cdot & \cdot & \frac{1}{2} & \cdot & \cdot & \cdot & \cdot & \cdot & \frac{1}{2} & \cdot & \cdot & \cdot & \cdot & \cdot & \cdot & \cdot  \\
                \cdot & \cdot & \cdot & \frac{1}{2} & \cdot & \cdot & \cdot & \cdot & \cdot & \cdot & \cdot & \cdot & \frac{1}{2} & \cdot & \cdot & \cdot  \\
                \cdot & \cdot & \cdot & \cdot & \cdot & 1 & \cdot & \cdot & \cdot & \cdot & \cdot & \cdot & \cdot & \cdot & \cdot & \cdot  \\
                \cdot & \cdot & \cdot & \cdot & \cdot & \cdot & \frac{1}{2} & \cdot & \cdot & \frac{1}{2} & \cdot & \cdot & \cdot & \cdot & \cdot & \cdot  \\
                \cdot & \cdot & \cdot & \cdot & \cdot & \cdot & \cdot & \frac{1}{2} & \cdot & \cdot & \cdot & \cdot & \cdot & \frac{1}{2} & \cdot & \cdot  \\
                \cdot & \cdot & \cdot & \cdot & \cdot & \cdot & \cdot & \cdot & \cdot & \cdot & 1 & \cdot & \cdot & \cdot & \cdot & \cdot  \\
                \cdot & \cdot & \cdot & \cdot & \cdot & \cdot & \cdot & \cdot & \cdot & \cdot & \cdot & \frac{1}{2} \cdot & \cdot & \cdot & \frac{1}{2} & \cdot  \\
                \cdot & \cdot & \cdot & \cdot & \cdot & \cdot & \cdot & \cdot & \cdot & \cdot & \cdot & \cdot & \cdot & \cdot & \cdot & 1  
            \end{bmatrix},
\end{equation}
now with columns running over $(ab)={(0,0),(0,1),(0,2),...,(3,3)}$ and rows running from $0$ to $9$. By this definition, we now have for the metric tensor field $v_1 \cong v_{01}, v_2 \cong v_{0,2},...$.
\begin{remark}
It might be tempting to define the pair of intertwiners in a way such that the matrix representation of $I$ equals the matrix transpose of the matrix representing $J$. Such a property could be achieved by not defining $I$ without fractions and introducing the factors that result from the multiplicities $\sigma$ in $J$, but by splitting them evenly and introducing factors of $\frac{1}{\sqrt{\sigma}}$ both in the definition of $I$ and $J$. Although this seems advantageous, there is one particular problem with this definition. The usage of square root factors might introduce irrational numbers into the intertwiners. As we are going to rely heavily on computer algebra later, and the appearance of irrational numbers severely obstructs the computational performance, we stick to the square root free approach.  
\end{remark}

In the end, we want to describe the gravitational field as sections of such a field bundle and construct for it a Lagrangian and the corresponding equations of motion. For that reason, it is necessary to introduce a notion of \textit{\textbf{deriving}} such sections. 

It is crucial to keep in mind that one of the aims of the Constructive Gravity approach is allowing for a deviation from metric gravity theories. Hence we are not dealing with spacetimes where the usual metric structures are present. In particular, there is, a priori, no distinguished connection on the frame bundle and therefore no covariant derivative that can be used to define derivatives of tensor fields in a coordinate-independent fashion. Note, however, that already in standard General Relativity those notions are strictly speaking not necessary. Although there, the fundamental object, the Riemann tensor might be interpreted as curvature 2-form associated to the metric compatible connection on the frame bundle, in coordinates it can be entirely specified in terms of the metric tensor its inverse and up to second partial derivatives of the corresponding component functions without ever mentioning any deeper structure that these expressions might define. The same holds for the Einstein-Hilbert-Lagrangian. 

We, therefore proceed as follows: We keep the involved structures at a minimum, aiming to construct a Lagrangian solely in terms of the gravitational field and partial derivatives of the gravitational field up to some finite order, just as it is the case for the Einstein-Hilbert-Action\footnote{In other words, we are not trying to construct structural analogies to General Relativity or some other known theory of gravity.}. In order to nevertheless be able to describe partial derivatives of the gravitational field in a well-defined, i.e., coordinate-independent fashion, we construct the $\textit{\textbf{jet bundle}}$ over the field bundle.

The jet bundle over a given bundle can be intrinsically defined in an entirely abstract fashion. Such an approach nicely reveals its geometric properties. This is, however, not the route we are going to take here, as a coordinate-based approach to jet bundles is intuitively often more accessible. Furthermore, such an approach emphasizes similarities to the construction of the tangent bundle of a manifold. Details regarding the more abstract access to jet bundles can be found in particular in the first chapters of \cite{seiler2009involution}. 

We start by defining the \textbf{\textit{1-jet}} of a \textbf{\textit{section}}.
\begin{definition}[1-jet] 
Given a bundle $(F, \pi_F, M)$, with adapted coordinates $(x^m, v_A)$ and sections $G,H \in \Gamma(F)$, we call $G$ and $H$ 1-equivalent at $p 
\in M$, if:
\begin{align}
    \begin{aligned}
    G(p) &= H(p) \\
    \text{and \ \ }
    \partial_i G_A \big \vert_p &= \partial_i H_A \big \vert_p.
    \end{aligned}
\end{align}
The corresponding equivalence class of a section $G$ at $p$ is then called the 1-jet of $G$ at $p$ and denoted by $j^1_pG$.
\end{definition}
Here $G_A$ and $H_A$ denote as before the chart representations of the two sections w.r.t. the specified adapted coordinates. One can easily check that this indeed defines an equivalence relation and further is independent of the chosen coordinate chart, i.e., well-defined. Details can be found in \cite{saunders_1989}. From this definition, one immediately sees the similarity to the construction of tangent vectors as equivalence classes of curves. Just as in the case of tangent vectors, where one collects all possible equivalence classes of curves to obtain the tangent bundle, we can now collect all such 1-jets of sections of $F$ to obtain the first-order jet bundle $J^1F$ over $F$. 
\begin{definition}[first jet bundle]
Let $(F, \pi_F, M)$ be a bundle, the first jet bundle over $F$, $(J^1F,\pi_1,M)$ is the bundle with total space $J^1F := \{j^1_pG \ \vert \  p \in M, G \in \Gamma(F)\}$ and bundle projection: 
\begin{align}
    \begin{aligned}
\pi_1 : J^1F &\longrightarrow M \\
j^1_pG &\longmapsto p.
    \end{aligned}
\end{align}
\end{definition}
\begin{remark}
Note that by this definition $J^1F$ is a bundle over the base manifold $M$. We can also equip $J^1F$ with the projection: 
\begin{align}
    \begin{aligned}
    \pi_{1,0} : J^1F &\longrightarrow F \\
    j^1_pG &\longmapsto G(p)
    \end{aligned}
\end{align}
By doing so we additionally provide $J^1F$ with the structure of a bundle over $F$, namely $(J^1F,\pi_{1,0},F)$. From the definition of the two projection we furthermore find that $\pi_1 = \pi_F \circ \pi_{1,0}$. 
\end{remark}
Given adapted coordinates $(x^m,v_A)$ on $F$ we get induced adapted coordinates $(x^m, v_A, v_{Ai})$ on $J^1F$ by specifying the new $4 \cdot n$ coordinate functions as: 
\begin{align}
v_{Ai}(j^1_pG) := \partial_iG_A \big \vert_p.
\end{align}
We could now define \textit{\textbf{higher-order jet bundles}} in a similar fashion, by calling two sections of $F$ \textit{\textbf{q-equivalent}} or equivalent in order $q$ at $p \in M$ if in any coordinate system covering $p$ they agree in their first $q$ partial derivatives evaluated at $p$. There are, however, arguments why we do not have to work with jet bundles higher than second-order if we wish to define a physically meaningful Lagrangian field theory.

As we want to use the jet-bundle formalism to define a Lagrangian and then obtain equations of motion (in short EOM) for the gravitational field as the corresponding Euler-Lagrange Equations, the order of the jet bundle is ultimately connected to the order of the highest derivative appearing in the EOM. In the following, we want to restrict to cases where the EOM are of no higher than second derivative order. The reason for this restriction is the famous \textbf{\textit{Ostrogratsky}} construction \cite{Ostrogradsky:1850fid}. 
% review this statement 

Simply stated, starting from Lagrangians that generate equations of motion with higher than second-order time derivatives, one obtains instabilities in the associated Hamiltonian formulation. From the perspective of covariant Lagrangian field theory, there is no distinct time direction, and hence the only way of preventing the theory from these kinds of instabilities once we pass to the Hamiltonian formulation is by restricting the Lagrangian such that the generated EOM are of second-derivative-order altogether. For further information see \cite{2015arXiv150602210W}. 

The appropriate setting for such Lagrangians is the \textbf{\textit{second-order jet bundle}}. Although completely analog to the definitions involved in the construction of the first-order jet bundle we quickly define the necessary ingredients. 
\begin{definition}[2-jet]
Given a bundle $(F, \pi_F, M)$ with adapted coordinate chart $(x^m, v_A)$ and sections $G,H \in \Gamma(F)$, we call $G$ and $H$ 2-equivalent at $p \in M$, if:
\begin{align}
    \begin{aligned}
    G(p) &= H(p), \\
    \partial_i G_A  \big \vert_p &= \partial_i H_A \big \vert_p, \\
    \text{and \ \ }
    \partial_i \partial_j G_A  \big \vert_ p &= \partial_i \partial_j H_A \big \vert_ p .
    \end{aligned}
\end{align}
The corresponding equivalence class of a section $G$ at $p$ is then called the 2-jet of $G$ at $p$ and denoted by $j^2_pG$.
\end{definition}
Again one can show that this is well-defined \cite{saunders_1989}. We collect all such 2-jets for all sections of $F$ and all points in $M$ to obtain the second-order jet bundle over $F$.
\begin{definition}[second jet bundle]
Let $(F, \pi_F, M)$ be a bundle, the second jet bundle $(J^2F,\pi_2,M)$ is the bundle with total space $J^2F := \{j^2_pG \  \vert \  p \in M, G \in \Gamma(F)\}$ and bundle projection: 
\begin{align}
    \begin{aligned}
\pi_2 : J^2F &\longrightarrow M \\
j^2_pG &\longmapsto p.
    \end{aligned}
\end{align}
\end{definition}
\begin{remark}
One can again show that this really defines a bundle over the base manifold $M$. In the case of the second-order jet bundle, we can obtain two further bundle projections: 
\begin{align}
    \begin{aligned}
    \pi_{2,1} : J^2F &\longrightarrow J^1F \\
    j^2_pG &\longmapsto j^1_pG ,\\[0.5ex]
    \pi_{2,0} : J^2F &\longrightarrow F \\
    j^2_pG &\longmapsto G(p).
    \end{aligned}
\end{align}
These bundle projections provide $J^2F$ with the additional structure of a bundle over $J^1F$ with projection $\pi_{2,1}$ or a bundle over $F$ with projection $\pi_{2,0}$. Note that from the various definitions we find $\pi_{2} = \pi_F \circ \pi_{2,0} = \pi_F \circ \pi_{1,0} \circ \pi_{2,1}$.
\end{remark}
As before we can obtain adapted coordinates $(x^m, v_A, v_{Ai},v_{Aij})$ of $J^2F$ from adapted coordinates $(x^m,v_A)$ on $F$ by defining:
\begin{align}
    v_{Aij}(j^2_pG) := \partial_i \partial_j G_A \big \vert _p.
\end{align} 
At this point, however, we run into similar troubles as before, when we tried to use coordinate-induced basis vectors to construct a chart on $F$. 
We have specified $4^2\cdot n$ new coordinate functions $v_{Aij}$ to construct adapted coordinates on $J^2F$. Due to the commutative property of the partial derivatives that are involved in the definition the $v_{Aij}$ actually only constitute $4\cdot (4+1)/2 \cdot n = 10\cdot n$ independent functions. To cope with such problems, one usually introduces multi-indices for the description of higher-order derivative coordinates on higher-order jet bundles. We will follow a slightly different approach by making use of the same intertwiner technique that we already have developed before, for the description of fiber coordinates on $F$. By doing so, it is easier for us to relate the derivative coordinates $v_{Aij}$ and possible higher jet coordinates to partial derivatives of sections in standard physics notation. 

We proceed along the same lines as before when we constructed a pair of intertwiners for the bundle of symmetric $(0,2)$ tensor fields, i.e., we provide a pair of intertwiners and thereby define $10$ new indices that label second-order derivative in terms of the $16$ possible pairs of spacetime indices:
\begin{align}
    \begin{aligned}
    v_{Aij} &= I^I_{ij} v_{AI}\\
    v_{AI} &= J_I^{ij} v_{Aij}.
    \end{aligned}
\end{align}
We divide the $16$ coordinates that label symmetric second-order derivatives $(ij)$ into the $10$ equivalence classes: 
\begin{align}
    \bigl [[(00)],[(01)],[(02)],[(03)],[(11)],[(12)],[(13)],[(22)],[(23)],[(33)] \bigr ].
\end{align}
We then define the intertwiners as before by (\ref{defI}) and (\ref{defJ}). 
This choice is given by the intertwiners specified by the matrices displayed in (\ref{interIMet}) and (\ref{interJMet}) that were obtained as a pair of intertwiners for a metric tensor field. The reason for this coincidence is apparent, as in both cases, we are dealing with symmetric index pairs. 

\begin{remark}
As stated above during this thesis, we will, in fact, never perform explicit computations on a jet bundle that is of higher than second-order. To nevertheless be able to provide some of the relevant definitions on jet bundles of arbitrary order we will in the following denote adapted coordinates on $J^qF$ by: 
\begin{align}
(x^m,v_A,v_{Am},v_{AI},v_{AI_{3}},...,v_{AI_{q}}),
\end{align}
where we introduced new indices $I_k$ that range from $0$ to $\binom{4+k-1}{k}-1$ and label the independent combinations of partial derivatives of order $k$. In the following, when displaying formulae that include such indices $I_k$ we might write $v_{AI_0} \equiv v_m, v_{AI_1} \equiv v_A$ and $v_{AI_2} \equiv v_{AI}$, such that we can concisely write expressions that involve sums involving all such indices:
\begin{align}
    a^mv_m + a^Av_A + a^{AI}v_{AI} + ... + a^{AI_q}v_{AI_q} = \sum _{k = 0}^q a^{AI_k}v_{AI_k}.
\end{align}
Note that we use no summation convention over $k$, but explicitly write out such sums.
As before we denote the corresponding intertwiners mapping between the representation of derivative in terms of such indices and the spacetime representation by:
\begin{align}
    \begin{aligned}
    v_{Ai_1...i_k} &= I^{I_k}_{i_1..i_k} v_{AI_k}\\
    v_{AI_k} &= J_{I_k}^{i_1...i_k} v_{Ai_1...i_k},
    \end{aligned}
\end{align}
with the obvious intertwiners for $k=0,..2$.
They can, in fact, be obtained entirely analog to the above example, we will, however, never need their explicit expression. 
The projections that project from a given $q$th-order jet bundle to the $k$th-order jet bundle with $k< q$ are denoted as before by $\pi_{q,k}$ with $\pi_q = \pi_F \circ \pi_{q,0}$.
\end{remark}

Certain operations naturally can be performed when working in the context of the jet-bundle formalism.
Given a section of the field bundle $G \in \Gamma(F)$, we can \textit{\textbf{prolong}} it to a section of $J^1F$, $J^2F$, or in fact any higher-order jet bundle $J^qF$ by, at each point $p \in M$, taking its 1-jet $j^1_pG$, 2-jet $j^2_pG$, or q-jet $j^q_pG$, respectively. The section of the $q$th-order jet bundle $J^qF$ that is obtained this way is called the $q$th \textbf{\textit{jet prolongation}} of $G$ and is denoted as $j^q(G)$. In adapted coordinates this is simply achieved by appending the appropriate derivatives of the section, for instance the coordinate representation of $j^2(G)$ is given by the map $x^m \mapsto (x^m, G_A, \partial_i G_A, \partial_I G_A)$, where $\partial_I := J_I^{ij} \partial_i \partial_j$.

Given a function $f \in C^{\infty}(F)$, with coordinate expression $f(x^m, v_A)$ we can compute its \textit{\textbf{total derivative}} as $D_if = \partial_i f + v_{Ai}\cdot \partial^A f$, which yields a function on $J^1F$. Here the expression $\partial^A f$ denotes the action of the coordinate-induced vector fields $\frac{\partial}{\partial v_A}$ on the coordinate expression of $f$. Similarly if $h \in C^{\infty}(J^1F)$ its total derivative $D_i h = \partial_i h + v_{Ai} \partial^A h + v_{AI}I^I_{ij} \partial ^{Aj} h$ defines a function on $J^2F$. In the same way we can compute total derivatives for functions on any higher $q$th-order jet bundle $J^qF$ to obtain a function on $J^{q+1}F$:
\begin{align}\label{totDer}
    D_j g = \sum _{k = 0}^{q}  v_{AI_{k+1}}I^{I_{k+1}}_{ji_1...i_k}J_{J_k}^{i_1...i_k}\partial^{AJ_k} g.
\end{align}
 

Lastly, with the definition of the total derivative at hand, given a function on some finite order jet bundle $f \in C^{\infty}(J^qF)$ with coordinate expression $f(x^m,v_A,v_{Am},v_{AI},v_{AI_{3}},...,v_{AI_{q}})$ we define its \textit{\textbf{variational derivative}} as: 
\begin{align}\label{varDer}
\frac{\delta f}{\delta v_A} := \partial^{A}g + \sum _{k = 1}^q (-1)^k D_{I_k}(\partial^{AI_k}f),
\end{align}
where $D_I := J_I ^{ij} D_i D_j$ and similar $D_{I_k} := J_{I_k}^{i_1...i_k} D_{i_1} ... D_{i_k}$.
Note that in particular the variational derivative of a function on the $q$th-order jet-bundle in general yields a function on the $2q$th-order jet bundle.

Now we have finally developed the necessary structure to give a precise definition of a \textbf{\textit{Lagrangian}} for a field theory whose dynamical objects are sections of the field bundle $(F, \pi_F,M)$.
\begin{definition}[Lagrangian]
A second-order Lagrangian on $(F,\pi_F,M)$ is a bundle map $\mathcal{L} : J^2F \rightarrow \Lambda^4M$ that covers $id_M$.
\end{definition}
Here $\Lambda^4 M$ denotes the volume bundle on $M$. Alternatively, we could have required $\mathcal{L}$ to take values in the bundle of scalar densities of weight 1 over $M$. In coordinates we have: 
\begin{align}
    \mathcal{L}(x^m,v_A,v_{Ai},v_{AI}) = L(x^m,v_A,v_{Ai},v_{AI}) \mathrm{d}x^0 \wedge \mathrm{d}x^1 \wedge \mathrm{d}x^2 \wedge \mathrm{d}x^3 = L \mathrm{d}^4x.
\end{align} 

As the Lagrangian is a volume form valued bundle map, we can take any section $G \in \Gamma(F)$, prolong it to $j^2(G) \in \Gamma(J^2F)$ and compose the result with $\mathcal{L}$ to obtain section of the volume bundle over $M$, $\mathcal{L}(j^2(G)) \in \Gamma(\Lambda^4M)$ which thus can be integrated over spacetime regions. Given a Lagrangian this procedure defines a \textit{\textbf{local action functional}} on the space of fields $\Gamma(F)$:
\begin{align}
\begin{aligned}
    \mathcal{S}_{\mathcal{L}} : \Gamma(F) &\longrightarrow \mathcal{R} \\
    G &\longmapsto \mathcal{S}_{\mathcal{L}}[G] := \int \mathcal{L}(j^2(G)).
\end{aligned}
\end{align}
The whole situation with the various maps that are involved in this construction is displayed in Figure \ref{diagram1}. 
\begin{figure}[hbt!]
\centering
\begin{tikzpicture}
\node (M) at (0,0) {$M$};
\node (F) at (0,2) {$F$};
\node (J1) at (0,4) {$J^1F$};
\node (J2) at (0,6) {$J^2F$};
\node (Vol) at (6,6) {$\Lambda^4M$};
\draw [-latex] (F) -- node[pos=0.35, right] {$\pi_F$} (M);
\draw [-latex] (J1) -- node[pos=0.35, right] {$\pi_{1,0}$}  (F);
\draw [-latex] (J2) -- node[pos=0.35, right] {$\pi_{2,1}$}  (J1);
\draw [-latex] (Vol.220) -- node[pos=0.4, left] {$\pi_{\Lambda^4M}$ \ }  (M.60);
\draw[-latex] (M) .. controls (-0.75,0.5) and (-0.75,1.5) .. node[pos=0.5, left] {$G$} (F);
\draw[-latex] (M) .. controls (-3,1) and (-3,5) .. node[pos=0.5, left] {$j^2(G)$} (J2);
\draw [-latex] (J2) -- node[pos=0.5, above] {$\mathcal{L}$}  (Vol);
\draw[-latex] (M.30) -- node[pos=0.5, right] { \ $\mathcal{L}\circ j^2(G)$} (Vol.250);
\end{tikzpicture}
\caption{Commutative Diagram: Lagrangian Field Theory on $J^2F$.} \label{diagram1}
\end{figure}
Note that by giving a proper definition of the Lagrangian as a bundle map between $J^2F$ and $\Lambda^4M$ we are able to fully encode the properties of the corresponding action functional in terms of a smooth map between finite-dimensional manifold. At no place, we actually have to work over infinite-dimensional spaces of sections and functionals defined on them. In particular, when working with the Lagrangian $\mathcal{L}$ we can apply the standard techniques of differential geometry. 
\section{Lagrangian Equivariance Equations}
One of the critical requirements that Constructive Gravity poses on the to-be-constructed dynamics for the gravitational field is their invariance under spacetime diffeomorphisms. In the following, we will show how this requirement can be incorporated in the previously developed framework of Lagrangian field theory. We start by computing the transformation behavior of 2-jets, $j_p^2G \in J^2F$, under spacetime diffeomorphisms.

The action of diffeomorphisms on $M$ naturally lifts to any tensor space over $M$ by the usual push-forward pull-back construction. To be more precise given $\phi \in \mathrm{Diff}(M)$ we can obtain vector bundle isomorphisms $\phi_{\ast} : TM \rightarrow TM $ and $\phi^{\ast} : T^{\ast}M \rightarrow T^{\ast}M$ that cover $\phi$ by taking its pushforward and pullback respectively. As the pull back is a contravariant functor, i.e., $(\phi \circ \psi)^{\ast}=\psi^{\ast}\circ \phi^{\ast}$, given $\phi \in \mathrm{Diff}(M)$ we simply take the pullback along $\phi^{-1}$ to compensate for the change of composition order. The action of diffeomorphisms on tensor spaces over $M$, $T^m_nM$ can then be obtained by tensoring m copies of the pushforward action and n copies of the inverse pull back action. In other words given $\phi \in \mathrm{Diff}(M)$ we get a vector bundle isomorphism on any $T^m_nM$ at wish by:
\begin{align}
\begin{aligned}
    &\underbrace{\phi_{\ast} \otimes ... \otimes \phi_{\ast}}_{\substack{m \  times}} \otimes \underbrace{(\phi^{-1})^{\ast} \otimes ... \otimes (\phi^{-1})^{\ast} }_{\substack{n \  times}} : T^m_nM \longrightarrow T^m_nM \\
    u_1 \otimes ... \otimes u_m &\otimes \omega_1 \otimes ... \otimes \omega_n \longmapsto \phi_{\ast} (u_1) \otimes ... \otimes \phi_{\ast}(u_m) \otimes (\phi^{-1})^{\ast}(\omega_1) \otimes ... \otimes (\phi^{-1})^{\ast}(\omega_n).
\end{aligned}
\end{align}
The action is then defined on all other elements of $T^m_nM$, i.e., on general linear combinations of product elements, by extending it linearly.

\begin{comment}
Note that there is a slight duplication in our notation: Before a superscripted $\ast$ denoted the dual of a linear map, the dual of a vector-space/bundle, etc. Here the superscripted $\ast$ denotes the pull back of a given diffeomorphism. The connection between the two notions and henceforth the reason for the apparent duplication in the notation can be explained regarding the standard definition of the pullback: Let $\phi \in \mathrm{Diff}(M)$, $p \in M$, $u \in T_pM$, and  $\omega \in T^{\ast}_{\phi(p)}M$ then pushforward and pullback of $\phi$ restrict to linear maps $(\phi_{\ast}) _p :  T_pM \rightarrow T_{\phi(p)}M$ and $\phi^{\ast}_{\phi(p)} : T^{\ast}_{\phi(p)}M \rightarrow T^{\ast}_pM$, where the pullback is defined by the requirement $\phi^{\ast}_{\phi(p)}\omega (u) := \omega ((\phi_{\ast})_p u)$. Comparing this with definition \ref{dual} the dual of a linear map we see that the pull back really defines the fiber wise dual of the pushforward. 
\end{comment}

To shorten the notation we will refer to actions of $\mathrm{Diff}(M)$ on the various tensor spaces $T^m_nM$ that can be obtained by means of this pushforward pullback construction simply as pullback actions and denote the corresponding bundle morphisms that cover a given $\phi \in \mathrm{Diff}(M)$ collectively by $\phi_{\ast}$. Which precise tensor bundle is meant can usually be inferred from the context quite easily.

Given $\phi \in \mathrm{Diff}(M)$, by this construction, we can obtain the corresponding action by bundle isomorphisms on any tensor bundle over $M$. As we have identified the field bundle $F$ with a subbundle of some $T^m_nM$ we can restrict the bundle isomorphisms to $F$ to get an action of $\mathrm{Diff}(M)$ on the field bundle in particular. 
To further lift the action of $\mathrm{Diff}(M)$ to the first jet bundle $J^1F$ and the second jet bundle $J^2F$,respectively, we need to develop additional techniques, the primary tool being the \textbf{\textit{prolongation of bundle morphisms}}. 
\begin{definition}[prolongation of morphisms]
Let $(F_1,\pi_{F_1},M)$ and $(F_2,\pi_{F_2},N)$ be bundles, $\phi : M \rightarrow N$ a diffeomorphism, $f : F_1 \rightarrow F_2$ a bundle morphism covering $\phi$ and $J^1F_1$, $J^1F_2$ the first-order jet bundles over $F_1$ and $F_2$, respectively. The first jet bundle lift or prolongation of the bundle morphism $(f,\phi)$ is given by the unique map\footnote{Just as it is the case with bundle morphisms being denoted by pairs $(f,\phi)$ their prolongation should be denoted by triples $(j^1(f),f,\phi)$. However, we will lighten their notation to simply $j^1(f)$ if the two remaining maps are clear from the given context.} $j^1(f)$ that lets the diagram in Figure \ref{ProlongMorph} commutes.
\begin{figure}[hbt!]
\centering
\begin{tikzpicture}
\node (M) at (0,0) {$M$};
\node (N) at (5,0) {$N$};
\node (F1) at (0,3) {$F_1$};
\node (F2) at (5,3) {$F_2$};
\node (JF1) at (0,6) {$J^1F_1$};
\node (JF2) at (5,6) {$J^1F_2$};
\draw [-latex] (M.10) -- node[pos=0.5, above] {$\phi$} (N.170);
\draw [<-] (M.350) -- node[pos=0.5, below] {$\phi^{-1}$} (N.190);
\draw [-latex] (F1) -- node[pos=0.5, above] {$f$} (F2);
\draw [-latex] (F1) -- node[pos=0.5, left] {$\pi_{F_1}$} (M);
\draw [-latex] (F2) -- node[pos=0.5, right] {$\pi_{F_2}$} (N);
\draw [-latex] (JF1) -- node[pos=0.5, left] {$(\pi_1)_{1,0}$} (F1);
\draw [-latex] (JF2) -- node[pos=0.5, right] {$(\pi_2)_{1,0}$} (F2);
\draw [-latex] (JF1) -- node[pos=0.5, above] {$j^1(f)$} (JF2);
\end{tikzpicture}
\caption{Commutative Diagram: Prolongation of Bundle Morphisms to First Jet Bundle.}\label{ProlongMorph}
\end{figure}
\end{definition}
The explicit expression for the prolongation of $(f,\phi)$ is then given by:
\begin{align}
    j^1(f)(j^1_pG) = j^1_{\phi(p)}(f\circ G).
\end{align}
We can now investigate how the prolongation $j^1(f)$ acts on the prolongation of sections of $F_1$.
Given a section $G \in \Gamma(F_1)$ we can map it to a section of $F_2$ by making use of the required invertibility of $\phi$: 
\begin{align}
\begin{aligned}
f_{sec} : \Gamma(F_1) &\longrightarrow \Gamma(F_2) \\
G & \longmapsto f \circ G \circ \phi^{-1}.
\end{aligned}
\end{align}
Aside from that we can also prolong the section to obtain a section $j^1(G) \in \Gamma(J^1F_1)$. Obviously we can now apply the same construction to the prolonged section to obtain:
\begin{align}
\begin{aligned}
j^1f_{sec}: \Gamma(J^1F_1) &\longrightarrow \Gamma(J^1F_2) \\
j^1G & \longmapsto  j^1(f) \circ j^1(G) \circ \phi^{-1}.
\end{aligned}
\end{align}
From the definition of the prolongation of bundle morphisms we then find:
\begin{align}
    j^1f_{sec} \circ j^1 \circ G = j^1 \circ f_{sec} \circ G.
\end{align}
In other words, the induced action of bundle morphisms on sections commutes with the operation of prolonging the involved section and bundle morphisms, respectively. We can either first map a section of $F_1$ by using the bundle morphisms $(f,\phi)$ to a section of $F_2$ and then prolong it to a section of $J^1F_2$ or we prolong it first to a section of $J^1F_1$ and then map it with the prolonged bundle morphism $(j^1(f),f,\phi)$ to a section of $J^1F_2$, the result is the same. The situation is displayed in Figure \ref{BundleMorphSec}. 

\begin{figure}[hbt!]
\centering 
\begin{tikzpicture}
\node (F1) at (0,0) {$\Gamma(F_1)$};
\node (F2) at (4,0) {$\Gamma(F_2)$};
\node (JF1) at (0,3) {$\Gamma(J^1F_1$)};
\node (JF2) at (4,3) {$\Gamma(J^1F_2)$};
\draw [-latex] (F1) -- node[pos=0.5, below] {$f_{sec}$} (F2);
\draw [-latex] (JF1) -- node[pos=0.5, above] {$j^1f_{sec}$} (JF2);
\draw [-latex] (F1) -- node[pos=0.5, left] {$j^1$} (JF1);
\draw [-latex] (F2) -- node[pos=0.5, right] {$j^1$} (JF2);
\end{tikzpicture}
\caption{Commutative Diagram: Prolongation of Bundle Morphisms Applied to Sections.}\label{BundleMorphSec}
\end{figure}

We can prolong bundle morphisms to any higher-order jet bundle by an equivalent definition, i.e., by defining $j^k(f)$ s.t. the diagram in Figure \ref{POrolongK} commutes:
\begin{figure}[hbt!]
\centering
\begin{tikzpicture}
\node (M) at (0,0) {$M$};
\node (N) at (5,0) {$N$};
\node (F1) at (0,3) {$F_1$};
\node (F2) at (5,3) {$F_2$};
\node (JF1) at (0,6) {$J^kF_1$};
\node (JF2) at (5,6) {$J^kF_2$};
\draw [-latex] (M.10) -- node[pos=0.5, above] {$\phi$} (N.170);
\draw [<-] (M.350) -- node[pos=0.5, below] {$\phi^{-1}$} (N.190);
\draw [-latex] (F1) -- node[pos=0.5, above] {$f$} (F2);
\draw [-latex] (F1) -- node[pos=0.5, left] {$\pi_{F_1}$} (M);
\draw [-latex] (F2) -- node[pos=0.5, right] {$\pi_{F_2}$} (N);
\draw [-latex] (JF1) -- node[pos=0.5, left] {$(\pi_1)_{k,0}$} (F1);
\draw [-latex] (JF2) -- node[pos=0.5, right] {$(\pi_2)_{k,0}$} (F2);
\draw [-latex] (JF1) -- node[pos=0.5, above] {$j^k(f)$} (JF2);
\end{tikzpicture}
\caption{Commutative Diagram: Prolongation of Bundle Morphisms to $k$th-order Jet Bundle.} \label{POrolongK}
\end{figure}
One can in particular show that the jet prolongation of bundle morphism behaves functorial w.r.t. their composition: $j^k(f\circ g) = j^k(f) \circ j^k (g)$. Details regarding the prolongation of bundle morphisms can be found in \cite{saunders_1989}.

Now we are in a position where we can formulate what we mean by requiring the gravitational theory to be invariant under spacetime diffeomorphisms in a precise manner. We start by recalling the definition of an \textit{\textbf{equivariant}} map. Further information regarding this can be found in \cite{doi:10.1142/3867}.
\begin{definition}[equivariance]
Let $M$ and $N$ be some spaces that both carry an action of a group $\mathcal{G}$, $\rho : \mathcal{G} \rightarrow \mathrm{Aut}(M)$ and $\sigma : \mathcal{G} \rightarrow \mathrm{N}$. We call a function $f : M \rightarrow N$ equivariant w.r.t. $\rho$ and $\sigma$ if it holds for all $g \in \mathcal{G}$ that: $f \circ \rho(g) = \sigma(g) \circ f(m)$, i.e., if the diagram displayed in Figure \ref{EquiDia} commutes for all $g \in \mathcal{G}$:
\begin{figure}[hbt!]
\centering
\begin{tikzpicture}
\node (M) at (0,0) {$M$};
\node (N) at (4,0) {$N$};
\node (M2) at (0,3) {$M$};
\node (N2) at (4,3) {$N$};
\draw [-latex] (M) -- node[pos=0.5, below] {$f$} (N);
\draw [-latex] (M2) -- node[pos=0.5, above] {$f$} (N2);
\draw [<-] (M) -- node[pos=0.5, left] {$\rho(g)$} (M2);
\draw [<-] (N) -- node[pos=0.5, right] {$\sigma(g)$} (N2);
\end{tikzpicture}
\caption{Commutative Diagram: Equivariant Map.}\label{EquiDia}
\end{figure}
\end{definition}
Finally, we provide a definition for a \textit{\textbf{diffeomorphism invariant Lagrangian field theory}}.
\begin{definition}
A Lagrangian field theory described by a second-order Lagrangian $\mathcal{L} : J^2F \rightarrow \Lambda^4 M$ is called diffeomorphism invariant if $\mathcal{L}$ is equivariant w.r.t. the action of $\mathrm{Diff}(M)$ that is obtained by prolonging diffeomorphisms to bundle isomorphisms of $J^2F$ and the pullback action of diffeomorphisms on $\Lambda^4M$, i.e., if it holds for all $\phi \in \mathrm{Diff}(M)$ that: 
\begin{align}
     \mathcal{L}\circ j^2(\phi_{\ast}) = \phi_{\ast} \circ \mathcal{L}.
\end{align}
\end{definition}

\textit{\textbf{Infinitesimally}}, on the \textbf{\textit{Lie algebra}} level, diffeomorphisms are described by vector fields $\xi \in \Gamma(M)$ with Lie bracket being provided by the commutator of vector fields. 
As usual, one can derive an action of a Lie algebra from a given action of the corresponding Lie group.
Given a Lie group action of some Lie group $\mathcal{G}$ on some manifold $M$, i.e., $\rho : \mathcal{G} \rightarrow \mathrm{Aut}(M)$, one defines for all $p \in M$ the orbit map:
\begin{align}
    \begin{aligned}
    \rho_p : \mathcal{G} &\longrightarrow M \\
    g &\longmapsto \rho(g)(p).
    \end{aligned}
\end{align}
This map is differentiable. For each element of the Lie algebra of $\mathcal{G}$ the pushforward of the orbit map at the identity of $\mathcal{G}$ yields a tangent vector at $p$ . Repeating this process for each $p \in M$ one obtains a vector field on $M$. It can then be shown that mapping each $\xi \in \mathrm{Lie}(\mathcal{G})$ to the negative of this vector field defines a morphism of Lie algebras $\mathrm{Lie}(\mathcal{G}) \rightarrow \Gamma(TM)$, i.e., a Lie algebra action of $\mathrm{Lie}(\mathcal{G})$ on $M$. For details see for instance \cite{boothby1989} and \cite{doi:10.1142/3867}.
Hence this procedure allows one to derive a Lie algebra action from a given Lie group action. Recall that we already have deduced how diffeomorphisms act on the field bundle $F$ and also on the jet bundles defined over it. Thus we can now use the construction outlined above to obtain from these actions of diffeomorphisms \textit{\textbf{Lie algebra morphisms}} that describe how infinitesimal diffeomorphisms act on these bundles.
Note that, for the special case of the pullback action of diffeomorphism on tensor bundles $T^m_nM$ over $M$, given a $\xi \in \Gamma(TM)$ the induced action of the corresponding vector field on $T^m_nM$ on sections $G \in \Gamma(T^m_nM)$ is usually called the \textit{\textbf{Lie derivative}} and denoted by $G \mapsto \mathcal{L}_{\xi}G$. 

We start by applying the above construction to the pullback action of diffeomorphisms on the field bundle in coordinates $x^m$ on $M$, adapted coordinates $(x^m, v_A)$ on $F$ and the corresponding induced coordinates on $TM$ and $TF$ respectively, this Lie algebra morphism takes the following form:
\begin{align}\label{LieF}
\begin{aligned}
    \mathcal{f} : \Gamma(TM) &\longrightarrow \Gamma(TF)\\
    \xi &\longmapsto \xi_F \\
    \smallskip
    \xi_F = \xi^m \frac{\partial}{\partial x^m} + \xi^A \frac{\partial}{\partial v_A} &= \xi^m \frac{\partial}{\partial x^m} + C_{An}^{Bm} v_B \partial_m \xi ^n \frac{\partial}{\partial v_A}. 
\end{aligned}
\end{align}
%
%
%
%
%
%Change the name of constant tensors
%
%
%
Here the constant tensor $C_{An}^{Bm}$ describes the vertical part of $\xi_F$, i.e., the part of the vector field that lies in the kernel of $(\pi_F)_{\ast}$. We will call $C_{An}^{Bm}$ the \textit{\textbf{vertical coefficient}} of the infinitesimal  diffeomorphism action on $F$. The precise form of the vertical coefficient clearly depends on the specific nature, i.e., rank and symmetries of the tensor fields in consideration.
It can most easily be computed by making use of the standard rules of computing Lie derivatives in coordinates and further using:
\begin{align}
    \mathcal{L}_{\xi} G_A = \partial_m G_A \xi^m + C_{An}^{Bm} G_B \partial_m \xi ^n.
\end{align}
The only feature that all possible vertical coefficients $C_{An}^{Bm}$ have in common, irrespective of the specific field bundle at hand is that they will always be given in a way that ensures that $\mathcal{f}$ defines a Lie algebra morphism, i.e,: 
\begin{align}
\mathcal{f}\bigl ( \bigl [\xi, \tilde{\xi} \bigr ]_{\Gamma(TM)}\bigr ) = \bigl [ \mathcal{f}\bigl (\xi\bigr ), \mathcal{f}\bigl (\tilde{\xi}\bigr ) \bigr]_{\Gamma(TF)}
\end{align}
Similarly, we can now compute the Lie algebra action that corresponds to the prolonged actions on $J^1F$ and $J^2F$. By doing this, we obtain \textbf{\textit{prolonged}} vector fields that describe the infinitesimal diffeomorphism action on the jet bundle. In coordinates $(x^m,v_A,v_{Ai})$ on $J^1F$ we obtain:
\begin{align}
    \begin{aligned}
    j^1(\mathcal{f}) : \Gamma(TM) &\longrightarrow \Gamma(TJ^1F)\\
    \xi & \longmapsto \xi_{J^1F}.
    \end{aligned}
\end{align}
Where we can compute the thus prolonged vector field $j^1(\xi)$ to be given by 
\begin{multline}\label{LieJ1}
    \xi_{J^1F} = \xi^m \frac{\partial}{\partial x^m} + C_{An}^{Bm} v_B \partial_m \xi ^n \frac{\partial}{\partial v_A}\\
    + C_{An}^{Bm} \partial_m \xi^n v_{Bi} \frac{\partial}{\partial v_{Ai}} - v_{An} \partial_m \xi ^n \frac{\partial}{\partial v_{Am}} + C_{An}^{Bm} v_B \partial_m \partial_p \xi^n \frac{\partial}{\partial v_{Ap}}.
\end{multline}
In precisely the same way one can now construct prolongations of vector field to higher-order jet bundles as for instance to the second-order jet bundle as it is relevant for our treatment of Lagrangian field theory:
\begin{align}
    \begin{aligned}
    j^2(\mathcal{f}) : \Gamma(TM) &\longrightarrow \Gamma(TJ^2F)\\
    \xi & \longmapsto \xi_{J^2F}.
    \end{aligned}
\end{align}
Performing a similar computation as before in adapted coordinates $(x^m,v_A,v_{Ai}, v_{AI})$ on $J^2F$, we find:
\begin{align}\label{LieJ2}
\begin{aligned}
    \xi_{J^2F} = &\hphantom{-} \xi^m \frac{\partial}{\partial x^m} + C_{An}^{Bm} v_B \partial_m \xi ^n \frac{\partial}{\partial v_A}
    + C_{An}^{Bm} \partial_m \xi^n v_{Bi} \frac{\partial}{\partial v_{Ai}}\\
    &- v_{An} \partial_m \xi ^n \frac{\partial}{\partial v_{Am}} + C_{An}^{Bm} v_B \partial_m \partial_p \xi^n \frac{\partial}{\partial v_{Ap}} 
    + C_{An}^{Bm} v_{BI} \partial_m \xi ^n \frac{\partial}{\partial v_{AI}}\\
    &- 2 v_{BJ} I^J_{an}J^{am}_I \partial_m \xi^n \frac{\partial}{\partial v_{AI}} + 2 C_{An}^{Bm} v_{Ba}J^{ap}_I \partial_m \partial_p \xi^n \frac{\partial}{\partial v_{AI}}\\
    &- v_{An} J^{pm}_I \partial_m \partial_p \xi^n\frac{\partial}{\partial v_{AI}} + C_{An}^{Bm} v_B J^{pq}_I \partial_m \partial_p \partial_q \xi^n \frac{\partial}{\partial v_{AI}}.
\end{aligned}
\end{align}
These two expressions describe the infinitesimal action of a diffeomorphism that is induced by a vector field $\xi$ on the first and second jet bundle over the field bundle, respectively. 
All the involved maps, $j^1(\mathcal{f})$ and $j^2(\mathcal{f})$ or in fact any prolongation to any higher jet bundle $j^q(\mathcal{f})$ define Lie algebra morphisms, and hence one, in particular, finds that they preserve the commutator:
\begin{align}
j^q (\mathcal{f})\bigl (  \bigl [\xi, \tilde{\xi}  \bigr ]_{\Gamma(TM)}\bigr) =  \bigl [ j^q(\mathcal{f})(\xi), j^q(\mathcal{f})(\tilde{\xi}) \bigr ]_{\Gamma(TJ^qF)}
\end{align}
Using the prolongation of vector fields to the jet bundle we can now state one of the main results regarding the implications of the diffeomorphism equivariance of bundle maps on the jet bundle. We present a first-order PDE system that
such an equivariant bundle map necessarily has to solve. 
This, of course, also applies to the special case of the bundle map in consideration being
the Lagrangian of a diffeomorphism invariant field theory. 
\begin{theorem}[equivariance equations]
Let $(F,\pi_f,M)$ be the field bundle and $J^2F$ denote the second-order jet bundle over $F$. Furthermore let $(E, \pi_E, M)$ be a second bundle over $M$ with adapted coordinates $(x^m, u_{\tilde{A}})$ that carries an action of $\mathrm{Diff}(M)$ which is infinitesimally described by a Lie algebra morphism $\mathcal{e}: \Gamma(TM) \rightarrow \Gamma(TE)$ with coordinate expression $\mathcal{e}(\xi) = \xi^m \frac{\partial}{\partial x^m} + K_{\tilde{A}n}^{\tilde{B}m} u_{\tilde{B}} \partial_m \xi^n \frac{\partial}{\partial u_{\tilde{A}}}$. Assume that we are given a bundle morphism $f : J^2F \rightarrow E$ covering $id_M$ that is equivariant w.r.t. the lifted action of $\mathrm{Diff}(M)$ on $J^2F$ and the action $\mathcal{e}$ on $E$, then in coordinates $f^{\tilde{A}}$ satisfies the following set of linear first-order partial differential equations:
\begin{align}
\begin{aligned}
    0 &= f^{\tilde{C}:m} \\
    0 &= f^{\tilde{C}:A} C_{An}^{Bm} v_B + f^{\tilde{C}:Ap} \bigl[ C_{An}^{Bm} \delta_p^q - \delta_A^B \delta_m^n \bigr] v_{Bq} + f^{\tilde{C}:AI} \bigl[ C_{An}^{Bm} \delta_I^J - 2 \delta_A^B J_I^{pm} I^J_{pn}  \bigr] v_{BJ} - f^{\tilde{A}}K_{\tilde{A}n}^{\tilde{C}m}\\
    0 &= f^{\tilde{C}:A(p\vert}C_{An}^{B \vert m)} v_B + f^{\tilde{C}: AI} \bigl[ C_{An}^{B(m\vert} 2 J_I^{\vert p) q} - \delta^B_A J_I ^{pm} \delta_n^q \bigr] v_{Bq} \\
    0 &= f^{\tilde{C}:AI} C_{An}^{B(m\vert} v_B J_I^{\vert p q )}.
\end{aligned}
\end{align}
\end{theorem}
Here we introduced the notation $h^{:A} := \partial^A h$. The four equations describe a system of first-order partial differential equations (PDEs) for the coordinate representation of $f$. 
\begin{proof}
The statement follows immediately from the definition of equivariance. Evaluating this definition infinitesimally, i.e., on the level of Lie algebra actions for some vector field $\xi$ we obtain an equality with left-hand side being given by the application of the prolonged vector field $\xi_{J^2F}$ and right-hand side given as: $f^{\tilde{A}}K_{\tilde{A}n}^{\tilde{C}m} \partial_m \xi^n$. As this equation has to hold for arbitrary vector fields, $\xi$, we can, in particular, use $\xi$ to isolate contributions from specific derivatives. 
For instance, by taking the components of the vector field as $\xi^a = \delta^a_2$, we only get contributions proportional to $\xi^2$. Thus we can conclude that these contributions have to vanish separately. Choosing further $\xi^a = \delta^a_2 x^3$ only terms that either contain $\xi^2$ or $\partial_3 \xi^2$ are non-zero. As priorly we already have deduced that those terms containing $\xi^2$ have to vanish separately we can now follow that also those that are proportional to $\partial_3 \xi ^2$ must vanish independently of the remaining contributions.
Hence we can equate the coefficients of different derivatives of $\xi$ to zero independently, which then yields the desired PDE system and therefore proves the statement.  
\end{proof}
The PDE system deduced from the equivariance of $f$ can, obviously, be projected to a PDE system encoding equivariance for functions on $J^1F$. This is done by discharging all terms that contain derivatives w.r.t. second-order jet coordinates $v_{AI}$. 
Similar PDEs can also be obtained for equivariant functions on higher-order jet bundles.
Note that from the last theorem we can conclude in particular that the Lagrangian of diffeomorphisms invariant field theories, in local coordinates given by $\mathcal{L} = L \mathrm{d}^4x$ on $J^2F$ satisfies the following linear first-order PDE system:  
\begin{align}\label{DiffeoEqn}
\begin{aligned}
    0 &= L^{:m} \\
    0 &= L^{:A} C_{An}^{Bm} v_B + L^{:Ap} \bigl[ C_{An}^{Bm} \delta_p^q - \delta_A^B \delta_m^n \bigr] v_{Bq} + L^{:AI} \bigl[ C_{An}^{Bm} \delta_I^J - 2 \delta_A^B J_I^{pm} I^J_{pn}  \bigr] v_{BJ} + L \delta^m_n \\
    0 &= L^{:A(p\vert}C_{An}^{B \vert m)} v_B + L^{: AI} \bigl[ C_{An}^{B(m\vert} 2 J_I^{\vert p) q} - \delta^B_A J_I ^{pm} \delta_n^q \bigr] v_{Bq} \\
    0 &= L^{:AI} C_{An}^{B(m\vert} v_B J_I^{\vert p q )}.
    \end{aligned}
\end{align}
The only ingredient that explicitly depends on the particular nature of the field is the vertical coefficient tensor $C^{Bm}_{An}$.
The remaining objects in (\ref{DiffeoEqn}) take precisely this form no matter what concrete field theory we wish to consider.
Considering field theories on $J^1F$ this was already presented in \cite{Gotay1992StressEnergyMomentumTA}, where the authors used a similar PDE system for the Lagrangian to construct a universal energy-momentum tensor that is conserved for any diffeomorphism invariant field theory.

In the context of Constructive Gravity, it is worth mentioning that as we required diffeomorphism invariance of the gravitational theory, any Lagrangian that might describe it necessarily needs to solve the PDE system (\ref{DiffeoEqn}). To put it differently, by finding the general solution to the Lagrangian equivariance PDE for a given gravitational field bundle we would obtain the broadest possible set of candidates for the Lagrangian of this theory of gravity. Without further requirements of the gravitational theory finding Lagrangians that describe the diffeomorphism invariant gravitational dynamics of a given gravitational field, therefore, boils down to solving the above PDE system (\ref{DiffeoEqn}). 

From the diffeomorphism equivariance equations, we easily see that in order to simplify the task of solving the system, we might proceed as follows:
We first solve the corresponding \textit{\textbf{invariance equation}}, the system (\ref{DiffeoEqn}) with the second equation being replaced by the corresponding homogeneous equation:
\begin{align}
    0 &= L_{scal}^{:A} C_{An}^{Bm} v_B + L_{scal}^{:Ap} \bigl[ C_{An}^{Bm} \delta_p^q - \delta_A^B \delta_m^n \bigr] v_{Bq} + L_{scal}^{:AI} \bigl[ C_{An}^{Bm} \delta_I^J - 2 \delta_A^B J_I^{pm} I^J_{pn}  \bigr] v_{BJ}.
\end{align}
A solution to this system describes a map $L_{scal}: J^2F \rightarrow \mathbb{R}$ that is invariant under diffeomorphisms, i.e., for all $\phi \in \mathrm{Diff}(M)$ it holds that:
\begin{align}
    L_{scal} \circ j^2(\phi) = L_{scal}.
\end{align}
Once we obtain the general solution to these diffeomorphism invariance equations, $L_{scal}$ multiplying by any particular solution of (\ref{DiffeoEqn}) yields a solution of the diffeomorphism equivariance equations. We might, for instance, take this particular solution to only depend on the coordinates of $F$ and not on the derivative coordinates. Such solutions can typically be found more easily as then all terms in (\ref{DiffeoEqn}) that contain derivatives w.r.t. the derivative coordinates on $J^2F$ drop out. Conversely given any two solutions to the equivariance equations taking their quotient yields a solution to the invariance equations. These arguments show that there really is a one to one correspondence between solutions of the equivariance and invariance equations. Whether or not solving the invariance equation gives advantages over solving the PDE (\ref{DiffeoEqn}) can only be decided with concrete examples at hand. Details regarding these considerations can be found in Theorem \ref{GeneralSol} and the corresponding proof. 

Note that the first equation in the PDE system (\ref{DiffeoEqn}) simply requires the Lagrangian to be independent of the coordinates of the spacetime manifold. This is, of course, a requirement posed on the Lagrangian as a bundle map on $J^2F$, not on the corresponding composition with a given section of $J^2F$ as all such compositions are necessarily $x^m$-dependent. The first equation can hence be trivially solved, and we might use its implications in the following. For instance, we can now write $L(v_A.v_{Ai},v_{AI})$ for the coordinate expression of the Lagrangian, discharging the explicit $x^m$-dependency.

Before proceeding with the next section, we quickly deduce the implications for the EOM that are generated by diffeomorphism invariant Lagrangians. The EOM of a given Lagrangian $\mathcal{L}$ can be obtained by taking its variational derivative, in coordinates $E^A := \frac{\delta \mathcal{L}}{\delta v_A}$, composing it with the prolongation of a section of the field bundle and then equating to zero. In the following, we will also refer to $E^A$ as equations of motion. If the Lagrangian is given as a function on the second jet bundle, then the EOM are generally functions on the fourth jet bundle $J^4F$. As mentioned before we are however particularly interested in the case where the Lagrangian is degenerate in the sense that also the EOM are functions on $J^2F$, and hence a meaningful Hamiltonian formulation exists.
\begin{theorem}[diffeomorphism equivariant EOM]
Given a Lagrangian on $J^2F$ that describes a diffeomorphism invariant field theory with second-order EOM, in the usual coordinates: $E^A := \frac{\delta \mathcal{L}}{\delta v_A}$, then the equations of motion satisfy the following linear first-order PDE system:
\begin{align}\label{EOM}
    \begin{aligned}
    0 &= E^{C:m} \\
    0 &= E^{C:A} C_{An}^{Bm} v_B + E^{C:Ap} \bigl[ C_{An}^{Bm} \delta_p^q - \delta_A^B \delta_m^n \bigr] v_{Bq} + E^{C:AI} \bigl[ C_{An}^{Bm} \delta_I^J - 2 \delta_A^B J_I^{pm} I^J_{pn}  \bigr] v_{BJ}\\
    &+ E^C \delta^m_n + E^A C_{An}^{Cm}  \\
    0 &= E^{C:A(p\vert}C_{An}^{B \vert m)} v_B + E^{C: AI} \bigl[ C_{An}^{B(m\vert} 2 J_I^{\vert p) q} - \delta^B_A J_I ^{pm} \delta_n^q \bigr] v_{Bq} \\
    0 &= E^{C:AI} C_{An}^{B(m\vert} v_B J_I^{\vert p q )}.
    \end{aligned}
\end{align}
\end{theorem}
\begin{proof}
Again we start with Lie algebra version of the equivariance condition for $\mathcal{L}$ and an arbitrary vector field $\xi$. This time the statement can be proven by first computing the variational derivative, i.e., acting with $\frac{\delta}{\delta v_A}$ on the whole equation and then equating the components in front of different derivatives of $\xi$ individually to zero. 
\end{proof}

Similar equations were obtained in the context of variational calculus and gauge symmetries of a given Lagrangian in \cite{article}. In the context of GR, although not in the form of a PDE system, Lovelock used comparable implications deduced from the required diffeomorphism invariance to prove that the Einstein-Hilbert-Lagrangian is unique in the sense that it is the only Lagrangian producing second-derivative-order EOM that solves these implications \cite{Lovelock1969}, \cite{doi:10.1063/1.1666069} and \cite{doi:10.1063/1.1665613}. 

If we loosen our requirements of the gravitational theory to the point where we are only interested in the EOM and not the Lagrangian, we could also work with the PDE system (\ref{EOM}). By solving this system, we obviously directly obtain EOM. Such an approach is described in \cite{TobiR}.

Summing up we have derived linear first-order PDE systems for the Lagrangian or the equations of motion respectively that encode the required diffeomorphism invariance of a given field theory. Any theory that is invariant under spacetime diffeomorphisms has to solve the two PDEs. Conversely, we might construct diffeomorphism invariant theories as solutions to either the system (\ref{DiffeoEqn}) on the level of the Lagrangian or the system (\ref{EOM}) on the level of the EOM. This is a massive improvement as now the notoriously difficult requirement of diffeomorphism invariance that we have posed on the gravitational theory that we wish to construct is translated into the rather simple quest of solving a linear first-order PDE system.  



\section{Canonical Kinematics and Hypersurface Deformations}
In the following two sections we are going to investigate implications of the required diffeomorphism invariance for the \textit{\textbf{canonical formulation}} of a given field theory. For that purpose we mainly concern field theories that are described by a \textit{\textbf{first-order}} Lagrangian, i.e., a function on $J^1F$. Working on $J^2F$ as before we would necessarily have to consider the case where the Lagrangian is degenerate such that the EOM are nevertheless of second-order and the Hamiltonian formulation is free of Ostrogratsky instabilities. The degeneracy of the Lagrangian would then, however, impede the canonical formulation of the theory\footnote{We remark that, although posing further technical difficulties, we expect a generalization of the following derivations to the case of degenerate Lagrangians on $J^2F$ to be straight forward.}.

The main idea that is used when transferring from a given Lagrangian field theory to the associated Hamiltonian formulation is that of decomposing the spacetime manifold $M$ into a disjoint set of 3-dimensional manifolds that are smoothly parametrized by a time parameter. As a consequence, one can then accordingly decompose any structure that is prescribed on $M$. In order to achieve this we follow \cite{2004math.ph..11032G} and introduce the notion of a \textit{\textbf{slicing}}, sometimes also called a \textit{\textbf{3+1-split}}, of the spacetime manifold $M$.
\begin{definition}[slicing]
A slicing of the spacetime manifold $M$ is a diffeomorphism: 
\begin{align}
\phi : \Sigma \times \mathbb{R} \longrightarrow M.
\end{align}
\end{definition}
By this definition for each $\lambda \in \mathbb{R}$ we obtain an embedding $\phi_{\lambda}$ of the 3-dimensional manifold $\Sigma$:
\begin{align}
\begin{aligned}
\phi_{\lambda}: \Sigma &\longrightarrow M \\
s &\longmapsto \phi_{\lambda}(s) := \phi(s,\lambda).
\end{aligned}
\end{align}
We define the images of this one-parameter family of embeddings as $\Sigma_{\lambda} := \phi_{\lambda}(\Sigma)$. Hence we can think of the embedded 3-dimensional manifolds $\Sigma_{\lambda}$ as modelling space at given parameter time $\lambda$.
Note that it is, a priori, not clear whether or not such a slicing exists for a given spacetime $M$. For the situation in Constructive Gravity, or physics in general, there are, however, certain results that guarantee its existence. If we require that the spacetime manifold can additionally be endowed with a predictive matter field theory --- predictive in the sense that the corresponding matter field EOM are of hyperbolic type\footnote{We will provide a rigorous version of this statement in the second chapter. } --- then one can show that there necessarily exists a slicing of spacetime with the embedded spaces $\Sigma_{\lambda}$ being initial data hypersurfaces for the hyperbolic EOM. This is undoubtedly a requirement that is necessary if we want to be able to make physical predictions. For further details see for instance \cite{2003CMaPh.243..461B} for the situation in GR, and also \cite{1996gere.conf...19G} for a more general context. 

\begin{figure}[hbt!]
\centering
\begin{tikzpicture}
\node (S) at (0,0) {$\Sigma \times \mathbb{R}$};
\node (M) at (4,0) {$M$};
\node (R) at (0,-4) {$\mathbb{R}$};
\draw [<-] (S) -- node[pos=0.5, above] {$\phi^{-1}$} (M);
\draw [-latex] (S) -- node[pos=0.5, left] {$\pi_{\mathbb{R}}$} (R);
\draw [-latex] (M) -- node[pos=0.5, right] {$t$} (R);
\end{tikzpicture} 
\caption{Commutative Diagram: Slicing induced Time Function.}\label{DiagrTime}
\end{figure}

In addition to the smoothly parameterized family of embeddings $\phi_{\lambda}$ a slicing on $M$ immediately induces a \textit{\textbf{time vector field}} and a \textit{\textbf{time differential}} one form on $M$. The one form can be obtained as differential $\mathrm{d}t$ of the function $t:=\pi_{\mathbb{R}} \circ \phi^{-1}$, where $\pi_{\mathbb{R}}$ denotes the canonical project of $\Sigma \times \mathbb{R}$ onto the second factor. The situation is displayed in Figure \ref{DiagrTime}.
The time vector field can be constructed by first defining for all $s \in \Sigma$ the following map: 
\begin{align}
\begin{aligned}
    \phi_s : \mathbb{R} &\longrightarrow M \\
    \lambda &\longmapsto \phi(s,\lambda).
\end{aligned}
\end{align}
These maps define curves on $M$. They track the path of specific spatial point $s \in \Sigma$ through spacetime, defined by the family of embeddings. The situation can be seen in Figure \ref{Slicing}.

\begin{figure}[hbt!]
\centering
\begin{tikzpicture}
\draw[thick, dashed] (0,0) .. controls (2,-1) and (4,0.5)  .. (6,0);
\draw[thick, dashed] (0,0) 
.. controls (0.5,0.8) and (0,2) ..
(0,2) .. controls (-0.3,2.6) and (0,4) ..
(0,4) .. controls (0.2,5) and (-0.2,5.5) .. (0,6);
\draw[thick, dashed] (6,0)  
.. controls (5.5,1.4) and (6,2) ..
(6,2) .. controls (6.2,3) and (6,4) ..
(6,4) .. controls (5.8,4.5) and (5.7,5.5) .. (6,6);
\draw[thick, dashed] (0,6) .. controls (2,5.5) and (4,7.5) .. (6,6);
\draw[thick, dashed] (0,0) .. controls (1,0.5) and (1.5,2) .. (2,2);
\draw[thick, dashed] (6,0)  .. controls (6.5,1) and (7.5,1.5) .. (8,2);
\draw[thick, dashed] (6,6) .. controls (6.5,6) and (7,8) ..  (8,8);
\draw[thick, dashed] (0,6) .. controls (1,5.5) and (1.5,8) .. (2,8);
\draw[thick, dashed] (2,2) .. controls (3,1) and (5,3) .. (8,2);
\draw[thick, dashed] (2,2) 
.. controls (2.5,2.8) and (2,4) ..
(2,4) .. controls (1.7,4.6) and (2,6) ..
(2,6) .. controls (2.2,7) and (1.8,7.5) .. (2,8);
\draw[thick, dashed] (8,2)   
.. controls (7.7,3.5) and (8,4) ..
(8,4) .. controls (8.2,4.8) and (8,6) ..
(8,6) .. controls (7.8,6.7) and (7.9,7.5) ..  (8,8);
\draw[thick, dashed] (2,8)  .. controls (4,7) and (7,9) .. (8,8);

\draw[thick, draw = black, fill = gray, fill opacity = 0.4] (0,2) .. controls (2,1) and (4,2.3) ..
(6,2)  .. controls (6.5,3) and (7.5,3.5)..
(8,4) .. controls (5,5) and (3,3) ..
(2,4) .. controls (1.5,3.8) and (1,2.5)  .. (0,2); 

\draw[thick, draw = black, fill = gray, fill opacity = 0.4] (0,4) .. controls (2,3.5) and (4,5) ..
(6,4)  .. controls (6.5,4) and (7,6)..
(8,6) .. controls (5,7) and (3,5) ..
(2,6) .. controls (1.5,6) and (1.5,4)  .. (0,4); 

\node (M) at (0.5,0.2) {$\boldsymbol{M}$};
\node (S1) at (1,2) {$\boldsymbol{\Sigma_{\lambda_1}}$};
\node (S2) at (1.5,4.3) {$\boldsymbol{\Sigma_{\lambda_2}}$};

\draw[fill] (5,3.2) circle [radius=0.05];
\node (P1) at (4.25,3)
{$\phi_s(\lambda_1)$};
\draw[fill] (5.2,5.5) circle [radius=0.05];
\node (P1) at (4.5,5.2)
{$\phi_s(\lambda_2)$};

\draw[thick, dotted] (5,-1) to [out = 90, in = 260] (5,3.2);
\draw[thick, dotted] (5,3.2) to [out = 80, in = 270] (5.2,5.5);
\draw[thick, dotted] (5.2,5.5) to [out = 90, in = 280] (4,9);
\node at (5.5,-0.5)
{$\phi_s$};
\draw[-latex] (5.2,5.5) to (5,7.5); 
\node at (5.75,7.25) {$\frac{\partial}{\partial t}\big \vert _{\phi_s(\lambda_2)}$};
\end{tikzpicture} 
\caption{Slicing of Spacetime into 3-dimensional Slices.}\label{Slicing}
\end{figure}
Therefore each such curve provides us with tangential vectors along it. Using this we can define a vector field on $M$ by defining the value at $p \in M$ to be given by the tangential vector of the specific such curve that is passing through $p$:
\begin{align}
\frac{\partial}{\partial t}\bigg \vert_p = \phi^{\prime}_{\pi_{\Sigma}\circ \phi^{-1}(p)} \left (\pi_{\mathbb{R}}\circ \phi^{-1}(p)\right ).
\end{align}
One readily finds that $\frac{\partial}{\partial t}$ and $\mathrm{d}t$ are dual objects, i.e., $\mathrm{d}t(\frac{\partial}{\partial t}) = 1$ and hence the notation. In the following, we call $t$ time function, $\mathrm{d}t$ time differential and $\frac{\partial}{\partial t}$ time vector field.


Given the fact that $\mathrm{d}t$ and $\frac{\partial}{\partial t}$ were obtained from the given slicing this immediately raises the question of how the time differential, time vector field, and all further objects that will subsequently be defined in terms of these, change under a change of slicing. If $\phi : \Sigma \times \mathbb{R} \rightarrow M $ and $\psi : \Sigma \times \mathbb{R} \rightarrow M$ are two slicings then we obtain a diffeomorphism on $M$ by taking $\psi \circ \phi^{-1}$. Conversely given a slicing $\phi$ any diffeomorphism $\rho \in \mathrm{Diff}(M)$ defines a second slicing by $\rho \circ \phi $. Hence changing the slicing is in one to one correspondence with diffeomorphisms on $M$. 

In addition to the time differential and time vector field on $M$ a slicing also induces a \textbf{\textit{direct sum split}} of the tangent bundle over $M$. Given a slicing $\phi : \Sigma \times \mathbb{R} \rightarrow M$ we can compute its pushforward to obtain a vector bundle isomorphism:
\begin{align}
\phi_{\ast}: T(\Sigma \times \mathbb{R}) \longrightarrow TM.
\end{align}
Any tangent space over a product of manifolds can be decomposed into a direct sum of the tangent spaces of the individual factors. More precisely one can show that $T(\Sigma \times \mathbb{R}) \cong \pi_{\Sigma}^{\ast}T\Sigma \oplus \pi_{\mathbb{R}}^{\ast} T\mathbb{R}$. And hence in total we have: $TM \cong \pi_{\Sigma}^{\ast}T\Sigma \oplus \pi_{\mathbb{R}}^{\ast} T\mathbb{R}$.
Here $\pi_{\Sigma}^{\ast}T\Sigma$ and $\pi_{\mathbb{R}}^{\ast}T\mathbb{R}$ denote the pullbacks\footnote{Given a bundle $(F,\pi_F,M)$ and a map $h: N \rightarrow M$, the pullback of the bundle $F$ along $h$ is the bundle over $N$ with total space $h^{\ast}F := \{ (n,f) \in N \times F \ \vert \  h(n) = \pi_F(f)\}$, and bundle projection $\pi_{h^{\ast}F}(n,f) = n$ (see \cite{doi:10.1142/3867}). Note that this construction makes the map $\tilde{h}: f^{\ast}F \rightarrow F$ defined by $\tilde{h}(n,f) = f$ a bundle morphism covering $h$, i.e the following diagram commutes. 
\begin{center}
\begin{tikzpicture}
\node (M) at (0,0) {$N$};
\node (N) at (4,0) {$M$};
\node (M2) at (0,2) {$f^{\ast}F$};
\node (N2) at (4,2) {$F$};
\draw [-latex] (M) -- node[pos=0.5, below] {$h$} (N);
\draw [-latex] (M2) -- node[pos=0.5, above] {$\tilde{h}$} (N2);
\draw [<-] (M) -- node[pos=0.5, left] {$\pi_{h^{\ast}F}$} (M2);
\draw [<-] (N) -- node[pos=0.5, right] {$\pi_F$} (N2);
\end{tikzpicture}
\end{center}} of the bundles $T\Sigma$ and $T\mathbb{R}$ over $\Sigma$ and $\mathbb{R}$ to the common base space $\Sigma \times \mathbb{R}$ by making use of the two canonical projections.
Moreover, given two vector bundles $(F,\pi_F,M)$ and $(E,\pi_E,M)$ over the same base space $M$, $F\oplus E$ denotes their direct sum or Whitney sum (for details see the first chapter of \cite{nla.cat-vn705150}). The Whitney sum of two vector bundles is the new vector bundle with fiber at $p \in M$ given by the direct sum of vector spaces $\pi_F^{-1}(p) \oplus \pi_E^{-1}(p)$. 
Note that the Whitney sum can only be constructed for two vector bundles with common base space.
Using the fact that all isomorphisms in the above construction are isomorphisms of vector bundles and thus preserve the Whitney sum decomposition we find that also the spacetime tangent bundle then necessarily decomposes into a Whitney sum. In the following we, write:
\begin{align}
    TM = \mathcal{T}\Sigma \oplus \mathcal{V}\Sigma
\end{align} 
for this direct sum decomposition of the tangent bundle and call the summands \textit{\textbf{tangential}} and \textit{\textbf{vertical subbundle}} respectively.  
It is important to note that up to technicalities such as the isomorphism $T(\Sigma \times \mathbb{R}) \cong \pi_{\Sigma}^{\ast}T\Sigma \oplus \pi_{\mathbb{R}}^{\ast} T\mathbb{R}$ and also pulling back the bundles along the canonical projections, the tangential and vertical subbundle of $TM$ are essentially given by the images of $T\Sigma$ and $T\mathbb{R}$ under the pushforward of the slicing $\phi_{\ast}$. 

We can now use this direct sum decomposition of the tangent bundle $TM$ to further decompose structure that is prescribed on it.
In particular, vector fields on $M$ can uniquely be \textit{\textbf{decomposed}} w.r.t. this split. Given a vector field $\xi \in \Gamma(TM)$ we write $\xi = \xi_{\parallel} + \xi_{\perp} $ for this decomposition, with $\xi_{\parallel} \in \Gamma(\mathcal{T}\Sigma)$ and $\xi_{\perp} \in \Gamma(\mathcal{V}\Sigma)$ called tangential and vertical part of $\xi$. The situation is drawn in Figue \ref{DecompPic}.

\begin{figure}[hbt!]
\centering
\begin{tikzpicture}
\path[fill = gray, opacity = 0.4] (0,0) -- (8,0) -- (10,4) -- (2,4);
\draw[-latex, dashed] (0.8,0.4) to (2.3,3.4);
\node at (1,3) {$\boldsymbol{\mathcal{V}_pM}$};
\node at (9,3.5) {$\boldsymbol{T_pM}$};
\draw[-latex, dashed] (0.8,0.4) to (7.8,0.4);
\node at (4,-0.5) {$\boldsymbol{\mathcal{T}_pM}$};
\draw[-latex] (0.8,0.4) --  node[pos=0.5, above] {$\boldsymbol{\xi_p}$} (5,2) ;
\draw[-latex] (0.8,0.4) --  node[pos=0.5, left] {$\boldsymbol{(\xi_{\perp})_p}$} (1.6,2) ;
\draw[dashed] (1.6,2) to (5,2);
\draw[-latex] (0.8,0.4) -- node[pos=0.5, below] {$\boldsymbol{(\xi_{\parallel})_p}$} (4.2,0.4);
\draw[dashed] (4.2,0.4) -- (5,2);
\end{tikzpicture}
\caption{Direct Sum Decompostion of Vectors.}\label{DecompPic}
\end{figure}

The vectors in $\mathcal{T}\Sigma$ satisfy: $\mathrm{d}t(\xi_{\parallel})=0$, the vectors in $\mathcal{V}\Sigma$ are parallel to $\frac{\partial}{\partial t}$  . Note that the direct sum split of $TM$ naturally induces a direct some split of the Lie algebra actions on $F$ and $J^1F$ (and also $J^2F$ if we worked on the second-order jet bundle) as the previously constructed Lie algebra morphisms (\ref{LieF}) and (\ref{LieJ1}) are in particular linear maps on the fibers of $TM$ and hence respect the direct sum structure.

We call a coordinate system $(x^m)$ on $M$ \textit{\textbf{adapted}} to the slicing $\phi$ if there exists an adapted coordinate system $(y^{\alpha},z)$ on $\Sigma \times \mathbb{R}$ s.t. $x^0 = z \circ \phi^{-1}$ and $x^{\alpha} = y^{\alpha} \circ \phi^{-1}$. Therefore in adapted coordinates the embedded hypersurfaces are characterized by $x^0 = const$ and the chart induced vector fields satisfy $\frac{\partial}{\partial t} = \lambda \cdot \frac{\partial}{\partial x^0}$ for some $\lambda$ that only depends on $x^0$. We further have $\mathrm{d}t\left(\frac{\partial}{\partial x^{\alpha}}\right) = 0$.
Note that in adapted coordinates we thus find in particular that the induced spatial coordinate fields $\frac{\partial}{\partial x^{\alpha}}$ are sections of the tangential subbundle $\mathcal{T}\Sigma$ whereas $\frac{\partial}{\partial x^0}$ defines a section of the vertical subbundle $\mathcal{V}\Sigma$. Hence in such coordinates, we can express the decomposition of vector fields as: 
\begin{align}
    \xi = \xi_{\perp} + \xi_{\parallel} = N  \frac{\partial }{\partial t} + N^{\alpha} \frac{\partial}{\partial x^{\alpha}} = N \cdot \lambda \frac{\partial}{\partial x^0} + N^{\alpha} \frac{\partial}{\partial x^{\alpha}},
\end{align}

In the following, we work mainly over one embedded hypersurface. Given a slicing $\phi$ take some $t_0 \in \mathbb{R}$ and consider the corresponding embedded hypersurface $\Sigma_{t_0}$ with embedding $\phi_{t_0}$. To this end we restrict all functions, fields, etc. on $M$ to $\Sigma_{t_0}$. These restrictions are in the following called \textbf{\textit{hypersurface quantities}}. To keep the notation concise, we might not explicitly denote the restriction every time. 
Furthermore, we find that the restriction of the direct sum split of the tangent bundle to $\Sigma_{t_0}$\footnote{It is important to observe the difference between the restricted tangent bundle $TM\vert_{\Sigma_{t_0}}$ and the tangent bundle of the embedded hypersurface $T\Sigma_{t_0}$. While the former contains a 4-dimensional vector space at each hypersurface point the latter only has 3-dimensional fibers. The difference lies precisely in the vertical bundle over the hypersurface $\mathcal{V}\Sigma_{t_0}$.} is given by $TM \vert _{\Sigma_{t_0}} = T\Sigma _{t_0} \oplus \mathcal{V}\Sigma _{t_0}$. Note that the restriction of the tangential bundle $\mathcal{T}\Sigma_{t_0}$ is precisely given by the tangent bundle over the embedded hypersurface $T\Sigma_{t_0}$.

We are then in particular interested in the question of how hypersurface quantities change under a change of slicing $\phi$ and hence an induced change of the embedding $\phi_{t_0}$. By previous arguments changing the slicing corresponds to the action of a diffeomorphism. Infinitesimally this is described by the Lie algebra action of a vector field $\xi$. Note that whether we can take the standard action of diffeomorphism and vector fields on $M$ or have to take the previously computed lifted actions (\ref{LieF}) on $F$, (\ref{LieJ1}) on $J^1F$, etc. depends on the kind of the objects that we act upon.  Given $\xi$, we decompose it into its tangential and vertical part $\xi_{\parallel}$ and $\xi_{\perp}$ to obtain derivative operators that describe the infinitesimal tangential and vertical change of hypersurface quantities.  In adapted coordinates, we obtain such tangential and vertical vector fields by specifying component functions
 $N^a : M \rightarrow \mathbb{R}$ and $N^a : M \rightarrow \mathbb{R}$. We define: 
\begin{align}
    \begin{aligned}
    \mathcal{D}(N^{\alpha}) &:= N^{\alpha} \frac{\partial}{\partial x^{\alpha}} \\
    \mathcal{H}(N) &:= N \frac{\partial}{\partial t} = N \cdot \lambda \frac{\partial}{\partial x^0}.
    \end{aligned}
\end{align} 
Note that we can interpret $\mathcal{D}(N^\alpha)$ and $\mathcal{H}(N)$ as \textit{\textbf{tangential}} and \textit{\textbf{vertical deformation operators}} that when acting on functions compute their infinitesimal change under a tangential respectively vertical deformation by an amount specified by the component functions $N^{\alpha}$ and $N$.
We are now going to compute the commutator algebra that such tangential and vertical vector fields satisfy. Doing this we obtain:
\begin{align}
    \begin{aligned}
    \left [ \mathcal{D}(N^{\alpha}), \mathcal{D}(M^{\alpha}) \right] &= \mathcal{D}(N^\alpha \partial_{\alpha}M^{\beta} - M^{\alpha} \partial_{\alpha} N^{\beta}) \\
    \left[ \mathcal{D}(N^{\alpha}), \mathcal{H}(M) \right] &= \mathcal{H}(N^{\alpha} \partial_{\alpha} M) - \mathcal{D}(M \lambda \partial_0 N^{\alpha})\\
    \left[ \mathcal{H}(N), \mathcal{H}(M) \right ] &= \mathcal{H}(N \lambda \partial_0 M - M \lambda \partial_0N)
    \end{aligned}
\end{align}
Moreover, we want to describe the change in tangential and vertical direction that is computed by the operators $\mathcal{D}(N^{\alpha})$ and $\mathcal{H}(N)$ intrinsically, in terms of quantities that only depend on the coordinates of $\Sigma_{t_0}$.
Hence we restrict the involved component functions $N^{\alpha}$ and $N$ and the deformation operators defined in terms of them such that they locally only depend on the coordinates $x^{\mu}$ and not on $x^0$.
\begin{comment}
The corresponding tangential and vertical vector fields are given by $\mathcal{D}(N^{\alpha})$ and $\mathcal{H}(N)$ can then be interpreted as tangential and vertical deformation operators defined on $\Sigma_{t_0}$. Acting on hypersurface quantities, they compute the infinitesimal change in tangential and vertical direction. Note that the action of vector fields on hypersurface quantities describes their infinitesimal change under diffeomorphisms. Hence by the arguments provided before the action of vector fields on hypersurface quantities describes their infinitesimal change under a change of slicing. The deformation operators $\mathcal{D}(N^{\alpha})$ and $\mathcal{H}(N)$ defined on $\Sigma_{t_0}$ encode contributions to this change in tangential and vertical direction.
\end{comment}
As a consequence contributions from $\partial_0$ derivatives of $N^{\alpha}$ and $N$ now drop out.
We thus obtain the following commutator algebra: 
\begin{align}\label{Alg}
    \begin{aligned}
    \left[ \mathcal{D}(N^{\alpha}), \mathcal{D}(M^{\alpha}) \right] &= \mathcal{D}(\mathcal{L}_{\vec{N}}M^{\beta}) \\
    \left[ \mathcal{D}(N^{\alpha}), \mathcal{H}(M) \right] &= \mathcal{H}(\mathcal{L}_{\vec{N}}M)\\
    \left[ \mathcal{H}(N), \mathcal{H}(M) \right] &= 0,
    \end{aligned}
\end{align}
where we denoted $\mathcal{L}_{\Vec{N}}M^{\beta} = N^\alpha \partial_{\alpha}M^{\beta} - M^{\alpha} \partial_{\alpha} N^{\beta}$ and $\mathcal{L}_{\Vec{N}}M = N^{\alpha} \partial_{\alpha} M$ as Lie derivative along the tangential hypersurface vector field $\vec{N} = N^{\alpha} \frac{\partial}{\partial x^{\alpha}}$ to display the commutator algebra in concise form.
We observe that the commutator of two tangential vector fields again yields a tangential vector field. The tangential vector fields hence constitute a Lie subalgebra. 
The commutator of a tangential and a vertical vector field yields a vertical vector field.
Finally, any two vertical vector fields commute. This commutator algebra encodes the change of hypersurface quantities on $\Sigma_{t_0}$ under a change of the slicing of spacetime. It is therefore often called the \textit{\textbf{hypersurface deformation algebra}}.

\begin{remark}
Note that in traditional terms the hypersurface deformation algebra is mostly obtained by computing commutators of vector fields on the infinite-dimensional manifold of embeddings of a given $\Sigma$ in $M$ (see \cite{HOJMAN197688} and \cite{doi:10.1063/1.522976}). We deliberately decided for an alternative approach here, as already in the formulation of Lagrangian field theory we saw how working with rigorously defined objects constructed over the spacetime manifold instead of a heuristic treatment involving infinite-dimensional function spaces led to a much clearer description that essentially involved the same information. In other words, if working on infinite-dimensional spaces can be avoided, we simply avoid it. We will re-encounter this guideline later on when we construct Poisson brackets to incorporate dynamics in terms of the Hamiltonian formulation of field theory.
\end{remark}

One often meets the situation where the direct sum split of the tangent bundle is not obtained from the time vector field $\frac{\partial}{\partial t}$ that is induced by the given slicing, but by means of some physically motivated notion that at each point of the embedded hypersurface $\Sigma_{t_0}$ distinguishes a vertical vector as "time" direction. For instance, this can be achieved by making use of a given \textit{\textbf{observer definition}} for the field theory at hand. For the embedded hypersurface $\Sigma_{t_0}$ one constructs the direct sum split by defining the vertical subbundle as the span of the "time" vector field seen as time direction by observers that travel through $p \in \Sigma_{t_0}$ and have a spatial frame lying in the tangent plane $T_p\Sigma_{t_0}$.

Taking the canonical formulation of General Relativity\footnote{For further information regarding some of the provided examples from General Relativity we kindly refer to the standard text books as for instance \cite{Misner1973}. For observer definitions in more general spacetime geometries see \cite{2011PhRvD..83d4047R}.} as an example, given $\Sigma_{t_0}$ and frame fields $e_{\alpha}$ that locally at each point $p \in \Sigma_{t_0}$ in consideration form a basis of $T_p\Sigma_{t_0}$ and satisfy $g(e_{\alpha},e_{\beta}) = \eta_{\alpha \beta}$, to complete the three frame fields to a viable observer frame\footnote{Recall that an observer in GR is defined as a curve $\gamma : \mathbb{R} \rightarrow M$ together with curves $e_{\alpha} : \mathbb{R} \rightarrow TM$, s.t. $ e_a(\lambda) :=(\dot{\gamma}(\lambda),e_{\alpha}(\lambda))$ forms a basis of $T_{\gamma(\lambda)}M$ and $g(e_a(\lambda),e_b(\lambda)) = \eta_{ab}$, , for each $\lambda$.} one needs to choose $e_0$ s.t.:
\begin{align}
    g(e_0,e_0) = 1 \ \text{and} \ g(e_0,e_{\alpha}) = 0.
\end{align}
These two conditions are then often referred to as frame conditions. One can then split the tangent space $T_pM$ at each $p \in \Sigma_{t_0}$ into the vertical part spanned by $e_0(p)$,  $\mathcal{V}_p\Sigma_{t_0} = \langle  e_0(p) \rangle$ and tangential part $T_p\Sigma_{t_0}$ . In particular, one sees that now by the two frame conditions, the vertical time direction and thereby the vertical subbundle and the split itself depend on the metric. Along the same lines given any other field theory with a notion for an observer, one can then use the observer definition to construct the vertical subbundle of $TM$ in a similar fashion. 

One way of defining observers from any given matter field theory that uses the given gravitational field as geometric background can be found in \cite{2018PhRvD..97h4036D} and \cite{2011PhRvD..83d4047R} and also \cite{Rivera}. There the authors construct an observer notion that suits a general gravitational field theory from the \textbf{\textit{Principal Polynomial}}\footnote{Loosely speaking the Principal Polynomial encodes the causal structure of the EOM. In the following chapter, we will encounter this object in more detail.} $\mathcal{P}_{mat}(v_A)$ of the given matter EOM. Note that the Principal Polynomial, in particular, is a function on the field bundle, i.e., is field dependent. The vertical vector field is then obtained similarly as in the standard GR case. The Principal Polynomial allows one to define a map that can be used to normalize the co-normal and map the result to a corresponding normal vector.

All these constructions have one thing in common: the vertical subbundle of the direct sum split of $TM$ is now field dependent. As a consequence, if one wants to deduce the hypersurface deformation algebra corresponding to a field-dependent direct sum split, it no longer suffices to consider the commutators of tangential and vertical vector fields over $\Sigma_{t_0}$. To account for the additional field dependence one now has to work with the induced vector fields on the field bundle\footnote{In the following we assume that the direct sum split only depends on the fields and not on their derivatives. Otherwise, we would need to work on the jet bundle $J^qF$ of an appropriate order.} $F$. To obtain the corresponding hypersurface deformation algebra one then computes commutators between the lifts of tangential and vertical vector fields according to the Lie algebra morphism (\ref{LieF}).

We again choose coordinates that are adapted to the given slicing. In contrast to the situation before, now, the vertical vector field $e_0$ that we use for the direct sum split additionally depends on the fiber coordinates of the field bundle $v_A$. The decomposition of vector fields then reads:
\begin{align}
    \xi = \xi_{\perp} + \xi_{\parallel} = N e_0 + N^{\alpha} \frac{\partial }{\partial x^{\alpha}} = N  e_0^a \frac{\partial}{\partial x^a} + N^{\alpha} \frac{\partial }{\partial x^{\alpha}}.
\end{align}
Here we expressed the field dependent vector field $e_0$ in terms of the holonomic basis fields and denoted the component functions by $e_0^a$.
The corresponding split of the induced vector field on the field bundle $\xi_F$ can now be computed to be given by:
\begin{align}
    \begin{aligned}
    \xi_{F,\parallel} &= \mathcal{f} \left ( \xi_{\perp} \right ) = N^{\alpha} \frac{\partial}{\partial x^{\alpha}} + C_{A \nu}^{B m} v_B \partial_{m} N^{\nu} \frac{\partial}{\partial v_A} \\
    \xi_{F, \perp} &= \mathcal{f} \left ( \xi_{\parallel} \right )=  N e_0^a \frac{\partial}{\partial x^a} + C_{A n}^{B m} v_B \partial_{m} (N \cdot e_0^n) \frac{\partial}{\partial v_A},
    \end{aligned}
\end{align}
where $\mathcal{f}$ is the Lie algebra morphism defined in (\ref{LieF}).
Again we define tangential and vertical deformation operators for arbitrary $N^a : M \rightarrow \mathbb{R}$ and $N : M \rightarrow \mathbb{R}$:
\begin{align}
    \begin{aligned}
    \mathcal{D}_F(N^{\alpha})&:= N^{\alpha} \frac{\partial}{\partial x^{\alpha}} + C_{A \beta}^{B m} v_B \partial_{m} N^{\beta} \frac{\partial}{\partial v_A} \\
    \mathcal{H}_F(N) &:=  N e_0^a \frac{\partial}{\partial x^a} + C_{A n}^{B m} v_B \partial_{m} (N \cdot e_0^n) \frac{\partial}{\partial v_A}
    \end{aligned}
\end{align}
and restrict to the case where the component functions $N^{\alpha}$, $N$ and $e^n$ do not depend on the coordinates $x^0$.

When computing commutators of $\mathcal{D}_F$ and $\mathcal{H}_F$ we can use the fact that the map $\mathcal{f}: \Gamma(TM) \rightarrow \Gamma(TF)$ that was defined in (\ref{LieF}) defines a Lie algebra morphism and hence preserves the commutator.
Comparing to the previous case (\ref{Alg}) we get however additional contributions whenever $\frac{\partial}{\partial v_A}$ acts on $e_0^n$. These contributions precisely account for the change of $e_0$ that is induced by a change of the fields in $F$ under the action of vector fields. The only commutator that does not involve such contributions is the one between two parallel vector fields. Therefore for this commutator, we simply get:
\begin{align}\label{FDD}
    \left [ \mathcal{D}_F(N^{\alpha}), \mathcal{D}_F(M^{\alpha})\right ] &= \mathcal{D}_F(\mathcal{L}_{\vec{N}}M^{\beta}).
\end{align}
The two remaining commutators, in general, contain additional contributions. As these additional contributions obviously depend on the specific form of the field dependency of $e_0$, we illustrate their appearance for the case of canonical GR. With slight modifications in the computations this will then of course also cover the case where for a given field theory the observer definition and thereby the vertical vector field $e_0$ is construction in analogy to the standard GR case, but with an effective metric that is constructed from the field instead of the fundamental metric field in GR, as it is, for instance, the case for the observer definition in \cite{2018PhRvD..97h4036D}, \cite{2011PhRvD..83d4047R}, and \cite{Rivera}.

Following the usual way of obtaining the vertical vector field, once the embedded hypersurface $\Sigma_{t_0}$ is specified, one starts by taking a co-normal vector field to $\Sigma_{t_0}$, i.e., a 1-form $\tilde{n} \in \Gamma(\Lambda^1M)$ that restricts to zero on each tangent plane of the hypersurface, $\tilde{n} \vert_{T\Sigma_{t_0}} = 0$. Then one uses the inverse metric to normalize the co-normal 1-form, in other words one defines a unit co-normal by $n := \frac{1}{\sqrt{ \vert g^{-1}(\tilde{n},\tilde{n}) \vert }} \cdot \tilde{n}$. The vertical vector field is then given by: 
\begin{align}
e_0 := g^{-1}(n, - ). 
\end{align}
Working in adapted coordinates, $\mathrm{d}x^0$ defines such a co-normal 1-form. Hence we get the vertical vector field by: 
\begin{align}
e_0 := \frac{1}{\sqrt{\vert g^{00} \vert }} g^{a0} \frac{\partial}{\partial x^a}.
\end{align}
In particular as one can readily check with this vertical vector field we have $g(e_0,e_0) = 1$ and $g(e_0,\frac{\partial}{\partial x^{\alpha}}) = 0$. If we chose any additional spatial observer vector fields $e_{\alpha}$ in the tangential subbundle of $\Sigma_{t_0}$, as the coordinate fields $\frac{\partial}{\partial x^{\alpha}}$ at each point of the embedded hypersurface constitute a basis of the tangential subspace, they are necessarily of the form $e_{\alpha} = e_{\alpha}^{\beta} \frac{\partial}{\partial x^{\beta}}$. Hence $e_0$ completes them to a valid GR observer frame.
The deformation operators can then be obtained as:
\begin{align}
    \begin{aligned}
    \mathcal{D}_F(N^{\alpha}) &= N^{\alpha} \frac{\partial}{\partial x^{\alpha}} + 2 g^{\mu (p\vert} \delta^{\vert q)}_{\nu} \partial_{\mu} N^{\nu} \frac{\partial}{\partial g ^{pq}} \\
    \mathcal{H}_F(N) &= N \frac{1}{\sqrt{\vert g^{00} \vert }} g^{a0} \frac{\partial}{\partial x^a} +2 g^{\mu (p\vert} \delta^{\vert q)}_{n} \partial_{\mu} \biggl (N \frac{1}{\sqrt{\vert g^{00} \vert }} g^{n0} \biggr )  \frac{\partial}{\partial g ^{pq}},
    \end{aligned}
\end{align}
where we inserted the expression for $C^{Am}_{Bn} \equiv C^{abm}_{cdn} = +2g^{m(a\vert}\delta^{\vert b)}_n$ for the case of the field being given by the inverse metric and further simplified the occurring expressions by employing the fact that $N^{\alpha},M$ and $e_0^a$ only depend on the hypersurface coordinates $x^\alpha$.

For this example, to put the focus on similarities to the standard formulation of canonical GR, we do not work in the intertwiner coordinates. We start with the commutator between a  tangential and a vertical vector field.
We get:
\begin{multline}\label{1stCom}
    \left[ \mathcal{D}_F(N^{\alpha}) , \mathcal{H}_F(M) \right] = \mathcal{H}_F(\mathcal{L}_{\vec{N}}M) - M \frac{1}{\sqrt{\vert g^{00} \vert }} g^{\mu0} \partial_{\mu} N^{\alpha} \frac{\partial}{\partial x^{\alpha}} \\
    +2M g^{\mu (p\vert} \delta^{\vert q)}_{\nu} \partial_{\mu} N^{\nu} \frac{\partial}{\partial g ^{pq}} \biggl(\frac{1}{\sqrt{\vert g^{00} \vert }} g^{a0} \biggr) \frac{\partial}{\partial x^a} + C^{pq} \frac{\partial}{\partial g^{pq}}.
\end{multline}
For the upcoming calculations, we want to restrict attention to the two additional terms that are proportional to $\frac{\partial}{\partial x^a}$ and $\frac{\partial}{\partial x^{\alpha}}$ respectively as their treatment nicely illustrates the spirit of the calculations without them getting too involved. In the equation above all additional extra terms that appear, i.e., those that are proportional to $\frac{\partial}{\partial g^{pq}}$ are collectively displayed by $C^{pq}$. We start by computing the derivative $\frac{\partial}{\partial g^{pq}}$ in the second extra term to obtain:
\begin{align}\label{metDer}
    \frac{\partial}{\partial g ^{pq}} \biggl (\frac{1}{\sqrt{\vert g^{00} \vert }} g^{a0} \biggr) = -\frac{1}{2} \frac{1}{\vert g^{00} \vert^{\frac{3}{2}}} g^{a0} \delta^0_p \delta^0_q +  \frac{1}{\sqrt{\vert g^{00} \vert }} \delta^a_{(p \vert} \delta^0 _{\vert q)}. 
\end{align}
Upon contraction with $g^{\mu ( p \vert} \delta ^{\vert q)}_{\nu}$ the first term of (\ref{metDer}) vanishes completely due to the appearance of $\delta^0_{\nu}$, the second term yields only one contribution which exactly cancels the first extra term in (\ref{1stCom}). Along similar lines, one then proceeds to show that the remaining extra contributions denoted by $C^{pq}$ vanish. In total, we find that:
\begin{align}\label{FDH}
    \bigl[ \mathcal{D}_F(N^{\alpha}) , \mathcal{H}_F(M) \bigr] = \mathcal{H}_F(\mathcal{L}_{\vec{N}}M).
\end{align}
We observe that the commutator between a tangential and a vertical vector fields is the same as in the previous case (\ref{Alg}).
Note that this commutation relation is not a general feature of direct sum decomposition with field-dependent vertical vector fields but depends on the precise observer definition. 

We proceed with the commutator of two vertical vector fields:
\begin{multline}
    \left[ \mathcal{H}_F(N), \mathcal{H}_F(M) \right] = \left( N\partial_{\mu} M - M \partial_{\mu}N  \right) \cdot \biggl[ \frac{1}{g^{00}} g^{\mu 0} g^{a0} \frac{\partial}{\partial x^a}  \\
    -2 g^{\mu (p\vert} \delta^{\vert q)}_{n} \frac{1}{\sqrt{\vert g^{00} \vert }} g^{n0}  \frac{\partial}{\partial g ^{pq}} \biggl( \frac{1}{\sqrt{\vert g^{00} \vert }} g^{a0} \biggr) \frac{\partial}{\partial x^a} \biggr] + \tilde{C}^{pq} \frac{\partial}{\partial g^{pq}}.
\end{multline}
Again we focus on the extra terms proportional to $\frac{\partial}{\partial x^a}$. Using the previously computed result for the derivatives w.r.t. the metric components one finds that the two terms of that kind collapse to:
\begin{align}
    \left[ \mathcal{H}_F(N), \mathcal{H}_F(M) \right] = \bigl( g^{\mu \alpha} - \frac{1}{g^{00}} g^{0\alpha} g^{0 \mu} \bigr) \left( N\partial_{\mu} M - M \partial_{\mu}N  \right) \frac{\partial}{\partial x^{\alpha}} + \tilde{C}^{pq} \frac{\partial}{\partial g^{pq}}.
\end{align}
Note that the expression appearing in the first bracket is exactly the inverse of the 3-metric on the embedded hypersurface that is defined in terms of the 4-metric by the component functions $g_3 = g_{\alpha \beta} \mathrm{d}x^{\alpha} \mathrm{d}x^{\beta}$ as it can be easily checked from:
\begin{align}
(g_3)_{\beta \mu } \cdot ( g^{\mu \alpha} - \frac{1}{g^{00}} g^{0\alpha} g^{0 \mu} ) = \delta_{\beta}^{\alpha}.
\end{align}
Proceeding along the same lines for the remaining contributions $\tilde{C}^{pq}$ we find that:
\begin{align}\label{FHH}
    \bigl[ \mathcal{H}_F(N), \mathcal{H}_F(M) \bigr] =  \mathcal{D}_F\bigl( (g_3)^{\mu \alpha}( N\partial_{\mu} M - M \partial_{\mu}N  ) \bigr).
\end{align}
Hence the commutator between two vertical vector fields yields a tangential vector field with components depending on the metric.
In particular, we observe that this last commutation relation is now field dependent. This field dependence can be traced back to the definition of the vertical vector field that is used for the direct sum split and involves the field. 
In particular, it is not contributed by some intrinsic property of the theory at hand but solely depends on the chosen observer definition and the thus defined vertical vector field.

The commutator algebra that is generated by the three commutation relations (\ref{FDD}), (\ref{FDH}) and (\ref{FHH}) is the form of the hypersurface deformation algebra that is usually presented in the literature\footnote{In most cases the algebra is presented with minus signs on the right-hand side of each of the three commutation relations. The sign discrepancy results from different ways of defining the Lie bracket structure for vector fields. If one wants to work with the usual definition of the Lie algebra as left-invariant vector fields on a given Lie group, then the Lie bracket of two vector fields is actually the negative of their commutator. Details can be found in \cite{1985AnPhy.164..288I}.}, most famously in \cite{HOJMAN197688}. In this context, the appearance of the field components in the last commutation relation is usually referred to by stating that the commutator of two vertical vector fields involves structure functions instead of structure constants. Much work regarding this aspect of the hypersurface deformation algebra was contributed in \cite{1985AnPhy.164..288I} and \cite{1985AnPhy.164..316I}. Note that already there the authors remark that the appearance of the metric in the last commutator results from the metric dependent definition of the vertical vector field. 

With the observer definition presented in \cite{2018PhRvD..97h4036D}, \cite{2011PhRvD..83d4047R} and \cite{Rivera} the authors derive a hypersurface deformation algebra similar to the one obtained in the standard case but with the appearance of different field-dependent quantities in the commutator of two vertical vector fields. With the direct sum split of this observer definition, the first two commutators remain the same whereas the commutator of two vertical vector fields now reads:
\begin{align}\label{PolyAlg}
    \left[\mathcal{H}(N), \mathcal{H}(M) \right] = \left(\mathrm{deg}(\mathcal{P}_{mat}(v_A)) -1\right ) \cdot  \mathcal{D}_F\left(\mathcal{P}_{mat}(v_A)^{\mu \alpha}( N\partial_{\mu} M - M \partial_{\mu}N  ) \right),
\end{align}
where $\mathcal{P}_{mat}(v_A)^{\mu \alpha}$ is a quantity that is obtained from the \textit{\textbf{Principal Polynomial}} $\mathcal{P}_{mat}(v_A)$ of the underlying matter theory used as starting point and $\mathrm{deg}(\mathcal{P}_{mat}(v_A))$ denotes its polynomial degree. Also here the appearance of these field-dependent quantities in the commutation relation results from the field-dependent definition of the vertical vector field in terms of the observer notion that is used for the direct sum split.
\section{Constrained Hamiltonian Dynamics}
In the previous section, we have developed the necessary kinematic tools for the canonical treatment of field theories, mainly the notion of a slicing of the spacetime manifold and the associated split of the tangent bundle. With these tools at hand, we will now proceed with our analysis of diffeomorphism invariant field theories by constructing \textbf{\textit{Hamiltonian dynamics}} from a given Lagrangian.
The diffeomorphism invariance of a given field theory will make its appearance in the Hamiltonian formulation by the form of \textit{\textbf{constraints}} on the conjugate momenta that will then force us to treat the system utilizing the techniques developed by Dirac to ensure consistency of Constrained Hamiltonian dynamics.
In particular, we will find that the Hamiltonian corresponding to a diffeomorphism invariant Lagrangian field theory is in fact fully constrained, i.e., vanishes on the constraint surface. Analyzing the \textit{\textbf{Poisson algebra}} that is generated by these constraints, we will finally nicely recover the underlying diffeomorphism invariance of the theory.

As before we want to focus on field theories that are described by a first-order Lagrangian $\mathcal{L} : J^1F \rightarrow \Lambda^4M$. Given a slicing of the spacetime manifold, we again restrict all fields in consideration to the embedded hypersurface $\Sigma_{t_0}$. Given a section of the field bundle $G \in \Gamma(F)$ we write: 
\begin{align}
     G_{t_0} := G \vert _{\Sigma_{t_0}},
\end{align}
for this restriction. As the slicing provides us with a time vector field $\frac{\partial}{\partial t}$ we can now consider the infinitesimal change of such sections in time direction. More precisely we define the \textbf{\textit{time derivative}} of such sections as Lie derivative along $\frac{\partial}{\partial t}$: 
\begin{align}
    \dot{G}_A := \mathcal{L}_{\frac{\partial}{\partial t}}G_A = \lambda \partial_0 G_A + C^{B0}_{A0} \partial_0 \lambda G_B,
\end{align}
where we used the priorly derived coordinate expression in charts adapted to the slicing for the last equality. Similarly we define the time derivative for the restrictions of section to the hypersurface as $\dot{G}_{t_0}:= \dot{G} \vert_{\Sigma_{t_0}}$, i.e., the time derivative of restrictions is the restriction of the corresponding time derivative.

The first step in our development of a Hamiltonian formulation of the given Lagrangian field theory consists of reexpressing the Lagrangian $\mathcal{L}(v_A,v_{Ap})$ not in terms of spacetime 1-jets of a given field but in terms of spatial jets, i.e., 3-dimensional 1-jets over $\Sigma_{t_0}$ of the field and its time derivative.

To this end, we follow \cite{2004math.ph..11032G} and define the following map:
\begin{align}
    \begin{aligned}
    \beta_{t_0} : (J^1F)_{t_0} &\longrightarrow J^1(F_{t_0}) \times (\mathcal{V}F)_{t_0} \\
    \beta_{t_0}((j^1_pG)_{t_0}) &\longmapsto (j^1_pG_{t_0}, (\dot{G}_{t_0})_p).
    \end{aligned}
\end{align}
Here a subscripted $t_0$ denotes the restriction of the involved maps, bundles, etc. to $\Sigma_{t_0}$. Hence $(J^1F)_{t_0}$ is the restriction of the previously constructed jet bundle over $F$ to $\Sigma_{t_0}$, with elements being spacetime 1-jets restricted to the hypersurface, i.e., for $p \in \Sigma_{t_0}$ these elements have adapted coordinate expressions: 
\begin{align}
(j^1_pG)_{t_0} = (x^0,x^{\mu},G_A,\partial_m G_A) \big \vert_p,
\end{align}
where on $\Sigma_{t_0}$ in adapted coordinates $x^0=const$. 

The bundle $J^1(F_{t_0})$ is the first-order jet bundle that is constructed over the restricted field bundle $F_{t_0}$. In particular, as the restricted field bundle is a bundle with base space $\Sigma_{t_0}$, this is now a bundle over the embedded hypersurface as well. The elements in $J^1(F_{t_0})$ are spatial 1-jets of sections of this bundle. They are obtained by adjoining spatial derivatives along $\Sigma_{t_0}$ to such a section. In adapted coordinates we have: 
\begin{align}
(j^1_pG_{t_0}) = (x^{\mu},G_A,\partial_{\mu} G_A) \big \vert.
\end{align}

Compared to the information contained in a spacetime jet, the spatial 1-jet of a section lacks the contribution of derivatives in a direction that is vertical to the embedded hypersurface, i.e., a direction described by a tangential vector in $\mathcal{V}\Sigma_{t_0}$. It can be shown (cf. \cite{1998physics...1019G}) that these missing derivative contributions are precisely encoded by the elements of
$(\mathcal{V}F)_{t_0}$, the restriction of the vertical subbundle $\mathcal{V}F$ of $TF$ to the embedded hypersurface. We will denote these elements by $(\dot{G}_{t_0})_p$, where the subscript $p\in \Sigma_{t_0}$ labels the base point on the embedded hypersurface to whose fiber the element belongs. The vertical subbundle is the subbundle of $TF$, and hence, in particular, a bundle over $F$, that is obtained from $TF$ as fiber wise kernel of the vector bundle morphism $(\pi_F)_{\ast} : TF \rightarrow
TM$. Details can be found in \cite{1998physics...1019G}. 
Note that only the vertical subbundle of $TF$ can be be obtained naturally from a given bundle $F$. Furthermore $\mathcal{V}F$ can also be given the structure of a bundle over $M$ with projection $\pi_{\mathcal{V}}:=\pi_{TM} \circ (\pi_F)_{\ast}$. We can think of the restriction of the vertical bundle to the embedded hypersurface as encoding the derivative contributions to a spacetime 1-jet $j^1_pG$ that are vertical to the embedded hypersurface.

The map $\beta_{t_0}$ allows us now to decompose a spacetime 1-jet over $\Sigma_{t_0}$ into a spatial 1-jet and a time derivative:
\begin{align}
    \beta_{t_0}\bigl ( (x^{\mu},(G_{t_0})_A, \partial_m(G_{t_0})_A) \big \vert_p \bigr ) = (x^{\mu},(G_{t_0})_A, \partial_{\mu}(G_{t_0})_A, (\dot{G}_{t_0})_A) \vert _p.
\end{align}
We denote adapted coordinates on $J^1(F_{t_0}) \times (\mathcal{V}F)_{t_0}$ by $(x^{\mu}, v_A, v_{A{\mu}}, \dot{v}_A)$. It can now be shown that the map $\beta_{t_0}$ provides a bundle isomorphism (see \cite{2004math.ph..11032G}). Hence we can either work with spacetime 1-jets of sections over the embedded hypersurface or we work with spatial 1-jets of a section and its time derivative, $\beta_{t_0}$ and its inverse provide the identification of the two approaches.

In the following, we use the decomposition map $\beta_{t_0}$ to construct an equivalent version of the Lagrangian that is defined in terms of such spatial 1-jets and time derivatives. 
\begin{definition}[time dependend Lagrangian]
Given a Lagrangian $\mathcal{L} : J^1F \rightarrow \Lambda^4M$ and a slicing $\phi : \Sigma \times \mathbb{R} \rightarrow M$ with $\phi_{t_0}(\Sigma) = \Sigma_{t_0}$ we define the \textbf{\textit{time dependent Lagrangian}} at $t_0$ w.r.t. the slicing $\phi$ as:
\begin{align}
\begin{aligned}
    \mathcal{L}_{t_0} &: J^1(F_{t_0}) \times (\mathcal{V}F)_{t_0} \longrightarrow \Lambda^3\Sigma_{t_0}\\
    \mathcal{L}_{t_0}(j^1pG_{t_0}, (\dot{G}_{t_0})_p) &:= i_{\Sigma_{t_0}}^{\ast} \bigl( \mathcal{L}\circ \beta_{t_0}^{-1}(j^1pG_{t_0}, (\dot{G}_{t_0})_p)(\frac{\partial}{\partial t},-)\bigr). 
\end{aligned}
\end{align}
\end{definition}

In other words, we obtain the time-dependent Lagrangian on $\Sigma_{t_0}$ by using the inverse of the decomposition map $\beta_{t_0}$ to glue the data consisting of a spatial jet and a time derivative together to a spacetime jet. This spacetime jet is then plugged into the given Lagrangian which yields a volume form, i.e., a 4-form over $M$. We let this 4-form act upon the time vector field to obtain a 3-form which we then pull back along the canonical embedding $i_{\Sigma_{t_0}}$ of $\Sigma_{t_0}$ in $M$, to construct a spatial volume form, i.e., a 3-form on the embedded hypersurface $\Sigma_{t_0}$. 

In adapted coordinates, expressing the Lagrangian as $\mathcal{L} = L \mathrm{d}^4x$, with $\frac{\partial}{\partial t} = \lambda \frac{\partial}{\partial x^0}$ we can easily compute the pullback to obtain the corresponding expression for the time dependent Lagrangian on $\Sigma_{t_0}$:
\begin{align}
    \mathcal{L}_{t_0}(x^{\mu}, v_A, v_{A{\mu}}, \dot{v}_A) = L_{t_0}(x^{\mu}, v_A, v_{\mu}, \dot{v}_A)\lambda \mathrm{d}^3x= L(x^0,x^{\mu}, v_A, v_{\mu}, \dot{v}_A)\lambda \mathrm{d}^3x,
\end{align}
where we defined the coordinate expression of the time dependent Lagrangian $L_{t_0}$ and as before $x^0$ denotes the constant value of the zeroth coordinate function on the embedded hypersurface. As always $\mathrm{d}^3x = \mathrm{d}x^1 \wedge \mathrm{d}x^2 \wedge \mathrm{d}x^3$ is the coordinate volume form on $\Sigma_{t_0}$.
Note that by definition, the fibers of the vertical bundle $(\mathcal{V}F)_{t_0}$ carry the structure of a vector space. 
Hence one can construct its dual by standard means and we can thus define the bundle\footnote{Note that the notation $(\mathcal{V}F)_{t_0}^{\dagger}$ is used to distinguish this bundle from the standard dual $(\mathcal{V}F)_{t_0}^{\ast}$ and in particular is not meant to provide any association to the usual use of the $\dagger$ symbol in functional analysis.}: 
\begin{align}
(\mathcal{V}F)_{t_0}^{\dagger} := (\mathcal{V}F)_{t_0}^{\ast} \otimes \Lambda^3\Sigma_{t_0}.
\end{align}
Sections of this bundle provide at each $p \in \Sigma_{t_0}$ a $\Lambda^3_pM$-valued linear map on the fiber of $(\mathcal{V}F)_{t_0}$ over $p$.

We are now going to use the given Lagrangian at time $t_0$, $\mathcal{L}_{t_0}$, to construct a bundle isomorphism between $J^1(F_{t_0}) \times (\mathcal{V}F)_{t_0}$ and $J^1(F_{t_0}) \times (\mathcal{V}F)_{t_0}^{\dagger}$ by means of the \textit{\textbf{fiber derivative}}. Details can be found in \cite{abraham2008foundations} and also \cite{FiberDer} for the standard case in particle mechanics and \cite{2000RpMP...45...67G} and \cite{AIF_1973__23_1_203_0} for a more sophisticated treatment.
As the two bundles at use are in particular products of manifolds, they both define trivial bundles with base space $J^1(F_{t_0})$ and projection being the canonical projection onto the second factor that is provided by the product structure. We now define the fiber derivative w.r.t $\mathcal{L}_{t_0} = L_{t_0}\mathrm{d}^3x$ as bundle map:
\begin{align}
\begin{aligned}
    \mathcal{D}\mathcal{L}_{t_0} : J^1(F_{t_0}) \times (\mathcal{V}F)_{t_0} &\longrightarrow J^1(F_{t_0}) \times (\mathcal{V}F)_{t_0}^{\dagger}\\
    (j^1_pG_{t_0},(\dot{G}_{t_0})_p) &\longmapsto (j^1_pG_{t_0},\pi_{(G_{t_0})_p}), 
\end{aligned}
\end{align}
where:
\begin{align}
\pi_{(G_{t_0})_p}((\dot{\widetilde{G}}_{t_0})_p) := \frac{\partial L_{t_0}(j_p^1G_{t_0}, (\dot{G}_{t_0})_p+\lambda \cdot (\dot{\widetilde{G}}_{t_0})_p)}{\partial \lambda } \bigg \vert _{\lambda = 0}\mathrm{d}^3x.
\end{align}
Given adapted coordinates $(x^{\mu},v_A, v_{A\mu}, \dot{v}_A)$ on $J^1(F_{t_0}) \times (\mathcal{V}F)_{t_0}$ we can construct adapted coordinates on $J^1(F_{t_0}) \times (\mathcal{V}F)_{t_0}^{\dagger}$ by:
\begin{align}
(x^\mu,v_A,v_{A\mu},\pi^A\mathrm{d}^3x) \ \ \text{where} \ \ \pi^A := \frac{\partial L_{t_0}}{\partial \dot{v}_A}.
\end{align}
In the following we call the thus introduced coordinates $\pi^A$ \textit{\textbf{canonical momenta}}.
Using this, we obtain the \textit{\textbf{time dependent Hamiltonian}} as always by:
\begin{align}\label{Ham}
\begin{aligned}
&\mathcal{H}_{t_0} : J^1(F_{t_0}) \times (\mathcal{V}F)_{t_0}^{\dagger} \longrightarrow \Lambda^3\Sigma_{t_0} \\
    &\mathcal{H}_{t_0}(j^1_pG_{t_0},\pi_{(G_{t_0})_p}) = \pi_{(G_{t_0})_p}((\dot{G}_{t_0})_p) - \mathcal{L}_{t_0}(j^1_pG_{t_0},(\dot{G}_{t_0})_p). 
\end{aligned}
\end{align}
Note that, just as the time dependent Lagrangian, the time-dependent Hamiltonian takes values in $\Lambda^3\Sigma_{t_0}$ and thus can be integrated once it is evaluated on a section of  $J^1(F_{t_0}) \times (\mathcal{V}F)_{t_0}^{\dagger}$.
In adapted coordinates, we obtain the following expression for the time-dependent Hamiltonian:
\begin{align}
    \mathcal{H}_{t_0}(x^\mu, v_A, v_{A\mu},\pi^A) = \pi^A \dot{v}_A \mathrm{d}^3x - L_{t_0}(x^\mu,v_A,v_{A\mu},\dot{v}_A) \mathrm{d}^3x = H_{t_0} \mathrm{d}^3x.
\end{align}

At this point, we would like to remark that the authors in \cite{2004math.ph..11032G} follow a slightly different route to nevertheless arrive at a similar expression for the Hamiltonian. They first integrate the Lagrangian evaluated for a given section of $J^1(F_{t_0}) \times (\mathcal{V}F)_{t_0}$ over $\Sigma_{t_0}$ to define a functional on the space of such sections. Then the fiber derivative is taken with respect to this function to obtain the corresponding integrated version of the Hamiltonian as Legendre transform. 
Hence compared with the approach here they start working on the usual canonical phase space\footnote{One usually defines the canonical phase space at a given time $t_0$ as the space of all local sections of the canonical configuration bundle.} and therefore dealing with functionals earlier. 

Although this is the standard approach, we decided to take an alternative route. The reason for this decision lies in our goal of deducing implications of the diffeomorphism invariance of a given Lagrangian on the corresponding Hamiltonian formulation. To be more precise, we are interested in the consequences of the PDE system (\ref{DiffeoEqn}) --- that such a Lagrangian then necessarily has to solve --- for the constructed Hamiltonian. In particular, we would like to make use of these PDEs for the Lagrangian when computing the time evolution a given Hamiltonian provides employing a yet to define Poisson structure. Just as it was the case for the Lagrangian picture, where we were able to avoid working on infinite dimensional spaces of sections we would then like to achieve something similar, i.e., avoid working on the canonical phase space and hence with functionals.

Furthermore, it is worth mentioning that here in our approach, the Legendre transform of the time-dependent Lagrangian is only taken w.r.t. the time derivative coordinates $\dot{v}_A$, in contrast to multi-momenta approaches, such a De Donder-Weyl theory (\cite{deDonder} and \cite{Weyl}) where one defines conjugate momenta for all spacetime derivative coordinates of the field, not only for the time derivatives. This is for instance also discussed in \cite{1998physics...1019G}. 

To proceed further, note that both $F_{t_0}$, and $(\mathcal{V}F)_{t_0}^{\dagger}$, define bundles over $\Sigma_{t_0}$. Hence we can construct their jet bundles of some given finite order $q$, $J^q(F_{t_0})$ and $J^q((\mathcal{V}F)_{t_0}^{\dagger})$. 
We define for all $q>0$:
\begin{align} 
\mathcal{K}^q_{t_0} := J^q(F_{t_0})\times J^q((\mathcal{V}F)_{t_0}^{\dagger}),
\end{align}
and set $\mathcal{K}^0_{t_0} := F_{t_0}\times (\mathcal{V}F)_{t_0}^{\dagger}$.
In the following we will call $\mathcal{K}^q_{t_0}$ the \textit{\textbf{canonical configuration bundle}} at time parameter $t_0$. 

\begin{remark}
For the following considerations the appropriate setting would be the infinite jet bundle $\mathcal{K}^{\infty}_{t_0} := J^{\infty}(F_{t_0})\times j^{\infty}((\mathcal{V}F)_{t_0}^{\dagger})$ construction which can be obtained as limit of finite jet bundles $\mathcal{K}^q_{t_0}F$ (see for instance \cite{saunders_1989}). The reason for this is that certain operations such as taking \textit{\textbf{total}} and \textit{\textbf{variational derivatives}} increase the differential order of functions that are defined on $\mathcal{K}^{q}_{t_0}$.

This subtlety does however not pose a problem for us as any function $f_q$ on $\mathcal{K}^q_{t_0}$ can be equivalently considered as a function that is defined on some higher-order jet bundle $\mathcal{K}^r_{t_0}$ with $r\geq q$ by using the canonical jet bundle projections $\pi_{r,q}$. In other words given such a function $f_q$ we might define an essentially equivalent function on $\mathcal{K}^r_{t_0}$ by $f_r:=f_q \circ \pi_{r,q}$. If $r$ is chosen large enough, we are then guaranteed that taking derivatives of $f_r$ does not leave the $r$th-order jet bundle. Hence we will simply assume that the order $q$ of the canonical configuration bundle $\mathcal{K}^q_{t_0}$ is chosen large enough s.t. all functions in use and all their total and variational derivatives, etc. can, by this identification, be considered as being defined on $\mathcal{K}^q_{t_0}$. 
\end{remark}

Our next step lies in the construction of \textit{\textbf{Poisson bracket}} to encode dynamical information in the Hamiltonian formulation. 
We follow \cite{1997hep.th....9164B} to some extend. Consider the space of sections  $\Gamma(\mathcal{K}^0_{t_0})$ that is traditionally referred to as \textit{\textbf{canonical phase space}}\footnote{One often restricts to sections that have compact support over $\Sigma_{t_0}$} at time $t_0$. 

Just as the Lagrangian might be evaluated on the prolongation $j^2\phi$ of sections $\phi \in \Gamma(F)$ and then be integrated to define a functional on the space of sections, we obtain a functional on $\Gamma(\mathcal{K}^0_{t_0})$ by specifying a bundle map on the canonical configuration bundle\footnote{As mentioned in the introduction we try to stick to the convention of denoting density valued maps by calligraphic letters $\mathcal{L}, \mathcal{f},..$ and the corresponding coordinate expressions in terms of the coordinate volume form by the respective letters in standard font $L,f,...$.} $\mathcal{f}: \mathcal{K}^q_{t_0} \rightarrow \Lambda^3\Sigma_{t_0}$. Given a section of $\Gamma(\mathcal{K}^0_{t_0})$ we compose it with this bundle map and then integrate the resulting 3-form over $\Sigma_{t_0}$. We call phase space functionals that are obtained in this way \textit{\textbf{local functionals}}.  
\begin{definition}[local functional]
Let $\mathcal{f}: \mathcal{K}^q_{t_0} \rightarrow \Lambda^3\Sigma_{t_0} $ be a bundle map covering $id_{\Sigma_{t_0}}$, in local coordinates\footnote{For simplicity of the notation in the following we will not denote the arguments of such coordinate expressions explicitly.} $\mathcal{f} = f(x^{\mu},v_A,v_{A\mu},v_{AI}, ... v_{AI_k},\pi^A,\pi^A _{\mu}, \pi^A_I ... \pi^A_{I_k})\mathrm{d}^3x$. We define the induced phase space functional as:
\begin{align}
\begin{aligned}
    \mathcal{S}_{\mathcal{f}} : \Gamma(\mathcal{K}^0_{t_0}) &\longrightarrow \mathbb{R}\\
    \phi &\longmapsto \mathcal{S}_{\mathcal{f}}[\phi] := \int_{\Sigma_{t_0}} \mathrm{d}^3x f \circ j^q(\phi),
\end{aligned}
\end{align}
where $j^q\phi$ denotes the prolongation of sections of $\mathcal{K}^0_{t_0}$ to sections of $\mathcal{K}^q_{t_0}$. Such phase space functionals are called local functionals.
\end{definition}
In the following we will only work with local functionals. 
Given this definition, we can now define a Poisson structure on the canonical phase space 
\begin{definition}[poisson bracket]
Let $\mathcal{f},\mathcal{g} : \mathcal{K}^q_{t_0} \rightarrow \Lambda^3\Sigma_{t_0} $ with coordinate expressions $\mathcal{f} = f\mathrm{d}^3x$ and $\mathcal{g} = g\mathrm{d}^3x$. Denote the corresponding local functionals as $\mathcal{S}_{\mathcal{f}}, \mathcal{S}_{\mathcal{g}}$. We define their Poisson bracket as the phase space functional given by:
\begin{align}
\left \{ \mathcal{S}_{\mathcal{f}}, \mathcal{S}_{\mathcal{g}} \right \}[\phi] := \int _{\Sigma_{t_0}} \mathrm{d}^3x \left ( \frac{\delta f}{\delta v_A} \frac{\delta g}{\delta \pi^A} - \frac{\delta g}{\delta v_A} \frac{\delta f}{\delta \pi^A}     \right ) \circ j^q(\phi)  .
\end{align}
\end{definition}
Here $\frac{\delta}{\delta v_A}$ and $\frac{\delta}{\delta \pi^A}$ denote the variational derivatives as they are defined in (\ref{varDer}) in terms of the total derivative (for the definition see (\ref{totDer})) on $\mathcal{K}^q_{t_0}$:
\begin{align}
    \begin{aligned}
    \frac{\delta f}{\delta v_A} &= \frac{\partial f}{\partial v_A} - D_{\mu}(\frac{\partial f}{\partial V_{A\mu}}) + D_{\mu}D_{\nu} (\frac{\partial f}{\partial v_{A\mu\nu}}) - ... \\
    \frac{\delta f}{\delta \pi^A} &= \frac{\partial f}{\partial \pi^A} - D_{\mu}(\frac{\partial f}{\partial \pi^{A}_{\mu}}) + D_{\mu}D_{\nu} (\frac{\partial f}{\partial \pi^{A}_{\mu\nu}}) - ... \\
    D_\mu f &= \partial _\mu f + \frac{\partial f}{\partial v_A} v_{A\mu} + \frac{\partial f}{\partial \pi^A } \pi ^{A}_{ \mu} + \frac{\partial f}{\partial v_{A\nu}} v_{A\mu \nu} + \frac{\partial f}{\partial \pi^{A}_ {\nu}}\pi^{A}_{ \nu \mu} ... \ \ .
    \end{aligned}
\end{align}

Given this definition we see that the poisson bracket of two local phase space functionals actually only depends on the \textit{\textbf{variational derivatives}} of the corresponding functions. This suggests that we might equivalently define a \textit{\textbf{bracket structure}} directly on the canonical configuration bundle $\mathcal{K}^q_{t_0}$:
\begin{definition}
Let $\mathcal{f},\mathcal{g} : \mathcal{K}^q_{t_0} \rightarrow \Lambda^3\Sigma_{t_0} $ with the usual coordinate expressions in terms of $f$ and $g$. We define their bracket as: 
\begin{align}
    \left \{ f \mathrm{d}^3x,g\mathrm{d}^3x\right \} := \left ( \frac{\delta f}{\delta v_A} \frac{\delta g}{\delta \pi^A} - \frac{\delta g}{\delta v_A} \frac{\delta f}{\delta \pi^A} \right ) \mathrm{d}x^3  .
\end{align}
\end{definition}
\begin{remark}
When it does not cause confusion, we will also write $\left \{ f ,g\right \}$ for this expression and also might drop the volume factor of the coordinate expressions $\mathrm{d}^3x$ in concrete computations to lighten the notation. 
\end{remark}
The idea to avoid working with functionals and instead directly work over the space of local functions was first developed by Gel'fand, Dickey and Dorfman (cf. \cite{Gelfand1976} and \cite{Gelfand1979}). For a review see \cite{doi:10.1142/5108}.
Details regarding a construction along the lines that we follow here can be found in \cite{1997hep.th....9164B} and \cite{Barnich1998}. 
Note that this bracket structure is now in particular completely specified in terms of functions on $\mathcal{K}^q_{t_0}$ and nevertheless fully encodes the standard Poisson bracket on local phase space functionals. 

Two distinct bundle maps $\mathcal{f},\mathcal{g} : \mathcal{K}^q_{t_0} \rightarrow \Lambda^3\Sigma_{t_0}$ might still define the same local functional, namely simply if they only differ by a total divergence of a function $\mathcal{h}: \mathcal{K}^q_{t_0} \rightarrow \Lambda^2\Sigma_{t_0}$, i.e., if it holds that:
\begin{align}
\mathcal{f} = \mathcal{g} +\mathrm{d}x^{\alpha}(D_{\alpha} \mathcal{h}).
\end{align}
We can equivalently express the 2-form valued bundle map $\mathcal{h}$ in coordinates as $\mathcal{h} = h^{\mu} \epsilon_{\mu \nu \alpha} \mathrm{d}x^{\nu} \otimes \mathrm{d}x^{\alpha}$. The total divergence is then expressed as $\mathrm{d}x^{\alpha}(D_{\alpha} \mathcal{h}) = D_{\alpha} h^{\alpha}$. \\
In the bracket structure that we defined on $\mathcal{K}^q_{t_0}$ this is reflected by the fact that the variational derivative annihilates total divergences, in other words we have $\frac{\delta}{\delta v_A}D_{\alpha} h^{\alpha} = 0$ and $\frac{\delta}{\delta \pi^A}D_{\alpha} h^{\alpha}=0$. Hence we are free to add arbitrary total divergences to the given functions \footnote{More rigorously one can also work with equivalence classes and the corresponding quotient, by defining two functions to be equivalent if they define the same functional in the above sense. This is for instance done in \cite{1997hep.th....9164B} and \cite{Barnich1998}.} once we compute the bracket of two functions on $\mathcal{K}^q_{t_0}$. In the standard functional derivative formulation of the Poisson bracket on the canonical phase space, this, of course, corresponds to integrating by parts. 

With this definition of a bracket on $\mathcal{K}^q_{t_0}$ one obviously recovers the standard properties of Poisson brackets in field theory.
One in particular finds that the bracket satisfies the Leipniz rule and hence defines a \textit{\textbf{derivation}} of functions $\mathcal{f},\mathcal{g},\mathcal{h} : \mathcal{K}^q_{t_0} \rightarrow \Lambda^3\Sigma_{t_0}$:
\begin{align}
\left \{\mathcal{f}, \mathcal{g}\cdot \mathcal{h} \right \} = \mathcal{g} \cdot \left \{ \mathcal{f}, \mathcal{h} \right \} + \left \{ \mathcal{f} , \mathcal{g}\right \} \cdot \mathcal{h},
\end{align}
where the product of $\Lambda^3\Sigma_{t_0}$-valued maps is defined by $f \mathrm{d}^3x \cdot g\mathrm{d}^3x = (fg)\mathrm{d}^3x$ (for details see \cite{1997hep.th....9164B} and \cite{Barnich1998}). Furthermore one readily computes: 
\begin{align}
\left \{ v_A, \pi^B\right \} = \delta_A^B.
\end{align}

Our main goal in the following section will be to translate the PDE system (\ref{DiffeoEqn}) that is obtained from the diffeomorphism equivariance of the Lagrangian into a language that allows us to use the resulting equations in the formulation of Hamiltonian dynamics. To this end we chose adapted coordinates on $J^1F$ that additionally satisfy $x^0 = t$ and hence we find $\frac{\partial}{\partial t} = \frac{\partial}{\partial x^0}$ and $\dot{v}_A = v_{A0}$. 

From the expression (\ref{LieJ1}) of the Lie algebra morphism $j^1(\mathcal{f})$ that lifts vector fields $\xi \in \Gamma(TM)$ to vector fields on $J^1F$ we then find that using the decomposition map $\beta_{t_0}$ the lift of vector fields to $J^1(F_{t_0}) \times (\mathcal{V}F)_{t_0}$ reads:
\begin{align}\label{LieJ1Dec}
\begin{aligned}
    \xi_{J^1(F_{t_0}) \times (\mathcal{V}F)_{t_0}} = \ &\xi^m \frac{\partial}{\partial x^m} + C_{An}^{Bm} v_B \partial_{m} \xi ^n \frac{\partial}{\partial v_A}
    + C_{An}^{Bm} \partial_{m} \xi^n v_{B\nu} \frac{\partial}{\partial v_{A\nu}}\\
    &+ C_{An}^{Bm} \partial_{m} \xi^n \dot{v}_{B} \frac{\partial}{\partial \dot{v}_A} - v_{A\nu} \partial_{\mu} \xi^{\nu} \frac{\partial}{\partial v_{A\mu}} 
     - v_{A\nu} \partial_{0} \xi^{\nu} \frac{\partial}{\partial \dot{v}_{A}} 
    - \dot{v}_{A} \partial_{\mu} \xi^{0} \frac{\partial}{\partial v_{A\mu}}\\
     &- \dot{v}_{A} \partial_{0} \xi^{0} \frac{\partial}{\partial \dot{v}_{A}}
    + C_{An}^{B\mu} v_B \partial_{\mu} \partial_{\nu} \xi^n \frac{\partial}{\partial v_{A\nu}}
    + C_{An}^{B0} v_B \partial_{0} \partial_{\nu} \xi^n \frac{\partial}{\partial v_{A\nu}}\\
    &+ C_{An}^{B\mu} v_B \partial_{\mu} \partial_{0} \xi^n \frac{\partial}{\partial \dot{v}_{A}}
    + C_{An}^{B0} v_B \partial_{0} \partial_{0} \xi^n \frac{\partial}{\partial \dot{v}_{A}}.
\end{aligned}
\end{align}
If a given Lagrangian on $J^1F$ is now diffeomorphism invariant from this action of vector fields on the space of functions on $J^1(F_{t_0}) \times (\mathcal{V}F)_{t_0}$ we obtain the following system of PDEs for the corresponding time-dependent Lagrangian with coordinate expression $\mathcal{L}_{t_0} = L_{t_0}\mathrm{d}^3x$:
\begin{align}\label{diffeoHam}
    \begin{aligned}
    0 = &\partial_m L_{t_0} \\
    0 = &L_{t_0}^{:A} C_{An}^{Bm} v_B + L_{t_0}^{:A\nu} C_{An}^{B m} v_{B\nu} + \pi^A C_{An}^{B m}\dot{v}_B - L_{t_0}^{:A \mu} v_{A\nu} \delta^{\nu}_n \delta^m_{\mu}
    - \pi^A v_{A\nu} \delta^{\nu}_n \delta^m_{0} \\
    &- L_{t_0}^{:A \mu} \dot{v}_A \delta^0_n \delta^m_{\mu}
    - \pi^A \dot{v}_A \delta^0_n \delta^m_0 + L_{t_0} \delta_n^m \\
    0 = &L_{t_0}^{:A (\nu \vert } C_{An}^{B \vert \mu)} v_B \delta_{\nu}^{p} \delta_{\mu}^{q}
    +\pi^A C_{An}^{B  \mu} v_B \delta_{0}^{p} \delta_{\mu}^{q}
    +L_{t_0}^{:A \nu } C_{An}^{B 0} v_B \delta_{\nu}^{p} \delta_{0}^{q}
    +\pi^A C_{An}^{B  0} v_B \delta_{0}^{p} \delta_{0}^{q},
    \end{aligned}
\end{align}
where we used that in the chosen coordinates $\lambda = 1$ together with the definition of the canonical momenta $\pi^A = \frac{\partial L_{t_0}}{\partial \dot{v}_A}$. 
From this PDE system, we deduce the first consequence of the underlying diffeomorphism invariance on the Hamiltonian formulation. If we evaluate the last equation in (\ref{diffeoHam}) for $p=0=q$ we get:
\begin{align}\label{ConstrPi}
    0=\pi^A C_{An}^{B0}v_B.
\end{align}
There exist four independent linear combinations of the canonical momenta that vanish. This poses a problem for the Hamiltonian formulation as the fiber derivative that allowed us to obtain canonical momenta from the given Lagrangian and hence define the Hamiltonian is then no longer injective. To put it simply, the number of independent fiber coordinates on the field bundle $F$ is given by its fiber dimension. On the other hand, due to the four vanishing linear combinations of canonical momenta, the image of the fiber derivative has dimension four less.

This is something that we could have already seen from the PDE system that encodes diffeomorphism invariance on the level of the EOM (\ref{EOM}). For theories that are described by a Lagrangian on $J^1F$ evaluating the last equation for $m=p=q=0$ we get: 
\begin{align}\label{Constr1}
    0 = \frac{\partial^2 L}{\partial \dot{v}_A \partial \dot{v}_C } C_{An}^{B0} v_B.
\end{align}
Hence we see that the Jacobian of the transformation $\dot{v}_A \rightarrow \pi^B$ is non-invertible but has a non-vanishing kernel that contains $C_{An}^{B0}v_B$. In particular, this shows that the transformation that maps the time derivatives $\dot{v}_A$ to canonical momenta defined by $L$ is locally non-invertible. 

If one now takes a closer look at the equations of motion that are obtained from a diffeomorphism invariant first-order Lagrangian we get:
\begin{align}
    0 = E^A = \frac{\delta L}{\delta v_A} = L^{:A} - L^{:Am:B} v_{Bm} - L^{:Am:Bp}v_{Bpm}.
\end{align}
Only the last term in this expression contains second derivatives.
Contracting the EOM with $C_{An}^{C0}v_C$ we get:
\begin{multline}
    E^A C_{An}^{C0}v_C = L^{:A0:B0}v_{B00}C_{An}^{C0}v_C + 2L^{:A\mu : B 0} v_{B\mu 0}C_{An}^{C0}v_C + L^{:A\mu : B \nu} v_{B\mu \nu}C_{An}^{C0}v_C\\
    + \text{lower derivative order}.
\end{multline}
Note that only the first term contains second-order time derivatives. Precisely this term, however, vanishes due to the last equation in (\ref{EOM}) evaluated for the case of $L$ being a first-order Lagrangian. Hence contracting the EOM with $C_{An}^{C0}v_C$ reveals four independent linear combinations of EOM that are of first derivative order in the time-derivatives.

In the case of second-derivative-order PDE systems that describe the time evolution of some field\footnote{More precisely we consider such PDE systems that are hyperbolic in the sense that the Cauchy problem is well-posed for appropriate Cauchy surfaces and initial conditions. We will provide a more detailed formulation regarding this in the next chapter.} over $M$ the values of the field together with its first derivatives must be specified on a suitable hypersurface of $M$ as \textbf{\textbf{initial values}} to uniquely determine solutions away from the hypersurface. 
Any individual equation appearing in the system that is of lower than second-derivative-order in the time derivatives then does not contribute dynamical information to the evolution process described by the system but instead poses a restriction on the possible choices of the initial values on the hypersurface. In other words, the previously obtained four linear combinations that were only of first-derivative-order in the time derivatives are no evolution equations but constrain the possible initial data one may specify. These PDEs are then called \textbf{\textit{constraints}}.
Similarly, one also calls the conditions on the canonical momenta in the Hamiltonian formulation (\ref{ConstrPi}) constraints.

The treatment of Hamiltonian systems with constraints requires specific care as one needs to ensure that once initial data in terms of values for the field and the canonical momenta $(v_A,\pi^B)$ at given time is chosen, the time evolution generated by the Hamiltonian is such that also for later times the thus obtained values of the field and momenta satisfy the constraints. One often states that \textbf{\textit{consistency}} of the Hamiltonian formulation requires that the time evolution does not leave the constraint surface, i.e., the vanishing set of the constraints.

Much of the treatment of such constrained Hamiltonian systems was developed by Dirac \cite{dirac_1950} and cast into the well-know \textit{\textbf{Dirac-Bergmann algorithm}} (\cite{PhysRev.83.1018}, \cite{doi:10.1063/1.523597}). An introduction that mainly deals with constrained Hamiltonian systems in the context of GR can be found in \cite{bojowald_2010}. The necessary computations are included in more detail in \cite{thiemann_2007}. 

Given the previously constructed Hamiltonian (\ref{Ham}) and the Poisson structure on $\mathcal{K}^q_{t_0}$ we now follow along the lines of the Dirac-Bergmann algorithm to render the Hamiltonian in a consistent form. We start with the constraints:
\begin{align}
C_n := \pi^A C_{An}^{B0}v_B
\end{align}
that according to the terminology of Dirac are referred to as \textbf{\textit{primary constraints}}. The system is then described by the constrained Hamiltonian which takes the form: 
\begin{align}
\begin{aligned}
H_1 &:= H_{t_0} + \lambda^n C_n = \pi^A v_A -L_{t_0} + \lambda^n C_n\\
\smallskip
\mathcal{H}_1 &= H_1 \mathrm{d}^3x.
\end{aligned}
\end{align}
The 4 functions $\lambda^n = \lambda^n(x^{\alpha})$ are up to now arbitrary functions that take the role of Lagrange multipliers. Such functions that do not depend on the phase space coordinates $(v_A, \pi^B)$ obey a trivial time evolution w.r.t. any Hamiltonian $\dot{\lambda}^n = \left \{H, \lambda^n \right \} = 0$. The reason for adding them to the previously obtained Hamiltonian is that one can show the unconstrained, usual Hamilton equations of motion obtained in the standard way from $\mathcal{H}_1$ to be equivalent to the constrained Hamilton equations of motion computed from $\mathcal{H}_{t_0}$ (see for instance \cite{bojowald_2010}).

We proceed by computing the time evolution of the primary constraints $\left \{ H_1, C_n \right \}$. Note that the bracket between two primary constraints can be computed to be: 
\begin{align}
    \left \{C_n, C_m \right \} = \frac{\delta C_n}{\delta v_A}\frac{\delta C_m}{\delta \pi^A} - m \leftrightarrow n = \pi^B C_{Bn}^{A0}C_{Am}^{C0} v_C - m \leftrightarrow n .
\end{align}
We now take the second equation in (\ref{diffeoHam}) evaluate it for $m=0$ and act with $\frac{\partial}{\partial \dot{v}_B}C_{Bm}^{D0}v_D$ upon it to obtain: 
\begin{align}\label{CC1}
    0 = L_{t_0}^{:A:B0}C_{An}^{C0}v_CC_{Bm}^{D0}v_D + L_{t_0}^{:A0}C_{An}^{B0}C_{Bm}^{D0}v_D,
\end{align}
where any additional terms vanish as they are prolongations of (\ref{ConstrPi}) once we reinsert the definition of the momenta in terms of $L_{t_0}$. We now take equation (\ref{ConstrPi}) reexpress $\pi^A = L_{t_0}^{:A0}$, act with $\frac{\partial}{\partial v_A}C_{Am}^{v_D}$ and interchange the free indices $m$ and $n$ to get:
\begin{align}\label{CC2}
    0 = L_{t_0}^{:A:B0}C_{An}^{C0}v_CC_{Bm}^{D0}v_D + L_{t_0}^{:A0}C_{Am}^{B0}C_{Bn}^{D0}v_D.
\end{align}
Subtracting (\ref{CC1}) and (\ref{CC2}) yields now $0=\pi^B C_{Bn}^{A0}C_{Am}^{C0} v_C - m \leftrightarrow n$ and we find:
\begin{align}
    \left \{C_n, C_m \right \} \approx 0.
\end{align}
Here and in the following $\approx$ denotes \textit{\textbf{weak equality}}, i.e., equality on the constraint surface, the vanishing set of the constraints. Hence $\left \{C_n, C_m \right \} \approx 0$ states that the bracket of two constraints vanishes not identically for all possible canonical coordinates $(v_A,\pi^A)$ but only if one uses the constraints (\ref{ConstrPi}) as we did in the previous derivation. We move on by computing the remaining part of the time evolution of the primary constraints:
\begin{align}
    \left \{H_{t_0}, C_m \right \} = -\frac{\delta L_{t_0}}{\delta v_A}C_{Am}^{B0}v_B - \pi^A C_{Am}^{B0}\dot{v}_B = (D_{\mu}(L_{t_0}^{:A\mu}) - L_{t_0}^{:A}) C_{Am}^{B0}v_B - \pi^A C_{Am}^{B0}\dot{v}_B.
\end{align}
We now use that the bracket is only defined up to a total divergence and subtract $D_{\mu}(L_{t_0}^{:A\mu} C_{Am}^{B0}v_B)$ from this result. Doing this we obtain:
\begin{align}
    \left \{H_{t_0}, C_m \right \} = -L_{t_0}^{:A} C_{Am}^{B0} v_B - L_{t_0}^{:A\mu} C_{Am}^{B0}v_{B \mu} - \pi^A C_{Am}^{B0} \dot{v}_B.
\end{align}
Re-expressing this result with the use of the second equation in (\ref{diffeoHam}) we find: 
\begin{align}
    \left \{H_{t_0}, C_m \right \} = -\pi^A v_{A\mu} \delta^{\mu}_m - \pi^A \dot{v}_A \delta^0_m + L_{t_0}\delta^0_m.
\end{align}
Seperating this equation for the cases $m=0$ and $m = \mu$ we obtain in particular:
\begin{align}
\begin{aligned}
    \left \{C_0 , H_{t_0}\right \} &= H_{t_0} =: \mathbf{H} \\
    \left \{C_{\mu} , H_{t_0}\right \} &= \pi^A v_{A\mu} =: \mathbf{D}_{\mu}.
\end{aligned}
\end{align}

Continuing the Dirac procedure we have to add also these in Dirac's terminology \textit{\textbf{secondary constraints}} with multiplier functions $N=N(x^{\alpha})$ and $N^{\mu}= N^{\mu}(x^{\alpha})$ to the Hamiltonian. We get:
\begin{align}
\begin{aligned}
H_{2} &= N \cdot \mathbf{H} + N^{\mu} \cdot \mathbf{D}_{\mu} + \lambda^0 C_0 + \lambda^{\mu}C_{\mu}\\
\smallskip
\mathcal{H}_2 &= H_2 \mathrm{d}^3x.
\end{aligned}
\end{align}
We call the thus computed time evolution of the primary constraints the \textit{\textbf{Hamiltonian constraint}} $\mathbf{H}$ and the \textit{\textbf{diffeomorphism constraint}} $\mathbf{D}_{\mu}$.
We find that the new Hamiltonian $H_{2}$ is given by a linear combination of the four primary and four secondary constraints and hence vanishes on the constraint surface. This observation is summarized in the following theorem.
\begin{theorem}
The Hamiltonian $H$ corresponding to any given diffeomorphism invariant Lagrangian field theory on $J^1F$ is a linear combination of constraints, i.e., \textbf{\textit{fully constraint}} and hence vanishes weakly:
\begin{align}
   H \approx 0.
\end{align}
\end{theorem}
\begin{proof}
The claim follows immediately from the above considerations.
\end{proof}
\begin{remark}
Although this consequence of diffeomorphism invariance for the Hamiltonian picture seems to be widely known and accepted, it is surprisingly hard to find concrete proofs of it in the literature. Furthermore, in most cases, the authors assume without deeper investigation that certain results and techniques from the case of classical Hamiltonian dynamics can be generalized to the case of a Hamiltonian treatment of field theories without running into further trouble. As we do not want to follow this practice, we provided the precise proof with its full derivation in the above.
\end{remark}

Moving on with the Dirac-Bergmann algorithm, we now have to compute the time evolution of the secondary constraints generated by $H_2$ and require it to vanish weakly. We have already shown that the bracket between $H_{t_0}$ and the primary constraints $C_n$ yields exactly the secondary constraints and hence $\left \{\mathbf{H}, C_n \right \} \approx 0$  . One also readily finds that the bracket between the diffeomorphism constraint $\mathbf{D}_{\mu}$ and the primary constraints $C_n$ yields the total derivative of the primary constraints and hence vanishes weakly:
\begin{align}
    \left \{\mathbf{D}_{\mu}, C_n \right \} = D_{\mu} (\pi^A C_{An}^{B0}v_B) \approx 0.
\end{align}
Given this the only contributions to the time evolution of the secondary constraints that do not immediately vanish weakly come from brackets between $\mathbf{D}_{\mu}$ and $\mathbf{H}$. 

We are now going to calculate the three possible brackets between the secondary constraints times multiplier. In order to compute not only their time evolution but also deduce the evolution they generate themselves we are not going to use any consequence of the constraints, i.e., we do not immediately drop terms that vanish on the constraint surface but keep them until the end of the calculation. This approach will turn out to be practical when we are going to interpret the result that we will now compute. We start with the bracket between two diffeomorphism constraints and find:
\begin{align}
    \left \{ N^{\mu}\mathbf{D}_{\mu}, M^{\nu} \mathbf{D}_\nu \right \} =
    -M^{\nu}v_{A\nu}D_{\mu}(\pi^AN^{\mu}) + M \leftrightarrow N  
\end{align}
We now add two total divergences, $D_{\mu}(M^{\nu}v_{A\nu} \pi^A N^{\mu})$ and the same expression with $M \leftrightarrow N$ to obtain:
\begin{align}
    \left \{ N^{\mu}\mathbf{D}_{\mu}, M^{\nu} \mathbf{D}_\nu \right \} = (N^\mu \partial_\mu M^\nu - M^\mu \partial_\mu N^\nu) \mathbf{D}_\nu,
\end{align}
Where we used that the total derivative acts on the multipliers by $D_\mu N^{\nu} = \partial_\mu N^\nu$ and the terms proportional to $D_\mu v_{A\nu} = v_{A\mu \nu}$ drop out as then $v_{A\mu \nu}$ which is symmetric under the exchange of $\mu$ and $\nu$ is contracted against $N^\mu M^\nu - M^\mu N^\nu$ which obviously is anti-symmetric. 

We continue with the bracket between diffeomorphism constraint and Hamiltonian constraint. The computation can be simplified significantly by using the linearity to split the bracket w.r.t. the two contributions from $\mathbf{H}$: 
\begin{multline}
    \left \{ N^{\mu}\mathbf{D}_{\mu}, M \mathbf{H} \right \} = \left \{ N^\mu \mathbf{D}_\mu , M \pi^A \dot{v}_A   \right \} - \left \{ N^\mu \mathbf{D}_\mu , M L_{t_0}  \right \} = \\
    - D_\mu (N^\mu \pi^A) M \dot{v}_A - N^\mu v_{A\mu} (M L^{:A} - D_\nu (M L^{:A \nu}))  .
\end{multline}
After adding the total divergence $D_{\mu}(N^\mu \pi^A M \dot{v}_A)$ the first term can readily be computed to equal $N^\mu \partial_\mu M \pi^A \dot{v}_A$. Similarly, one finds that adding a total divergence to the second term yields an expression containing the total derivative of $L$. Adding one further total divergence this expression equals $-N^\mu \partial _\mu M L$. Hence in total we find that the only non-vanishing contribution is given by:
\begin{align}
    \left \{ N^{\mu}\mathbf{D}_{\mu}, M \mathbf{H} \right \} = N^\mu \partial_\mu M \mathbf{H}.
\end{align}
We finally compute the bracket between two Hamiltonian constraints. The computation can be simplified to a large extent by making the Leibniz rule satisfied by the bracket. Doing so one trivially finds:
\begin{align}
    \left \{N \mathbf{H}, M \mathbf{H} \right \} = N M \left \{ \mathbf{H}, \mathbf{H}\right \} = 0.
\end{align}
Summing up we have found that the secondary constraints obey the following relations of brackets between them:
\begin{align}
    \begin{aligned}
    \left \{N^\mu \mathbf{D}_\mu, M^\nu \mathbf{D}_\nu \right \} &= (N^\mu \partial_\mu M^\nu - M^\mu \partial _\mu N^\nu)\mathbf{D}_{\nu}\\
    \left \{ N^\mu \mathbf{D}_\mu , M \mathbf{H}\right \} &= (N^\mu \partial_\mu M) \mathbf{H}\\
    \left \{N \mathbf{H , M \mathbf{H}} \right \} &= 0.
    \end{aligned}
\end{align}
In particular we find that all these bracket relations are weakly zero and hence we do not need to add any further constraints to the Hamiltonian $H_2$. $H_2$ is consistent and the Dirac-Bergmann algorithm thus terminates. 

Besides that, we observe that the evolution generated by the secondary constraints satisfies the same algebra as the deformation of Hypersurface quantities displayed in (\ref{Alg}). This is something that we could have expected as in the context of Dirac's treatment of constrained Hamiltonians the secondary constraints are said to generate gauge transformations which obviously in the case of diffeomorphism invariant field theory precisely correspond to diffeomorphisms. 

When we deviated from the covariant picture by means of slicing spacetime into spatial 3-manifolds, the underlying action of the diffeomorphism was translated into the hypersurface deformation algebra (\ref{Alg}).
Hence, if the associated Hamiltonian formulation reflects the diffeomorphism invariance of the Lagrangian field theory in the context of constrained Hamiltonian dynamics, we would expect the secondary constraints to generate diffeomorphisms and thereby their algebra to resemble the hypersurface deformation algebra. This is precisely what happened here.

Note in particular that the previous considerations about possible changes in the algebra relations if one uses a differently constructed vertical vector field to obtain the direct sum decomposition of the tangent bundle, for instance by a field-dependent observer construction, would also affect the constraint algebra computed above. Such a different vertical vector field would, for instance, influence the decomposition of the spacetime 4-jet of fields by changing the definition of $\dot{v}_A$. In the end, this would probably yield a different split of the 4 secondary constraints into diffeomorphism constraint, which from the comparison to the deformation algebra can be thought of as generating spatial diffeomorphisms and Hamiltonian constraint which we might interpret similarly as the generator of diffeomorphism in the chosen time direction. 

Nevertheless, even when choosing a different split by the above reasoning, we still expect the correspondence between constraint algebra and hypersurface deformation algebra to be valid. Taking again GR as an example there the constraint algebra in the canonical formulation (see for instance \cite{thiemann_2007} for a rigorous derivation) features precisely the same metric dependent contribution in the bracket of two Hamiltonian constraints that we already observed in the commutator of two vertical vector fields. \\

Before we proceed with any further developments, we quickly summarize the results of this section in particular with their meaning towards Constructive Gravity in mind. 
Foremost we have developed a rigorous framework for Lagrangian field theories in terms of the jet bundle formalism.  Given this framework, we provided a precise definition of diffeomorphism invariance of a Lagrangian field theory, which at least infinitesimally turned out to be equivalent to the requirement that the Lagrangian solves a specific first-order linear PDE system.  This, in particular, constitutes a massive achievement from the Constructive Gravity point of view. The requirement of finding diffeomorphism invariant dynamics for a given gravitational field theory is now translated into a merely mathematical problem, namely obtaining solutions to the corresponding PDE system. Furthermore solving PDEs is a widely occurring problem and henceforth the underlying theory is quite extensively developed\footnote{In particular we will see in the following sections how we can always solve such equations at least perturbatively.}.

In the second part, we investigated the implications of the required diffeomorphism invariance on the associated Hamiltonian formulation. Above all, we have proven that the Hamiltonian of any diffeomorphism invariant field theory is fully constraint and hence vanishes weakly. Additionally, we revealed the underlying diffeomorphism group in the form of the hypersurface deformation algebra in the Poisson algebra satisfied by the secondary constraints of the Hamiltonian. Although this treatment of field theories in the Hamiltonian formulation will play a minor role in the future elaborations yet to come in this thesis, it nevertheless is essential for the conceptual understanding of Constructive Gravity. 

The precise meaning of this last statement can be understood best by taking a closer look at the three different ways the dynamics underlying GR, i.e., the Einstein Field Equations or equivalently the Einstein-Hilbert-Lagrangian or any other equivalent formulation, were reconstructed from first principles that we have already mentioned in the introduction. There we have Lovelock who constructed GR from the direct requirement of diffeomorphism invariant dynamics (see \cite{Lovelock1969}, \cite{doi:10.1063/1.1665613} and \cite{doi:10.1063/1.1666069}). 
We have Hojman, Kuchař, and Teitelboim who derived the GR dynamics from the requirement that the canonical dynamics represent the hypersurface deformation algebra. Moreover, last but not least we have contributions mainly due to Deser (\cite{1970GReGr...1....9D}) nicely reviewed in \cite{2008IJMPD..17..367P} that recover Einsteinian dynamics starting from a linearized version of metric gravity by the requirement that the self-coupling of the gravitational field be consistent in the sense that the total energy-momentum is conserved.

With the formerly developed deeper understanding at hand, all these methods can be understood as incorporating the principle of diffeomorphism invariance on different levels. Lovelock directly required diffeomorphism invariant dynamics. Hojman, Kuchař, and Teitelboim used the hypersurface deformation algebra as a guiding principle and required to obtain a representation thereof in the canonical formulation of dynamics. As we have seen in the above, this is a necessary consequence fulfilled by the Hamiltonian dynamics of any diffeomorphism invariant field theory. Finally, Deser's self-coupling approach uses energy-momentum conversation as the driving force. The energy-momentum tensor obtained from diffeomorphism invariance via the usual Noether construction presented in \cite{Gotay1992StressEnergyMomentumTA}, that furthermore in the case of GR boils down to the known formula, is always conserved. Hence also here the central requirement can be understood as a consequence of diffeomorphism invariance. 

Finally we wish to comment on more recent work \cite{2018PhRvD..97h4036D}, \cite{2012PhRvD..85j4042G} and also \cite{2017arXiv170803870S}, where the authors use a similar approach as the one used by Hojman, Kuchař, and Teitelboim, however comprehensively generalised to not only recover Einsteinian dynamics but provide a framework that essentially allows one to compute dynamical equations to any given geometry at wish. Also here the fundamental requirement is the representation of the hypersurface deformation algebra in terms of the constraint algebra of the canonical formulation of dynamics. There, the observer definition that is used depends on the causal structure of given matter dynamics that use the gravitational field as background geometry, more precisely the Principal Polynomial of the matter EOM. This quantity, to be more precise a certain sub-expression of it, also appears in one of the commutation relations of the hypersurface deformation algebra. The precise commutation relation is displayed in equation (\ref{PolyAlg}).   Requiring the constraint algebra to mimic this algebra, the authors then conclude that the guiding principle of their framework is the \textit{\textbf{causal compatibility}} between matter gravitational dynamics, encoded by the appearance of the sub-expression of the Principal Polynomial.

Although the ideas underlying the causal compatibility between matter and gravitational dynamics presented in the referenced work is unarguable of fundamental importance\footnote{We will, in fact, return to precisely these requirements in the following section.}, the interpretation that this constitutes the main ingredient of their framework solely from the appearance of parts of the Principal Polynomial --- already here the authors of \cite{2018PhRvD..97h4036D} remark that this appearance can, in fact, be traced back to the used observer frame --- in one of the commutation relations of the hypersurface deformation algebra seems a bit ad hoc.
With our developments in mind, we would argue that also in their framework, the one and only guiding principle is the requirement of the to-be-constructed dynamics to be invariant under spacetime diffeomorphisms.


\chapter{PDE Theoretic Approach to Constructive Gravity}\label{chapter2}
\dictum{
We gather main results and techniques from the formal approach to PDE theory with the focus lying in particular on the construction of power series solutions to a given PDE. These techniques allow us to formulate a second and final requirement that is posed on the gravitational dynamics if spacetime is additionally inhabited by a matter field: the causal compatibility between matter and gravitational EOM. Thus PDE theory bridges the gap between gravitational and matter dynamics.
Moreover, we develop a concise framework for the perturbative computation of gravitational Lagrangians that we formulate in terms of an explicit perturbative construction algorithm.
}
\section{Formal PDE Theory, Symbols and Involution}
Given the achievement of translating the first requirement we wish to pose on any gravitational dynamics --- the invariance under spacetime diffeomorphisms --- into an equivalent linear, first-order PDE system that we presented in the previous chapter, any further considerations that we provide in this chapter aim to solve one of the following two problems:
\begin{itemize}
    \item Assume we are given a concrete matter theory that employs a certain tensor field as background geometry.
    Such a matter theory could, for instance, be phenomenologically motivated. As we do not want to specify the values of the tensorial background geometry by hand we pursue the goal of supplementing the matter theory with dynamical equations that allow us to predict values of the tensor field, i.e., we complete the matter theory with an appropriate theory of gravity for the tensorial geometry. To that end, we try to solve the equivariance equations (\ref{DiffeoEqn}) specifically for the given tensor field. 
    Once we have solved the stated PDE system, we obtain a gravitational Lagrangian that describes the dynamics of this tensor field. The gravitational dynamics is however not independent of the matter field, as the latter will appear as source term in the gravitational EOM, describing how gravity is generated.
    The matter and gravity EOM actually constitute a coupled system (see Figure \ref{MatterGrav}), neither of the EOM can be solved independently. This raises the question of how we can ensure the two theories are, in fact, compatible in the sense that the given EOMs can be solved together.
    \item The above consideration obviously requires that we are able to obtain solutions to the equivariance PDE. This itself in no way constitutes a simple task and certainly demands specific techniques.
\end{itemize}
Although these two problems at first glance seem to have not much in common, it turns out that the key to the solution of either one of them lies in an in-depth treatment of partial differential equations. 
To that end, we will develop some additional tools and techniques that are indispensable for the  \textit{\textbf{formal theory}} PDEs.

\begin{figure}[hbt!]
\centering 
\begin{tikzpicture}[main node/.style={shape=rectangle, rounded corners,draw,fill=gray!40,font=\sffamily\large\bfseries}, side node/.style = {shape=rectangle, rounded corners,draw, font = \sffamily\large, text width = 2.5cm, align = center}]
\node[main node] (M) at (0,0) {matter field};
\node[main node] (G) at (10,0) {gravitational field};
\node[side node] (S) at (5,2) {sources};
\node[side node] (B) at (5,-2) {provides \ \  geometric
background};
\draw [thick] (M) to [out = 90, in = 180] (S);
\draw [-latex, thick] (S) to [out = 0, in = 90] (G);
\draw [thick] (G) to [out = 270, in = 0] (B);
\draw [-latex, thick] (B) to [out = 180, in = 270] (M);
\end{tikzpicture}
\caption{Matter Gravity Coupling}\label{MatterGrav}
\end{figure}

The formal theory of partial differential equations is phrased in terms of differential geometry. A PDE is defined as a submanifold of a jet bundle that is constructed over the bundle with base space coordinates being provided by the \textit{\textbf{independent variables}} and fiber coordinates given by the \textit{\textbf{dependent variables}} of the given problem. The advantage of such a description of partial differential equations does not only lie in the fact that one works now with coordinate independent geometric objects, but the jet bundle approach also enables one to treat the derivatives of a given function as algebraically independent new variables. Thus many problems that arise in the context of PDEs can be stated and solved in terms of basic \textit{\textbf{linear algebra}}. This, in particular yields the additional convenience that many of those arising problems can be tackled by using \textbf{\textit{computer algebra}}.

An introduction to the formal theory of partial differential equations can be found in \cite{saunders_1989}. A comprehensive treatment is provided in \cite{seiler2009involution} were also deeper results regarding the algebraic and homological aspects of the formal PDE theory are included. We will especially follow along the lines of \cite{seiler1994analysis} as there the relevant notions are concisely described with just the right level of rigor that is necessary for our subsequent developments. We start by stating the needed definitions, most of which can be found in \cite{seiler1994analysis}.
\begin{definition}[PDE]
Let $(F,\pi_F,M)$ be a bundle and $J^qF$ the $q$th-order jet bundle over $F$. A $q$th-order PDE $R_q$ on $F$ is a  submanifold of $J^qF$. A solution to a given PDE is a section $G \in \Gamma(F)$ s.t. the jet prolongation $j^q(G)$ lies entirely in $R_q$.  
\end{definition}
Sometimes one additionally requires $R_q$ to be fibered. Thereby one ensures that the PDE poses no restrictions on the independent variables, i.e., the base space coordinates on $M$ (this version is for instance used in \cite{seiler1994analysis}).
For many of the following computations, it is crucial to know the dimension of the jet bundle of order $q$ over a bundle $F$ with base space dimension $m$ and fiber dimension $n$:
\begin{align}
    \mathrm{dim}J^qF = n\binom{m+q}{q}
\end{align}
From this definition of a PDE one can obtain a PDE in the traditional sense by specifying an additional vector bundle $(E,\pi_E,M)$ over the same base space together with a bundle morphism:
\begin{align}
    \begin{aligned}
    \Phi : J^qF &\longrightarrow E\\
    (x^a, v_A, v_{Ap},...,v_{AI_q}) &\longmapsto \Phi^{\tilde{B}}(x^a, v_A, v_{Ap},...v_{AI_q}),
    \end{aligned}
\end{align}
such that the PDE is given as kernel of $\Phi$, i.e.
the PDE is described by the individual component  equations:
\begin{align}
    \Phi^{\tilde{B}}(x^a, v_A, v_{Ap},...v_{AI_q}) = 0.
\end{align}
Whenever we will work with such a representation of a PDE, w.l.o.g. we will always assume that the bundle morphism $\Phi^{\tilde{B}}$ is specified such that the individual component equations are really independent, i.e., the bundle morphism is non-degenerate. 

Given such a representation of a PDE, we can apply the usual jet bundle operations to the individual component functions $\Phi^{\tilde{B}}$, i.e., we can \textit{\textbf{prolong}} them to higher derivative orders by taking the previously defined \textit{\textbf{total derivative}} of the representation $\Phi^{\tilde{B}}$ along any base space direction and we can use the jet bundle projections to \textit{\textbf{project}} them to any lower derivative order at wish. We denote the PDE that is described by the combined set of the given equations $\Phi^{\tilde{B}}=0$ and all possible prolongations $D_i\Phi^{\tilde{B}}=0$ by $R_{q+1} \subset J^{q+1}F$. Similarly we denote higher prolongations of a PDE by $R_{q+r}$. We call $R_{q+r}$ the \textit{\textbf{prolongation}} of $R_q$ to $(q+r)$th-order or short the $r$th prolongation of $R_q$. 

Considering this, it is particularly interesting to first prolong a given PDE and then use the jet projections to define $R_q^{(1)} := \pi_{(q+1),q}\left ( R_{q+1} \right ) \subset J^qF $. As the following example taken from \cite{seiler1994analysis} illustrates this in general does not recover the original equation but only a subset $R_q^{(1)} \subset R_q$, i.e., prolonging and projecting might reveal additional equations.
\begin{example}\label{ExamplePDE}
Consider the PDE on the first-order jet bundle over the trivial bundle $\mathbb{R}^3 \times \mathbb{R}$ with jet bundle  coordinates $(x,y,z,u,u_x,u_y,u_z)$ defined by: 
\begin{align}
    R_1 : \begin{cases} u_z + y \cdot u_x &= 0 \\
                        u_y &= 0.
            \end{cases}
\end{align}
prolonging the first equation w.r.t. $y$, i.e., applying the total derivative: 
\begin{align}
D_y = \partial_y + u_{xy} \cdot \partial_{u_x} + u_{yy} \cdot \partial_{u_y} + u_{yz} \cdot \partial_{u_z}
\end{align}
to it we obtain the equation $u_{yz} + y \cdot u_{xy} + u_x =0$. Prolonging the second equation w.r.t. $x$ and $z$, however, shows that all second-order derivatives in the newly obtained equation vanish and we are left with $u_x = 0$. Inserting this equation into the first equation we furthermore find $u_z = 0$. Hence we find that the prolonged and projected system is given by: 
\begin{align}\label{prolo}
    R_1^{(1)} : \begin{cases} u_x = 0 \\
                        u_y = 0\\
                        u_z = 0 .
            \end{cases}
\end{align}
Note that the involved prolongations were mandatory to get this result; the additional equations in (\ref{prolo})  cannot be obtained by purely algebraic manipulations. 
\end{example}
Such additional independent equations that can only be obtained by prolonging to a higher-order and then projecting the PDE again to the previous order are called \textit{\textbf{integrability conditions}}. 
Usually, they might be found by constructing a specific linear combination of prolonged equations and thereby eliminating appropriate terms that contain derivatives of leading order such that one ends up with an equation of sub-maximal derivative order. In our example, we prolonged the first equation w.r.t. $y$ and then subtracted the prolongation of the second equation w.r.t $x$ and $z$ to discard the second derivatives $u_{xy}$ and $u_{yz}$ respectively. 

If a PDE $R_q$ already contains all its integrability conditions, i.e., if for all $n\geq 0$ it holds that $R_{q+n}^{(1)} = R_{q+n}$ we call it \textit{\textbf{formally integrable}}.
Formal integrability is of particular importance when computing \textbf{\textit{power series solutions}} to a given PDE. The essence of the computation of power series solutions is making a general power series ansatz with arbitrary expansion coefficients for the solution of the PDE and then, by successively inserting the ansatz into the PDE and all necessary prolongations, and evaluating the resulting equations at the expansion point, deriving linear equations for the expansion coefficients. 
To be more precise, we are not interested in computing such solutions up to infinite order and then summing up the series to obtain a real solution of the PDE, but we will simply abort the power series construction after some finite order, neglecting all further contributions that are obtained in higher orders.
If for some reason, the individual contributions are known to decrease with an increase in their order, the aborted power series solution yields a reasonable approximation to the real situation. 
In other words we will consider a \textit{\textbf{perturbative}} treatment of the PDE.
Thus we will only construct the power series ansatz\footnote{Such an ansatz should probably better be called a polynomial ansatz, but to emphasis the similarities to the case of computing real power series solutions we will stick to calling it a (finite) power series ansatz.} up to the desired finite order.

Using adapted coordinates on $F$, such a finite power series solution to a given PDE $R_q$, described by the equations $\Phi^{\tilde{A}} = 0$, can be computed as follows: We take a point $p_0 \in M$ --- in the adapted coordinates on $F$ we define: $x^m(p_0) =: x_0^m$ --- and construct a general section $G\in\Gamma(F)$ as arbitrary, finite power series around $x_0$:
\begin{align}
\begin{aligned}
    G_A(x^m) :=  \sum_{k=0}^{r} a_{AI_k}I^{I_k}_{i_1...i_k}(x^{i_1}-x_0^{i_1}) \cdot ... \cdot (x^{i_k}- x_0^{i_k}), 
\end{aligned}
\end{align}
where $a_{AI_k}$ are constants. Note in particular that the $q$th jet prolongation of $G$ evaluated at $p_0$ then yields, up to combinatorial\footnote{This is a result from our factor-less definition of the finite power series. The factor-less definition is used as it is in closer relation to the later treatment of perturbative Lagrangians.} factors, the following coordinate expression:
\begin{align}
    j^k(G)(p_0) \equiv \left ( x_0^m, a_A, a_{Am}, a_{AI}, ... a_{AI_q} \right ).
\end{align}
Hence plugging in the finite power series into the PDE $\Phi^{\tilde{A}} =0$ and evaluating at $x^m=x^m_0$ yields a purely algebraic equation system for the expansion coefficients up to order $q$, that we can readily solve. Prolonging the equation to order $q+r$ and then plugging in the power series ansatz we obtain algebraic equations for the higher expansion coefficients up to order $q+r$ that feature the priorly solved coefficients $a_{AI_0},...,a_{AI_q}$ as inhomogeneities. We then can solve these equations for the new expansion coefficients $a_{AI_{q+1}}...a_{AI_{q+r}}$. The solutions, in general, will be given by expressions that contain the lower-order coefficients.
The algebraic equation system that we obtain looks as follows:
\begin{align}
\begin{aligned}
R_q &: \Phi^{\tilde{A}}(x_0^m,a_A,...,a_{AI_q}) &&= 0 \\
R_{q+r} &:  D_{J_r}\Phi^{\tilde{A}}(x_0^m,a_A,...,a_{AI_{q+r}}) &&= 0,
\end{aligned}
\end{align}
where as before $D_{J_r} = J^{i_1...i_r}_{J_r} D_{i_1} ... D_{i_r}$.

The occurrence of integrability conditions during prolongations of the PDE then hinders one to construct such a power series solution order by order. Potentially occurring integrability conditions yield further equations for lower-order coefficients that only are present after higher-order prolongations are computed. Thus, once one has solved the problem up to some finite order $r$, it might happen that integrability conditions of higher prolongation order yield additional equations that further restrict the obtained solution. 
Hence the computed solution is, in fact, too general, i.e., it includes fake solutions.
One can, however, in most cases, not easily predict during which prolongations these integrability conditions appear. Therefore, if a PDE is not formally integrable, i.e., if it generates integrability conditions, one can never be sure that obtained perturbative solutions are not too general. To put it differently, one can never be sure if all information that is contained in the PDE is also included in the construction of the power series solution, or if information remains hidden in integrability conditions.

Although the perturbative treatment of PDEs compromises a vast field in theoretical physics, techniques that allow the proof of formal integrability of a given PDE that we will subsequently present, seem to be to a large extent unknown in this area of research. 

Summing up, we have thus seen that the occurrence of integrability conditions thoroughly disturbs the order by order construction of power series solutions. When treating a specific PDE perturbatively, it is, therefore, essential to first check whether or not the PDE is formally integrable; if it is not any obtained solution is practically meaningless.
As we, in particular, want to develop a framework for the perturbative treatment of the equivariance equation (\ref{DiffeoEqn}), we hence have to establish techniques for proofing formal integrability.
Explicitly checking formal integrability of a given PDE is, however, extensively involved, as in principle one would need to probe the infinite number of conditions $R_{q+r}^{(1)} = R_{q+r}$ corresponding to an infinite number of prolongations and projections that one would need to compute. Thus subsequent developments concentrate on how we can predict in advance whether or not integrability conditions occur during specific prolongations of a given PDE, without actually carrying out the explicit computation.

To that end, we take a closer look at a specific step that is involved in  the order by order construction of power series solutions, namely the step of solving the various linear systems.
In each order $q+r$ only a certain number of the newly obtained highest order coefficients $a_{AI_{q+r}}$ is in fact determined by the prolonged PDE $R_{q+r}$, some of the coefficients are not governed by the PDE at all and can hence be chosen freely.
For the particular case of \textit{\textbf{quasilinear}}\footnote{Note that although if $R_q$ is not quasilinear, all its prolongations will be quasilinear as the total derivative always yields quasilinear equations. For the same reason PDEs that are obtained as EOMs to a given Lagrangian will always be quasilinear.} PDEs, i.e., PDEs that are linear in the highest order derivatives that occur, the expansion coefficients that can be chosen freely are given by the kernel of a matrix that is obtained from the highest derivative part of the PDE. This matrix is called the \textit{\textbf{symbol}} of the PDE.
\begin{definition}[symbol]
Given a PDE $R_q$ with representation $\Phi^{\tilde{B}}=0$, its symbol is the following matrix that only contains the highest-derivative-order contributions to $R_q$:
\begin{align}
    M_q : M_q^{(\tilde{B})({AI_q})} := \left ( \frac{\partial \Phi^{\tilde{B}}}{\partial v_{AI_q}} \right ).
\end{align}
The rows of the symbol are labeled by the independent equations $\tilde{B}$, i.e., the components of $(\Phi^{\tilde{B}})$, the columns are labeled the different highest-derivative-order contributions $(AI_q)$.
Similarly we can directly obtain the symbol of the prolonged equations $R_{q+r}$ as:
\begin{align}\label{proSym}
    M_{q+r} : M_{q+r} ^{(\tilde{B}I_r) (AI_{q+r})}:= \left ( \frac{\partial D_{I_r}\Phi^{\tilde{B}
    }}{\partial v_{AI_{q+r}}} \right ). 
\end{align}
Note that now the rows are labeled by order $r$ prolongations of individual equations $(\tilde{B}I_r)$\footnote{As the notation of: $M_{q+r} ^{(\tilde{B}I_r) (AI_{q+r})}$ for the prolonged symbol really only is used to indicate the way we arrange the different contributions in the matrix the reader should not be confused by the appearance of $I_r$ in upper position on the left-hand-side and in lower position on the right-hand-side.}.
\end{definition}
It is important to observe that the symbol of a quasilinear PDE really only collects its highest-derivative-order coefficients in matrix form.
If the PDE $R_q$ contains a contribution proportional to $\frac{\partial \Phi^{\tilde{B}}}{\partial v_{AI_q}}$ , recalling the definition of the total derivative (\ref{totDer}), we find that the prolonged PDE $R_{q+r}$ involves precisely the same contribution, but multiple times and possibly in multiple of its individual equations. 
To be more precise, the matrix components of the prolonged symbol matrix are given by:
\begin{align}
M_{q+r}^{(\tilde{B}I_r) (AI_{q+r})} = 
\begin{cases}
M_q^{(\tilde{B}) (AI_q)}  & \text{if \ \ }  
    v_{AI_{q+r}} = D_{I_r} v_{AI_q} \\
0 & \text{otherwise}.
\end{cases}
\end{align} 
We now use that: 
\begin{align}
    D_{I_r}v_{AI_q} = v_{AI_{q+r}}I_{I_q I_r}^{I_{q+r}}.
\end{align}
Thus we see that we can easily express the prolonged symbol $M_{q+r}$ by the matrix elements of the unprolonged symbol $M_q$ as:
\begin{align}
    \begin{aligned}
    M_{q+r} : M_{q+r} ^{(\tilde{B}I_r) (AI_{q+r})}  = \left ( \frac{\partial \Phi^{\tilde{B}}}{\partial v_{AI_q}} \right ) I_{I_q I_r}^{I_{q+r}} .
    \end{aligned}
\end{align}
We will synonymously call the matrix and the corresponding linear equation system the symbol of the PDE. The precise meaning can be inferred from the context. 


Although only containing contributions from the highest derivative order, the symbol already incorporates considerable amounts of the PDE's overall information.
When constructing order by order power series solutions
a general power series ansatz contains
$n\binom{m+q-1}{m-1}$ undetermined expansion coefficients $a_{AI_q}$ of order $q$th. 
Precisely those expansion coefficients $a_{AI_q}$ that lie in the kernel of the symbol $M_q$ are not restricted by the PDE $R_q$ and thus can be specified arbitrarily. In particular, the number of such arbitrary expansion coefficients equals the dimension of the kernel of $M_q$. 
Hence the rank of the symbol $M_q$ contains the complete information regarding how many undetermined expansion coefficients of order $q$ appear in the general power series solution.
Similarly, the ranks of the prolonged symbols $M_{q+r}$ yield this information for higher-order expansion coefficients $a_{AI_{q+r}}$.

If we restrict attention to analytic solutions of a given PDE, and further disregard questions concerning the convergence of the power series solutions for a moment, then such undetermined expansion coefficients in the general power series solution are in one-to-one correspondence with arbitrary Taylor coefficients in the expansion of the general analytic solution. Thus the symbol $M_q$ and all its prolongations $M_{q+r}$ govern all the information of how many arbitrary Taylor coefficients the general analytic solution of the PDE contains, i.e., these matrices contain the entire information about the size of the formal\footnote{As we disregarded the question of convergence of power series solutions such power series are also called formal power series. } solution space of the given PDE. 

For our future developments, the symbol, however, foremost attains its significance due to its close relation to the occurrence of integrability conditions.
Note that for integrability conditions to occur during a prolongation, it must be possible to form a certain linear combination of prolonged equations that does not contain contributions in the highest derivative order. Such linear combinations only exist if the symbol has sub-maximal rank, i.e., suffers from rank defects, as the specific linear combination of equations that does not contain contributions in highest derivative order then precisely produces a zero-row in the symbol matrix. This can be seen in more detail by prolonging a given PDE and then taken its linearization, i.e., the Jacobi matrix of its representation:
\begin{align}
\def\arraystretch{2.5}
\begin{bmatrix}
      \ \ \mathlarger{{\frac{\partial D_i\Phi^{\tilde{A}}}{\partial v_{AI_{q+1}}}}} \ & \vline & \mathlarger{\frac{\partial D_i \Phi^{\tilde{A}}}{\partial v_{AI_k}}}, \ \ k \leq q \ \  \\
        \cmidrule(lr){1-3}
        \mathlarger{0} & \vline & \mathlarger{\frac{\partial \Phi^{\tilde{A}}}{\partial v_{AI_k}}}, \ \ k \leq q \ \
\end{bmatrix}.
\end{align}
The lower right block is simply the Jacobi matrix of the unprolonged PDE. The upper left block is precisely the prolonged symbol matrix $M_{q+1}$. Only if $M_{q+1}$ has sub-maximal rank there exist certain linear operations that produce zero rows in $M_{q+1}$ and hence yield equations of sub-maximal derivative order. If such linear combinations exist, the prolongation, however, does not necessarily produce integrability conditions.
There exist essentially three possibilities:
\begin{itemize}
    \item If the corresponding linear operation also produces a zero row in the upper right block, we are left with an overall zero row which thus can be removed. 
    Hence, in this case, we do not end up with an integrability condition.\\
    \item If the obtained row in the upper right block is not equal to zero, there might nevertheless exist a linear combination of equations in the original system $R_q$ that is equivalent to this row. 
    If this is the case the newly obtained equation does not add information to the system and hence can be discarded. Thus also then we do not obtain an integrability condition but merely find a redundant description of the system.\\
    \item Only if the row that is produced in the upper right block yields a new equation that is independent of the original system we have found an integrability condition.  
\end{itemize}
Hence for an integrability condition to occur, it is necessary, but not sufficient to have a sub-maximal rank in the prolonged symbol.

The above discovery that qualitatively relates rank defects in the prolonged symbol to the possible occurrence of integrability conditions further allows one  to also quantitatively deduce the number of integrability conditions that occur during a prolongation.
The following formula relates the manifold dimension of the prolonged and projected PDE $R_q^{(1)}$, i.e., the system that contains all integrability conditions that occur during the prolongation to the next higher order, to the manifold dimension of the prolonged PDE $R_{q+1}$ and the dimension of the solution space of the prolonged symbol $\mathrm{dim(}M_{q+1})$, i.e., the dimension of its kernel. The formula is proven in \cite{seiler1994analysis}:
\begin{align}
    \mathrm{dim}(R_{q}^{(1)}) = \mathrm{dim}(R_{q+1}) - \mathrm{dim}(M_{q+1}).
\end{align}

Given this observation, we deduce that we can avoid the possible occurrence of integrability conditions entirely by performing only such prolongations that are certain not to produce rank defects in the prolonged symbol. Such prolongations obviously only constitute a subset of all possible prolongations and hence, in general, do not yield all independent equations of the prolonged PDE. In special cases, they, however, nevertheless contain at least the entire information about the highest derivative order equations in $R_{q+1}$.

We follow \cite{seiler1994analysis} in introducing the \textit{\textbf{class}} of a \textit{\textbf{derivative index}} $I_k$. We label the coordinates of $M$ from $1$ to $m$ and call the label of a specific coordinate $x^j$ its class. The thus introduced notion class labels can be extended to higher derivative indices $I_k$, by defining the label of such an index as the according to the base space coordinate labeling, smallest $i$ s.t. there exist $j_2,...,j_k\geq i$ with $I^{I_k}_{ij_2...j_k} \neq 0$, i.e., the "smallest" base space derivative label that occurs in the higher derivative index $I_k$.

We proceed by sorting the columns of the solved symbol $M_q$ according to classes without demanding a particular order of the elements inside a given class, i.e., we sort the derivative indices with the largest possible class to the leftmost columns of the matrix and those belonging to the smallest class to its rightmost columns.
The following considerations only involve linear algebra. Therefore we are free to apply linear operations to the symbol $M_q$ to equivalently work with its reduced row-echelon form. We call the thus obtained row reduced symbol matrix the solved form of $M_q$. 
It is crucial that we only perform row-operations on the symbol, such that the class order of its columns remains intact.
We call a row of $M_q$ with pivot\footnote{The pivot of a row is its first non-zero element.} belonging to a column of class $k$ a row of class $k$, and we denote the number of rows in $M_q$ that are of class $k$ by $\beta_q(k)$. Note that there is no deeper meaning to the class of a derivative index. We could, in particular, apply a simple change of coordinates on $J^qF$ to thoroughly mix up the class order. The reason why we nevertheless sort the columns of $M_q$ according to classes is that we will use this labeling to introduce a systematic way of computing all possible prolongations that are guaranteed not to produce rank defects in the symbol and therefore are also certainly free of integrability conditions.

This can be achieved by proceeding as follows:
Given an equation that corresponds to a row of class $k$, we only prolong it w.r.t. $D_j$ for  $j \leq k$. The corresponding independent variables $x^j$ where $j \leq k$ are then called the \textbf{\textit{multiplicative variables}} of this equation. If we proceed like this for the whole PDE, i.e., prolonging each equation only with respect to its multiplicative variables, we are guaranteed not to produce integrability conditions. All prolongations that are obtained in such a way will have distinct pivot elements in $M_{q+1}$ and hence are independent. Therefore we get no rank defects in the prolonged symbol and hence no integrability conditions.

For the above reason it would be particularly advantageous to be able to prolong the equations of a given PDE $R_q$ only w.r.t. to their multiplicative variables and nevertheless know that one obtains all independent equations of $R_{q+1}$ by doing so, as this would  guarantee that during the prolongation to order $q+1$ the PDE never generates integrability conditions that are of maximal order, i.e., of order $q+1$.
Note that prolongations w.r.t. non-multiplicative multiplicative variables might nevertheless contribute independent equations, but these are then necessarily of lower order, i.e., these precisely are integrability conditions.
In particular, \textit{\textbf{all}} non-multiplicative prolongations either are redundant or yield integrability conditions.
The idea to distinguish PDEs $R_q$ for which all independent equations can be obtained by multiplicative prolongations only leads to the notation of an \textit{\textbf{involutive symbol}}. 
\begin{definition}[involutive symbol]
The symbol $M_q$ of a PDE $R_q$ is called involutive if:
\begin{align}\label{sumBeta}
    \sum_{k=1}^m k\beta_q(k) = \mathrm{rank}(M_{q+1}).
\end{align}
\end{definition}
\begin{remark}
Whereas the rank of the prolonged symbol obviously does not change under a change of coordinates,
as we have outlined above the class of a derivative index and hence also the sum of the beta numbers that are involved in this definition are coordinate dependent notions. One can, however, show that the sum is the same in a specific class of coordinates, so-called \textit{\textbf{$\boldsymbol{\delta}$-regular}} coordinates. These are characterized by the requirement that the sum admits its maximal value.
We thus have to restrict to $\delta$-regular coordinates for the above definition to be well defined.
Hence for concrete computations in principle, one would have to check that the chosen coordinates are $\delta$-regular.  We will however not be too much concerned by that restriction as in the future developments we will only have to deal with linear PDEs, and for such we will trivially obtain the maximum value of the sum from the dimension of the underlying space of independent variables, without ever having to introduce classes of derivative indices.
\end{remark}
Note that the sum of these beta numbers is a lower bound for the rank of the prolonged symbol $M_{q+1}$. $\sum_{k=1}^m \beta(k)$ is precisely the number of independent equations of order $q+1$ that can be obtained by prolongations with respect to multiplicative variables only.  
Hence for a PDE with an involutive symbol, we can obtain all independent equations of order $q+1$ by prolongations with respect to multiplicative variables only. Any prolongation w.r.t. a non-multiplicative variable then is necessary either redundant or of sub-maximal order $\geq q$ and produces integrability conditions. 
Thus, if the symbol of a given PDE is involutive, for each equation, we can split the possible prolongations into two disjoint sets, those w.r.t. a multiplicative variable that are hence free of integrability conditions, and those w.r.t. a non-multiplicative variable which if the resulting equation is independent of the PDE must produce integrability conditions.

There are further exciting properties of involutive symbols, most of which can be found in \cite{seiler2009involution} and \cite{seiler2009involution}. It is also worth noting that the idea of analyzing the generation of integrability conditions along the lines presented above can be traced back to early works of Janet \cite{janet1920systemes}, \cite{MSM_1927__21__1_0} and Riquier \cite{bateman_1910} in the context of Janet-Riquier theory.

\begin{example}
We consider again the PDE from Example (\ref{ExamplePDE}):
\begin{align}
    R_1 : \begin{cases} (i)\hphantom{i} \ \ \ \ u_z + y \cdot u_x &= 0\\
                        (ii) \ \ \ \ u_y &= 0.
            \end{cases}
\end{align}
We label the coordinates as: $x \equiv 3, y \equiv 2, z \equiv 1$. The symbol matrix $M_1$ of $R_1$ is given as:
\begin{align}
M_1 : \begin{cases}
\ \ 
\begin{blockarray}{cccc}
\text{class 3} & \text{class 2} & \text{class 1} \\
u_x & u_y & u_z \\
\begin{block}{(c|c|c)l}
  y & 0 & 1 & (i) \\
  0 & 1 & 0 & (ii) \\
\end{block}
\end{blockarray}
\end{cases}
\end{align}
The symbol matrix is already in solved form. We thus have one row of class $3$ and one row of class $2$ and hence find:
\begin{align}
  \sum_{k=1}^3 k\beta_1(k) = 5.  
\end{align}
The prolonged system $R_2$ is given by the following $8$ equations:
\begin{align}
    R_2 : \begin{cases} (i)\hphantom{vii} \ \ \ \   u_z + y \cdot u_x &= 0 \\
                        (ii)\hphantom{vi} \ \ \ \  u_y &= 0. \\
                        (iii)\hphantom{v} \ \ \ \  u_{xz} + y \cdot u_{xx} &= 0 \\
                        (iv)\hphantom{ii} \ \ \ \  u_x + u_{yz} + y \cdot u_{xy} &= 0 \\
                        (v)\hphantom{iii} \ \ \ \  u_{zz} + y \cdot u_{xz} &= 0 \\
                        (vi)\hphantom{ii} \ \ \ \ u_{xy} &= 0 \\
                        (vii)\hphantom{i} \ \ \ \ u_{yy} &= 0 \\
                        (viii) \ \ \ \ u_{yz} &= 0 
            \end{cases}
\end{align}
We can use equation $vi$ and equation $viii$ to remove all second derivative order contributions from equation $iv$.
The prolonged symbol $M_2$ thus reads:
\begin{align}
M_2 : \begin{cases}
\ \ 
\begin{blockarray}{ccccccc}
\BAmulticolumn{3}{c}{\text{class 3}} & 
\BAmulticolumn{2}{c}{\text{class 2}} & 
\text{class 1} \\
u_{xx} & u_{xy} & u_{xz} & u_{yy} & u_{yz} & u_{z} \\
\begin{block}{(ccc|cc|c)l}
  y & 0 & 1 & 0 & 0 & 0 & (iii) \\
  0 & 1 & 0 & 0 & 0 & 0 & (vi) \\
  0 & 0 & y & 0 & 0 & 1 & (v) \\
  0 & 0 & 0 & 1 & 0 & 0 & (vii) \\
  0 & 0 & 0 & 0 & 1 & 0 & (viii) \\
\end{block}
\end{blockarray}
\end{cases}
\end{align}
Here we reordered the rows and removed all zero-rows. The rank of matrix $M_2$ clearly is $5$. Therefore the symbol of this example is involutive.
\end{example}

The involution of the symbol of a PDE is of particular importance as it allows for an efficient criterion whether or not the given PDE if formally integrable, i.e., whether or not it generates integrability conditions during its prolongations.
The basic idea is to distinguish the special class of PDEs $R_q$, for which all equations contained in the prolongation $R_{q+1}$ can be reached by prolongations w.r.t. multiplicative variables only. As such multiplicative prolongations are free of integrability conditions, we then know fur sure that the system does not produce integrability conditions during the prolongation to order $q+1$. The astonishing fact, however, is that this then holds also true for all further prolongations, the PDE is then, in fact, formally integrable. 
If a PDE does not only feature an involutive symbol, but further also is formally integrable, we call the \textit{\textbf{PDE involutive}}.
\begin{definition}[involutive PDE] \label{invol}
We call a PDE $R_q$ involutive if it is formally integrable and its symbol $M_q$ is involutive.
\end{definition}
Unlike formal integrability, the stronger requirement of an involutive PDE, which in particular encompasses formal integrability, can actually be checked in a finite number of steps.
To illustrate this we first supplement the following result that is proven in \cite{seiler1994analysis}:
\begin{theorem}\label{invoCons}
Let $R_q$ be a PDE with involutive symbol $M_q$ then it holds that:
\begin{itemize}
    \item $M_{q+1}$ is involutive too.
    \item $(R_{q}^{(1)})_{+1} = R_{q+1}^{(1)}$ .
\end{itemize}
\end{theorem}
\begin{proof}
Can be found in \cite{seiler1994analysis}.
\end{proof}
It is essential to understand the second implication of the involution of $M_q$. The left-hand side of the equality is obtained by taking the prolonged and projected equation $R_q^{(1)}$ and prolonging it to order $q+1$. In particular, any integrability conditions that is obtained during the projection is hence also prolonged. The right-hand side is given by taking the prolonged equation $R_{q+1}$, prolonging it to order $q+2$, and then projecting it down to order $q+1$. Therefore the equality states that for the case of PDEs with involutive symbol all integrability conditions that might be found during the projection from the second prolongation with order $q+2$ to order $q+1$ are actually prolongations of integrability conditions that were obtained one order before, during the prolongation to order $q+1$.
Using these two implications of the involution of $M_q$ we can proof the following statement.
\begin{theorem}
A PDE $R_q$ is involutive according to definition (\ref{invol}) if its symbol $M_q$ is involutive and performing one prolongation and projection reveals no additional integrability conditions: $R_q^{(1)} = R_q$ .
\end{theorem}
\begin{proof}
The following proof can also be found in (\cite{seiler1994analysis}). As it is quite short we nevertheless also provide it here explicitly.
Looking at the definition of an involutive PDE we have to proof that the assumption $R_q^{(1)} = R_q$ already guarantees formal integrability, i.e., it holds for all $r \in \mathbb{N}$ that $R_{q+r}^{(1)} = R_{q+r}$. We proof the statement by induction and start with $r=0$. Here the requirement reads $R_q^{(1)}=R_q$ which is precisely the assumption of the theorem. Assuming it holds that $R_{q+r}^{(1)}=R_{q+r}$, as $M_q$ is involutive, theorem (\ref{invoCons}) states that $M_{q+r}$ is involutive too and hence using again the above theorem (\ref{invoCons}) we find that $R_{q+r+1}^{(1)}= (R_{q+r}^{(1)})_{+1}$. Using the induction hypotheses $(i.h.)$ we thus have in total:
\begin{align}
   R_{q+r+1}^{(1)} \overset{(\ref{invoCons})}{=} (R_{q+r}^{(1)})_{+1} \overset{(i.h.)}{=} (R_{q+r})_{+1} = R_{q+r+1} 
\end{align}
which completes the proof. 
\end{proof}
Hence to proof involution of a given PDE, which in  particular compromises formal integrability, we only have to proof the involution of the symbol ---which only encompasses standard linear algebra--- and then check whether one further prolongation and projection reveals integrability conditions. Therefore in concrete applications we proceed as follows: We first compute the symbol $M_q$ and its prolongation to check whether $M_q$ meets the requirement: 
\begin{align}
        \sum_{k=1}^m k\beta_q(k) = \mathrm{rank}(M_{q+1}).
\end{align}
If it does, we proceed by computing the prolonged PDE $R_{q+1}$ to investigate whether or not one further prolongation reveals additional integrability conditions. This can be achieved by comparing dimensions:
\begin{align}\label{dims}
    \mathrm{dim}(R_q^{(1)}) = \mathrm{dim}(R_{q+1}) - \mathrm{dim}(M_{q+1}) \stackrel{?}{=} \mathrm{dim}(R_q).
\end{align}
If this requirement is also met the PDE is involutive and hence in particular formally integrable. We are then in a position where we can apply perturbative techniques like the construction of power series solutions to solve the given PDE order by order without running into trouble due to the generation of integrability conditions.

\begin{example}
We consider again the PDE from Example (\ref{ExamplePDE}):
\begin{align}
    R_1 : \begin{cases} u_z + y \cdot u_x &= 0\\
                        u_y &= 0.
            \end{cases}
\end{align}
With prolonged system being given as:
\begin{align}
    R_2 : \begin{cases} u_z + y \cdot u_x &= 0 \\
                        u_y &= 0. \\
                        u_{xz} + y \cdot u_{xx} &= 0 \\
                        u_x + u_{yz} + y \cdot u_{xy} &= 0 \\
                        u_{zz} + y \cdot u_{xz} &= 0 \\
                        u_{xy} &= 0 \\
                        u_{yy} &= 0 \\
                        u_{yz} &= 0 
            \end{cases}
\end{align}

We have already shown that the symbol is involutive. 
The rank of the prolonged symbol $M_2$ was computed to be $5$. As there exist $6$ independent $2$nd-order derivatives, the dimension of its solution space is given by:
\begin{align}
    \mathrm{dim}(M_2) = 1.
\end{align}
The prolonged PDE $R_2$ is described by $8$ independent equations for $10$ coordinate functions $u,u_x,...,u_{zz}$ and hence has dimension:
\begin{align}
    \mathrm{dim}(R_2) = 2.
\end{align}
Thus, by using the formula (\ref{dims}), we find that:
\begin{align}
    \mathrm{dim}(R_1^{(1)}) = 2 - 1 = 1,
\end{align}
i.e., there exists one integrability condition and the PDE is not involutive. This is consistent with the previous treatment (\ref{ExamplePDE}).
\end{example}


If either one of the above conditions is not met, one needs to work somewhat harder. It can be proven that any given PDE can be completed to an equivalent involutive PDE that has the same solution space by a finite number of prolongations and projections. This fact is famously known as \textbf{\textit{Cartan-Kuranishi Theorem}}. A proof of it can be found in \cite{sweeney1968}. Further details are included in \cite{seiler2009involution} and \cite{seiler1994analysis}.

In this thesis, we will never have to complete a PDE to involution. This is a considerable advantage as it will turn out that the PDE (\ref{DiffeoEqn}) encoded the diffeomorphism invariance of a given Lagrangian is rather sizable, and therefore any concrete computation involving it poses a real technical problem not to mention the technicalities arising if we had to work with its prolongation. 

Before we proceed with a more in-depth investigation of the PDE (\ref{DiffeoEqn}) from the formal PDE theoretic point of view, we would like to return to the priorly mentioned fact that the extensive use of linear algebra in formal PDE theory allows one to solve many of the associated problems utilizing computer algebra systems. There already exist several implementations that were developed for precisely that task. One such can be found in the last chapters of \cite{seiler1994analysis} although it seems like there is not much recent contribution to this project. Another such tool consists of the Maple package Janet (see \cite{Janet2} and \cite{Janet}) that implements similar techniques stemming mainly from Janet-Riquier theory in the computer algebra system Maple. 
Although for our case the treatment of the PDEs that will arise once we consider concrete examples of our developed framework in chapter 4 employing computer algebra will be inevitable, we won't use either of the implementations presented above. The reason for that is the fact that the sheer dimension of the systems that we will encounter forces us to use computer algebra that is more specialized towards efficient memory usage. To that end, the computer algebra that we used for these examples was newly developed to a large extent. Details can be found in chapter \ref{computerAlg}.    

\section{Causal Compatibility of Matter and Gravitational Dynamics}
In the previous developments of techniques for the construction of power series solutions to a given PDE, the essential information was contained in the highest derivative part of the PDE, its symbol. 
It further turns out that this highest derivative part of the PDE also contains the entire information regarding the causal structure of the PDE. To that end, we introduce a second way of collecting the information of the highest-derivative-order contributions of a PDE, the so-called \textbf{\textit{Principal Symbol}}.
\begin{definition}[Principal Symbol] \label{PSym}
Let $ k \in T^{\ast}M$ be a  1-form, in holonomic coordinates on $T^{\ast}M$ we write: $k_{a} \mathrm{d}x^a$. The Principal Symbol to a given $q$th order PDE $R_q$ that is described as kernel of the map $\Phi^{\tilde{A}}$, is the matrix:
\begin{align}
    T^{\tilde{A} B}(k_a) = \left ( \frac{\partial \Phi^{\tilde{A}}}{\partial v_{BI_q}} \right ) J_{I_q}^{i_1...i_q} k_{i_1} \cdot ... \cdot k_{i_q}.
\end{align}
\end{definition}
\begin{remark}
Note that it is essential to complete a given PDE to its equivalent involutive form before one can obtain its Principal Symbol in a meaningful way. The reason for that lies in the fact that lower derivative order terms that therefore a priory are not present in the Principal Symbol might nevertheless generate integrability conditions that enter in the highest derivative order. Obviously, the completion to involution depends on the given PDE. Hence in the following treatment, we will assume that all involved PDEs are already involutive.
\end{remark}
The principal symbol does not only depend on the 1-form $k$ but of course also defines a map on $J^qF$. Thus we should more appropriately write:
\begin{align}
    T^{A\tilde{B}} = T^{A\tilde{B}}\underbrace{(k_a,x^m,v_A,...,v_{AI_q})}_{T^{\ast}M \times J^qF},
\end{align}
i.e., understand the principal symbol as a matrix-valued map on $T^{\ast}M \times J^qF$. 
For notational simplicity we will however briefly write $T^{A\tilde{B}}(k_a)$ most of the time, as this is the crucial dependency of the principal symbol during all further considerations.

The entries of the Principal Symbol are homogeneous polynomials in the components of the 1-form $k_a$.
The connection of the Principal Symbol to the causal structure of a given PDE can be illustrated best by considering \textit{\textbf{wave like}} solutions to the PDE in the \textit{\textbf{infinite frequency}} limit (cf. \cite{2012arXiv1211.1914K}, \cite{2011PhRvD..83d4047R} and \cite{2018PhRvD..97h4036D}). 
The infinite frequency limit $\lambda \rightarrow 0 $ is sometimes also called \textit{\textbf{geometric optical limit}}. In terms of standard electrodynamics on a possibly curved background, it treats light propagation by approximating light waves as geometrical rays. Also, in the context of any other PDE at hand, this limit can be interpreted as describing the ray approximation to the propagation of wave-like solutions.
More precisely, we consider the following formal WKB\footnote{Wentzel-Kramers-Brillouin.} expansion as an ansatz for solutions of the PDE $R_q$:
\begin{align}\label{waveAns}
    G_A(x^m) = \mathrm{Re}\left \{ e^{\frac{iS(x^m)}{\lambda}} \cdot   \bigl [ a_A(x^m) + \mathcal{O}(\lambda) \bigr ]\right \},
\end{align}
Plugging this ansatz into the PDE and taking the infinite frequency limit $\lambda \rightarrow 0$ one obtains in leading order, i.e., in $\lambda^{-q}$th order the following equation
\begin{align}
    \underbrace{\left ( \frac{\partial \Phi^{\tilde{A}}}{\partial v_{BI_q}} \right ) J_{I_q}^{i_1...i_q} k_{i_1} \cdot ... \cdot k_{i_q}}_{T^{\tilde{A} B}(k_a)} a_B(x^m) = 0,
\end{align}
where now $k_a = - \partial_aS(x^m)$ is the wave covector\footnote{The wave covector $k_a(x^m)$ is the conormal of the wave front at $x^m$.} of the ansatz, and thus encodes the direction the wave travels. Hence if the ansatz shall provide a solution to the PDE in the desired limit, it, in particular, has to solve 
\begin{align}\label{solvabilityCond}
    T^{\tilde{A} B}(k_a) a_B(x^m) = 0
\end{align} 
Therefore, if we want to obtain non-trivial solutions with $a_A(x^m) \neq 0$ the Principal Symbol
$T^{\tilde{A} B}(k_a)$, with $k_a$ being the components of the wave covector of the ansatz, must be non-injective.

Non-injectivity of a general, non-square matrix can be expressed by the vanishing of several of its sub determinants. As we will restrict attention to PDEs that are obtained as Euler-Lagrange equations of a given Lagrangian, we will, however, have the special situation where the Principal Symbol matrix is square. We thus can write: 
\begin{align}
T^{A B}(k_a) =  \left ( \frac{\partial E^{A}}{\partial v_{BI_q}} \right ) J_{I_q}^{i_1...i_q} k_{i_1} \cdot ... \cdot k_{i_q}.
\end{align}
In the simplest case, the non-injectivity can now be imposed as the vanishing of the determinant of this square matrix.

There is, nonetheless, a slight obstruction to this. If the Lagrangian field theory at hand satisfies gauge symmetries then $T^A_B(k_a)$ is necessarily non-injective, irrespective of the particular 1-form $k_a$. The reason for this is the following: In the infinite frequency limit the presence of an $s$-dimensional gauge symmetry establishes itself by the fact that given a solution of (\ref{solvabilityCond}), we immediately get $s$ additional, independent functions $\tilde{a}^{(i)}_A(x^m)$ that are connected to $a_A(x^m)$ via the $s$ independent gauge transformations, i.e. that lie on the same gauge-orbit as $a_A(x^m)$. We call two functions that are related by a gauge transformation gauge-equivalent. As the field theory was assumed to be gauge invariant and $a_A(x^m)$ was assumed to be a solution to (\ref{solvabilityCond}), the $s$ additional functions necessarily also solve (\ref{solvabilityCond}). In particular there exist $s$ independent functions that are gauge-equivalent to the trivial solution $a_A(x^m) = 0$.    
These functions thus lie in the kernel of the principal symbol matrix, independent of the specific 1-form $k_a$.
Hence whenever the field theory features a $s$-dimensional gauge symmetry, the principal symbol of the corresponding EOM has at least $s$-dimensional kernel (cf. \cite{2018PhRvD..97h4036D}). 

Returning to the relevant example of $E^A$ describing the second derivative order EOM of a diffeomorphism invariant theory of gravity, we find that the Principal Symbol takes the following form:
\begin{align}
    T^{A B} (k_a) = \left (\frac{\partial E^A}{\partial v_{BI}} \right )J_I^{pq} k_p k_q = E^{A: BI} J_I^{pq} k_p k_q.
\end{align}
Taking now a closer look at the PDE (\ref{DiffeoEqn}) that such EOM then necessarily  satisfy we find from the last such equation:
\begin{align}\label{symbolDef}
\begin{aligned}
    T^{A B} (k_a) C_{An}^{Cm}v_Ck_m &= 0 \\
    T^{B A} (k_a) C_{An}^{Cm}v_Ck_m &= 0 .
\end{aligned}
\end{align}
Hence, due to the diffeomorphism symmetry, there exist four independent vectors:
\begin{align}
   \chi_{An}(k_a) =  C_{An}^{Cm}v_Ck_m
\end{align} that are in the kernel of the Principal Symbol and $T^{AB}(k)$ is not injective but as a 4-dimensional rank defect\footnote{Note that this is closely related to the previous four constraints that any diffeomorphism invariant field theory necessarily contains amongst its EOM (see chapter \ref{ConstrainedDyn}).}. To compensate for that, we have to require the 1-form $k_a$ to be given s.t. $T^{AB}(k_a)$ is not only non-injective but has at least a 5-dimensional kernel. Then there necessarily exists at least a one dimensional subspace in this kernel of $T^{AB}(k_a)$ that is not gauge-equivalent to the trivial solution $a_A(x^m) = 0$. Hence, $k_a$ represents the wave covector of a physically non-trivial wave solution of the PDE (at least in the $\lambda \rightarrow 0$ limit).

The condition of having at least a $5$ dimensional kernel of the principal symbol can be expressed more formally by requiring that its adjunct matrix of order four vanishes. Given the $n \times n$ matrix $T^{AB}(k_a)$, its adjunct matrix of order four is the $\binom{n}{4} \times \binom{n}{4}$ square matrix that has entries equal to the order four sub determinants of $T^{AB}(k_a)$:
\begin{align}\label{MinorDef}
    Q_{(A_1...A_4) (B_1...B_4)}(k_a) := \frac{\partial^4 (\mathrm{det}(T^{AB}(k_a)))}{\partial T^{A_1 B_1}(k_a) ... \partial T^{A_4 B_4}(k_a)}.
\end{align}
Consequently, we index an entry of the order four adjunct matrix $Q_{(A_1...A_4) (B_1...B_4)}(k_a)$ by a symmetric $4$-tuples of rows $(A_1...A_4)$ and columns $(B_1...B_4)$ that encode the rows and columns that are removed from the matrix $T^{AB}(k_a)$ to obtain a certain $(n-4) \times (n-4)$ submatrix of it. The value of $Q_{(A_1...A_4) (B_1...B_4)}(k_a)$ is then given by the determinant of this submatrix. 

One can show (\cite{2018PhRvD..97h4036D} and \cite{2009JPhA...42U5402I}) that the requirement of a vanishing order four adjunct matrix\footnote{Obviously one can obtain similar results for other gauge symmetries \cite{2018PhRvD..97h4036D}. The only difference then lies in the particular expression for the null vectors and the dimension of the rank defect in the symbol.} boils down to the vanishing of a certain homogeneous polynomial in the components $k_a$. More rigorously one obtains the following general expression for the entries of the order four adjunct matrix of the Principal Symbol that holds for any diffeomorphism invariant field theory at hand:
\begin{align}\label{diffeoMinor}
    Q_{(A_1...A_4) (B_1...B_4)}(k_a) = \epsilon^{i_1...i_4} \epsilon^{j_1...j_4} \chi_{A_1i_1}(k_a) \cdot ... \cdot \chi_{A_4i_4}(k_a) \cdot \chi_{B_1j_1}(k_a) \cdot ... \cdot \chi_{B_4j_4}(k_a) \cdot \mathcal{P}(k_a),
\end{align}

Hence the non trivial zeros of the adjunct matrix are completely determined by the zeros of $\mathcal{P}(k_a)$. We call this homogeneous degree $2n-16$ polynomial in the components $k_a$ the \textit{\textbf{Principal Polynomial}} $\mathcal{P}(k_a)$ of the EOM.
This explicit form of the entries of the adjunct matrix of order four is also vital for concretely computing the Principal Polynomial. We simply have to find one single $(n-4) \times (n-4)$ submatrix of the Principal Symbol with non-vanishing determinant by removing four rows $(A_1...A_4)$ and four columns $(B_1...B_4)$. From the choice of removed rows and columns, we can compute the occurring prefactor expression 
\begin{align}\label{prefacF}
f_{(A_1...A_4)(B_1...B_4)}(k_a) := \epsilon^{i_1...i_4} \epsilon^{j_1...j_4} \chi_{A_1i_1}(k_a) \cdot ... \cdot \chi_{A_4i_4}(k_a) \cdot \chi_{B_1j_1}(k_a) \cdot ... \cdot \chi_{B_4j_4}(k_a).
\end{align}
Then we compute the determinant of the chosen submatrix, $Q_{(A_1...A_4)(B_1...B_4)}(k_a)$. The Principal Polynomial can now be obtained by dividing the determinant by the computed prefactor:
\begin{align}
    \mathcal{P}(k_a) = \frac{Q_{(A_1...A_4)(B_1...B_4)}(k_a)}{f_{(A_1...A_4)(B_1...B_4)}(k_a)}.
\end{align}
In particular, the only technical complexity in this computation is provided by calculating one order four subdeterminant of the Principal Symbol. 

Note that if the theory at hand incorporates different gauge symmetries than the discussed diffeomorphism invariance one can obtain similar formulae with the particular order of the appropriate submatrices and the expression for the prefactor obviously depending on the specific form of the gauge symmetry.
\begin{definition}[Principal Polynomial of EOM]
Given a field bundle $(F,\pi_F,M)$ with $n$-dimensional fibers and a Lagrangian on $F$ that generates EOM with derivative order $q$, let  $T^{AB}(k_a)$ be the Principal Symbol of the EOM. Assume further that the $s$ vectors $(\chi_{A\sigma}(k_a))_{\sigma=1,...,s}$ generate the $s$ independent gauge transformations. The Principal Polynomial $\mathcal{P}(k_a)$
is the homogeneous degree $q\cdot n - (q+2)s$ polynomial in the components $k_a$:
\begin{align}
\mathcal{P}(k_a) := \frac{Q_{(A_1...A_s)(B_1...B_s)}(k_a)}{\epsilon^{\sigma_1...\sigma_s} \epsilon^{\tau_1...\tau_s} \chi_{A_1\sigma_1}(k_a)\cdot ... \cdot \chi_{A_s\sigma_s}(k_a) \cdot \chi_{B_1\tau_1}(k_a) \cdot ... \cdot \chi_{B_s\tau_s}(k_a)},
\end{align}
where $Q_{(A_1...A_s)(B_1...B_s)}(k_a)$ is any specific component of the order $s$ adjunct matrix of $T^{AB}(k_a)$.
\end{definition}
Hence, for a wave ansatz of the form (\ref{waveAns}) to provide a non trivial solution to the given PDE, the corresponding gradients $k_a = - \partial_aS(x^m)$ must be zeros of the Principal Polynomial, i.e., yield $\mathcal{P}(k_a) = 0$.
Such 1-forms are called \textit{\textbf{characteristic}} 1-forms. We denote the set of 1-forms that are characteristic in $p \in M$, i.e., the $k_a \in T_p^{\ast}M$, for which the principal  polynomial $\mathcal{P}(k_a)$ vanishes by $V_p(\mathcal{P})$. Returning to our wave ansatz, we now see that exactly those $k_a \in V_p(\mathcal{P})$ are admissible wave covectors for a non-trivial solution of the given PDE. Hence the Principal Polynomial encodes the complete information regarding the propagation of such wave-like solutions in the geometric optical limit.

In addition to encoding information about the propagation of wave-like solutions of infinite frequency, the Principal Polynomial also allows one to decide whether or not a given PDE is \textit{\textbf{predictive}}, in the sense that the PDE allows one to specify initial data on a suitable hypersurface of $M$ and then use the PDE to uniquely predict the values of the involved fields away from the initial data hypersurface. We follow along the lines of the second chapter of \cite{Rivera}.
\begin{definition}[Cauchy problem]
We call the Cauchy problem to a given PDE well-posed if specifying suitable initial data on a suitable initial data hypersurface yields a unique solution of the PDE that satisfies the initial data and depends continuously on it.  
\end{definition}
As explained in \cite{Rivera}, this should really be a fundamental requirement fulfilled by any meaningful physical field theory. Roughly speaking, it states that the EOM of the given theory can be used to evolve collected, i.e., measured data in order to predict future values of the appropriate fields.
Before we can finally explore the implications of a well-posed initial value problem on the corresponding Principal Polynomial, we quickly recall the following definition of a hyperbolic polynomial. For further information regarding hyperbolic polynomials see also (\cite{2012arXiv1212.6696K} and \cite{Bauschke98hyperbolicpolynomials}).
\begin{definition}[hyperbolic polynomial]
Let $\mathbb{R}[x]$ denote the ring of polynomials in the variables $(x^1,...,x^d)$ with real coefficients.
A homogeneous degree $d$ polynomial $P \in \mathbb{R}[x]$ is called hyperbolic w.r.t. $h\in \mathbb{R}^d$ with $P(h) \neq 0$ if for all $t\in \mathbb{R}^d$, the univariat polynomial
$P(t + \lambda h) \in \mathbb{R}[\lambda]$ has only real roots, i.e.,
the equation $P(t + \lambda h)=0$ only has real solutions $\lambda \in \mathbb{R}$.
\end{definition}
If a given polynomial $P \in \mathbb{R}[x]$ is hyperbolic and hence we find such $h\in \mathbb{R}^d$ one actually readily obtains further $\tilde{h}\in \mathbb{R}^d$, such that $P$ is hyperbolic w.r.t. those $\tilde{h}$, as well. In fact one can show (see for instance \cite{Rivera} and \cite{10.2307/24900665}) that $P$ is then also hyperbolic w.r.t any: 
\begin{align}
    C(P,h) := \{ \tilde{h} \in \mathbb{R}^d \ \vert \ P(\tilde{h}- \lambda h) = 0 \implies \lambda > 0\}.
\end{align}
We call $C(P,h)$ the \textbf{\textit{hyperbolicity cone}} of $P$, containing $h$. One can also show that $C(P,h)$ indeed constitutes an open and convex cone \cite{10.2307/24900665}.

The set of hyperbolic directions $h\in \mathbb{R}^d$ of a given hyperbolic polynomial $P$ might very well be composed of several connected components. For instance, one can readily show that whenever $P$ is hyperbolic w.r.t. $h\in \mathbb{R}^d$ it necessarily is also hyperbolic w.r.t. $-h$. For further developments, we will thus select one connected component of the set of such hyperbolic directions, and subsequently call it the hyperbolicity cone of $P$: $C(P)$. We will later comment on the deeper physical meaning that such a choice of hyperbolicity cone provides.

The notion of hyperbolic polynomials and the corresponding hyperbolicity cones is best illustrated by a picture (Figure \ref{hyperbol}). 
\begin{figure}[hbt!]
    \centering
    \includegraphics[width=\textwidth]{Poly.pdf}
    \caption{Hyperbolicity Cone and Vanishing Set of a Second and a Fourth Degree Polynomial (see \cite{Rivera}).}
    \label{hyperbol}
\end{figure}
One can now see the underlying geometric interpretation of  hyperbolic directions  $h\in \mathbb{R}^d$ for a given polynomial $P$. If $P$ is hyperbolic w.r.t. $h$ than any affine line in direction $h$ intersects $\mathrm{deg}(P) = d$ times with the vanishing set $V(P)$.

Finally, we can state the connection between hyperbolic polynomials and the well-posedness of the Cauchy problem.
\begin{theorem}
If the Cauchy-Problem of a given PDE is well-posed in a region of $M$, then the Principal Polynomial necessarily restricts to a hyperbolic polynomial on $T_p^{\ast}M$ for every $p$ contained in that region. Furthermore, exactly those hypersurfaces that have at every point a conormal which is hyperbolic w.r.t. $\mathcal{P}$ are admissible \textit{\textbf{initial data hypersurfaces}}, i.e., serve the purpose of specifying initial data.
\end{theorem}
\begin{proof}
The proof can be found in \cite{Hormander1977}\footnote{Much information is also contained in Hörmander's book series on partial differential equations (\cite{hormander1994analysis}, \cite{hormander2004analysis}, \cite{hormander2009analysis}, and \cite{hormander2015analysis}).} and also in \cite{Ivrii_1974}. 
\end{proof}
In the following, we proceed as before and select at each spacetime point a connected component of the set of hyperbolic covectors that is encoded by $\mathcal{P}$. We denote the thus provided hyperbolicity cone of the principal polynomial $\mathcal{P}$ at $p \in M$ as $C_p(\mathcal{P}) \subset T_p^{\ast}M$.
Moreover, we require that the choice of hyperbolicity cone varies smoothly w.r.t. the individual spacetime points, i.e., it is made in such a way that the following map:
\begin{align}
\begin{aligned}
\pi_{\mathcal{P}}: \{ C_p(\mathcal{P}) \ \vert \ p \in M\}=: C(\mathcal{P}) &\longrightarrow M\\
C_p(\mathcal{P}) &\longmapsto p,
\end{aligned}
\end{align}
defines a smooth subbundle $(C(\mathcal{P}), \pi_{\mathcal{P}}, M) \subset T^{\ast}M$.
Choosing such a hyperbolicity cone at each spacetime point distinguishes a specific, connected subset of possible initial data hypersurface conormals. These then can be used to define a smooth distribution of cones on the tangent bundle $C^{\#} \subset TM$. Details how is construction can be carried out are provided in \cite{Rivera} and also \cite{2012arXiv1211.1914K} for a slightly different approach.
The distribution of tangent space cones then encodes feasible future directions at the particular spacetime points.
Thus, in conclusion,  the choice of hyperbolicity cones endows the spacetime manifold $M$ with a \textit{\textbf{time orientation}} (cf. \cite{2012arXiv1211.1914K} and \cite{Rivera}).

In total, we can sum up the situation as follows: If we are given a predictive PDE ---in the sense that there exist hypersurfaces such that the corresponding Cauchy problem is well-posed--- then we can calculate the Principal Polynomial of this PDE. This Principal Polynomial is then necessarily hyperbolic. As shown above, the zero variety of the Principal Polynomial describes the propagation of waves in the infinite frequency limit. Furthermore, from the Principal Polynomial, we can compute the hyperbolicity cones. These then precisely encode the admissible initial data hypersurface of the given PDE.

The above consideration immediately opens up a problem that we are going to outline in the following. In the first section, we developed the necessary techniques that allowed us to phrase the requirement of diffeomorphism invariance in terms of a linear, first-order PDE. Given a particular spacetime geometry, i.e., given a field bundle $F_{grav}$, computing solutions to the corresponding equivariance equation (\ref{DiffeoEqn}) would yield a diffeomorphism invariant theory of gravity described by a Lagrangian for the gravitational field and the corresponding gravitational EOM.
If the theory of gravity is additionally required to be predictive, then the corresponding gravitational principal polynomial $\mathcal{P}_{grav}$ is hyperbolic.

On the other hand, in order to really put that theory to use and derive predictions meaningful predictions from it, that in particular can be probed in experiments, we additionally need a theoretical description of the matter that couples to that particular gravitational field and foremost also generates it.
This additional matter theory is then prescribed on top of the geometric background that is provided by the gravitational field. If we also require matter to be governed by predictive EOM than also the matter principal polynomial $\mathcal{P}_{mat}$ is necessarily hyperbolic.

More precisely, the situation now looks as follows:
As before, gravitational Lagrangian is given by a bundle map:
\begin{align}
    \mathcal{L}_{grav} : J^2F_{grav} \longrightarrow \Lambda^4M
\end{align}
Where $F_{grav}$ is the gravitational field bundle. Additionally we now also have a matter field described by a Lagrangian that depends on the values of the matter field and its first derivatives\footnote{We might also here consider the case where the Lagrangian also depends on second derivatives of the matter field and is thus required to be degenerate s.t. the matter EOM are again of second derivative order, but for simplicity, we restrict to first-order matter Lagrangians.}, but also exploits the spacetime geometry provided by the gravitational field. Thus the matter Lagrangian is a bundle map
\begin{align}
    \mathcal{L}_{mat} : F_{grav} \times J^1F_{mat} \longrightarrow \Lambda^4M,
\end{align}
where $F_{mat}$ is the matter field bundle. In the following we will denote adapted coordinates on $J^1F_{mat}$ by $(x^m,\phi^{\tilde{A}},\phi^{\tilde{A}}_m)$ Hence the total Lagrangian is given by:
\begin{align}
\begin{aligned}
    \mathcal{L}_{tot} : J^2F_{grav} \times J^1F_{mat} &\longrightarrow \Lambda^4M \\
    \mathcal{L}_{tot} &= \mathcal{L}_{grav} + \mathcal{L}_{mat}.
\end{aligned}
\end{align}
We now get two sets of EOM one from taking the variational derivative of $\mathcal{L}_{tot}$ w.r.t the gravitational field bundle coordinates, i.e., the gravitational EOM and one from taking the variational derivative w.r.t. the matter field bundle coordinates $\phi^{\tilde{A}}$, the matter EOM.
The gravitational EOM now feature an inhomogeneous term that describes how the matter sources the gravitational field.
\begin{align}
    0 = \frac{\delta \mathcal{L}_{tot}}{\delta v_A} = \frac{\delta \mathcal{L}_{grav}}{\delta v_A} + \frac{\delta \mathcal{L}_{mat}}{\delta v_A}.
\end{align}
The matter EOM is given by 
\begin{align}
    0 = \frac{\delta \mathcal{L}_{mat}}{\delta \phi^{\tilde{A}}}.
\end{align}
We can then compute the two Principal Polynomials that correspond to the two EOMs, $\mathcal{P}_{grav}$ and $\mathcal{P}_{mat}$. Note in particular that these will depend both on the values of the gravitational field. We denote the corresponding vanishing sets by $V_{p,grav}$ and $V_{p,mat}$. As we required both theories to be predictive, i.e., have well-posed Cauchy problems we can also compute the two hyperbolicity cones $C_{p,grav}$ and $C_{p,mat}$ that at each spacetime point $p\in M$ encode the possible choices of initial data hypersurfaces for the two theories. 

Doing this at given $p \in M$ we would now end up with one of the following three situations of which we will show in that only one can serve the purpose of describing a meaningful physical theory:
\begin{itemize}
    \item $C_{p,grav} \neq C_{p,mat}$ and $V_{p,grav} \neq V_{p,mat}$
    
This situation obviously incorporates the special case where the two hyperbolicity cones are disjoint. This case can be immediately ruled out as then there would not exist a single initial data hypersurface that is common to both theories and hence could serve as a starting point for solving the coupled matter gravity system.   

Also, if the two hyperbolicity cones are not disjoint but nevertheless do not coincide we immediately get problems. Then we would either find a suitable matter initial data hypersurface that is no admissible initial data hypersurface for the gravitational EOM, or vice versa we would find a gravitational initial data hypersurface that can not serve the purpose of specifying initial data for the matter EOM. As can be seen, for instance in \cite{Rivera}, and also in \cite{2011PhRvD..83d4047R} the hyperbolicity cones of a given EOM are in close relation with feasible observer definitions for the underlying theory. Any physically meaningful observer must be able to collect data in his spatial surroundings and use the EOM provided by the theory to evolve this initial data to future values and thereby make physical predictions. To allow for this process, the spatial surrounding of any possible observer must serve as initial data hypersurface. Hence if there exist initial data hypersurfaces that are exclusive to either the matter or the gravitational EOM, we would end up with certain observers that are limited to one of the two theories. Such observers could in particular meaningfully measure data and predict processes in the corresponding free theory, being either the matter theory or gravity, but become meaningless once the coupled case is concerned.

In the following we want to restrict to situations that allow for a unified observer definition, i.e., the observers of gravity and matter theory obey the same laws and in particular posses the same properties no matter if the two theories are coupled or one considers the case of a free treatment of either one independently. Hence we dismiss the case where $C_{p,grav} \neq C_{p,mat}$.

\item $C_{p,grav} = C_{p,mat}$ and $V_{p,grav} \neq V_{p,mat}$ 

In this case, all initial data hypersurfaces are common to both theories. Nevertheless, the vanishing sets of the two Principal Polynomials differ. We have seen that these vanishing sets govern the propagation behavior of wave-like solutions in the infinite frequency limit. It might hence be possible that, if $V_{p,grav} \neq V_{p,mat}$, the wave propagation of matter waves and gravitational waves shows quite different properties. The structure provided by the vanishing sets of the two Principal Polynomials, in particular, governs the information regarding future domains that such a propagating wave might causally influence — doing so it particular encodes the speed of such waves. 

With the recent detection of gravitational waves \cite{2017ApJ...848L..12A}, \cite{2017PhRvL.119n1101A} and \cite{2016PhRvL.116f1102A} also further insight regarding their propagating speed is gained rapidly. 
To provide an example using the observed time difference between the gravitational wave event GW170817 and the gamma-ray burst GRB 170817A emitted by a Binary Neutron Star Merger the propagation speed of gravitational waves has already be constrained to deviate no more than $-3\cdot 10^{{-}15}c$ and $+7\cdot 10^{{-}16}c$ from the speed of light $c$ (see \cite{2017ApJ...848L..13A}). Hence it seems reasonable to additionally require that $V_{p,grav} = V_{p,mat}$, in order to incorporate the thus already observed similarities in the propagation of gravitational and matter waves into our framework. We are therefore left with the final option.
\item $C_{p,grav} = C_{p,mat}$ and $V_{p,grav} = V_{p,mat}$ 

This situation is the only option left and henceforth precisely what we will require in the following. Not only all possible initial value hypersurfaces and therefore all possible observers of the matter and gravity EOM then coincide, but this option also requires that the causal structure in the form of wave propagation in the infinite frequency limit is the same for the two theories.  
\end{itemize}
\begin{remark}
Note that the requirement $V_{p,grav} = V_{p,mat}$ could be loosened to the point where we only impose that $V_{p,mat} \subset V_{p,grav}$. This would still be consistent with the measured propagation speed of gravitational waves, as then, in particular, every wave covector of the matter theory serves as wave covector of gravitational waves. Further wave covectors that are exclusive to gravitational waves do not necessarily contradict observations. Such waves might have simply not been detected, yet. This could, for instance, be explained if such covectors correspond to massive modes which thus decay faster and therefore are harder to observe in large distances from the source. Nevertheless restricting to $V_{p,grav} = V_{p,mat}$ for the future developments yet to come seems reasonable.
We will even see that at least in the two perturbative examples that we are going to treat in the next chapter, the two requirements are in fact equivalent.
\end{remark}
The previous investigation yields the second and last requirement that we wish to pose on the theory of gravity that we want to construct. We require that given a matter theory that employs the theory of gravity as geometric background, the two theories are \textit{\textbf{compatible}} in their \textit{\textbf{causal structure}} in the sense that the two Principal Polynomials at each point yield the same vanishing set, i.e., it holds for all $p \in M$ that: 
\begin{align}
    V_{p,grav} = V_{p,mat}.
\end{align}
Note that then $C_{p,grav} = C_{p,mat}$ immediately follows. Further note that the above condition, the equality of the two vanishing sets, simply requires the two polynomials to be compromised of the same irreducible polynomial factors. In most relevant cases, the factorization of the polynomials can easily be obtained.


Summing up the previous achievements, it is now conceptually apparent how we must proceed to find the most general gravitational Lagrangian that is compatible with any given matter theory. We first solve the equivariance equations (\ref{DiffeoEqn}) to implement the required diffeomorphism invariance of the theory of gravity paying close attention to make sure that the resulting EOM are of no higher than second derivative order. Then we compute the gravitational Principal Polynomial and compare it to the one obtained from the matter theory. We require that at each spacetime point the two polynomials define the same vanishing set and by this get further conditions the gravitational Lagrangian has to solve, this time encoding the causal compatibility between matter theory and gravity.
We end this section by formulating this results in terms of a definite construction manual that is displayed as the Algorithm \ref{Algo1}.
\begin{algorithm}[hbt!]
\SetAlgoLined
\KwData{Matter theory: $ \mathcal{L}_{mat} : F_{grav} \times J^1F_{mat} \longrightarrow \Lambda^4M$.
}
\KwResult{Most general diffeomorphism invariant, causal compatible theory of gravity: $\mathcal{L}_{grav} : J^2F_{grav} \longrightarrow \Lambda^4M$ .}
Compute the vertical coefficients $C^{Bm}_{An}$ of the infinitesimal diffeomorphism action on $F_{grav}$. \\
Set up the equivariance equations (\ref{DiffeoEqn}). \\
Solve the equivariance equations (\ref{DiffeoEqn}) to obtain the most general diffeomorphism invariant $L_{grav}(x^m,v_A,v_{Am},v_{AI})$.\\
Compute the EOM $\frac{\delta L_{grav}}{\delta v_A}$.\\
Consider the most general subtheory that has 2nd-order EOM.\\
Calculate the Principal Polynomials $\mathcal{P}_{grav}$ and $\mathcal{P}_{mat}$.\\
Determine the arbitrary quantities in $L_{grav}$ s.t. for all $p \in M$ : $V_{p,grav} = V_{p,mat}.$
 \caption{Construction of Gravitational Lagrangian}\label{Algo1}
\end{algorithm}


\section{Perturbative Approach to Constructive Gravity}
In the previous section, we have completed the required diffeomorphism invariance of the gravitational theory with the second requirement $V_{p,grav}=V_{p,mat}$ ensuring the compatibility with a given matter theory. Conceptually, it is therefore entirely clear how one can solve the problem posed by Constructive Gravity. Practically obtaining solutions to the diffeomorphism equivariance equations (\ref{DiffeoEqn}) is, however, a different question. Although the relevant PDE is linear in the unknown Lagrangian and furthermore only of first derivative order --- methods for solving  such PDEs have actually already been known for a long time (cf. \cite{Hilbert} and also \cite{Han2015}) --- already treating the standard case of finding the most general such Lagrangian that can be constructed from a metric tensor field and its first and second derivatives is surprisingly hard. The reason is the sheer size of the resulting PDE system. For the stated example of metric gravity we will later see that the PDE system is compromised of a total of 136 partial differential equations and the Lagrangian is a function of 150 independent variables.  

Regarding this, we are essentially left with one of two options that avoid the mammoth task of solving the equivariance equations in fully general form and nevertheless furnish us with access to two exceptionally relevant realms of gravitational physics. We can either apply \textbf{\textit{symmetry}} methods to the equivariance equations and thereby for a given gravitational field obtain solutions, i.e., theories of gravity that describe the relevant phenomenology under these symmetry assumptions. We could, for instance, solve the equivariance equation assuming spatial homogeneity and isotropy to obtain a description of cosmology in the generalized spacetime geometry.
Such an approach will be discussed in \cite{NilsPHD}.
Alternatively --- and this is the path that we will take in the following --- we can perturbatively solve the equivariance equations by employing the previously developed framework for computing \textit{\textbf{power series solution}} in some finite order. We, in particular, chose this path with the treatment of gravitational waves in mind.  The recent developments in the detection of gravitational waves makes them an excellent tool to test alternatives to GR \cite{2010PhRvD..81f4008Y}, \cite{2011PhRvD..83j4022B}, \cite{2017PhRvD..95j4027Z}, \cite{2013LRR....16....9Y}.

We have already gathered the necessary techniques that allow us to construct perturbative solutions to the equivariance equations in a rigorous fashion in the previous section. Primarily we have seen that such an approach only yields meaningful results if the relevant PDE is \textit{\textbf{involutive}}. We are now going to apply these techniques to the equivariance equations (\ref{DiffeoEqn}). To that end, it is vital to take a closer look at the jet bundle that underlies the geometric treatment of the PDE provided by (\ref{DiffeoEqn}). Recall that the Lagrangian in consideration is given by a bundle map:
\begin{align}
\mathcal{L} : J^2F \longrightarrow \Gamma^4M.
\end{align}
Abstractly the PDE (\ref{DiffeoEqn}) is then defined as a submanifold of the first-order jet bundle $J^1(J^2F \times \Lambda^4M)$ over the trivial bundle $J^2F \times \Lambda^4M$. 
We denote adapted coordinates on $J^1(J^2F \times \Lambda^4M)$ by $(x^m,v_A,v_{Am},v_{AI},l,l^{m},l^{A},l^{Am},l^{AI})$ such that we obtain the formal representation of the PDE (\ref{DiffeoEqn}) by replacing the bundle map $L$ the derivatives $L^{:m},L^{:A},...$ with fiber coordinates $l$ and derivative coordinates $l^m,l^A,...$. Hence in terms of formal PDE theory the equivariance equation reads 
\begin{align}\label{DiffeoEqnFormal}
\begin{aligned}
    0 &= l^{m} \\
    0 &= l^{A} C_{An}^{Bm} v_B + l^{Ap} \bigl[ C_{An}^{Bm} \delta_p^q - \delta_A^B \delta_m^n \bigr] v_{Bq} + l^{AI} \bigl[ C_{An}^{Bm} \delta_I^J - 2 \delta_A^B J_I^{pm} I^J_{pn}  \bigr] v_{BJ} + l \delta^m_n \\
    0 &= l^{A(p\vert}C_{An}^{B \vert m)} v_B + l^{ AI} \bigl[ C_{An}^{B(m\vert} 2 J_I^{\vert p) q} - \delta^B_A J_I ^{pm} \delta_n^q \bigr] v_{Bq} \\
    0 &= l^{AI} C_{An}^{B(m\vert} v_B J_I^{\vert p q )}.
    \end{aligned}
\end{align}

We begin the perturbative treatment of Constructive Gravity with one of the main results that in the end justifies the perturbative approach:
\begin{theorem}
PDE (\ref{DiffeoEqnFormal}) is involutive.
\end{theorem}
\begin{proof}
We start by proofing that the symbol of the equation (\ref{DiffeoEqnFormal}) is involutive. Since the PDE is of first derivative order its symbol simply consists of its homogeneous part, namely the PDE with the second equation being replaced by the homogeneous counterpart:
\begin{align}
    0 &= l^{A} C_{An}^{Bm} v_B + l^{Ap} \bigl[ C_{An}^{Bm} \delta_p^q - \delta_A^B \delta_m^n \bigr] v_{Bq} + l^{AI} \bigl[ C_{An}^{Bm} \delta_I^J - 2 \delta_A^B J_I^{pm} I^J_{pn}  \bigr] v_{BJ}.
\end{align}
This homogeneous system describes functions that are invariant under the action of any lifted vector field (\ref{LieJ2}). 

The first step in proving involution of the symbol is computing the sum of beta numbers. We have to pay attention as we have to compute this sum in coordinates that maximize it, i.e., are $\delta$-regular. However, because we are dealing with first-order equations, the classes of the derivative indices all posses exactly one member and range from 1 to $\mathcal{k}$ where 
\begin{align}
    \mathcal{k} := \mathrm{dim}(J^2F) = 4+n+4n+10n,
\end{align}
$n$ being as before the fiber dimension of $F$. 
In the following, we are going to assume that $\mathrm{dim}(J^2F)$ is bigger than the number of independent equations in (\ref{DiffeoEqnFormal}) which is given by 140 as otherwise, in general, there will not exist solutions in the first place. 
The maximum value for the sum of betas is obtained when we transform the coordinates on $J^2F$ s.t. the first equation is solved w.r.t. the derivative coordinate with maximal class $\mathcal{k}$ and the $i$th equation is solved w.r.t. the derivative coordinate of class $\mathcal{k}-i$ such that the last equation is solved w.r.t. the derivative coordinate of class $\mathcal{k}-139$.
Note that this immediately renders the symbol in solved form.
To simplify the notation in the following we briefly denote coordinates on $J^2F$ that bring the symbol to this form by $y_a$ for $a = 1,...,\mathcal{k}$, where the ordering is taken as $y_{\sigma} < y_{\tau}$ for $\sigma < \tau$. Moreover we denote the corresponding derivative coordinates by $l_{\sigma}$. 
Note that such coordinates always exist. 
If we display the symbol $M_1$ as matrix and sort its columns by class starting from the highest class $\mathcal{k}$ and further label its rows by the individual equations, from 1 to 140, this transformation this coordinate transformation on $J^2F$ corresponds to bringing the matrix to the following upper triangular form
\begin{align}\label{symbolMat}
\begin{blockarray}{cccccccccc}
\underset{\longrightarrow}{\text{class}} & \mathcal{k} & \mathcal{k} -1 & \hdots & \hdots & \hdots & \mathcal{k}- 140 & \hdots & 1 & \downarrow \text{Equation} \\
\begin{block}{c(cccccccc)c}
   & 1 & a^1_{\mathcal{k}-1} & \hdots & \hdots & \hdots & a^1_{\mathcal{k} -140} & \hdots & a^1_1 & 1 \\
    & 0 & \ddots & \ddots & &  & \vdots & & \vdots & \vdots \\
    & \vdots & \ddots  & \ddots & \ddots &  & \vdots & & \vdots & \vdots \\
    & \vdots & & \ddots & \ddots & \ddots & \vdots & & \vdots &  \vdots\\
    & 0 & \hdots  & \hdots & 0 & 1 & a^{140}_{\mathcal{k}-140} & \hdots & a^{140}_{1} &  140 \\
\end{block}
\underset{\longrightarrow}{\text{coordinate}} & l_{\mathcal{k}} & l_{\mathcal{k} -1} & \hdots & \hdots & \hdots & l_{\mathcal{k}- 140} & \hdots & l_1 &
\end{blockarray}
\end{align}
for some $a^{\alpha}_{\tau}$ that are functions on $J^2F$ . The sum of betas can then be computed to be given by:
\begin{align}
    \sum_{i=1}^{\mathcal{k}} i \beta_1(i) = \sum_{i = 0}^{139} \mathcal{k} - i .
\end{align}
We now have to show that this equals the rank of the prolonged symbol $M_2$. By previous arguments, this is the case if we obtain all equations of $2$nd derivative order by prolongations w.r.t. multiplicative variables only. We again consider the symbol solved to the form given by matrix (\ref{symbolMat}). 
We take an arbitrary row $\sigma$, for $\sigma=1,...,140$, and consider the corresponding equation $E_{\sigma}$. The multiplicative variables of this $\sigma$th equation are $y_1,...,y_{\mathrm{cls}(\sigma)}$ where we introduced 
\begin{align}
\mathrm{cls}(\sigma):= \mathcal{k} - \sigma +1.
\end{align}
In other words $\mathrm{cls}$ returns the class of a given row
We now show that prolonging this equation w.r.t. an arbitrary non multiplicative variable $y_b$ for $b>\mathrm{cls}(\sigma)$ yields no additional independent contribution to the prolonged symbol. This is achieved by showing that we can add prolongations w.r.t. multiplicative variables to such a non multiplicative prolongation $D_bE_{\sigma}$ and thereby obtain an equation of first derivative order which is thus not present in $M_2$. Note that in the matrix (\ref{symbolMat}) multiplicative variables of a given row are exactly those corresponding to the columns that are right to the normalized pivot elements (including these) and thus contain the functions $a^{\alpha}_{\tau}$. Non-multiplicative variable correspond to the columns left to the pivots that hence contain only zeros.

Prolonging $E_{\sigma}$ w.r.t. such a $y_b$, i.e., computing $D_bE_{\sigma}$ we get second-order derivatives of the form $y_{bc}$ for $c < \mathrm{cls}(\sigma)$, where $y_c$ is multiplicative for $E_{\sigma}$. Considering a particular such $c$ as an example, we can now get rid of this second order derivative contribution by utilizing the equation with pivot of class $b$. As $b>c$ such an equation necessarily exists; this can easily be seen from the upper triangular structure of the solved symbol. More precisely this equation is explicitly given by $E_{\mathrm{cls}(b)}$.
Since $c<b$, $y_c$ is multiplicative for $E_{\mathrm{cls}(b)}$. We prolong $E_{\mathrm{cls}(b)}$ w.r.t. $y_c$, scale it appropriately, by the negative of the coefficient of $l_{bc}$ in the previously computed non multiplicative prolongation $D_bE_{\sigma}$ and add it to this equation. Doing so we remove the entire contribution from $y_{bc}$. Similarly, one can get rid of all further second derivative contributions contained in $D_bE_{\sigma}$. In total we therefore see that we can add multiplicative prolongations to remove the whole second derivative order of such a non multiplicative prolongation $D_bE_{\sigma}$.
Hence any non multiplicative prolongation does not contribute an independent new second-derivative-order equation to the prolonged symbol $M_2$ and the symbol. The symbol is therefore involutive. 

To proof that not only the symbol but also the PDE (\ref{DiffeoEqnFormal}) itself involutive, we have to show that the PDE generates no integrability conditions after one prolongation. We have already discussed that prolonging each equation only w.r.t. its multiplicative variables never produces integrability conditions. Furthermore, we have just shown that any prolongation w.r.t. non-multiplicative variables can be reduced to first derivative order by adding further multiplicative prolongations in the way it is outlined above. Hence the question is if the first derivative order contributions that we get from such a procedure are already included in $R_1$ or contribute additional independent equations. 

We first consider the homogeneous case. We take again the $m$th equation and work as before in the coordinates $y^a$ on $J^2F$ that solve the symbol $M_1$ to (\ref{symbolMat}). Note that this homogeneous linear first-order PDE can then be concisely written as
\begin{align}
    E_{\sigma} = l_{\mathrm{cls}(\sigma)} + \sum_{i = \sigma}^{\mathcal{k}} a^{\sigma}_{\mathrm{cls}(i)} l_{\mathrm{cls}(i) } .
\end{align}
We can equivalently write this in the form of vector fields on $J^2F$. We define for $\sigma = 1,...,140$ the vector fields $\zeta_{\sigma} \in \Gamma(J^2F) $ corresponding to $E_{\sigma}$:
\begin{align}
    \zeta_{\sigma} := \frac{\partial}{\partial y_{\mathrm{cls}(\sigma})} + \sum_{i = \sigma}^{\mathcal{k}} a^{\sigma}_{\mathrm{cls}(i)} \frac{\partial}{\partial y_{\mathrm{cls}(i)}}.
\end{align}
The solutions to the given homogeneous PDE are then precisely those functions on $J^2F$ that are invariant under these vector fields, i.e., that satisfy $\zeta_{\sigma} f = 0$ for all values of $\sigma$.
Note that we obtained the homogeneous PDE from precisely such an invariance requirement. The only difference to the previous case is that we now work coordinates that render the symbol particularly simple. 
In terms of vector fields this corresponds to the fact that the $\zeta_{\sigma}$ commute (see also \cite{seiler1994analysis}):
\begin{align}
    \left [ \zeta_{\sigma}, \zeta_{\tau}\right ] =0.
\end{align}

Carefully analyzing the above procedure, we now find that the remaining first-order equation that is produced when eliminating all second-order contributions of a non-multiplicative prolongation $D_bE_{\sigma}$ by multiplicative prolongations is simply given by the commutator equation
\begin{align}
    \left [\zeta_{\sigma}, \zeta_{\mathrm{cls}(b)} \right] = 0.
\end{align}
Hence for the system to not produce integrability conditions this commutator must now be given by a linear combination of unprolonged equations. For the particular case of the $\zeta_{\sigma}$, this is trivially true as the commutator of two such fields vanishes. This can, however, already be seen from the original system (\ref{DiffeoEqnFormal}).
Recalling that we obtained homogeneous counterpart to (\ref{DiffeoEqnFormal}) by requiring invariance under the prolonged vector fields $\xi_{J^2F}$ (\ref{LieJ2}) that were in particular constructed via a Lie algebra morphism. It is therefore clear that also these vector fields close under the operation of taking commutators. Hence no integrability conditions are generated, and the homogeneous system is involutive.

This is no special property of the homogeneous PDE in consideration but is actually general to all linear homogeneous first-order PDEs. Such PDEs that can be described entirely by vector fields are called \textit{\textbf{complete}} if the vector fields that generate the given system form a Lie algebra w.r.t. the commutator. If the vector fields commute, the system is furthermore called a \textit{\textbf{Jacobian}} system. One can show along the same lines that we followed her that in fact, any complete system is involutive. For details regarding the treatment of complete systems and in particular for a proof of this last statement see \cite{seiler1994analysis} and also for further information \cite{Clebsch1866}, \cite{caratheodory1956variationsrechnung} and \cite{lie1970theorie}.

What remains to show is that the presence of the inhomogeneity in $(\ref{DiffeoEqnFormal})$ does not change this observation, i.e., does not generate integrability conditions. We again work in the coordinates $y_a$. The inhomogeneity is only present in the equations $5-20$ and only when the indices in (\ref{DiffeoEqnFormal}) are such that $m = n$. We label the equations s.t. it contributes to equation $5,9,13$ and $17$. In each case, the contribution is simply given by an extra term of $+l$ and the respective equations $\Tilde{E}_{\sigma}$ of the inhomogeneous PDE read now 
\begin{align}
     \tilde{E}_{\sigma} = \begin{cases}
     E_{\sigma} + l \ &\text{for} \ \sigma \in \{ 5,9,13,17 \}\\
     E_{\sigma} \ &\text{else}.
     \end{cases}
\end{align}
One can now show along the same lines we followed in the homogeneous case that the inhomogeneous system is involutive if for any two such equations $\tilde{E}_{\sigma}$ and $\tilde{E}_{\tau}$ the so-called \textit{\textbf{Jacobi brackets}} or sometimes also called Mayer bracket 
\begin{align}
  \llbracket \tilde{E}_{\sigma}, \tilde{E}_{\tau} \rrbracket = \sum _{i = 1}^{\mathcal{k}} \biggl(\frac{\partial \tilde{E}_{\sigma}}{\partial l_i}\biggr) D_{i}\tilde{E}_{\tau} -  \biggl(\frac{\partial \tilde{E}_{\tau}}{\partial l_i}\biggr) D_{i}\tilde{E}_{\sigma} 
\end{align}
vanish on $R_1$ (see  \cite{seiler1994analysis} and also also Example 2.3.12 in \cite{seiler2009involution}), i.e., can be obtained by linear combinations of unprolonged equations. Comparing this to the previous case where the possible integrability conditions where built from the commutator we now get extra terms whenever either $\sigma$, $\tau$ or both are contained in $\{ 5,9,13,17 \}$. Consider first the case where $\sigma \in \{5,9,13,17\}$ but not $\tau$, then we have $\tilde{E}_{\sigma} = E_{\sigma} + l$ and $\tilde{E}_{\tau} = E_{\tau}$ and hence using the linearity of the Jacobi bracket
\begin{align}
    \llbracket \tilde{E}_{\sigma}, \tilde{E}_{\tau} \rrbracket = \llbracket E_{\sigma}, E_{\tau} \rrbracket - \sum_{i=1}^{\mathcal{k}} \frac{\partial E_{\tau}}{\partial l_i} l_i.
\end{align}
We can now use that the first contribution vanishes, $\llbracket E_{\sigma}, E_{\tau} \rrbracket = 0$ as this again essentially nothing different but the commutator of the two vector fields $[\zeta_{\sigma},\zeta_{\tau}]$ expressed in terms of the quantities $E_{\tau}$ and $E_{\sigma}$. We are thus left with the second term $- \sum_{i=1}^{\mathcal{k}} \frac{\partial E_{\tau}}{\partial l_i} l_i$, but as the equation $E_{\tau}$ is in particular linear in the $l_i$ this again yields $E_{\tau}$ and hence vanishes on $R_1$. Along the same lines one can proceed with the case of $E_{\tau}$ also containing extra contributions from $+l$. In total we thus find that also the inhomogeneous PDE is involutive which proofs the statement.
\end{proof}

We can now apply the previously developed techniques to obtain a power series solution to the PDE (\ref{DiffeoEqnFormal}). This then yields a finite order power series expansion of the requested gravitational Lagrangian of which we can, of course, compute the Euler Lagrange equations in the appropriate perturbative order. In other words, we thus get a perturbative theory of gravity. We consider the result that the PDE (\ref{DiffeoEqnFormal}) is involutive to be essential for this approach as only now as we have proven that we can be sure the contribution to the perturbative Lagrangian that we compute up to some finite order will not change in any higher-order of the power series procedure. Differently stated only with the PDE being involutive we can be sure to really extract all the for the particular order relevant information from it and not end up with a solution that is too general. 

Note that from the point of view of formal theory, involution of a PDE is not only essential for the construction of power series solution. Previously we have already stated that one of the key ideas behind a PDE with involutive symbol lies in the fact that one is then able to predict the rank of the prolonged symbol by means of (\ref{sumBeta}) without actually having to compute the relevant prolongations. As for an involutive symbol according to (\ref{invoCons}) also all prolonged symbols are involutive one can thus obtain formulae that also allow for the prediction of the ranks of higher-order prolongations of the symbol. Recall that the rank of the symbol $M_q$ essentially determines the number of expansion constants of order $q$ the one might specify arbitrarily when constructing a power series solution to the PDE  and thereby the various ranks of the symbol govern the entire information regarding the solution space of the given PDE.
Thus the involution of the PDE (\ref{DiffeoEqnFormal}) allows us to formulate the following statement about its general solution for arbitrary field bundles $F$.
\begin{theorem}\label{GeneralSol}
Given any field bundle $F$ with fiber dimension $n$. The general solution to the invariance equation, i.e., the homogeneous version of (\ref{DiffeoEqnFormal}) admits the form:
\begin{align}
    \mathcal{F} \left (\Psi_1,...,\Psi_r \right ),
\end{align}
where $r:=\mathcal{k}-140$, $\Psi_1,...\Psi_r$ are $r$ functionally independent solutions of the homogeneous PDE and $\mathcal{F}$ is an arbitrary function of these independent solutions. The general solution to the equivariance equation, i.e., the inhomogeneous PDE (\ref{DiffeoEqnFormal}) is given by:
\begin{align}
    \omega \cdot \mathcal{F} \left (\Psi_1,...,\Psi_r \right ),
\end{align}
where $\omega$ is any explicit solution of the PDE. 
\end{theorem}
\begin{proof}
The proof of the first statement is given by Proposition 7.1 in \cite{seiler1994analysis}. Further information can in particular be found in chapter 3 of \cite{seiler2009involution} and \cite{articleCH}. 

The second statement then simply follows from the previously stated fact that given an arbitrary solution $\mathcal{F}$ of the homogeneous PDE, multiplying by any solution $\omega$ of the inhomogeneous system yields again a solution of the inhomogeneous system and vice versa the quotient of any two solutions of the inhomogeneous PDE defines again a solution of the homogeneous system. The first part of this statement is simply a result of the linearity of the given PDE. For simplicity, we consider the case of a single equation 
\begin{align}
    0=D_i l + l 
\end{align}
and the corresponding homogeneous version. The generalization to finitely many equations is then really straight forward. 
We assume that we have the general solution of the homogeneous equation, i.e., $0 = D_i \mathcal{F}$ and also an explicit solution to the inhomogeneous version: $0 = D_i \omega + \omega$.
We first show that $\omega \cdot \mathcal{F}$ then also solves the inhomogeneous system. Inserting $\omega \cdot \mathcal{F}$ into the inhomogeneous PDE we get  
\begin{align}
    0 = \left ( D_i \mathcal{F} \right ) \cdot \omega + \mathcal{F} \cdot \left ( D_i \omega \right) + \mathcal{F} \cdot \omega. 
\end{align}
As $\mathcal{F}$ is supposed to provide a solution of the homogeneous equation, the first term vanishes. As $\omega$ solves the inhomogeneous equation the second term yields $\mathcal{F} \cdot \left ( - \omega \right )$,
and thus cancels the third term. We thus have shown that $\omega \cdot \mathcal{F}$ solves the inhomogeneous system. To complete the proof we still have to show that any solution of the inhomogeneous PDE is necessary of this form. We proof this by first taking an arbitrary solution of the inhomogeneous system $\rho$. We now take any other solution $\omega$ and insert the quotient $\rho/\omega$ into the homogeneous PDE
\begin{align}
    0 = D_i \left (\frac{\rho}{\omega} \right) = - (\frac{\rho}{\omega}) - \frac{\rho}{\omega^2} \cdot (-\omega) ,
\end{align}
Where we used $D_i \rho = -\rho$ and the same for $\omega$ as they both are supposed to solve the inhomogeneous system. Hence the quotient solves the homogeneous system. Obviously, we can write the solution $\rho$ as 
\begin{align}
    \rho = \omega \cdot  \frac{\rho}{\omega}.
\end{align}
However, as the first factor is taken to be a solution of the inhomogeneous system and we have now shown that the second factor solves the homogeneous PDE this is precisely the form that was claimed. 
Hence in total, we have thus shown that any solution of the inhomogeneous PDE is of the form $\omega \cdot \mathcal{F}$.
The straight forward generalization of this to the case finitely many PDEs then completes the proof.
\end{proof}
It is essential to observe that the previous theorem not only tells us the general form any solution to the equivariance or invariance equations respectively will admit but also provides us with information regarding the number of functionally independent solutions to these systems. Concretely for any field bundle, there will be 
\begin{align}
    \mathrm{dim}(J^2F) - 140 
\end{align}
functionally independent solution.
Therefore we see that the size of the space of potential candidates for a particular diffeomorphism invariant theory of gravity described in terms of a specific field bundle essentially depends only on the fibre dimension of the field bundle. 
This is obviously something that one might intuitively guess as it is merely the difference between the number of independent variables in $J^2F$ and the number of equations in the PDE. However only now that we know the equations are involutive we can be sure that there is no further information hidden in them yielding possibly additional restrictions on the unknowns and thereby reducing the number of functionally independent solutions. 
The proof of the involution of the equivariance equations is in particular not a mere technicality, but involution of this PDE could have made the difference between finitely many functionally independent solutions and not a single solution at all. 

Note that these solutions now, in general, will correspond to Lagrangians that might generate $4$th-derivative-order EOM. 
In the context of GR, these functionally independent solutions of the homogeneous equations are called \textit{\textbf{curvature invariants}}. In most cases, they are constructed from the Riemann curvature tensor. Observe, however, that also her the Riemann curvature is strictly speaking not necessary to obey such a notion of curvature invariants. 
In GR it is a well-known result that there exist 14 functionally independent curvature invariant, yet already for this rather simple case their concrete expressions are surprisingly difficult. Obtaining possible generating sets that allow one to express arbitrary other invariants in terms of them is still a topic of research. 
Moreover, note that computing a set of $14$ functionally independent curvature invariants is completely equivalent to solving the equivariance equation for the case of the metric field bundle.
In the context of GR the curvature invariants and also possibly higher invariants involving higher derivatives of the metric tensor are then mostly used to classify spacetimes. Details can be found in
\cite{2009CQGra..26b5013C}, \cite{Zakhary1997}, \cite{2002IJMPD..11..827C} and also \cite{doi:10.1063/1.531425}.

With our developed framework we can easily reproduce the known result of $14$ curvature invariants for GR. The fiber dimension of the bundle of symmetric $(0,2)$ tensors $F_{GR}$ is obviously 10 yielding for this case 
\begin{align}
    \mathrm{dim}(J^2F_{GR}) = 4 + 10 + 40 + 100 = 154,
\end{align}
which according to the above considerations, yields $154-140=14$ functionally invariant solutions.
Furthermore, we are now in a position where we can easily determine the number of curvature invariants for any other spacetime geometry, i.e., any other possible gravitational field. In order to do so we, in particular, do not need to rely on structural analogues to the Riemann curvature tensor.
When constructing particular alternative gravity theories,
such a prediction regarding the number of possible curvature invariants that a given spacetime geometry might admit yields vital information about the richness of the spacetime structure. 
Doing so we can obtain a first hint on how many undetermined quantities such a theory might feature, be it undetermined functions in a full formulation or undetermined parameters in a perturbative treatment.
Also we can then immediately compare the strength of any further condition that is posed on the theory of gravity --- in the sense of how many otherwise unknown quantities the condition  determines ---
to the fundamental requirement of diffeomorphism invariance.
We will further treat these ideas in a particular example that we will consider in the next chapter.

Before we concretely work out a general framework for computing perturbative solutions to the equivariance equations we concern ourselves for a moment with the question of what points $p_0 \in J^2F$ might serve as expansion points for such a power series. From the sheer mathematical point of view, there is really no restriction on the possible expansion points. Yet from a physical point of view, the specific expansion point will, in the end, determine the interpretation and foremost the range of validity of the thereby constructed perturbative theory of gravity. 

As we want to work towards the treatment of gravitational waves, we are going to restrict ourselves to certain expansion points that describe a \textit{\textbf{flat}} variant of the theory of gravity at hand. Solutions of the obtained perturbative equations can then be thought of as corrections to that flat background theory with the special case of gravitational waves being wave-like solutions that propagate on the given background. More precisely with a slight adaption of the usual meaning in Riemannian geometry\footnote{In Riemannian geometry one call a Riemannian manifold flat if the associated Riemannian curvature tensor vanishes everywhere. The existence of coordinates in which the derivatives of the metric are zero is then a consequence (see \cite{petersen2006riemannian}).} we call a section $G \in \Gamma(J^2F)$ \textit{\textbf{flat}} if there exist adapted coordinates on $J^2F$ s.t. the coordinate expression of $G$ satisfies $\partial_mG_{A}=0$.
Hence we chose an expansion point with adapted coordinates $(x_0^m,N_A,0,0)$. note that such a point is essentially already determined by a point in $F$. 

We are in particular interested in the case where the expansion point is not only flat but furthermore supports our intuition that the geometry of spacetime is in quite good approximation provided by the flat Minkowski metric $\eta_{ab} = \mathrm{diag}(-1,+1,+1,+1)$. 
%check signature !!!
%
%
In other words, we want to interpret solutions of the to-be-constructed perturbative theory of gravity as corrections to a flat Minkowski background. Note we are not restricting our treatment to metric theories of gravity but deliberately want to allow for more general tensor fields as spacetime geometry.
The interpretation of doing perturbation theory around Minkowski spacetime will nevertheless become meaningful once the additional matter theory $\mathcal{L}_{mat}(x^m,\phi^{\tilde{B}},\phi^{\tilde{B}}_m,G_A)$ is provided. We then require from our flat expansion point that the specific matter theory defined on this flat expansion background $\mathcal{L}_{mat}(x^m,\phi^{\tilde{B}},\phi^{\tilde{B}}_m,N_A)$ is equivalent to the counterpart that we get when placing the matter field $\phi^{\tilde{B}}$ on a Minkowski background. 
In most cases this is achieved by constructing $N_A = N_A(\eta_{ab})$ from the Minkowski metric. We call such expansion points in the following \textit{\textbf{$\boldsymbol{\eta}$-induced}}.

Note that this clearly restricts the set of field bundles $F$ we can possibly treat to those that are compromised of tensors with an even total rank, as we simply cannot write down expressions with odd rank that are solely constructed from $\eta_{ab}$. This will not affect any further developments as in all examples we will treat, the gravitational field is described by an even rank tensor field. There are even certain arguments mainly stemming from QFT that the gravitational field must be of even rank in order to describe an attractive force (see for instance \cite{vecchiato2017variational}). 

We start our development of a framework that will allow us to construct perturbative expansions of diffeomorphism invariant Lagrangians for arbitrary fields by writing down the general finite power series expansion of an arbitrary such Lagrangian up to some order $r > 0$ around a flat expansion point $p_0 \in J^2F$ with adapted coordinates $p_0 \equiv (x_0^m,N_A, 0, 0)$. Note again that such a Lagrangian is already a bundle map on the second-order jet bundle $J^2F$, the equivariance PDE (\ref{DiffeoEqnFormal}) is then constructed as a submanifold of $J^1(J^2F \times \Lambda^4M)$. It is important not to confuse the two different ways the jet bundle construction is involved therein. In order to concisely display such a power series Lagrangian, we need to agree on some further notation. We denote the adapted coordinates on $J^2F$ collectively by
\begin{align}
    (v_{AI_k}) := (x^m,v_A,v_{Ap},v_{AI}).
\end{align}
In order to distinguish the appearing derivative indices $I_k,J_k,...$ that label higher spacetime derivatives from those that label higher derivatives w.r.t. the fiber coordinates of $J^2F$ we denote the latter ones by letters
$\tilde{I}_k, \tilde{J}_k$, ... . Finally we introduce the coordinate expression of the deviation from the expansion point 
\begin{align}
    (H_{AI_k}) := (v_{AI_k}) - (N_{AI_k}) = (x^m-x_0^m,v_A-N_A,v_{Ap},v_{AI}).
\end{align}

Before we write down the concrete expression for the expansion of the Lagrangian note that the first equation in (\ref{DiffeoEqnFormal}) simply states that the Lagrangian must not explicitly depend on $x^m$, hence we can discard any explicit $x^m$ dependency from the very beginning and therefore also exclude such from the power series expansion. In total, with the introduced notation, a general power series expansion reads as follows. 
\begin{align} \label{generalPowerSL}
    \begin{aligned}
    &L_{per}  \mathrm{d}^4x = \mathcal{L}_{per} : J^2F \longrightarrow \Lambda^4M \\
    &L_{per} = \sum_{n=0}^r \sum_{k_1,...,k_n = 0}^2 a^{\tilde{I}_k} \cdot J_{\tilde{I}_k}^{A_1I_{k_1}...A_nI_{k_n}} H_{A_1I_{k_1}} \cdot ... \cdot H_{A_nI_{k_n}}
    \end{aligned}
\end{align}
%rewrite this ???
%
%
As before the $a^{\tilde{I}_k}$ are constants.
We further define a second way of displaying these constants:
\begin{align}
    a^{A_1I_{k_1}...A_nI_{k_n}} := a^{\tilde{I}_k} \cdot J_{\tilde{I}_k}^{A_1I_{k_1}...A_nI_{k_n}}
\end{align}
In order to clarify this last expression further, we now explicitly provide the first three orders of this expansion. For that particular case, we drop any numerical factors due to the intertwiners as this simply corresponds to redefining the constants. Furthermore, we split the sums over the $v_{AI_{k_i}}$ into the different contributions of $(v_A,v_{Ap},v_{AI})$:
\begin{align}\label{LPert}
\begin{aligned}
    L_{per} = \  &a_0 + a^A H_A + a^{Ap} H_{Ap} + a^{AI}H_{AI} + a^{AB} H_{A}H_{B} + a^{ABp}H_A H_{Bp} + a^{ABI} H_{A} H_{BI}\\
    &+a^{ApBI}H_{Ap} H_{BI} + a^{AIBJ} H_{AI}H_{BJ} + a^{ABC} H_a H_B H_C 
    + a^{ABCp} H_A H_B H_{Cp} \\
    &+a^{ABCI} H_A H_B H_{CI} + a^{ABpCq} H_{A}H_{BP}H_{Cq} + a^{ABpCI} H_A H_{Bp} H_{CI}\\
    &+ a^{ABICJ} H_A H_{BI}H_{CJ} 
    + a^{ApBqCm} H_{Ap} H_{Bp} H_{Cm}+ a^{ApBq CI} H_{Ap} H_{BP} H_{CI}\\
    &+ a^{Ap BI CJ} H_{Ap} H_{BI} H_{CJ} + a^{AIBJCK} H_{AI} H_{BJ} H_{CK} + \mathcal{O}(4),
\end{aligned}
\end{align}
where $\mathcal{O}(4)$ denotes terms that are of order four or higher in any $(H_{AI_k})$. We now proceed exactly as outlined before in calculating the perturbative solution to solution to (\ref{DiffeoEqn}) from the above power series ansatz of the Lagrangian. We start by plugging in the expansion (\ref{LPert}) into the PDE (\ref{DiffeoEqnFormal}) and evaluate at the expansion point $p_0$. Note that evaluating at $p_0$ in coordinates corresponds to evaluating at $(H_{AI_k})=0$. This then yields equations for the expansion constants $a_0, a^A, a^{Ap}$ and $a^{AI}$. To obtain equations for the remaining higher-order expansion coefficients, we simply prolong the PDE and again insert the series expansion and evaluate at $p_0$. As we have shown that PDE (\ref{DiffeoEqnFormal}) is involutive, there will not occur any integrability conditions.

There is, however, one obstruction to this procedure that will become noticeable by providing additional lower-order equations that are revealed only after prolongations, yet they have nothing to do with integrability conditions. 
The reason for this obstruction is that the chosen $\eta$-induced expansion point in general features higher symmetry than an arbitrary point in $J^2F$. More precisely since we required $N_A(\eta_{ab})$ to be constructed from the Minkowski metric through products, sums and possibly contractions, it is \textit{\textbf{Lorentz invariant}}. To further investigate the consequences thereof we consider first an arbitrary flat expansion point $\tilde{p}_0 \in J^2F$ with coordinates $(\tilde{x}_0^m,M_A,0,0)$. Evaluating the second equation in (\ref{DiffeoEqnFormal}) for this point yields:
\begin{align}
    0 = l^A \vert _{\tilde{p}_0} C^{BM}_{An}M_B.
\end{align}
When evaluated at a general point, these are 16 independent equations. 
Conversely evaluating the same equation at the $\eta$-induced expansion point we find that:
\begin{align}\label{RankDef}
    0 = l^A \vert_{p_0} C^{BM}_{An}N_B
\end{align}
provides only 10 independent equations. The reason for this rank defect and foremost, its connection to the observed Lorentz invariance can be observed by considering the expression $K_{m[rs]}^n :=\eta_{m[r}\delta_{s]}^n$. 
Taking a closer look at the six individual expressions that can be obtained from the six possible values of the anti symmetric index pair  $\{K_{m[rs]}^n \ \vert \ r < s \} $ we find that they all are by construction anti symmetric w.r.t. the Minkowski metric, i.e., they satisfy 
\begin{align}
K_{m[rs]}^n\eta_{n p} - m \leftrightarrow p = 0.
\end{align}
Furthermore, they are all linearly independent when identified as real $4 \times 4$ matrices. Considering the dimension of the vector space of $4 \times 4$ matrices that are anti symmetric w.r.t. the matrix expression of $\eta_{ab}$ we find that  the expressions $K_{m[rs]}^n$ even constitute a basis of the vector space of $\eta$-antisymmetric matrices. 

One can show that the Lie algebra of the Lorentz group $SO(1,3)$ is isomorphic to the Lie algebra that is obtained by equipping this vector space of $\eta$-antisymmetric matrices with the standard matrix commutator. Hence these six matrices actually yield a basis of the Lie algebra $\mathrm{Lie}(SO(1,3))$. One often calls such matrices the \textit{\textbf{generators}} of the corresponding group as finite group elements can be obtained by applying the exponential map to linear combinations of them. Much information regarding the Lorentz group can be found in \cite{doi:10.1142/p199} and \cite{naimark2014linear}.

It is now clear that it henceforth holds that
\begin{align}
    0 = C^{Bm}_{An}N_B K_{m[rs]}^n,
\end{align}
as this is nothing but the infinitesimal change of $N_A$ under Lorentz transformations, but as $N_A$ is solely constructed from $\eta_{ab}$ and hence Lorentz invariant this infinitesimal change obviously vanishes. As there exist six independent generators of the Lorentz group we can thus obtain six independent vanishing linear combinations of the 16 equations in (\ref{RankDef})
\begin{align}
    0 = a^A C^{BM}_{An}N_B K_{m[rs]}^n,
\end{align}
leaving us with 10 independent equations. At first sight, it seems like due to the higher symmetry of the expansion points the equations are now weaker. This is, however, not true as the Lorentz invariance of the expansion point $N_A$ also influences the power series solution in higher-order. To illustrate this effect we consider the prolongation of the second equation in (\ref{DiffeoEqnFormal}) w.r.t. $v_B$, i.e., we apply the total derivative $D_B$ to this equation. We get the second-derivative-order equation:
\begin{multline}
    0 = l^AC_{An}^{Bm} + l^{AB}C_{An}^{Cm}v_C + l^{BAp} \bigl[ C_{An}^{Cm} \delta_p^q - \delta_A^C \delta_m^n \bigr] v_{Cq}\\
    + l^{BAI} \bigl[ C_{An}^{Cm} \delta_I^J - 2 \delta_A^C J_I^{pm} I^J_{pn}  \bigr] v_{CJ} + l^{B} \delta^m_n.
\end{multline}
When evaluating at $p_0$ only terms that have no $v_{Ap}$ or $v_{AI}$ contribute and we are left with 
\begin{align}\label{prolongE}
    0 = a^A C_{An}^{Bm} + a^{AB} C_{An}^{Cm} N_C +  a^B \delta^m_n.
\end{align}
If we had chosen a different prolongation, i.e., $D_{Bp}$ or $D_{BI}$, we would have got a similar expression. All these prolongations of the second equation of (\ref{DiffeoEqnFormal}) have in common that the only second-derivative-order contribution is proportional to $C^{Cm}_{An} N_C$ and hence all allow for the following construction. We simply take the whole prolonged equations (\ref{prolongE}) and contract it with the Lorentz generators $K_{m[rs]}^n$. Note that such a contraction is equivalent to a certain linear combination of equations of the prolonged equations. Then obviously the second-derivative-order contribution vanishes, as already $C_{An}^{Bm} N_B K_{m[rs]}^n = 0$.  Therefore the resulting equation is now again of first derivative order. Furthermore one readily finds that the contribution from $a^B \delta^m_n K_{m[rs]}^n$ vanishes. Hence we are left with 
\begin{align}\label{ansatz1}
    0 = a^A C^{Bm}_{An}  K_{m[rs]}^n.
\end{align}
Note that we can obtain similar equations from arbitrary other prolongations of the second equation in (\ref{DiffeoEqnFormal}), in particular also for higher-order prolongations, we will always obtain equations similar to (\ref{ansatz1}). Ultimately comparing the newly obtained first-derivative-order equation with the condition for $N_A$ to be Lorentz invariant also the meaning of the above equation (\ref{ansatz1}) becomes clear: the equation simply states that the expansion constants $a^A$ must be the components of a Lorentz invariant tensor. With these new first-order equations at hand, we, therefore, see that the Lorentz invariance of the expansion point $N_A$, in fact, does not yield weaker but stronger equations than a general expansion point, with the only obstruction lying in the fact the additional first-order equations could only be obtained after a prolongation and subsequent evaluation. Note that we could carry through a similar construction for any expansion point $M_A$ that is invariant under any arbitrary subgroup of the infinitesimal $GL(4)$ transformations that are locally induced by the lifted vector fields $\xi_F$. 

Before we proceed with a further investigation of these additional first-order equations, we want to reinforce the remark that these, in fact, have nothing in common with the integrability conditions discussed in the previous sections. Whereas integrability conditions really yield new independent partial differential equations, that in particular can be prolonged again and thereby can themselves contribute to further integrability conditions in the situation above we only obtain lower-order equations once we evaluate the whole equation at a certain point. Such already evaluated equations can obviously not be prolonged. The additional equations are hence not a feature of the PDE itself but of the expansion point. Similar effects can actually already occur during the treatment of a single ordinary differential equation, in short ODE, where the construction of true integrability conditions is clearly not possible. We illustrate the above arguments by considering the following single ODE
\begin{align}
    0 = x \cdot f^{\prime}(x) - 2x^2.
\end{align}
The general solution is given by $f(x) = x^2 +c$ where $c=const$. We construct a finite power series solution up to some $k \geq 0$ around $x_0 = 0$. Therefore we insert the following ansatz into the ODE:
\begin{align}
    f_{per} = \sum_{i=0}^k a_i x^i.
\end{align}
We evaluate the equation at $x_0=0$ to find in the first-order the trivial equation $0=0$. Note that the ODE is of first derivative order. Hence, in general, we would expect to obtain equations for $a_1$ from inserting the power series ansatz. We proceed by prolonging the ODE to obtain the second-derivative-order equation
\begin{align}
    0 = f^{\prime}(x) + x \cdot f^{\prime \prime}(x) - 4 x.
\end{align}
Inserting the series ansatz and evaluating at $x_0=0$ now yields the equation $a_1=0$. Although the prolonged ODE is of second derivative order, after evaluating it at the expansion point, we now obtained an equation for the first-order expansion coefficient. Proceeding with the next orders we will find $a_2 = 1$ and $a_i = 0$ for $i < 2$. Hence the power series solution reads $f_{per} = a_0 + x^2$, which obviously is the correct solution. Nevertheless, during each step of the construction of this power series solution, we had to prolong the ODE one order further than it is usually required for the given order of the expansion constants. This was necessary due to the fact that although the ODE is of first derivative order, at the expansion point $x_0=0$ the only term that contains first-order derivatives vanishes. If we had constructed a power series solution around any other point $x_0 \neq 0$ this would not have happened. Note that this vanishing of the highest derivative order at the expansion point is in one to one correspondence with the rank defects that were featured by (\ref{DiffeoEqnFormal}) when evaluated at the expansion point provided by $N_A$. 

Returning now to our original problem the construction of power series solutions to (\ref{DiffeoEqnFormal}) we thus can proceed as usual, by successively prolonging the PDE and inserting the series expansion, we only have to keep in mind that in order to obtain the general solution up to a given order we actually have to prolong to one order higher in order to take the additional lower-order equations stemming from the rank defect at $N_A$ into account. 

At first sight, this might sound like a massive disadvantage as with each prolongation order the PDE gets increasingly magnified in its size. We are, however, in the fortunate situation that we can predict the precise form of the additionally occurring lower-order equations in arbitrary order. All these equations simply encode the Lorentz invariance of the various expansion constants. In particular, note that by taking appropriate prolongations of the second equation in (\ref{DiffeoEqnFormal}) we can really construct such lower-order equations that govern the Lorentz invariance for all expansion constants in the general power series expansion of the Lagrangian. 
Considering this we can severely reduce the dimensions of any linear equation systems that may arise during the construction of such power series solutions by not taking the expansion constants $a^{\tilde{I}_k} \equiv a_0, a^A, a^{Ap},...$ as arbitrary constants and then solve the appropriate newly obtained lower-order equations to ensure that they describe the components of Lorentz invariant tensors, but by including from the very beginning only such expansion coefficients in the power series that are Lorentz invariant.

As we are dealing with linear equations, it suffices to construct a basis of the appropriate space of Lorentz invariant constant tensors of the corresponding valence and symmetry. This can be achieved by making use of some well-known results from classical invariant theory, more precisely the so-called \textit{\textbf{first fundamental theorem}} (see \cite{Aslaksen1995InvariantTO} and also \cite{PROCESI1976306}). Roughly speaking and restricting attention to the special orthogonal groups $SO(n)$ it states all objects that are invariant under the $(p,q)$ tensor representation -- the tensor product of $p$ copies of the fundamental $SO(n)$ representation and $q$ copies of its dual, for $p \neq q$ --- can be obtained by forming expressions that solely involve the $SO(n)$ invariant metric\footnote{Which is given by $\eta_{ab}$ for the special case of the Lorentz group $SO(1,3)$.}, the $n$-dimensional Levi-Civita symbol $\epsilon_{a_1...a_n}$ and the corresponding contravariant objects. 

Applying this to the special case of the Lorentz group $SO(1,3)$ at hand we find that for a given kind of expansion coefficient, for instance $a^{AB}$ we can obtain a basis of Lorentz invariant expressions by first writing down the most general expression for $a^{AB}$ that can be constructed from the invariant metric $\eta_{ab}$, its inverse $\eta^{ab}$, $\epsilon_{abcd}$ and $\epsilon^{abcd}$ that is consistent with the symmetries and the index structure of $a^{AB}$. To that end it is best to transform $a^{AB}$ back to the form involving spacetime indices for instance $a^{abcdefgh} := a^{AB}J_A^{abcd}J_B^{efgh}$, if the gravitational field is described by a rank $(0,4)$ tensor field. It is then straight forward to reduce the most general expression obtained in this fashion to a basis by removing linear dependencies.

Details regarding how one needs to proceed step by step to achieve the above are provided when we discuss particular examples. As this endeavor is straight forward but extensively laborious, it is best done relying on computer algebra. For precisely that purpose, we developed a highly performant computer program. 
%cite own computer program here
Further information regarding the underlying mathematics but also the concrete implementation and a short how-to-use guide can be found in chapter \ref{computerAlg}. 

Finally, we would like to emphasize the practical advantage of this approach. When computing power series solutions to (\ref{DiffeoEqnFormal}), the arising linear equations rapidly increase in size. Assuming $F$ has fiber dimension $21$ to provide an example that we will, in fact, treat in the next chapter the PDE compromises of $136$ equations for a function of $315$ independent variables (when the explicit $x^m$ dependency is dismissed from the very beginning and hence the first equation in (\ref{DiffeoEqnFormal}) removed.).  When inserting the power series ansatz into the prolonged PDE obviously some of the obtained linear equations for the expansion constants decouple into subsystems that can be solved independently. Nevertheless, due to the sheer dimensionality of the expansion constants, even solving the obtained subsystems poses a real problem. In our example, for instance, we would encounter the expression $a^{AIBJCK}$, which now includes $210\cdot 211\cdot212/6=1565620$ constants. Hence even if the obtained linear system decouples in such a way that allows us to treat all expressions that involve $a^{AIBJCK}$ independently from the rest we still have to solve a linear system with roughly $1.5$ Mio constants. On the other hand, the space of Lorentz invariant expressions that we can obtain for $a^{AIBJCK}$ only has a dimension of several hundred. Hence working with the Lorentz invariant expression from the very beginning reduces the size of the problem from roughly $1.5$ Mio involved constants to several hundred. 

Moreover the additional restriction to Lorentz invariant expansion coefficients also reduces the number of terms that are present in the power series Lagrangian (\ref{LPert}). Any Lorentz-invariant tensor must necessarily be of even rank, i.e., in our case have an even number of spacetime indices. As we further assumed the gravitational field to be of even rank, we can reduce the power series expansion by dismissing all terms that contain an odd number of spacetime derivative indices.

In addition to that, from the expansion (\ref{LPert}), one can readily compute the contribution the individual terms yield in the EOM.  As we required these to be of second derivative order, such that the associated Hamiltonian formulation is free of instabilities, we can additionally drop terms in the power series expansion that contain either contributions from $H_{AI_k}$ with $k>2$, or that feature expressions of the form $H_{AI_2}\cdot H_{AI_k}$ where $k\geq 1$, as all such expressions contribute higher that $2$nd derivative orders once we take apply the variational derivative to them.

Finally, the requirement of causal compatibility between matter and gravitational EOM demands that the gravitational Principal Polynomial generates the same distribution of vanishing sets as the matter Principal Polynomial. As the matter theory is prescribed by a Lagrangian on $J^1F_{mat} \times F_{grav}$ also the matter Principal Polynomial only depends on the gravitational fiber coordinates $v_A$, not on any gravitational derivative coordinates $v_{AI_k}$ for $k\geq 1$.
As a consequence of the causal compatibility requirement also the gravitational Principal Polynomial must hence only depend on the coordinates on $F_{grav}$. Thus we must discard all terms from (\ref{LPert}) that contains a total number of derivatives\footnote{Here we do not refer to the specific derivative order but really mean the total number of derivatives. For instance $H_{AI}$ and $H_{Ap} \cdot H_{Bq}$ both contain a total number of two derivatives, whereas $H^{Ap}H_{Bq}H_{CI}$ includes four derivatives.} higher than two, as for the case of second derivative order EOM these terms then necessarily must contain derivative contributions in the Principal Symbol.
In the following we will call this quantity, the total number of derivative indices that occur in an ansatz $a^{\tilde{I}_k}$, its \textit{\textbf{cumulative derivative order}}.  

After all, we find that the following general expansion remains: 
\begin{multline}\label{LperRed}
     L_{per} = a_0 + a^A H_A + a^{AI}H_{AI} + a^{AB} H_{A}H_{B} + a^{ApBq} H_{Ap}H_{Bq} + a^{ABI} H_{A} H_{BI} \\
    + a^{ABC} H_a H_B H_C + a^{ABpCq} H_{A}H_{Bp}H_{Cq} +
    + a^{ABCI} H_A H_B H_{CI} 
    + \mathcal{O}(4).
\end{multline}
Inserting this in the last 3 equations of the PDE (\ref{DiffeoEqnFormal}) and evaluating at the flat expansion point yields:
\begin{align}\label{order1}
    \begin{aligned}
    &0 = a^A C_{An}^{Bm}N_B + a_0 \delta^m_n\\
    &0 = a^{AI}C_{An}^{B(m\vert }N_B J^{\vert pq)}_I.
    \end{aligned}
\end{align}
doing the same for the prolonged PDE we find 
\begin{align}\label{order2}
    \begin{aligned}
    &0 = a^A C_{An}^{Bm} + 2 a^{AB}C_{An}^{Cm}N_C + a^B\delta^m_n\\
    &0 = a^{AI}\left [C_{An}^{Bm}\delta^I _J- 2 \delta^A_B J_I^{pm}I^J_{pn} \right ] + a^{ABJ}C_{An}^{Cm}N_C + a^{BJ} \delta^m_n \\
    &0 = 2a^{A(p\vert Bq}C_{An}^{C\vert m)}N_C + a^{AI} \left [C_{An}^{B(m\vert} 2 J_{I}^{\vert p)q} - \delta_A^BJ_I^{pm}\delta^q_n \right ]\\
    &0 = a^{BAI}C_{An}^{C(m\vert}N_CJ_I^{\vert pq)} + a^{AI}C_{An}^{B(m \vert} J_I^{\vert pq)}.
    \end{aligned}
\end{align}
Finally, prolonging the PDE to third derivative order inserting the power series ansatz and evaluating at $p_0$ yields
\begin{align}\label{order3}
\begin{aligned}
&0 = 2 a^{AC}C_{An}^{Bm} + 2a^{AB}C_{An}^{Cm} + 6 a^{ABC}C_{An}^{Dm} N_D + 2a^{BC} \delta^m_n\\
&0 = 2 a^{BqCr} \left [ C_{An}^{Bm} \delta ^q_p - \delta^B_A \delta^m_n \right ] +2 a^{A Bq Cr} C_{An}^{Dm} N_D + 2 a^{BqCr} \delta^m_n\\
&0 = a^{CAI} \left [C_{An}^{Bm}\delta^I _J- 2 \delta^A_B J_I^{pm}I^J_{pn} \right ] + 2 a^{ACBJ} C_{An}^{Dm} N_D + a^{CBJ} \delta ^m _n \\
&0 = 2 a^{C A(p \vert B q} C_{An}^{D \vert m )} N_D + a^{CAI} \left [C_{An}^{B(m\vert} 2 J_{I}^{\vert p)q} - \delta_A^BJ_I^{pm}\delta^q_n \right ]\\
&0 = 2 a^{BCAI}C_{An}^{D(m\vert}N_DJ_I^{\vert pq)} + a^{CAI}C_{An}^{B(m \vert} J_I^{\vert pq)}.
\end{aligned}
\end{align}
%go one order further ??
Together with the requirement that the expansion constants are Lorentz invariant these equations really contain all information that we can extract from the PDE (\ref{DiffeoEqnFormal}) for the construction of the power series Lagrangian. In particular, as we have shown that the PDE is involutive, we are now sure that we do not miss hidden information.  By means of this perturbative power series solution, the requirement of constructing diffeomorphism invariant Lagrangians for any given field theory hence boils down to the much simpler quest of solving the above system of linear equations.
Note that we could now easily derive a similar expression for the higher-order contributions to such a power series expansion. As the resulting linear systems then however become extensively complicated and in most cases practically unsolvable we only provide the linear systems contributing to the first three orders of the power series Lagrangian here. 

Besides these equations take precisely this form no matter what specific field is in consideration. The only quantities in the above linear equations (\ref{order1}), (\ref{order2}) and (\ref{order3}) that explicitly depend on the chosen field bundle are the vertical coefficients $C^{Bm}_{An}$ of the infinitesimal diffeomorphism action on $F$. For that reason, we have also cast the treatment of such equations into a computer program. Once we have specified the particular field bundle that we wish to work on and the expansion point we therefore only need to set up the computer with the expression for $C_{An}^{Bm}$ and the appropriate ranges for the fiber indices of the field bundle $A$ as initial input, the construction of the appropriate Lorentz invariant expansion coefficients and the subsequent solution of the above linear equations is then treated fully automatically. Details can again be found in chapter \ref{computerAlg}. 

We end this chapter by examining how we can incorporate the requirement of causal compatibility of matter and gravitational equations in the perturbative approach.
We assume again that we are handed a matter theory 
\begin{align}
    \mathcal{L}_{mat} : F_{grav} \times J^1F_{mat} \longrightarrow \Lambda^4M,
\end{align}
that exploits the gravitational field as geometric background. Along the previously prescribed lines we can now construct a perturbative expansion of the requested gravitational Lagrangian
\begin{align}
    \mathcal{L}_{grav,per} : J^2F_{grav} \longrightarrow \Lambda^4M,
\end{align}
and implement the required diffeomorphism invariance by solving the linear systems for the expansion coefficients that are obtained from inserting the power series around $p_0 \in J^2F_{grav}$ into (\ref{DiffeoEqnFormal}). 
Following along the lines of the construction recipe Algorithm \ref{Algo1} we would now have to compute the matter and gravitational Principal Polynomials.
These obviously must now also be computed perturbatively. 
To lighten the notation, we will drop the explicit $k$ dependency in the following and write $\mathcal{P}_{mat}$ instead of $\mathcal{P}_{mat}(k)$ and analogously for the gravitational Principal Polynomial. It is straight forward how we can expand $\mathcal{P}_{mat}$ in coordinates around the chosen expansion point $p_0$ up to some arbitrary order, as we can easily derive the corresponding full, i.e., non-perturbative expression $\mathcal{P}_{mat}$ from the matter EOM. We get the following expansion of $\mathcal{P}_{mat}$ up to and including the quadratic order in $H_A$:
\begin{align}
    \mathcal{P}_{mat}(v_A) = (P_{mat})_{0} + (P_{mat})^A_1 H_A+ (P_{mat})^{AB}_2 H_A H_B +\mathcal{O}(3),
\end{align}
where $(P_{mat})_0 = \mathcal{P}_{mat}(N_A)$, $(P_{mat})_1^A = \partial^A \mathcal{P}_{mat} \vert _{N_A}$, and similar for $(P_{mat})^{AB}_2$. Of course one readily computes similar expression for expansion up to higher-order. 

Finding the expansion of the gravitational Principal Polynomial requires somewhat more work as we do not have access to the full expression for $\mathcal{P}_{grav}$. All we can provide is the perturbative expansion of the Lagrangian. From this, we can compute perturbative EOM and also a perturbative expression for the Principal Symbol of the EOM.
Note that when expanding the Lagrangian up to order $k$ from applying the variational derivative $\frac{\partial}{\partial v_A}$ we will get the corresponding EOM up to order $k-1$. They are then necessarily quasilinear and by requirement of second derivative order. According to definition (\ref{PSym}) we can obtain the Principal Symbol by deriving the EOM w.r.t. the highest occurring derivative  --- as the EOM are linear in the highest derivative this simply admits to taking the coefficients in front of $v_{AI}$ --- and then contracting with $J_I^{pq} k_p, k_q$, for $k_p$ being the components of some 1-form on the base manifold $M$. 
The explicit formula for the EOM corresponding to the expansion (\ref{LperRed}) can be computed as:
\begin{align}\label{EOMPert}
    \begin{aligned}
    E_{per}^A = \frac{\partial L_{per}}{\partial v_A} &= a^A + 2 a^{AB}H_B + 3a^{ABC}H_B H_C \\
    &\hphantom{=}+ \left [a^{ABI} + a^{BAI} - 2 a^{ApBq}I_{pq}^I  \right ] H_{BI} \\
    &\hphantom{=}+ \left[a^{ABpCq} -2a^{BApCq} +2a^{BCAI} J_I^{pq} \right]H_{Bp}H_{Cq} \\
    &\hphantom{=}+ \left [2a^{ACBI} -2a^{CApBq}I_{pq}^I + 2a^{BCAI} \right ]H_C H_{BI} \\
    &\hphantom{=}+ \mathcal{O}(3),
    \end{aligned}
\end{align}
where we used the symmetries of the expansion coefficients that are enforced by the way they are contracted in (\ref{LperRed}).
In particular, observe that the thus obtained Principal Symbol is now given up to order $k-2$. Therefore we can only compute the Principal Polynomial up to order $k-2$. In the following, we denote the thereby obtained expansion of the Principal Symbol by 
\begin{align}
    T(v_A) = T_0 + T_1^CH_C + T_2^{CD}H_CH_D + \mathcal{O}(3).
\end{align}
Reading of the values in constant and linear order from the expansion of the EOM we find:
\begin{align}
    \begin{aligned}
    T_0^{AB} &= \left [a^{ABI} + a^{BAI} - 2 a^{ApBq}I_{pq}^I  \right ]\\
    \\
    (T_1^{AB})^C &= \left [2a^{ACBI} -2a^{CApBq}I_{pq}^I + 2a^{BCAI} \right ]. 
    \end{aligned}
\end{align}
Note that the Principal Symbol and thus also the Principal Polynomial could in general also depend on the coordinates $v_{Ap}$ as the EOM would then be still of second derivative order. We however excluded such contributions already in the general power series Lagrangian (\ref{LperRed}),  as the matter polynomial cannot contain such and in the end we want to require that the two polynomials describe at each $p \in M$ the same vanishing set. 

Recall that before in the exact setting we obtained the Principal Polynomial from the Principal Symbol by computing any non-vanishing order four sub determinant that was obtained by removing rows $(A_1...A_4)$ and columns $(B_1...B_4)$ from the symbol, and then dividing the expression by the appropriate prefactor (\ref{prefacF})
Note that also the expression $\chi_{An}(v_A) = C_{An}^{Bm}v_Bk_m$ is linear in $v_A$ and hence also the prefactor contributes in different orders. We expand $\chi_{An}$ around $p_0$ to obtain
\begin{align}
\chi_{An}(v_A) =  C^{Bm}_{An} N_B k_m + C^{Bm}_{An} H_B k_m =: (\chi_0)_{An} + (\chi_1)^B_{An}H_B
\end{align}
Inserting this into (\ref{prefacF}) we denote the thereby induced expansion by
\begin{multline}\label{prefacExp}
    f_{(A_1...A_4)(B_1...B_4)}(v_A) = (f_0)_{(A_1...A_4)(B_1...B_4)} + (f_1)^C_{(A_1...A_4)(B_1...B_4)}H_C\\ + (f_2)^{CD}_{(A_1...A_4)(B_1...B_4)}H_CH_D
    + \mathcal{O}(3).
\end{multline}
The determinant of a $(n-4) \times (n-4)$ submatrix of the expansion of the Principal Symbol with entries being $m$th-order expressions in $H_A$ will in general be of order $m\cdot(n-4)$. This would in general pose no problem as we could simply compute the closed-form expression of the determinant of this submatrix end then again expand the result up to the required order in $H_A$. This method, however, will generate technical problems. After solving the perturbative equivariance equations (\ref{order1}), (\ref{order2}) and (\ref{order3}) the Lagrangian and therefore also the Principal Symbol will contain undetermined constants. Hence the entries of the Principal Symbol are not only expressions that are linear in $H_A$ but also contain additional constants. Using standard computer algebra to calculate an algebraic expression for the determinant of such a matrix that contains symbolic entries one usually reaches a limit of either the available memory or the required computation time once the dimension of the matrix exceeds $15 \times 15 $, with the precise limit obviously depending on the used machine, the precise form of the matrix and the specific algorithms at use. 

Note that all these complications only arise once we try to compute the closed form of the determinant. This is not even what we try to achieve as we are only interested in the expansion of the determinant up to some order. Hence we can at least partly avoid such technical problems by directly expanding the determinant. Recall that when interpreted as a map on the column vectors that specify a given matrix, the determinant is multilinear. Therefore we can simply expand it (see for instance \cite{2008CoTPh..49..801Z} and also the following nice collection of matrix formulae \cite{IMM2012-03274}), with an expansion around the identity matrix taking the particularly simple form 
\begin{align}\label{detExp}
    \mathrm{det}(I+M) = 1 + \mathrm{Tr}(M) + \frac{(\mathrm{Tr}(M))^2- \mathrm{Tr}(M^2)}{2} + \mathcal{O}(3). 
\end{align}
As we required the chosen $(n-4)\times (n-4)$ submatrix of the Principal Symbol\footnote{Obviously concretely finding such a submatrix involves some trial and error. Nevertheless, as here we are only interested in the distinction of whether the determinant is zero or not, the involved computations do not yield technical problems. This is true as we can simply evaluate all symbolic entries of the matrix at some random integers and compute its rank using, for instance, a fraction free implementation of Gaussian elimination. The thus computed rank always provides a lower bound for the true symbolic rank as in the worst case by randomly evaluating the symbolic entries we generate additional linear dependencies of the rows or columns of the matrix. Hence when the randomly evaluated matrix has full rank, we know that the symbolic matrix must have non zero determinant. This is a huge advantage over computing the rank symbolically as now the computation only involves integer arithmetic which is not only much more efficient than performing symbolic computations but also remarkably stable, as no round offs are performed.} to have non-vanishing determinant, the constant order of this submatrix will be invertible. We denote the chosen submatrix and the corresponding expansion by 
\begin{multline}
    T_{(A_1...A_4)(B_1...B_4)}(v_A) = (T_0)_{(A_1...A_4)(B_1...B_4)} + (T_1)_{(A_1...A_4)(B_1...B_4)}^{C} H_C \\
    +(T_2)_{(A_1...A_4)(B_1...B_4)}^{CD} H_C H_D + \mathcal{O}(3),
\end{multline}
where the symmetric 4-tuples $(A_1...A_4)$ and $(B_1...B_4)$ denote the removed rows and columns respectively. The inverse of the constant order is then given by $(T_0)^{-1}_{(A_1...A_4)(B_1...B_4)}$. We can now compute the determinant of this submatrix as follows:
\begin{multline}
    \mathrm{det}\left(T_{(A_1...A_4)(B_1...B_4)}(v_A)\right)
    = \mathrm{det}\left((T_0)_{(A_1...A_4)(B_1...B_4)}\right)\\ 
    \cdot \mathrm{det}\left (I +(T_0)^{-1}_{(A_1...A_4)(B_1...B_4)}
    \cdot \left [ (T_1)_{(A_1...A_4)(B_1...B_4)}^{C} H_C+(T_2)_{(A_1...A_4)(B_1...B_4)}^{CD} H_C H_D \right ]  \right ) \\
    + \mathcal{O}(3)  
\end{multline}
Using now the expansion of the determinant around the identity matrix for the second factor in the above expression and the linearity and cyclicity of the trace we
find the following expansion for this order four sub determinant 
\begin{multline}
    \mathrm{det}\left(T_{(A_1...A_4)(B_1...B_4)}(v_A)\right) = (D_0)_{(A_1...A_4)(B_1...B_4)} + (D_1)^C_{(A_1...A_4)(B_1...B_4)}H_C\\
    +(D_2)^{CD}_{(A_1...A_4)(B_1...B_4)}H_CH_D
    + \mathcal{O}(3),
\end{multline}
with the contributions in the individual orders being given as 
\begin{align}\label{polyMatrices}
\begin{aligned}
  (D_0)_{(A_1...A_4)(B_1...B_4)} &=  \mathrm{det}\left((T_0)_{(A_1...A_4)(B_1...B_4)}\right) \\
  \\
  (D_1)^C_{(A_1...A_4)(B_1...B_4)} &= \mathrm{det}\left((T_0)_{(A_1...A_4)(B_1...B_4)}\right) \cdot \mathrm{Tr} \left ( (T_0)^{-1}_{(A_1...A_4)(B_1...B_4)}
    \cdot (T_1)_{(A_1...A_4)(B_1...B_4)}^{C} \right) \\
    \\
    (D_2)^{CD}_{(A_1...A_4)(B_1...B_4)} &= \mathrm{det}\left((T_0)_{(A_1...A_4)(B_1...B_4)}\right)
     \cdot \Bigl [ \mathrm{Tr} \left ( (T_0)^{-1}_{(A_1...A_4)(B_1...B_4)}
    \cdot (T_2)_{(A_1...A_4)(B_1...B_4)}^{CD} \right ) \\
     &\hphantom{=}
    + \frac{1}{2} \cdot \Bigl \{ \mathrm{Tr}\left ( (T_0)^{-1}_{(A_1...A_4)(B_1...B_4)} \cdot (T_1)_{(A_1...A_4)(B_1...B_4)}^{C} \right )\\
     &\hphantom{=} \cdot \mathrm{Tr}\left ( (T_0)^{-1}_{(A_1...A_4)(B_1...B_4)} \cdot (T_1)_{(A_1...A_4)(B_1...B_4)}^{D} \right )  \\
      &\hphantom{=} 
    - \mathrm{Tr}\Bigl  (((T_0)^{-1}_{(A_1...A_4)(B_1...B_4)})^2 \cdot (T_1)_{(A_1...A_4)(B_1...B_4)}^{C} \cdot (T_1)_{(A_1...A_4)(B_1...B_4)}^{D}  \Bigr )    \Bigr \} \Bigr ]
    \end{aligned}
\end{align}
On the other hand, using (\ref{diffeoMinor}) we can express the different order contributions to the determinant of such a submatrix by means of an expansion of the gravitational Principal Polynomial:
\begin{align}
    \mathcal{P}_{grav}(v_A) = (P_{grav})_{0} + (P_{grav})^A_1 H_A+ (P_{grav})^{AB}_2 H_A H_B +\mathcal{O}(3).
\end{align}
Together with the expansion of the prefactor (\ref{prefacExp}), we thus get from (\ref{diffeoMinor}) the following contributions in the different orders
\begin{align}\label{minorPoly}
    \begin{aligned}
    (D_0)_{(A_1...A_4)(B_1...B_4)}  &= (f_0)_{(A_1...A_4)(B_1...B_4)} \cdot (P_{grav})_0 \\
    \\
    (D_1)^C_{(A_1...A_4)(B_1...B_4)}  &= (f_0)_{(A_1...A_4)(B_1...B_4)} \cdot (P_{grav})^C_1 + (f_1)^C_{(A_1...A_4)(B_1...B_4)} \cdot (P_{grav})_0  \\
    \\
    (D_2)^{CD}_{(A_1...A_4)(B_1...B_4)}  &=  (f_0)_{(A_1...A_4)(B_1...B_4)} \cdot (P_{grav})_2^{CD} \\
     & \hphantom{=} +
     (f_1)^C_{(A_1...A_4)(B_1...B_4)} \cdot (P_{grav})_1^D +(f_2)^{CD}_{(A_1...A_4)(B_1...B_4)} \cdot (P_{grav})_0 
    \end{aligned}
\end{align}
Hence once we know the removed rows and columns $(A_1...A_4)$ and $(B_1...B_4)$ we can compute the constant, linear and quadratic-order contributions to $f_{(A_1...A_4)(B_1...B_4)}(v_A)$, and the constant, linear and quadratic-order contributions to the determinant of the submatrix and from these obtain the constant order of the gravitational Principal Polynomial as:
\begin{align}\label{POLY1}
(P_{grav})_0 = \frac{(D_0)_{(A_1...A_4)(B_1...B_4)}}{(f_0)_{(A_1...A_4)(B_1...B_4)}}.
\end{align}
Using now the constant-order Principal Polynomial that we thus obtain we get the linear order by
\begin{align}\label{POLY2}
    (P_{grav})^C_1= \frac{(D_1)^C_{(A_1...A_4)(B_1...B_4)} - (f_1)^C_{(A_1...A_4)(B_1...B_4)} \cdot (P_{grav})_0}{(f_0)_{(A_1...A_4)(B_1...B_4)}}.
\end{align}
Moreover, from this expression for the linear-order, we can obtain the quadratic-order of the gravitational Principal Polynomial by 
\begin{multline}\label{POLY3}
    (P_{grav})_2 = \\
    \frac{(D_2)^{CD}_{(A_1...A_4)(B_1...B_4)}-\left [ (f_1)^C_{(A_1...A_4)(B_1...B_4)} \cdot (P_{grav})^D_1  +(f_2)^{CD}_{(A_1...A_4)(B_1...B_4)} \cdot (P_{grav})_0 \right ]}{(f_0)_{(A_1...A_4)(B_1...B_4)}}.
\end{multline}
In total proceeding as outlined above, we there obtain the expansion of the gravitational Principal Polynomial from the power series expansion of the Lagrangian.
Note that we could now proceed along the same lines to obtain the corresponding contributions to the Principal Polynomial in higher-order. This is in principal straight forward with the only exception being that the expressions get increasingly involved. For that reason, we did not provide these here. When tackling practical problems in most constructing the perturbative Lagrangian beyond fourth-order is computationally intractable and hence providing the formulae for computation of the Principal Polynomial up to second-order almost always suffices.

Before we proceed further, we would like to quickly remark on some consequences of the Lorentz invariance of the chosen expansion point $p_0$ on the possible values that we can obtain for the contributions to the gravitational Principal Polynomial in the individual orders. Recall that for second-derivative-order, diffeomorphism invariant theories the Principal Polynomial is a homogeneous polynomial in the components of some 1-form $k_m$ on $M$ in degree $r := 2n-16$. We hence can write it as 
\begin{align}
    \mathcal{P}_{grav} = \mathcal{P}_{grav}^{{p_1}...{p_{r}}} k_{p_1} \cdot ... \cdot k_{p_r}.
\end{align}
Where $\mathcal{P}_{grav}^{{p_1}...{p_r}}$ is are function on $F$ that is totally symmetric in $(p_1...p_r)$. As a consequence of the required equivariance of the Lagrangian it holds that:
\begin{align}\label{polyEqn}
    0 = \partial^A\mathcal{P}_{grav}^{{p_1}...{p_r}}C_{An}^{Bm}v_B - r \cdot \mathcal{P}_{grav}^{({p_1}...\vert m} \delta_{n}^{\vert p_r) }  + (r/2) \cdot \mathcal{P}_{grav}^{{p_1}...{p_r}} \delta^m_n.
\end{align}
%check sign !!!
%
%
%
This can be seen by starting from the PDE that the EOM have to satisfy as an implication of the equivariance (\ref{EOM}) and applying the total derivative $D_{AI}$ to obtain a similar PDE for subexpression $\left ( \frac{\partial E^A}{\partial V_{BI}} \right )$ that is involved in the Principal Symbol. Proceeding along the same lines for the remaining steps that are involved in the computation of the Principal Polynomial, one then finds the above equation. 

The numerical factor $r/2 = (n-8)$ in front of the $\delta^m_n$ term is obtained as this factor encodes the density weight of the Principal Polynomial. Using the language of tensor densities, PDE (\ref{EOM}) states that the EOM $E^A$ defines a density of weight one and further prolonging this PDE w.r.t. $v_{AI}$ reveals that also the Principal Symbol has density weight one. Computing the determinant of a $(n-4) \times (n-4)$ submatrix of the Principal Symbol changes the weight to $n-4-2$. This can, for instance, be seen when expressing the determinant of the Principal Symbol w.r.t. the Levi-Civita symbols and then obtaining the order four sub determinant by deriving four times w.r.t. the components of the Principal Symbol, as explained in (\ref{MinorDef}). To achieve this, we need $n$ copies of the Principal Symbol, each of which has weight one and two covariant Levi-Civita symbols of dimension $n$ that carry density weight $-1$. This yields a total weight of $n-2$. Deriving four times w.r.t. the components of the Principal Symbol reduces the factor to $n-6$. Finally, according to (\ref{diffeoMinor}) this sub determinant equals the product between Principal Symbol and the corresponding prefactor. This prefactor involves two $4$-dimensional contravariant Levi-Civita symbols that hence contribute $+2$ to the weight. Therefore the weight of the Principal Polynomial is given by $n-8$, which thus explains the origin of the numerical factor in the above PDE. 

The all-important observation is now that any perturbative expansion of the Principal Symbol around the Lorentz-invariant expansion point $p_0$ that we might obtain necessarily provides a power series solution of the form
\begin{align}
    \mathcal{P}_{grav}^{{p_1}...{p_{r}}} = (P_{grav})^{{p_1}...{p_{r}}}_0 + (P_{grav})_1^{C{p_1}...{p_{r}}} H_C + \mathcal{O}(2)
\end{align}
to the above PDE ($\ref{polyEqn}$). Therefore we can apply the same prolongation trick as before to deduce that the expansion coefficients $(P_{grav})_0, (P_{grav})_1^C,...$ must by given as components of constant Lorentz invariant tensors. Hence as before we can try to construct them utilizing $\eta_{ab}$, $\epsilon_{abcd}$ and the corresponding contravariant expressions. As now, however, the expansion coefficients are totally symmetric in the indices $p_1...p_r$ the requirement of Lorentz invariance is much stronger than in the general case. The total symmetry for instance, hugely restricts possible ways $\epsilon_{abcd}$ can contribute to the expansion coefficients. In particular for any theory at hand $(P_{grav})_0^{{p_1}...{p_{r}}}$ cannot contain any contribution from $\epsilon^{abcd}$ and in fact one readily finds that the only Lorentz invariant totally symmetric tensor is a totally symmetrized product of inverse Minkowski metrics $\eta^{ab}$. Hence due to the required diffeomorphism invariance of the theory of gravity, for any gravitational field at wish already the Lorentz invariant expansion point fixes the constant-order contribution to the Principal Polynomial and therefore the causal structure of the linearized EOM uniquely.

Note that although the above discovery nicely illustrates the astonishing influence of the required diffeomorphism invariance on the causal structure of the linearized theory of gravity --- recall that such a linearized theory of gravity for instance already contains enough information to predict the propagation of gravitational waves --- there is no way how we can use this fact to obtain a computational advantage. In contrast to the case before, where we noticed that we might use the Lorentz invariance of the expansion coefficients of the gravitational Lagrangian to reduce the dimensionality of the arising linear systems now the further Lorentz invariance of the expansion coefficients of $\mathcal{P}^{p_1...p_r}$ is not deduced as consequence of a requirement that we wish to impose but stems from a  requirement that we already have imposed. If up to this point, we made no mistakes, this simply will come out. It thus can serve as a nice consistency check of any perturbative EOM that we might construct.

Given now the expansion of the gravitational Principal Polynomial we can compute the vanishing set, where obviously we also have to compute this perturbatively, by dropping terms of higher than the desired order. Doing the same also for expansion of the matter Principal Polynomial, we finally impose that the obtained perturbative vanishing sets of matter and gravity polynomial coincide in the appropriate perturbative order: 
\begin{align}
    \sum_{i=0}^{k-2} (V_{mat})_i = \sum _{i=0}^{k-2}(V_{grav})_i + \mathcal{O}(k-2),
\end{align}
where $k$ is the desired order of the power series expansion of the gravitational Lagrangian. This then ultimately also includes our second requirement, the causal compatibility between matter and gravitational dynamics perturbatively. 

We complete the chapter by providing an explicit Algorithm \ref{Algo2} for the perturbative construction of diffeomorphism invariant gravitational theories that are compatible with a given matter theory. 
Unlike the full construction algorithm (\ref{Algo1}), that suffered from the difficulty of computing the general solution of the equivariance equation, this perturbative construction algorithm now really is applicable. 
With this achievement the construction of alternative matter theories, at least perturbatively, boils down to choosing physically meaningful gravitational field bundles for the description of the gravitational field, computing the appropriate dynamics then merely corresponds to feeding the algorithm with the necessary input data.  

\begin{algorithm}[hbt!]
\SetAlgoLined
\KwData{Matter theory: $\mathcal{L}_{mat} : F_{grav} \times J^1F_{mat} \rightarrow \Lambda^4M$, expansion order: $k \in \mathbb{N}$, flat Lorentz invariant expansion point: $p_0 \in F_{grav}$}
\KwResult{Most general diffeomorphism invariant, causal compatible gravitational Lagrangian expanded as finite power series $\mathcal{L}_{grav,per}$ to order $k$ around $p_0$.}
Compute the vertical coefficients $C^{Bm}_{An}$ of the infinitesimal diffeomorphism action on $F_{grav}$. \\
Expand the Lagrangian as described in (\ref{generalPowerSL}) around $p_0 \equiv (x_0^m,N_A,0,0)$ with expansion coefficients $a^{\tilde{I}_k}$.\\
Restrict to those $a^{\tilde{I}_k}$ with cumulative derivative order $\leq 2$.\\
Insert the most general Lorentz invariant expressions for the $a^{\tilde{I}_k}$.\\
Solve the perturbative equivariance equations such as (\ref{order1}), (\ref{order2}) and (\ref{order3}) for the $a^{\tilde{I}_k}$ by plugging in $L_{grav,per}$ in (\ref{DiffeoEqnFormal}) and all necessary prolongations and evaluating the resulting expressions at $p_0$.\\
Compute the induced expansion of the Principal Symbol.\\
Chose a $(n-4) \times (n-4)$ full ranked submatrix of the Principal Symbol by removing rows $(A_1...A_4)$ and columns $(B_1...B_4)$ from it. \\
Compute the expansion of the determinant of the chosen submatrix.\\
Compute the induced expansion of the corresponding prefactors $f_{(A_1...A_4)(B_1...B_4)}$. \\
Compute the expansion of the gravitational Principal Polynomial with the use of the general expression (\ref{diffeoMinor}) with expansion (\ref{minorPoly}), i.e., by solving (\ref{POLY1}), (\ref{POLY2}), (\ref{POLY3}) and similar for higher-orders. \\
Compute the expansion of the matter Principal Polynomial up to order $k-2$.\\
Solve $\sum_{i=0}^{k-2} (V_{mat})_i = \sum _{i=0}^{k-2}(V_{grav})_i + \mathcal{O}(k-2)$ w.r.t. the remaining undetermined constants in $L_{grav,per}$.
 \caption{Perturbative Construction of Gravitational Lagrangian}\label{Algo2}
\end{algorithm}

Summing up the achievements of this chapter, firstly with the help of several notions of formal PDE theory, we have discovered a second and final fundamental requirement that we wish to pose on any gravitational Lagrangian, namely its compatibility with a given matter theory. As one might argue that the prediction of future processes really lies at the heart of theoretical physics this causal compatibility really is an indispensable requirement that any meaningful coupled matter gravity theory must incorporate. As stated in the last chapter, this requirement also compromises a cornerstone of the framework that was contributed in (\cite{2018PhRvD..97h4036D}). 

Besides the second main ingredient of our framework, we mainly focused on techniques of solving the equivariance equations with the concentration, in particular, lying on the construction of power series solutions. To justify this perturbative approach, it was essential to prove the involution of the equivariance PDE. In addition to that, the further developed techniques from formal PDE theory that we provided in this chapter ultimately allowed us to prescribe a detailed instruction how one will always obtain perturbative solutions to the problem of Constructive Gravity.  All that is left for this thesis consists now in concretely testing the thereby developed framework. This is precisely what we intend to do in the next chapter.
 

\chapter{Concrete Applications of the Developed Framework}\label{chapter3}
\dictum{
In this chapter, we investigate two concrete examples of possible gravitational fields each one motivated by a particular matter theory that exploits the thereby provided spacetime geometry as background. Concretely we will study the well-known case of gravity being described by a metric tensor field but also deviate from this by considering the case of spacetime geometry being provided by a so-called area metric. For each case, we will then apply the previously developed framework to not only qualitatively examine possible gravitational dynamics but also explicitly construct the most general meaningful perturbative theory of gravity.
}
\section{Perturbative Metric Gravity around Minkowski Spacetime}
As a first example, we consider the case where the gravitational field is provided by a symmetric $(0,2)$ tensor field, the metric tensor $g_{ab}$. The field bundle is hence given by the bundle of such tensors and as before labeled as $F_{GR} := S^0_2M \subset T^0_2M$. For some of the following considerations, we might more appropriately restrict $F_{GR}$ to the subbundle of $(0,2)$ tensors that are non-degenerate in the sense that the induced map between tangent and cotangent bundle that is obtained by partially applying the metric to a vector field defines an isomorphism. Furthermore, we additionally might restrict future considerations to those non-degenerate metrics that have signature $(-,+,+,+)$. As both of these restrictions define open conditions, this does not cause any problems.

Note that the case of gravity being described by a metric tensor field not only is the standard example and hence excellent for testing our framework, from the point of view of Constructive gravity it is also unique as the metric tensor is the "smallest" structure --- in the sense that the corresponding field bundle has the least dimensional fibers --- that allows for the support of a non trivial matter theory.

In the simplest possible case the matter field is described by a scalar field, i.e., a spacetime function $\phi \in C^{\infty}(M,\mathbb{R})$ which corresponds to the trivial matter field bundle $M \times \mathbb{R}$. We are again restricting to matter theories that are described by first-derivative-order Lagrangians and hence generate second-derivative-order EOM. In the simplest case, the matter EOM are linear. This corresponds to a quadratic matter Lagrangian. Denoting adapted coordinates on $J^1(M \times \mathbb{R})$ by $(x^m,\phi,\phi_m)$ the most general, such matter Lagrangian is given as:
\begin{align}\label{KGL}
    \mathcal{L}_{KG} = \frac{1}{2} \left ( B^{ab} \phi_a \phi_b - m^2 \phi^2\right )\omega \mathrm{d}^4x.
\end{align}
This is the so called Lagrangian \textit{\textbf{Klein-Gordon}} Lagrangian. It is important to observe that this really described the most general quadratic, first-order Lagrangian for a scalar field. 
In particular note that specifying such a Lagrangian forces us to prescribe $10$ spacetime functions $B^{ab}$ that tell us how the product $\phi_a\cdot \phi_b$ has to be contracted and furthermore one scalar density of weight one $\omega$ that ensures that the Lagrangian really defines a volume form valued bundle map and thus provides us with a well-defined action functional and EOM.
These additional ingredients are precisely provided by a metric tensor field $g_{ab}$. We then simply take $B^{ab}$ as the corresponding inverse metric $(g^{-1})^{ab}$. Note that in the case of non-degenerate metrics mapping a metric to its inverse is a smooth bundle isomorphism. Furthermore, one can in fact show that up to an overall constant there is exactly one scalar density of weight $1$ that can be obtained from the metric, namely $\omega = \sqrt{ \vert \mathrm{det}(g) \vert }$. 
Thus we see that the simplest possible non-trivial Lagrangian one can write down for a scalar field forces us to additionally provide exactly the amount of structure that is provided by a metric tensor field. If this particular metric, i.e., its the values of $g_{ab}$ at the various spacetime points, is not specified by hand, we need further equations that allow us to determine it.  

To that end, we are going to apply the previously developed perturbative framework to construct a perturbative expansion of a diffeomorphism invariant gravitational Lagrangian $\mathcal{L}_{GR} = L_{GR} \mathrm{d}^4x$ that is causally compatible with the Klein-Gordon Lagrangian (\ref{KGL}) and generates second-derivative-order EOM. 
We have already remarked that for the field bundle $F_{GR}$ there exist precisely 14 functionally independent curvature invariants that each solve the homogeneous invariance equation. Due to Lovelock (\cite{Lovelock1969}, \cite{doi:10.1063/1.1665613} and also \cite{doi:10.1063/1.1666069}) we know that out of the 14 independent gravitational Lagrangians that can be constructed from these up to contributions that vanish once we compute the EOM and equate to zero the Einstein-Hilbert Lagrangian is the only one that generates second-derivative-order EOM. 
Hence we expect from our perturbative construction recipe to precisely recover a perturbative expansion of the Einstein-Hilbert Lagrangian from first principles. Note that thereby, we get an excellent test for our framework. 


We start by introducing the necessary structure on $F_{GR}$. We have already seen that a pair of intertwiners for this field bundle can simply be obtained by the matrices  (\ref{interIMet}) and (\ref{interJMet}). These are, of course exactly the same intertwiners as the ones we use for describing second spacetime derivatives. We thus obtain coordinates on $J^2F_{GR}$ as $(x^m,v_A,v_{Ap},v_{AI})$ where now both $A$ and $I$ run from 0 to 9. For a metric tensor, i.e., a section $g \in \Gamma(F_{GR}))$ we further get the relations:
\begin{align}
    g_{ab} = I^A _{ab} g_A \ \ \text{and} \ \ g_A = J^{ab}_{A} g_{ab},
\end{align}
and similar for the inverse metric $g^{-1} \in \Gamma(F_{GR}^{\ast})$.
Following along the lines described in algorithm \ref{Algo2} we choose $J^2F_{GR} \ni p_0 \equiv (x_0^m,\eta_A,0,0) =: (N_{AI})$ where $\eta_A = J^{ab}_A \eta_{ab}$ as $\eta$-induced expansion point. Furthermore, we choose to construct the perturbative gravitational Lagrangian up to order $k=3$. 
Next, we compute the coordinate expression of the Lie derivative of a metric tensor field to obtain the vertical coefficients $C_{An}^{Bm}$. We get:
\begin{align}
    \mathcal{L}_{\xi} g_{ab} = \partial_m g_{ab} \cdot \xi^m + \left (-2 \delta_n^{(c\vert} \delta_{(a}^m \delta_{b)}^{\vert d)} \right ) g_{cd} \partial_m \xi^n.
\end{align}
Inserting $g_{cd} = I^B_{cd} g_B$ further contracting the whole expression with $J^{ab}_A$ to convert the free $ab$ indices to a $A$ index we find:
\begin{align}
    \mathcal{L}_{\xi} g_A = \partial_m g_A \xi^m + \left (-2 I^B_{nb}J^{mb}_{A} \right )g_B \partial_m \xi ^n. 
\end{align}
Here we also used that the two intertwiners $I^A_{ab}$ and $J_A^{ab}$ are symmetric in their spacetime indices. Thus we can simply read off from the last expression:
\begin{align}
    C_{An}^{Bm} = -2 I^B_{nb}J_A^{mb}.
\end{align}
As before we define $(H_{AI}) := (v_{AI}) - (N_{AI})$ to obtain the most general expansion of the Lagrangian that generates second-derivative-order EOM by:
\begin{align}\label{LGR}
    L_{GR,per} =  a_0 + a^A H_A + a^{AI}H_{AI} + a^{AB} H_{A}H_{B} + a^{ApBq} H_{Ap}H_{Bq} + a^{ABI} H_{A} H_{BI} \\
    + a^{ABC} H_a H_B H_C + a^{ABpCq} H_{A}H_{Bp}H_{Cq} +
    + a^{ABCI} H_A H_B H_{CI} 
    + \mathcal{O}(4).
\end{align}
We again restrict attention to those expansion coefficients with an even number of spacetime indices, as any coefficients with an odd number of such cannot be Lorentz invariant and thus are forbidden by the equivariance equations.

According to algorithm \ref{Algo2}, the next step now consists of inserting the most general Lorentz invariant tensors with appropriate index structure for the expansion coefficients. This is done entirely in terms of the developed computer algebra. The explicit expressions can be found in the appendix (\ref{LorentzGR1}).
Nevertheless providing an illustrative example we consider the expansion coefficient $a^{AB} = I^{A}_{ab}I^{B}_{cd}a^{abcd}$. We clearly see the symmetries in the pairs $(ab)$ and $(cd)$ and the additional symmetry under the exchange $(ab) \leftrightarrow (cd)$ as the coefficient only appears contracted against $H_AH_B$ in $L_{GR,per}$. Writing down the most general Lorentz invariant tensor with these symmetries we readily find that we cannot obtain non zero contributions from $\epsilon^{abcd}$ to any tensor that features these symmetries. The inverse Minkowski metric a priory allows for $3$ different expressions with $4$ upper case indices:
\begin{itemize}
    \item[(i)] $\eta^{ab} \eta^{cd}$ 
    \item[(ii)] $\eta^{ac} \eta^{bd}$ 
    \item[(iii)] $\eta^{ad} \eta^{bc}$.
\end{itemize}
Enforcing the symmetries on these $3$ expressions, we find that (i) already features the desired symmetries and (ii) and (iii) both provide the symmetry once we symmetrize in $(ab)$. These two terms then are precisely the same. We thus find that:
\begin{align}\label{ansatzExample}
    a^{AB} = I^{A}_{ab}I^{B}_{cd} \left ( 8\mu_3 \cdot \eta^{ab}\eta^{cd} + 8\mu_4 \cdot \eta^{c(a} \eta^{b)d}   \right ).
\end{align}
Here the factor $8$ is a result of the factor less symmetrization that we implement in our computer program, i.e., when symmetrizing the expression w.r.t. $(ab),(cd)$ and $(ab) \leftrightarrow (cd) $ we do not divide by $2$ each time and hence get an overall factor of $2^3=8$. This approach is equivalent to the symmetrization including these factors as we are free to absorb such factors by redefining the constants. It furthermore provides the advantage that we can then deal with integer factors throughout the whole computation. Details are explained in section (\ref{LorentzGen}).
Thus we have obtained an expression that features $2$ constants for $a^{AB}$.

Note that computing the remaining expansion coefficients works precisely along the same lines as the above with the only difference being that the expressions will become increasingly involved and thus are best obtained by using computer algebra.
Doing this the dimensions, i.e., the number of arbitrary constants in the expansion coefficients are displayed in table \ref{GRExp}.
\begin{table}
\centering 
\begin{tabular}{lll}\toprule
    expansion coefficient & dimension & constants   \\ \midrule
    $a_0$ & 1 & $\{\mu_1\}$ \\
    $a^A$ & 1 & $\{\mu_2\}$ \\
    $a^{AI}$ & 2 & $\{\nu_1, \nu_2\}$ \\
    $a^{AB}$ & 2 & $\{\mu_3, \mu_4 \} $ \\
    $a^{ApBq}$ & 6 & $\{\nu_3,...,\nu_8\}$ \\
    $a^{ABI}$ & 5 & $\{ \nu_9,...,\nu_{13} \}$ \\
    $a^{ABC}$ & 3 & $\{ \mu_5,...\mu_7 \}$\\
    $a^{ABpCq}$ & 21 & $\{\nu_{14},...,\nu_{34} \}$ \\
    $a^{ABCI}$ & 13 & $\{ \nu_{35},...,\nu_{47}\}$\\ \bottomrule
\end{tabular}
\caption{Dimensions of the Lorentz Invariant Expansion Coefficients for $\mathcal{L}_{GR,per}$.}\label{GRExp}
\end{table}
Here we separated the appearing constants in those that appear in front of terms including derivatives of the metric tensor field and those that appear in terms that do not include such. We adopt some terminology more present in QFT related subjects and refer to the former expressions as \textbf{\textit{kinetic terms}} ans the latter as \textit{\textbf{mass terms}}. The constants are then called \textit{\textbf{kinetic parameters}} $(\nu_1,...)$ and \textit{\textbf{masses}} $(\mu_1,...)$ respectively.   

According to algorithm \ref{Algo2} the next step consists of inserting these expansion coefficients and the previously computed expression for the constant tensor $C_{An}^{Bm}$ into the perturbative equivariance equations (\ref{order1}), (\ref{order2}) and (\ref{order3}). These are then linear equations for the unknown constants $\{ \mu_1,...,\mu_7\}$ and $\{\nu_1,...,\nu_{47}\}$. The solution we get from doing this by using our computer program is displayed in equation (\ref{GRSol}) in the appendix. Note that the expansion (\ref{LGR}) in generall features $15.906$ undetermined constants in the $9$ expansion coefficients. The restriction to those that define the components of Lorentz invariant tensor that is demanded by the perturbative equivariance equations reduces these to $54$ constants. Solving the perturbative equivariance equations then further decreases the number of undetermined parameters to $2$. Also, observe that in the solution, there is no contribution from terms that involve $\epsilon^{abcd}$. In fact, the constant parameters in front of expansion coefficients that contain the Levi-Civita symbol are the only ones that the equivariance equations set to zero.  

We continue with construction algorithm \ref{Algo2} by computing the perturbative expansion of the gravitational principal polynomial. Inserting the solution for the expansion coefficients (\ref{GRSol}) into (\ref{LGR}) now yields the most general diffeomorphism invariant perturbative Lagrangian. Normally we would now proceed by plugging the solved expansion coefficients into the perturbative EOM (\ref{EOMPert}) and then compute the corresponding expansion of the principal symbol and polynomial as outlined in the last chapter. It is, however, worth noting that for the simple case of the gravitational field being provided by a metric tensor it is actually not necessary to follow along with these general steps. We can, in fact, save ourselves much work by taking a closer look at equation (\ref{polyEqn}) that the principal polynomial has to satisfy as a consequence of the diffeomorphism equivariance. For the metric tensor field, this equation takes the particularly simple form:
\begin{align}\label{metricPoly}
    0 = \partial^A \mathcal{P}_{GR}^{abcd} C_{An}^{Bm} v_B - 4\mathcal{P}_{GR}^{(abc\vert m} \delta_n^{\vert d)} + 2 \mathcal{P}_{GR}^{abcd} \delta^m_n.
\end{align}
We now expand $\mathcal{P}_{GR}^{abcd}$ around $(N_{A}) = (x_0^m, \eta_A)$ and construct a power series solution to (\ref{metricPoly}). From before we know that given the expansion of the Lagrangian up to third order we only need to expand the polynomial up to first order and hence also the totally symmetric expression $\mathcal{P}_{GR}^{abcd}$ is expanded up to linear order. We get:
\begin{align}
    \mathcal{P}_{GR}^{abcd} = (P_0)_{GR}^{abcd} + (P_1)_{GR}^{abcdA} H_A + \mathcal{O}(2).
\end{align}
Evaluating equation (\ref{metricPoly}) and its first prolongation at the expansion point $\eta_A$ and contracting it against the Lorentz generators yields additional zero and first-order equations that state that the expansion constants must be the components of constant Lorentz invariant tensors. One readily finds that for the constant-order expansion coefficient, there exists exactly one such expression, namely:
\begin{align}
   (P_0)_{GR}^{abcd} = a \cdot \eta^{(ab} \eta^{cd)}. 
\end{align}
For the linear-order coefficients we obtain the following two possible expression:
\begin{align}
    (P_1)_{GR}^{abcdA} = I^A_{ef} \left (b \cdot \eta^{(ab} \eta^{cd)}  \eta^{ef} + c \cdot \eta^{(ab} \eta^{c \vert e} \eta^{f \vert d)} \right ).
\end{align}
Upon inserting the thereby obtained expansion in (\ref{metricPoly}) and evaluating at the expansion point $(N_A)$, we get:
\begin{align}
    b = a \ \ \text{and} \ \ c = -2a.
\end{align}
Inserting this into $\mathcal{P}_{GR}^{abcd}$ and contracting again against $k_a\cdot ...\cdot k_d$ in total we find the following expression for the most general linear-order expansion of the principal polynomial that is consistent with the diffeomorphism equivariance of $L_{GR,per}$:
\begin{align}
\begin{aligned}
    \mathcal{P}_{GR} &= a \cdot  \eta(k) \cdot \left [\eta(k) + H \cdot \eta(k)  -
    2 \cdot  \eta(k) H(k) \right ] + \mathcal{O}(2)\\
    &= a \cdot (1 + H) \cdot (\eta(k) - H(k))^2 + \mathcal{O}(2).
\end{aligned}
\end{align}
where  $H:=\eta^{ab} I^A_{ab} H_{A}$ and $H(k) := \eta^{ap} \eta^{bq} I_{pq}^A H_A k_a k_b$ and $\eta(k) := \eta^{ab}k_a k_b$. 
\begin{comment}
Taking a section $g \in \Gamma(F_{GR})$ and the corresponding inverse metric $g^{-1} \in \Gamma(F_{GR}^{\ast})$ with $(g^{-1})^{ab}=:g^{ab}$ and the usual relation $g^{ab}g_{bc}= \delta^a_c$ and evaluating the thus obtained most general expression for the principal polynomial we find
\begin{align}
    \mathcal{P}_{GR,per} (g_{ab}) = -a \cdot \mathrm{det}(g) g^{ab} g^{cd} k_ak_bk_ck_d + \mathcal{O}(2).
\end{align}
%check notation v vs g
\end{comment}
Note in particular that the vanishing set of this principal polynomial is entirely independent of the choice of the remaining parameter $a$. Thus the causal structure of the perturbative metric gravity EOM is already uniquely fixed by requiring diffeomorphism invariance.

Furthermore, one readily finds that the principal polynomial of the Klein-Gordon equation yields the following linear-order expansion:
\begin{align}
    \mathcal{P}_{KG} = (1 + H/2) \cdot (\eta(k) - H(k)) + \mathcal{O}(2) .
\end{align}
To relate the perturbative vanishing sets of $\mathcal{P}_{GR,per}$ and $\mathcal{P}_{KG}$ note that we can change the first factor of the two polynomials $(1+H)$ and $(1+H/2)$ respectively at wish by multiplying with the perturbative expansion of the following non-vanishing scalar density of weight $2n$:
\begin{align}
    f_{dens}(n) := \vert \mathrm{det}\left (I^A_{pq}v_A \right )\vert ^n = 1 + nH + \mathcal{O}(2).
\end{align}
This obviously does not alter the vanishing set described by either one of the polynomials. Thus multiplying the gravitational polynomial by $f_{dens}(-1)$ and dividing by the prefactor $a$ and multiplying the matter polynomial by $f_{dens}(-1/2)$ we find:
\begin{align}
    \mathcal{P}_{GR,per} = (\eta(k)-H(k))^2 = \mathcal{P}_{KG}^2.
\end{align}
Therefore the matter and gravitational polynomial $\mathcal{P}_{KG}$ and $\mathcal{P}_{GR,per}$ define the same vanishing set in linear order.
Hence the two theories are causal compatible, and our construction procedure is finished. We are left with a perturbative Lagrangian that contains $2$ undetermined constants describing a $2$ parameter family of perturbative metric gravity theories.  

As mentioned earlier in this section Lovelock has shown that the only diffeomorphism invariant metric theory of gravity with second-order EOM is provided by the \textit{\textbf{Einstein-Hilbert}} Lagrangian (\cite{Lovelock1969}, \cite{doi:10.1063/1.1665613}, \cite{doi:10.1063/1.1666069}) that reads:
\begin{align}
    L_{EH} := \kappa \sqrt{\vert \mathrm{det} \left ( I^A_{pq}v_A \right ) \vert }  \left( R + \Lambda \right ),
\end{align}
where the two constant parameters, $\kappa$, and $\Lambda$ are called gravitational constant and cosmological constant respectively. Here $R$ is the so-called Ricci scalar. As we have mentioned earlier in this thesis, one usually first constructs the Riemann tensor, the curvature  $2$-form of the unique metric compatible connection on the frame bundle over $M$. From this one obtains the Ricci tensor by computing its trace. Finally one contracts the Ricci-tensor with the inverse metric to end up with the Ricci scalar $R$. From the point of view of classical field theory, all the deeper connections to Riemannian geometry that are present in the above procedure are, however, actually not necessary. The only relevant fact one really needs is that the Ricci scalar defines a function on $J^2F_{GR}$. To that end recall that any non-degenerate metric defines a smooth bundle isomorphism:
\begin{align}\label{music}
\begin{aligned}
\flat : TM &\longrightarrow T^{\ast}M\\
X &\longmapsto g(X,-) .
\end{aligned}
\end{align}
Through this construction we therefore also obtain an isomorphism\footnote{At least if we restrict to the subbundles of non-degenerate, symmetric, $(0,2)$ tensors.} between the field bundle $F_{GR}$ and its dual $F_{GR}^{\ast}$ that with a slight abuse of notation will also be called $\flat$. 
Using the dual coordinates $(x^m,v^A)$ on $F_{GR}^{\ast}$ we define $v^A_{\flat} := v^A \circ \flat$. Thus applying $v^A_{\flat}$ on any metric exactly yields the components of its inverse. Using this we get the following explicit expression for $R$ understood as a map on $J^2F$ as:
\begin{align}
R = v_{\flat}^A J_A^{ab} \left ( D_p \Gamma^p_{ba} - D_b \Gamma^p_{pa} + \Gamma^p_{pq} \Gamma^q_{ba} - \Gamma^p_{bq} \Gamma^q_{pq} \right ),
\end{align}
where we also defined the Christoffel symbols that are given by the following functions on $J^1F_{GR}$:
\begin{align}
\Gamma^a_{bc} = \frac{1}{2} v_{\flat}^A J_A^{ap} \left ( I^B_{bq}\delta^p_b + I^B_{cq}\delta^p_c - I^B_{bc}\delta^p_q  \right ) v_{Bq}.
\end{align}
To compare the computed gravitational Lagrangian with the Einstein-Hilbert Lagrangian expand $L_{EH}$ to third-order around the chosen eta-induced expansion point $(N_{AI})$. The expansion of the determinant can be obtained from (\ref{detExp}). Doing so, we find that:
\begin{align}
   \sqrt{\vert \mathrm{det} \left ( I^A_{pq}v_A \right ) \vert } = 1 + H/2 +1/8 \cdot H^2 - 1/4 \cdot I^A_{ab}I^B_{cd} \eta^{ac} \eta^{bd} H_A H_B + \mathcal{O}(3).   
\end{align}
In order to compute the expansion of $v^A_{\flat}$ one uses that $-1/2 \cdot v_{\flat}^AC_{An}^{Bm}v_B = v_{\flat}^A J_A^{nb}I^B_{bm}v_B = \delta^m_n$. Deriving this equation w.r.t. $v_B$ and performing some simplifications one thus finds the usual expression:
\begin{align}
    \partial^Bv_{\flat}^A = 1/4 \cdot C_{Cn}^{Am}C_{Dm}^{Bn}v_{\flat}^C v_{\flat}^{D}.
\end{align}
Using this one obtains the following perturbative expansion for $v_{\flat}^A$:
\begin{align}
    v_{\flat}^A = \eta^A - \eta^{ap}\eta^{bq} I^B_{ab} I^A_{pq} \cdot H_B + \eta^{ac}\eta^{bd}\eta^{ef} I^B_{ab} I^C_{ce} I^A_{df} \cdot H_BH_C + \mathcal{O}(3).  
\end{align}
With these results computing, this expansion is straight forward yet quite laborious.
Starting with the linear-order of the Einstein-Hilbert Lagrangian, we find:
\begin{align}
        L_{EH} = \kappa \cdot (\eta^{ap}\eta^{bq} - \eta^{ab}\eta^{pq}) I^{A}_{ab}I^{I}_{pq} H_{AI} + \kappa \Lambda \cdot (1 + 1/2 \cdot \eta^{ab} I_{ab}^A H_A) + \mathcal{O}(2).
\end{align}
On the other hand inserting the explicit Lorentz invariant expressions for the expansion coefficients (\ref{LorentzGR1}) with the appropriate multiplicities from the factor free symmetrizations and the computed solution to the perturbative equivariance equations (\ref{GRSol}) into $\mathcal{L}_{GR,per}$, we find:
\begin{align}
    \mathcal{L}_{GR,per} = \mu_1(1 + 1/2 \cdot \eta^{ab} I_{ab}^A H_A) - 8 \nu_1 \left(\eta^{ap}\eta^{bq} - \eta^{ab}\eta^{pq} \right )I^{A}_{ab}I^{I}_{pq} H_{AI} + \mathcal{O}(2).
\end{align}
Thus we see that the obtained expansion for $\mathcal{L}_{GR,per}$ up to linear order precisely yields the Einstein-Hilbert Lagrangian with gravitational constant $\kappa = -8 \nu_1$ and cosmological constant $\Lambda = -1/8 \cdot \frac{\mu_1}{\nu_1}$. Carefully proceeding with the comparison one then finds that also in second and third order the computed perturbative Lagrangian exactly coincides with the expansion of $\mathcal{L}_{EH}$, once we identify the constants as described above. 
Concluding, we constructed a $3$rd-order perturbative expansion of the most general metric gravity Lagrangian that is compatible with a Klein-Gordon scalar field. By employing the perturbative construction algorithm (\ref{Algo2}) and the developed computed algebra (see chapter \ref{computerAlg}) this procedure was really straight forward. 
Further, we saw that, except for the restriction to EOM with principal symbol depending only on the metric, not its derivatives, for this particular case the requirement of causal compatibility did not yield further conditions on top of the imposed diffeomorphism invariance on the metric theory of gravity.
Already the requirement of diffeomorphism invariant dynamics sufficed to recover the perturbative expansion of the Einstein-Hilbert Lagrangian.

\section{Perturbative Area Metric Gravity around a Flat Minkowski Induced Spacetime}
Whereas the first example theory we constructed mainly served the purpose of testing our framework the perturbative theory of gravity that we are going to construct now really is intended to provide a viable alternative to GR in describing the geometry of our spacetime.  
To that end, we are going to construct a perturbative theory of gravity that is compatible with the most general linear theory of Electrodynamics. 

We start by providing some results from the premetric treatment of Electrodynamics that can be found in \cite{1999PhLB..458..466O}, \cite{1999gr.qc....11096H}, (\cite{hehl2003foundations}, \cite{2006physics..10221H}, \cite{2004PhRvD..70j5022L} and also \cite{Hehl2005}). The basic idea of this premetric approach lies in describing all fields that are involved in the treatment of Electrodynamics without referring to an additional spacetime geometry. This is most easily illustrated using the calculus of exterior forms. One starts by introducing the 4-dimensional electric charge current density $J$. As one wishes to integrate $J$ over 3 dimensional submanifold $\Omega$ of the spacetime $M$ to obtain the total electric charge current --- the total electrical current flowing through the boundary of $\Omega$ and the electric charge inside --- that is confined therein the current density is most naturally described as a 3-form $J \in \Gamma(\Lambda^3M)$. Demanding charge conservation in the sense that the integral of $J$ over any closed 3-dimensional submanifold of $M$ vanishes, by using the theorem of stokes on arrives at the requirement $dJ =0$. By de Rham's first theorem we then find that $J$ is necessarily given as exterior derivative of some 2-form $H \in \Gamma(\Lambda^2M)$:
\begin{align}
    J = d H.
\end{align}
From considerations involving the Lorentz force one similarly obtains that the magnetic analog to $J$ the field strength $F$ which plays the role of a 4-dimensional magnetic flux current density should be described by a 2 form $F \in \Gamma(\Lambda^2M)$. Its integral over a 2 dimensional submanifold $C \subset M$ describes the total magnetic flux current in terms of the total magnetic flux flowing through $C$ and the total magnetic flux current\footnote{Note that by this interpretation rewriting $F$ in terms of a 3+1 split of spacetime and the usual $E$ and $B$ field the magnetic flux is as usual given by $B$ whereas $E$ is now interpreted as the corresponding magnetic flux current (see \cite{2006physics..10221H}).} flowing through the boundary of $C$. The corresponding conversation law deduced in similar fashion as before then reads $dF =0$. Thus we conclude using again de Rham's first theorem the existence of a 1 form $A \in \Gamma(\Lambda^1M)$ that satisfies:
\begin{align}
    F = d A.
\end{align}
Finally one assumes that the two 2-forms $F$ and $H$ are related via a local and linear spacetime relation. Writing the 2 forms in terms of some chart induced basis $F = F_{ab} = \mathrm{d}x^a \otimes \mathrm{d}x^b$ with $F_{ab} = - F_{ba}$ and similar for $H$ specifying such a relation requires an additional $(0,4)$ tensor field $G$ that obeys the following symmetries:
\begin{align}\label{areaSym}
    G_{abcd} = -G_{bacd} = G_{cdab}.
\end{align}
Note that such a tensor field at each spacetime point $p\in M$ defines a symmetric inner product on the space of $2$-forms $\Lambda^2_pM$. Consequently we will call such a tensor field an \textbf{\textit{area metric}} tensor field and denote the bundle of these area metric tensors by $F_{Area}$. Due to (\ref{areaSym}) one need $21$ independent component functions to uniquely specify such an area metric. Thus $F_{Area}$ has $21$ dimensional fibers.   In the following we restrict the bundle of such area metric tensor fields $G$ by requiring that the inner product they define is non degenerate for all $p$ in the sense that $G$ provides us with a bundle isomorphism: 
\begin{align}
\flat_{Area} : \Lambda^2M \longrightarrow (\Lambda^2M)^{\ast} = A^2_0M,
\end{align}
that is defined exactly the same way as in the metric case (\ref{music}). Here $A^2_0M$ denotes the bundle of antisymmetric $(2,0)$ tensors. 

The linear relation between $H$ and $F$ can now be obtained by first using the isomorphism provided by the area metric to map the 2-form $F$ to an antisymmetric $(2,0)$ vector field and then applying the Hodge star operator\footnote{Details regarding the construction and properties of the Hodge start operator can be found in \cite{Abraham:1988:MTA:50877}, and also in \cite{nlab:Hodge}.} to again end up with a 2 form. In coordinates we therefore get:
\begin{align}
    H_{ab} = 1/2 \omega_G \epsilon_{abcd} G^{cdef} F_{ef},
\end{align}
Note that in absence of a (pseudo)-Riemanian metric the Hodge star operator requires the choice of a volume form (or equivalently a scalar density of weigh 1) $\omega_G$. One possible such choice is for instance given by $\omega_G = \epsilon^{abcd}G_{abcd}$ which obviously only works if this expression is nowhere vanishing.

From considering the appropriate dimensions of $F$ and $H$ on finds that the only viable Lagrangian constructed from $F$ and $H$ is given by $F \wedge H$. Expressing this in terms of the constitutive relation between $H$ and $F$ we finally arrive at the Lagrangian of so-called \textbf{\textit{General Linear Electrodynamics}} (in short GLED):
\begin{align}
    \mathcal{L}_{GLED} = \omega_G G^{abcd}F_{ab}F_{cd}\mathrm{d}^4x.
\end{align}
Note that this really defines the most general quadratic Lagrangian which thus generates linear EOM that one can specify in terms of the electromagnetic field strength 2-form $F$. In particular this Lagrangian contains the standard Maxwell electrodynamics on a metric background for the special case $G^{abcd} = 2 g^{c^[a}g^{b]d}$ and $\omega_{G}=\sqrt{\vert -\mathrm{det}(g) \vert}$.

Just as it was the case for the Klein-Gordon Lagrangian for the Lagrangian of GLED there exist now two possibilities: either one specifies the area metric tensor field by hand, or --- and this is the route that we want to take --- one supplements the GLED matter theory with a suitable theory of area metric gravity that then allows one to determine the area metric by solving the corresponding area metric gravity EOM. 
In the following, we are going to construct the perturbative expansion of such a theory of are metric gravity by following our construction recipe algorithm \ref{Algo2}.

To that end we quickly want to translate $\mathcal{L}_{GLED}$ into the language of jet bundles. Note that the electromagnetic field strength can be expressed in terms of the 1-form $A \in \Gamma(\Lambda^1(M))$ as:
\begin{align}
F = F_{ab} \cdot  \mathrm{d}x^a \otimes \mathrm{d}x^b = 2 \partial_{[a} A_{b]} \cdot \mathrm{d}x^a \otimes \mathrm{d}x^b.
\end{align}
Thus $F$ can be understood as a function on the first-order jet bundle $J^1(\Lambda^1M)$. We again use adapted coordinate $(x^m,A_a,A_{am})$ on this bundle. Together with the area metric field bundle $F_{Area}$ and the bundle isomorphism $\flat_{Area}$ we thus see that the GLED Lagrangian describes a bundle map:
\begin{align}
\mathcal{L}_{GLED} : F_{Area} \times J^1(\Lambda^1M) \rightarrow \Lambda^4M,
\end{align}
where now also $\omega_G$ is understood as a bundle map on $\Lambda^1M$.  

We start the derivation of perturbative area metric gravity by constructing the pair of intertwiners for the area metric field bundle $F_{Area} \subset T^0_4M$. Recall from chapter 1 that given coordinates $(x^m)$ on $M$ one can obtain a pair of intertwiners for such a subbundle of $T^0_4M$ by dividing the set of the $4^4$ fiber coordinates of $T^0_4M$ $v_{abcd}=\frac{\partial}{\partial x^a} \otimes ... \otimes \frac{\partial}{\partial x^d}$ in equivalence classes defined by the symmetries (\ref{areaSym}) sorting them according to some order relation and then constructing the pair of intertwiners as described in (\ref{defI}) and (\ref{defJ}). We will use the following ordered set of $21$ equivalence classes $[v]$ that are specified in terms of their representative $v$ as:
\begin{align}
\begin{aligned}
    \bigl [ [v_{0101}], [v_{0102}], [v_{0103}], [v_{0112}], [v_{0113}], [v_{0123}], [v_{0202}], [v_{0203}], [v_{0212}], [v_{0213}], [v_{0223}],\\
     [v_{0303}], [v_{0313}], [v_{0323}], [v_{1212}], [v_{1203}], [v_{1213}], [v_{1223}], [v_{1313}], [v_{1323}], [v_{2323}]  \bigr ].
\end{aligned}
\end{align}
We label the equivalence classes by indices $A$ that now run from $0$ to $20$.
There are now two possibilities for the individual classes: either we have an equivalence class with representative of the form $v_{ijij}$, i.e., the equivalence class consists of coordinate functions that only have two distinct index values $i$ and $j$. Such a class then consists of the following $4$ elements:
\begin{align}
    [v_{ijij}] = \{ v_{ijij}, v_{jiji}, -v_{ijji}, -v_{jiij} \}.
\end{align}
Thus the multiplicity $\sigma$ is 4 for these classes.
The other possibility is given when the representative of the equivalence class takes the general form $v_{ijkl}$ where $i<j$ or $(i=j) \land (k<l)$. The equivalence class consists then of 8 elements:
\begin{align}
    [v_{ijkl}] = \{v_{ijkl},v_{jikl}, v_{klij}, v_{lkji}, -v_{jikl}, -v_{ijlk}, -v_{lkij}, v_{klji} \}.
\end{align}
Such classes have multiplicity 8. The intertwiners $I^A_{abcd}$ and $J_A^{abcd}$ can now be constructed according to (\ref{defI}) and (\ref{defJ}). The explicit non-vanishing components of these intertwiners are displayed in the appendix (\ref{AreaI}) and (\ref{AreaJ}). Using these we define the new adapted coordinate functions on $F_{Area}$ and its dual $F_{Area}^{\ast}$:
\begin{align}
    \begin{aligned}
    v_A = J_A^{abcd}v_{abcd} \\
    v^A = I^A_{abcd}v^{abcd}.
    \end{aligned}
\end{align}
As usual we are going to label the corresponding adapted coordinates on $J^2F_{Area}$ by $(x^m,v_A,v_{Ap},v_{AI})$. 
Note that from the fact that the fiber dimension of $F_{Area}$ equals $21$ whereas $F_{GR}$ only has fiber dimension $10$ one readily sees that the area metric indeed compromises a richer structure than the metric. It is, however, not a priory clear if this is also reflected in the corresponding theory of gravity. Recall that the metric allowed for $14$ functionally independent curvature scalars on $J^2F_{GR}$ and hence also for $14$ functionally independent diffeomorphism invariant Lagrangians. With theorem (\ref{GeneralSol}) at hand, we can now immediately compute the corresponding quantity for the area metric. Doing so we find that there exist $179$ functionally independent diffeomorphism invariant Lagrangians that can be constructed from the area metric and its first two derivatives. Thus, in general, we expect the perturbative theory of area metric gravity to contain more undetermined parameters than the gravitational and cosmological constant that we obtained for the metric case.


Next we are going to chose the $\eta$-induced expansion point $(N_{AI}) = (x_0^m, N_A, 0,0)$ to be provided by:
\begin{align}
N_A := 2 \eta_{c[a} \eta_{b]d} - \epsilon_{abcd}.
\end{align}
Not that assuming that $\omega_G(N_A) = const \neq 0$ inserting this into $\mathcal{L}_{GLED}$ one really obtains the Lagrangian of standard Maxwell Electrodynamics on a flat Minkowskian background:
\begin{align}
    \mathcal{L}_{GLED}\bigl((N_{AI})\bigr ) = 2 \omega_G(N_A) \eta^{ac}\eta^{bd}F_{ab}F_{cd} \mathrm{d}^4x.
\end{align}
Thus expanding the to-be-constructed Lagrangian of area metric gravity around $(N_{AI})$ serves the purpose of interpreting the result as perturbative theory around a flat spacetime. Note that we included the totally antisymmetric $\epsilon_{abcd}$ in the definition of $N_A$ to ensure that  the possible density choice $\omega_G = \epsilon^{abcd}G_{abcd}$ does not vanish when evaluated at the flat $\eta$-induced background. 

We have already stated that we want to construct the perturbative are metric gravity theory up to order $k=3$. Following algorithm \ref{Algo2}, we proceed by computing the coordinate expression of the Lie derivative of $G_{abcd}$ to read of the vertical coefficient tensor $C_{An}^{Bm}$:
\begin{align}\label{LieArea}
    \mathcal{L}_{\xi} G_{abcd} = \partial_m G_{abcd} \cdot \xi^n + \left (-4 \delta_n^{[e\vert} \delta_{[\mathnode{a}}^m \delta_{b]}^{\vert \mathnode{f} ]} \delta_{[c}^{[g} \delta_{\mathnode{d}]}^{\mathnode{h}]} \right ) \cdot G_{efgh} \partial_m \xi^n.
\end{align}
\begin{tikzpicture}[overlay]
   \path [>=stealth, <->, shorten <= 3pt, shorten >=3pt]
     (a) edge [bend left=-80] (d);
    \path [>=stealth, <->, shorten <= 2pt, shorten >=2pt]
     (f) edge [bend left=80] (h);
\end{tikzpicture}

Here the additional arrows denote the symmetrization w.r.t. the interchange of the antisymmetric pairs $[ab] \leftrightarrow [cd]$ and $[ef] \leftrightarrow [gh]$ respectively.

Similar to before we now insert several intertwiners $I^A_{abcd}$ and $J_A^{abcd}$ to rewrite the second term in (\ref{LieArea}) in terms of abstract area metric indices $A$. Here we can use the fact that the two intertwiners already feature the full set of area metric symmetries.  Doing so we simply can read of the constant tensor as:
\begin{align}\label{areaGotayMInter}
    C_{An}^{Bm} = -4 I^B_{nbcd} J_A^{mbcd}.
\end{align}
The most general power series expansion of $\mathcal{L}_{Area,per} = L_{Area,per}\mathrm{d}^4x$ that generates second-derivative-order EOM is again given by:
\begin{align}\label{LArea}
    L_{Area,per} =  a_0 + a^A H_A + a^{AI}H_{AI} + a^{AB} H_{A}H_{B} + a^{ApBq} H_{Ap}H_{Bq} + a^{ABI} H_{A} H_{BI} \\
    + a^{ABC} H_a H_B H_C + a^{ABpCq} H_{A}H_{Bp}H_{Cq} +
    + a^{ABCI} H_A H_B H_{CI} 
    + \mathcal{O}(4),
\end{align}
where as before we introduced the coordinate deviation from the expansion point: 
\begin{align}
(H_{AI}) = (x^m,H_A,H_{Ap},H_{AI}) = (v_{AI}) - (N_{AI}).
\end{align}
Note that although we use the same symbols as in $L_{GR,per}$ to denote $L_{Area,per}$ this Lagrangian defines a fundamentally different object. In particular the $21$ dimensional fibers of  $F_{Area}$ result in the fact that there will be more freedom in constructing the expansion coefficients of (\ref{LArea}). Again the perturbative equivariance equations demand that these expansion coefficients define the components of constant Lorentz invariant tensors. The corresponding expressions that we obtained by means of our computer program are displayed in the appendix in table \ref{LorentzArea}. The dimensions and parameters of the individual expansion coefficients are displayed in table \ref{AreaExp}.
\begin{table}
\centering 
\begin{tabular}{lll} \toprule
    expansion coefficient & dimension & constants   \\ \midrule
    $a_0$ & 1 & $\{\mu_1\}$ \\
    $a^A$ & 2 & $\{\mu_2,\mu_3\}$ \\
    $a^{AI}$ & 3 & $\{\nu_1,..., \nu_3\}$ \\
    $a^{AB}$ & 6 & $\{\mu_4,..., \mu_9 \} $ \\
    $a^{ApBq}$ & 15 & $\{\nu_4,...,\nu_{18}\}$ \\
    $a^{ABI}$ & 16 & $\{ \nu_{19},...,\nu_{34} \}$ \\
    $a^{ABC}$ & 15 & $\{ \mu_{10},...\mu_{24} \}$\\
    $a^{ABpCq}$ & 110 & $\{\nu_{35},...,\nu_{144} \}$ \\
    $a^{ABCI}$ & 72 & $\{ \nu_{145},...,\nu_{216}\}$ \\ \bottomrule
\end{tabular}
\caption{Dimensions of the Lorentz Invariant Expansion Coefficients for $\mathcal{L}_{Area,per}$.}\label{AreaExp}
\end{table}
In total, we thus get from solving this first implication of the perturbative equivariance equations, the Lorentz invariance of the expansion coefficients, $240$ yet undetermined constant parameters. These can be divided into $24$ masses and $216$ kinetic parameters.

It is worth noting that the perturbative equivariance equations (\ref{order1}), (\ref{order2}) and (\ref{order3}) decouple into the two subsystems containing only the mass term and kinetic term coefficients, respectively. Once we insert the obtained Lorentz invariant expression, the former yields a linear system that is compromised of $14$ independent equations for the $24$ masses. The latter contains $174$ independent equations for the $216$ kinetic parameters. Thus in total, we get $188$ independent linear equations that therefore reduce the number of undetermined parameters that are contained in $\mathcal{L}_{Area,per}$ from $240$ to $52$. We get a total of $10$ masses and $42$ kinetic parameters. The solution to the linear system in terms of the $52$ parameters (\ref{AreaParas}) is displayed in the equations table \ref{AreaSol}. Although the perturbative Lagrangian that we thus obtain is clearly more complicated then the expansion of the Einstein-Hilbert action and in particular now contains $52$ undetermined parameters compared to $2$ gravitational constants that contribute to $\mathcal{L}_{GR,per}$ we can now see the real strength of the required diffeomorphism invariance that is encoded in the perturbative equivariance equations (\ref{order1}), (\ref{order2}) and (\ref{order3}). To that end, it is vital to observe that a priori the expansion (\ref{LArea}) allows for a total of $133694$ undetermined parameters. It is remarkable that solely the requirement of diffeomorphism invariance reduces this number to the $52$ parameters that are left in the obtained solution. 

Following algorithm \ref{Algo2} we now compute the principal polynomials of the EOM generated from $\mathcal{L}_{GLED}$ and $\mathcal{L}_{Area,per}$ to deduce whether the two theories are already causally compatible or this second requirement yields further conditions for the $52$ gravitational constants in the perturbative area metric gravity Lagrangian. This time we start by computing the GLED polynomial. It was first computed by Rubilar (see \cite{2009JPhA...42U5402I}) in the context of premetric Electrodynamics and is given by the following expression with corresponding first-order expansion: 
\begin{align} \label{GLEDPoly}
\begin{aligned}
    \mathcal{P}_{GLED} &= -\frac{1}{24}\omega_G^2\epsilon_{mnpq}\epsilon_{rstu}J_A^{mnra}J_B^{bpsc}J_C^{dqtu} v_{\flat_{Area}}^A v_{\flat_{Area}}^B v_{\flat_{Area}}^C k_ak_bk_ck_d \\
                &= \bigl[ 1 -  A \eta(H)- \frac{1}{2} A \epsilon(H) + \frac{1}{12} \epsilon(H) \bigr] \eta(k)^2 - H(k)\eta(k) + \mathcal{O}(2)\\
                &= \bigl\{  \bigl[ 1 - \frac{1}{2} A \eta(H) - \frac{1}{4} A \epsilon(H) +  \frac{1}{24} \epsilon(H) \bigr] \eta(k) - \frac{1}{2} H(k)       \bigr\}^2 + \mathcal{O}(2).
\end{aligned}
\end{align}
Here we defined $v_{\flat_{Area}}^A = v^A \circ \flat_{Area}$ and used $v_{\flat_{Area}}^A J_A^{abcd} I^B_{cdef}v_B = 4 \delta^{[a}_e \delta^{b]}_f$ to expand $v^A_{\flat_{Area}}$ around $N_A$. Further, we defined:
\begin{align}
\begin{aligned}
\eta(H) &:= \eta^{ac}\eta^{bd} I^A_{abcd} H_A, \\ 
\epsilon(H) &:=\epsilon^{abcd}I^A_{abcd}H_{A},\\ H(k) &:=\eta^{ac}\eta^{bp}\eta^{cq} I^A_{abcd}H_Ak_pk_q, \\ \eta(k)&:=\eta^{pq}k_pk_q.
\end{aligned}
\end{align}
Besides, $A$ denotes a constant that depends on the specific choice of density $\omega_G$ in the GLED Lagrangian.  

We now compute the principal polynomial of the constructed diffeomorphism invariant perturbative theory of area metric gravity. To that end, we start by inserting the calculated solution of the expansion coefficients in to (\ref{EOMPert}) to obtain the explicit expression of the perturbative EOM. From this, we can simply read off the principal symbol. Following further along the lines described in the last chapter, we find the following expression for the area metric gravity polynomial:
\begin{align} \label{AreaPoly}
\begin{aligned}
    \mathcal{P}_{Area} &= \bigl[1 - \frac{1}{2} \tilde{C}\eta(H) - \frac{1}{4} \tilde{C} \epsilon(H) + \frac{7}{12} \epsilon(H) \bigr] \eta(k)^{13} - \frac{13}{2}H(k) \eta(k)^{12} + \mathcal{O}(2) \\
    &=\bigl\{  \bigl[ 1 - \frac{1}{2} C \eta(H) - \frac{1}{4} C \epsilon(H) +  \frac{7}{12\cdot13} \epsilon(H) \bigr] \eta(k) - \frac{1}{2} H(k)       \bigr\}^{13} + \mathcal{O}(2).
\end{aligned}
\end{align}
Here $C$ is a constant the depends on the parameters that are left in (\ref{LArea}) and $\Tilde{C} = 13 C$.
There is a slight caveat in the computation of this expression for the area metric gravity polynomial. 
To explain this further first note that since there still appear $52$ constant parameters in the area metric gravity Lagrangian in a rather complicated fashion, it is virtually impossible to derive the principal polynomial without employing efficient computer algebra. Once we have selected a $17 \times 17$ submatrix of the $21 \times 21$ principal symbol matrix we have to compute the two expressions, $D_0$ and $D_{1}^C$ in (\ref{polyMatrices}) in order to find the constant and linear order of the expansion of the principal polynomial. Note that the involved matrices no longer depend on $H_A$. To compute $D_0$ and $D_1^A$ we now need the determinant and the inverse of the constant order $T_0$ of the chosen $17 \times 17$ symbol submatrix. These two steps precisely represent the bottleneck of the computation of the polynomial. Although $T_0$ does not depend on $H_A$, it still includes some of the $52$ parameters, namely precisely those that are left in $a^{ApBq}$ and $a^{ABI}$ after solving the equivariance equations and contribute to the principal symbol. We find that these still consist of $7$ parameters. Although this might not sound much computing, the determinant or the inverse of a $17 \times 17$ matrix with $7$ symbolic parameters that appear in rather complicated fashion turns out to be surprisingly hard. We tried to tackle this problem by using a fraction free implementation of Gaussian elimination in the computer algebra system Maple\footnote{We expect that one might achieve even superior performance in computing the determinant by using techniques from multivariate polynomial interpolation as it is for instance suggested in \cite{Qin2018}, \cite{MARCO2004749} and also \cite{articleDet}} \cite{Maple}.

Nevertheless, with the available resources, the program did not terminate in adequate time. Thus the only way we nevertheless were able to obtain an expression for the polynomial was by evaluating the $7$ parameters that appear in $T_0$ at randomly generated integer values and then only computing the principal polynomial with the remaining parameters from the two third-order coefficients $a^{ABpCq}$ and $a^{ABCI}$. Taking a closer look at (\ref{polyMatrices}) these parameters only appear in traces, not in determinants or inverses. We then proceeded by performing this computation to obtain the principal polynomial several times, each time taking newly obtained random integers for the $7$ parameters in $T_0$. Doing so, we observed that they only contribute an overall factor to the principal polynomial (\ref{AreaPoly}). We have to admit that this was not done often enough to rigorously interpolate the precise way the prefactor depends on these $7$ parameters. This is however not necessary as we are only interested in the vanishing set of the area metric gravity polynomial. 

Note that, as we are only interested in the vanishing set of the principal polynomial, we can multiply a given polynomial with a non-vanishing scalar density of arbitrary weight. The expansion to linear order of such a density admits the general form:
\begin{align}\label{dens}
\omega = 1+ b_1 \cdot (\eta(H) + \frac{1}{2} \epsilon(H)) + \frac{b_2}{12}\epsilon(H) + \mathcal{O}(2),
\end{align}
for arbitrary constants $b_1$ and $b_2$.
Multiplying  (\ref{GLEDPoly}) with such a density we see that perturbatively the GLED polynomial describes the same vanishing set as:
\begin{align} \label{GLEDPoly2}
\begin{aligned}
    \widetilde{\mathcal{P}}_{GLED}(k) = \omega \cdot \mathcal{P}_{GLED} = 
    \bigl\{  \bigl[ 1 - \frac{1}{2} (A-b_1) \eta(H) - \frac{1}{4} (A-b_1) \epsilon(H)) + \\ \frac{1+b_2}{24} \epsilon(H) \bigr] \eta(k)
    -\frac{1}{2} H(k)       \bigr\}^2 + \mathcal{O}(2).
\end{aligned}
\end{align}
Comparing this expression with the previously obtained area metric gravity polynomial (\ref{AreaPoly}), we find that multiplying the GLED polynomial with a density with constants $b_1 = A -C$ and $b_2 = \frac{1}{13}$, the two polynomials are products of the same factor and therefore, in particular, describe the same vanishing set in $\mathcal{O}(2)$. Thus also for the case of area metric gravity requiring causal compatibility with the GLED mater theory in addition to diffeomorphism invariance at least perturbatively yields no further conditions for the parameters of $\mathcal{L}_{Area,per}$.

Finally, we wish to investigate if this was, in fact, a coincidence of the chosen GLED matter theory or a general feature of perturbative area metric gravity independent on the specific matter theory. To this end, we take a closer look at the consequences of the required diffeomorphism invariance on the causal structure of the corresponding area metric gravity EOM by studying general solutions to (\ref{polyEqn}). This equation now takes the following form:
\begin{align}\label{AreaPolyEqn}
    0 = \partial^A \mathcal{P}_{Area}^{p_1...p_{26}} C_{An}^{Bm} v_B - 26\mathcal{P}_{Area}^{(p_1...p_{25}\vert m} \delta_n^{\vert p_{26})} + 13 \mathcal{P}_{Area}^{p_1...p_{26}} \delta^m_n.
\end{align}
Proceeding as before for the metric in computing the general power series solution up to linear order in $H_A$ for this equation and contracting it against $k_{p_1} \cdot k_{p_{26}}$ we get the most general linear-order expansion of $\mathcal{P}_{Area}$ that is consistent with the required diffeomorphism invariance:
\begin{align}
\begin{aligned}
    \mathcal{P}_{Area} &= a \cdot \bigl \{  \eta(k)^{13} + \tilde{b} \cdot \eta(H) \cdot \eta(k) - \frac{13}{2} \cdot  H(k) \cdot \eta(k)^{12} + (\frac{\tilde{b}}{2}+\frac{14}{24}) \cdot \epsilon(H) \eta(k)^{13}  \bigr \} + \mathcal{O}(2)\\
    &=a \cdot \bigl\{  \bigl[ 1 + b \eta(H) + \frac{b}{2} \epsilon(H) +  \frac{7}{12\cdot13} \epsilon(H) \bigr] \eta(k) - \frac{1}{2} H(k)       \bigr\}^{13} + \mathcal{O}(2),
\end{aligned}
\end{align}
with constants $a$, $b$ and $\tilde{b} = 13b$. It is clear that the overall constant $a$ does not influence the vanishing set. Additionally we can multiply this expression with with the general form of a non-vanishing scalar density of arbitrary weight (\ref{dens}) that in linear order allows for $2$ arbitrary parameters $b_1$ and $b_2$. Doing so we find:
\begin{multline}
    \tilde{\mathcal{P}}_{Area} = \bigl\{  \bigl[ 1 + (b+ b_1) \eta(H) + \frac{b+b_1}{2} \epsilon(H) +  (\frac{7}{12\cdot13}+b_2) \epsilon(H) \bigr] \eta(k) - \frac{1}{2} H(k)       \bigr\}^{13} \\
    + \mathcal{O}(2).
\end{multline}
This polynomial now obviously perturbatively describes the same vanishing set. However, now we see that we can, in fact, specify the remaining constant $b$ at wish just by multiplying with an appropriate density and hence in particular without changing the vanishing set. 
Thus the fact that the requirement of causal compatibility between are metric gravity and GLED let to no further condition was no coincidence but a general feature of the perturbative expansion of area metric gravity
We conclude that for the case of the third-order perturbative area metric gravity Lagrangian already the required diffeomorphism invariance determines the vanishing set of the principal polynomial and thereby the causal structure if the theory uniquely. 

Hence the construction procedure is completed. The most general meaningful third-order expansion of the Lagrangian of area metric gravity thus involves $52$ undetermined parameters. Its explicit expression can be obtained by inserting the relations between the parameters displayed in table \ref{AreaSol} together with the computed Lorentz invariant expansion coefficients in the third-order expansion (\ref{LArea}). It is of great importance to be aware of the meaning of this result. This really compromises the most general perturbative, diffeomorphism invariant, second-derivative-order theory of gravity that is compatible with a linear theory of Electrodynamics. As such, it is a particularly exciting candidate to test against nature, for instance, by predicting gravitational wave emission\footnote{Such predictions will be discussed in \cite{NilsPHD}.} of some known binary system and then comparing this prediction with experiments. It is essential to note that such a prediction of the emission of gravitational waves really requires the third order of the gravitational Lagrangian. Although often stated differently, this is already true for standard GR (see for instance \cite{1984grra.book.....S} chapter 4.5).

Summing up the results that we developed in this chapter, not only did our perturbative Constructive gravity framework pass its first test in successfully recovering the third-order expansion of the Einstein-Hilbert Lagrangian, with the perturbative expansion of area metric gravity we also have developed the most general perturbative theory of gravity that is consistent with linear Electrodynamics. Although the expansion of such a theory to second order in the Lagrangian has already been computed in \cite{2017arXiv170803870S}, the expansion to third order that we derived here for the first time provides access to the computation of gravitational wave in such a theory of gravity and thereby allows us to concretely test the area metric against measurements. For precisely that reason, the computation of the emission of gravitational waves should compromise the main focus of future developments in this area of research. 

\chapter{Computer Algebra}\label{computerAlg}
\dictum{
After providing the reader with a concise introduction to functional programming in Haskell, focusing on the main differences with more traditional imperative programming languages, we proceed by illustrating the developed Haskell tensor algebra library sparse-tensor in detail. Not only will we explain the basic functionality contained therein, but the following chapters are also intended as a short introduction manual for potential future users. Finally, we also clarify how the construction of Lorentz invariant basis tensors is implemented in sparse-tensor.
The sparse-tensor library is available via \cite{sparse-tensor}.}   

\section{A Concise Introduction to Functional Programming in Haskell}
For the following chapter to constitute a self-contained description of the developed computer algebra also for readers that are unfamiliar with the programming language Haskell \cite{Marlow_haskell2010} we now furnish a quick introduction to it. We mainly treat its particularities compared with more traditional imperative languages.
A more in-depth description of the Haskell programming language, including 
further details can be found in any good textbook as for instance in \cite{Thompson99thecraft}, \cite{bird_2014}, \cite{hutton_2007}. Introductions to Haskell that rather resemble user guides aiming to allow the reader to write his own programs rapidly can be found in  \cite{OSullivan2008} and  \cite{Lipovaca:2011:LYH:2018642}. Much information is also collected in the HaskellWiki \cite{wiki:xxx}.

Compared with traditional imperative programming languages such as $C$, the most striking difference of Haskell lies in the fact that it employs a \textbf{\textit{purely functional}} programming paradigm. In imperative languages, a computer program consists of a sequence of commands for the computer to execute step by step. Thus not only the individual steps but also the specific order of their execution, the control flow, is explicitly provided by the programmer. Typically this is achieved by using control structures such as loops or conditional statements. At each time during its execution, the computer program can be said to occupy a known state that can be described by the explicit information that is stored in the entirety of its variables. A single computation step then precisely tells the program how to transfer from a given state to the subsequent one. This corresponds to a change in the values of some of the variables that describe the program's state. Apart from invoking the desired modifications of its variables that are necessary for the specific computation additionally such a computer program might undesirably change values of further variables. This is referred to as \textit{\textbf{side effects}}. As a consequence executing such a computer program that contains side effects multiple times might not return the same result each time as side effects that possibly occur during the execution can alter the outcome. Needless to say that unexpected side effects compromise a broad source of potential errors and are hence best avoided.

One way of avoiding side effects is by employing a functional programming paradigm. In contrast to an imperative programming style, a computer program no longer consists of a step by step instruction of state changes but is given by a mathematical function that maps some input data to the desired output. The execution of the program then simply corresponds to the evaluation of this function. As such a function obviously only depends on the data it is given as input, it is completely free of side effects.

In practice, one can build a computer program using such a functional programming style by composing it of several individual functions that describe sub-parts of the program via usual function composition. Besides of composition in most functional languages functions of multiple variables can also be  \textit{\textbf{curried}} to obtain an equivalent higher-order function that only takes a single argument but now yields a function as a result:
\begin{center}
    \mintinline{haskell}{f :: (A,B) -> C} $\ \xrightarrow{ \ currying \ } \ $
    \mintinline{haskell}{f' :: A -> (B -> C)}
\end{center}
Evaluating the curried function \mintinline{haskell}{f'} on some value \mintinline{haskell}{a :: A} simply corresponds to the partial evaluation of the original function \mintinline{haskell}{f}, i.e., leaving free the second slot of \mintinline{haskell}{f} and thus returning a function that maps from \mintinline{haskell}{B}  to \mintinline{haskell}{C}. Note that currying really establishes an equivalence between functions that take multiple arguments and higher-order functions of a single argument.
In fact in Haskell all functions take precisely one argument, i.e., functions that take several arguments are always represented in curried form. The general form of a function definition in Haskell reads:
\begin{center}
\begin{cminted}{haskell}
f :: A -> (B -> C)
f a b = ...  
\end{cminted}
\end{center}
Here
\mintinline{haskell}{f a b = (f a) b} denotes the evaluation of the function \mintinline{haskell}{f} first on \mintinline{haskell}{a} and then applying the resulting function to \mintinline{haskell}{b}. Note that the function evaluation is left associative.
In the above, we already used the second remarkable feature of the  Haskell programming language to some extent, the Haskell \textbf{\textit{type system}}. Every Haskell expression has a type that is denoted by \mintinline{haskell}{expr :: type}. Hence the first line in the above function definition simply declares   the type of the function \mintinline{haskell}{f :: A -> (B -> C)}.
Function evaluation behaves type sensitive, i.e., we might only evaluate a given function on a specific value if the type of the value matches the type of the function.
Note that whereas the evaluation of functions behaves left-associative, the corresponding types are right-associative. The type of the above function can thus also be written as \mintinline{haskell}{f :: A -> B -> C}.

New data types can be defined by specifying \textbf{\textbf{data constructors}}, functions that describe how the new data type is constructed in terms of known data types. We can illustrate this by constructing a data type that represents a pair of integers:
\begin{center}
\begin{cminted}{haskell}
data IntPair = IntPair Int Int 
\end{cminted}
\end{center}
The left-hand side of this definition simply contains the name of the new data type. The right-hand side declares the type of the constructor function, which is usually given the same name as the data type. 
Note that the constructor \mintinline{haskell}{IntPair} by definition yields a function that when applied to two integers returns an expression with type IntPair. 

When defining functions in Haskell one can \textbf{\textit{pattern match}} against data constructors. In other words one may define the function value in terms of values that are given to the constructor of the input argument. To provide an example we can define a function that computes the sum of the two integers contained in an \mintinline{haskell}{IntPair}:
\begin{center}
\begin{cminted}{haskell}
sumIntPair :: IntPair -> Int 
sumIntPair (IntPair x y) = x + y
\end{cminted}
\end{center}
where we used the infix notation for the addition of integers. 

The unique property of the Haskell type system, however, is that already when compiling Haskell code the compiler checks whether the types of the various function applications contained therein match. If not, the program is immediately rejected by the compiler. Thus potential type errors are already avoided at compile time and therefore cannot lead to runtime errors. Languages that employ this form of type checking are also called \textit{\textbf{statically typed}}. Also, our developed tensor algebra package intensively relies on the Haskell type system. It is built such that typos that might occur when using it to compute tensorial expressions almost always yield a type error. Especially when dealing with longer computations, this is a massive advantage as such typos or similar mistakes are then already detected when compiling the code and not after possibly several hours of runtime.  

Finally, the last special feature that Haskell admits is its \textit{\textbf{lazy evaluation}} strategy. Functions are only explicitly evaluated on their arguments once the result is needed. This does not only allow for performance improvements as by employing lazy evaluation, unnecessary function calls can be avoided; it also provides further advantages. For instance, lazy evaluation enables the use of conditional control structures in a purely functional style. The usual conditional if statement, for instance, can be given by a function:
\begin{center}
\begin{cminted}{haskell}
if' :: Bool -> a -> a -> a
if' False _ y = y 
if' True x _ = x 
\end{cminted}
\end{center}
Note that when the boolean evaluates to \mintinline{haskell}{True} this function simply returns the expression \mintinline{haskell}{x}. Thus when using lazy evaluation in this case, \mintinline{haskell}{y} is never evaluated. Similarly \mintinline{haskell}{x} is not evaluated when the boolean is \mintinline{haskell}{False}. 

Besides the above particularities of Haskell as a programming language in order to fully grasp the essence of the algorithms that will be outlined in the following section, it is crucial to accustom oneself with some essential functions that are heavily used in functional programming languages to iteration over data structures. These techniques can be understood best by considering a particular example. Amongst the simplest data structures that can be used for the iteration are lists. In Haskell, the list data structure is defined as follows: 
\begin{center}
\begin{cminted}{haskell}
data [a] = [ ] | a : [a] 
\end{cminted}
\end{center}
Thus it has two constructors \mintinline{haskell}{[ ]} constructing the empty list and \mintinline{haskell}{(:)} appending an element to the front of a given list. Note that \mintinline{haskell}{(:)} is written as an infix operator.
The first heavily used function is the \mintinline{haskell}{map}function:
\begin{center}
\begin{cminted}{haskell}
map :: (a -> b) -> [a] -> [b]
map f [] = [] 
map f (x:xs) =  f x : map f xs 
\end{cminted}
\end{center}
Hence the map function takes a function and a list and applies he function to each element of the list returning the resulting list.
The concept of mapping a function over a data structure is not restricted to lists but can also be employed for more general structures. Data types that can be mapped over must be instances of the \mintinline{haskell}{Functor} type class.

The second important concept are the fold operators \mintinline{haskell}{foldl} and \mintinline{haskell}{foldr}. They exist in several versions. The main idea consists of taking a binary function, a start argument and a list, or any other instance of the \mintinline{haskell}{Foldable} type class, containing additional arguments and then successively reducing the list by first applying the function to the start argument and the first argument of the list and then proceeding by evaluating the function on the thus obtained result and the next list element. One can distinguish folds by whether they reduce the list starting from its first (left fold, \mintinline{haskell}{foldl}) or its last (right fold, \mintinline{haskell}{foldr}) value. A left fold operator can be implemented in Haskell according to:
\begin{center}
\begin{cminted}{haskell}
foldl :: (a -> b -> a) -> a -> [b] -> a 
foldl f x [] = x 
foldl f x (y:ys) = foldl f (f x y) ys
\end{cminted}
\end{center}
Obviously, there exist many more such functions that allow each particular problem at hand to be treated in a suitable way using a purely functional programming paradigm. Details can be found in the provided literature.

\section{sparse-tensor: A Typesafe Tensor Algebra Library }
With the short introduction to Haskell at hand, we now explain the main ideas that underlie the developed Haskell library sparse-tensor. Ultimately the aim consists of solving the perturbative equivariance equations (\ref{order1}), (\ref{order2}) and (\ref{order3}). Recall that these three systems are merely compromised of linear equations. In principle, it should thus be straight forward to re-express these in suitable matrix from and use standard matrix computer algebra to solve them. Practically however extracting the appropriate matrices from the systems is a different question as for doing so one needs to evaluate the tensorial expressions in all free indices and to that end also
explicitly express all occurring contractions and symmetrizations.
Thus using solely standard matrix computer algebra to solve the perturbative equivariance equations is hardly possible.

On the other hand existing tensor algebra systems (such as Cadabra (see \cite{cadabra1} and \cite{cadabra2}) and the Physics package \cite{MaplePhysics} currently provided by the computer algebra system Maple to name a few) are mostly tailored towards a flexible, symbolic treatment of tensors, not an efficient retrieval of their particular components and thus are not entirely suited for the specific purpose of solving the perturbative equivariance equations. 

These observations led us to develop our own computer algebra framework. We quickly summarize the main ideas that underlie its construction.
As a first observation note that except for parameters that label the independent Lorentz invariant expressions in the expansion coefficients and thus occur linearly in the perturbative equivariance equations all remaining expressions only contain rational numbers. In particular, the components of the tensors involved in these equations are explicitly known. Besides, when solving the perturbative equivariance equations, we need to evaluate the tensorial equation for all its free indices, and thus, the explicit components are also all needed. Hence, in fact, we never have to treat the occurring tensors symbolically as abstract objects but can think of them as containers that are simply used to store values in a particular way. 

Furthermore, it is crucial to grasp that the use of abstract indices $I, A, ...$ leads to a severe reduction in computation costs and is thus best included in the computer program. For instance, contracting two blocks with $4$ indices each that both admit the symmetry of an area metric tensor against each other requires the summation of $4^4 = 256$ expressions. In contrast to that introducing an abstract index $A = 0,...,20$ that runs over the independent components such an area metric block contains, the contraction only requires the summation of $21$ expressions.

Finally, we can additionally derive a benefit from the fact that the tensors that occur in the treatment of the perturbative equivariance equations are to a large extent only sparsely occupied. Hence it is best to also only work with the non zero components in the computer program. Providing an example the area metric intertwiner with components displayed in (\ref{AreaI}) only contains $144$ non-vanishing components out of a total of $21 \cdot 4^4 = 5376$ total components which admits to an occupation level of around $2.68 \%$.

Summing up, we want to incorporate the following functionality in our computer program:
\begin{itemize}
    \item treatment of tensors with multiple abstract indices each running over an individual index range. 
    \item Sparse storage of tensor components.
    \item Optimization towards computing explicit component operations over abstract algebraic manipulation of tensors.
\end{itemize}

Additionally to the three requirements displayed above we are going to encode the generalized rank of a given tensor --- for standard tensors described by contravariant and covariant indices the rank is expressed as pair ($\#$ of contravariant indices, $\#$ of covariant indices), consequently we express the rank of a tensor that takes $n$ contravariant and $n$ covariant index types as the corresponding $2n$-tuple of numbers of indices --- directly in its type. Doing so Haskell's type system guarantees the prevention of possible errors that might, for instance, occur when by mistake adding tensors of different ranks. The detection of such errors then already happens during compile time. During our usage of the developed computer algebra framework, this feature really turned out to be vital as entering the complicated equations involving multiple indices that one wishes to solve in the perturbative framework of Constructive Gravity often led to minor mistakes that where then immediately detected when compiling the program and not after possibly several hours of runtime.

We start with the underlying data type that we defined to represent a tensor of arbitrary rank that only takes a single index type, i.e., We can think of such a type as representing tensors with only contravariant indices of the given type. Using Haskell's generalized algebraic data types, in short \textit{\textbf{GADT}}s the data type definition of such a single index tensor is displayed in listing \ref{TensorDat}. 

\begin{listing}[hbt!]
\begin{minted}[frame = lines, framesep = 2.5mm, baselinestretch = 1.2, bgcolor=LG!40
]{haskell}
-- | Basic tensor data types.
data Tensor n k v where
    -- | Constructor of leaf values.
    Scalar :: v -> Tensor 0 k v
    -- | Constructs a @'Tensor'@ from a @'TMap'@ of index sub
    --   tensor pairs.
    Tensor :: TMap k (Tensor n k v) -> Tensor (n+1) k v
    -- | Represents a @'Tensor'@ that is identical zero.
    ZeroTensor :: Tensor n k v
\end{minted} 
\caption{Tensor Data Type.}\label{TensorDat}
\end{listing}

Hence the tensor data type \mintinline{haskell}{Tensor n k v} contains additional type information consisting of a type-level natural number \mintinline{haskell}{n} representing the rank of the tensor, i.e., its number of indices and types \mintinline{haskell}{k} and \mintinline{haskell}{v} that encode the type of index the tensor takes and the type of values it stores. The data type then provides three constructors: \mintinline{haskell}{Scalar} constructs a rank 0 tensor, i.e., a scalar out of a given value, the constructor \mintinline{haskell}{Tensor} takes an ordered list of (index, sub tensor) pairs:
\begin{center}
\begin{cminted}{haskell}
type TMap k v = [(k,v)]
\end{cminted}
\end{center}
where the sub tensors have rank \mintinline{haskell}{n} and constructs from it a \mintinline{haskell}{Tensor (n+1) k v} with rank \mintinline{haskell}{n+1} and \mintinline{haskell}{ZeroTensor} simply represents a tensors of arbitrary rank with all components being identical zero. Thus a tensor is represented as ordered forest\footnote{A collection of disjoint ordered trees.} with additional type information regarding rank, index type, and stored values. Including an ordering of the individual sibling sets \footnote{We call a collection of nodes that have a common parent node a sibling set.} allows for faster insertion and lookup operations. All further functions that we define for the types \mintinline{haskell}{Tensor} and \mintinline{haskell}{TMap} will maintain this ordering.

To retrieve a value of a tensor of rank \mintinline{haskell}{n} we need to specify \mintinline{haskell}{n} values of the appropriate type \mintinline{haskell}{k}. It is important to understand that the distinction between the individual indices is not provided by some abstract labels that are attached to them but simply by their position. To provide an example, figure \ref{ExampleTens} displays the forest structure of a rank $2$ tensor that uses \mintinline{haskell}{Ind3} types, i.e., spacetime indices with potential values between $0$ and $3$ as index type and stores rational numbers as values.
\begin{figure}[hbt!]
\centering
\begin{tikzpicture}[roundnode/.style={rectangle, draw=black, fill=LG!40, very thick, minimum size=7mm},]
\node[roundnode]  (I1) at (0,0) {\mintinline{haskell}{Ind3 0}};
\node[roundnode]  (I2) at (0,-3) {\mintinline{haskell}{Ind3 2}};
\node[roundnode]  (I3) at (0,-7) {\mintinline{haskell}{Ind3 3}};
\node[roundnode]  (J1) at (4,1) {\mintinline{haskell}{Ind3 1}};
\node[roundnode]  (J2) at (4,-1) {\mintinline{haskell}{Ind3 2}};
\node[roundnode]  (J3) at (4,-3) {\mintinline{haskell}{Ind3 2}};
\node[roundnode]  (J4) at (4,-5) {\mintinline{haskell}{Ind3 0}};
\node[roundnode]  (J5) at (4,-7) {\mintinline{haskell}{Ind3 1}};
\node[roundnode]  (J6) at (4,-9) {\mintinline{haskell}{Ind3 3}};
\node  (K1) at (8,1) {\mintinline{haskell}{Scalar 1 % 2}};
\node  (K2) at (8,-1) {\mintinline{haskell}{Scalar 3 % 1}};
\node  (K3) at (8,-3) {\mintinline{haskell}{Scalar -17 % 1}};
\node  (K4) at (8,-5) {\mintinline{haskell}{Scalar 1 % 3}};
\node  (K5) at (8,-7) {\mintinline{haskell}{Scalar 1 % 1}};
\node  (K6) at (8,-9) {\mintinline{haskell}{Scalar 1 % 2}};


\draw [-] (I1) -- (J1);
\draw [-] (I1) -- (J2);
\draw [-] (I2) -- (J3);
\draw [-] (I3) -- (J4);
\draw [-] (I3) -- (J5);
\draw [-] (I3) -- (J6);
\draw [-] (J1) -- (K1);
\draw [-] (J2) -- (K2);
\draw [-] (J3) -- (K3);
\draw [-] (J4) -- (K4);
\draw [-] (J5) -- (K5);
\draw [-] (J6) -- (K6);


\end{tikzpicture}
\caption{Forest Structure of Rank 2 Spacetime Tensor with Rational Values.}\label{ExampleTens}
\end{figure}
Providing the non-vanishing components of the displayed tensor forest explicitly the tensor would read:
\begin{align}
\begin{alignedat}{3}
T^{01} &= \frac{1}{2}, \ \  &  \ \ T^{02} &= 3, \ \  & \ \ T^{22} &= -17,\\
T^{30} &= \frac{1}{3}, & T^{31} &= 1, & T^{33} &= \frac{1}{2}.
\end{alignedat}
\end{align}

In order for tensors, represented this way to satisfy the usual tensor algebra, some restrictions on the possibly stored values are necessary. In particular, we need to be able to add and subtract the stored values as this is necessary when adding and subtracting tensors.
More precisely, the types that we might use as value must constitute an additive group. 
We collect such types that constitute a group and the corresponding group methods in the \mintinline{haskell}{TAdd} \textit{\textbf{type class}} displayed in (\ref{TAdd}).

\begin{listing}[hbt!]
\begin{minted}[frame = lines, framesep = 2.5mm, baselinestretch = 1.2, bgcolor=LG!40]{haskell}
class TAdd a where
    -- | Test whether the given element is zero, i.e., the neutral
    --   element.
    scaleZero :: a -> Bool
    -- | Addition of two elements.
    addS :: a -> a -> a
    -- | Maps an element to its additive inverse.
    negateS :: a -> a
    -- | Subtraction of two elements.
    subS :: a -> a -> a
    subS a b = a `addS` negateS b
\end{minted} 
\caption{Addition Type Class.}\label{TAdd}
\end{listing}

Furthermore, when forming the product of two tensors or also scaling a given tensor with a number, we also need to be able to compute products of the tensor values or between tensors values and scalars. This requirement is encoded in the additional \mintinline{haskell}{TProd} type class (\ref{Prod}).

\begin{listing}[hbt!] 
\begin{minted}[frame = lines, framesep = 2.5mm, baselinestretch = 1.2, bgcolor=LG!40]{haskell}
class Prod v v' where
    -- | Type level function that returns the type of the result of
    --   @'prod'@.
    type TProd v v' :: *
    -- | Product function.
    prod :: v -> v' -> TProd v v'
\end{minted} 
\caption{Product Type Class.}\label{Prod}
\end{listing}

Note that the above type class not only requires the existence of a function \mintinline{haskell}{prod} that computes the products of any two stored values of types \mintinline{haskell}{v} and \mintinline{haskell}{v'}, it also requires its instances to provide a type level function that computes the new type of such a product \mintinline{haskell}{TProd v v'} once the types \mintinline{haskell}{v} and \mintinline{haskell}{v'} are specified.
The first instance that we are going to provide for the two introduced type classes is used to represent arbitrary number-types that allow for the required functions:
\begin{center}
\begin{cminted}{haskell}
newtype SField a = SField a deriving (Show, Eq, Ord)
\end{cminted} 
\end{center}
\mintinline{haskell}{SField} simply provides a wrapper for any number-type at wish.

Finally, the indices of any given tensor also must satisfy specific properties that can be described by class constraints that constraint their possible types. For instance, tensor indices should employ an order relation of their possible values, and also it should be possible to test two index values for equality. The necessary class constraints are combined in the \mintinline{haskell}{TIndex} type class:
\begin{center}
\begin{cminted}{haskell}
class (Eq a, Ord a, Enum a) => TIndex a where
\end{cminted} 
\end{center}



Restricting now to tensors with indices and values being instances of these type classes, we can define the usual tensor algebra operations.
Note that these operations provide the \mintinline{haskell}{Tensor} type with the structure of the usual graded\footnote{A graded algebra is an algebra whose elements can be labeled by elements of some monoid (such as positive integers equipped with addition) or group such that the algebra multiplication is consistent with the group operation of the labels. The label of an algebra element is called its grade.
For details see for instance \cite{bourbaki1998algebra} and also \cite{nlab:gradedAlg}.} tensor algebra on the type level. In other words, the grades of the individual tensors will be directly reflected by their type.
The source code for achieving the addition of two tensors is provided in listing (\ref{Addition}).
\begin{listing}[hbt!]
\begin{minted}[frame = lines, framesep = 2.5mm, baselinestretch = 1.2, bgcolor=LG!40]{haskell}
(&+) :: (TIndex k, TAdd v) => Tensor n k v -> Tensor n k v ->
                              Tensor n k v
(&+) (Scalar a) (Scalar b) = Scalar $ a `addS` b
(&+) (Tensor m1) (Tensor m2) = Tensor $ addTMaps (&+) m1 m2
(&+) t1 ZeroTensor = t1
(&+) ZeroTensor t2 = t2
\end{minted} 
\caption{Addition Function of Tensors.}\label{Addition}
\end{listing}
Here \mintinline{haskell}{addTMaps} (see listing \ref{addTmaps}) is a function that adds the two \mintinline{haskell}{TMaps k v} of the two tensors with a combiner function that is called to obtain the appropriate value if an index is present in both lists of type \mintinline{haskell}{TMaps k v}. It further ensures that the order of the indices is still valid when combining the two lists. In the case above the combiner function is again the addition of tensors, this time, however, only applied to the appropriate sub tensors. 

\begin{listing}[hbt!] 
\begin{minted}[frame = lines, framesep = 2.5mm, baselinestretch = 1.2, bgcolor=LG!40]{haskell}
addTMaps :: (Ord k) => (v -> v -> v) -> TMap k v -> TMap k v ->
                                        TMap k v 
addTMaps f m1 [] = m1 
addTMaps f [] m2 = m2 
addTMaps f ((k1,v1):xs) ((k2,v2):ys) 
            | k1 < k2 = (k1,v1) : (addTMaps f xs ((k2,v2):ys))
            | k2 < k1 = (k2,v2) : (addTMaps f ((k1,v1):xs) ys)
            | k1 == k2 = (k1, f v1 v2) : (addTMaps f xs ys) 
\end{minted} 
\caption{Helper Function: Addition of Tensor Lists.}\label{addTmaps}
\end{listing}

Scalar multiplication (see listing \ref{ScalarProd}) and subtraction (see listing \ref{SubTens}) can now be implemented straight forwardly.
\begin{listing}[hbt!] 
\begin{minted}[frame = lines, framesep = 2.5mm, baselinestretch = 1.2, bgcolor=LG!40]{haskell}
(&.) :: (TIndex k, Prod s v) => s -> Tensor n k v ->
                                     Tensor n k (TProd s v)
(&.) scalar = fmap (prod scalar)
\end{minted} 
\caption{Scalar Multiplication of Tensors.}\label{ScalarProd}
\end{listing}

\begin{listing}[hbt!] 
\begin{minted}[frame = lines, framesep = 2.5mm, baselinestretch = 1.2, bgcolor=LG!40]{haskell}
(&-) :: (TIndex k, TAdd v) => Tensor n k v -> Tensor n k v ->
                              Tensor n k v
(&-) (Scalar a) (Scalar b) = Scalar $ subS a b
(&-) (Tensor m1) (Tensor m2) = Tensor $ addTMaps (&-) m1 m2
(&-) t1 ZeroTensor = t1
(&-) ZeroTensor t2 = negateS t2
\end{minted} 
\caption{Subtraction of Tensors.}\label{SubTens}
\end{listing}
Where \mintinline{haskell}{fmap} similar to the discussed case of mapping a function over a list of values maps a function over the values of a given tensor and is provided by the functor instance (see (\ref{Functor})) of the tensor data type:

\begin{listing}[hbt!] 
\begin{minted}[frame = lines, framesep = 2.5mm, baselinestretch = 1.2, bgcolor=LG!40]{haskell}
instance Functor (Tensor n k) where 
        fmap f (Scalar x) = Scalar (f x)
        fmap f (Tensor m) = Tensor (mapTMap (fmap f) m)
        fmap f ZeroTensor = ZeroTensor 
\end{minted}
\caption{Functor Instance of Tensor Data Type.}\label{Functor}
\end{listing}

The product of two tensors displayed in (\ref{TensorProd}) can be implemented by appending the second tensor to each of the leaves of the first tensor that is specified. This is done such that the indices of the first tensor always are included in the result before the indices of the second tensor, i.e., the order of indices in the newly obtained tensor is the same as one would expect from naively writing down such a tensor product.
\begin{listing}[hbt!]
\begin{minted}[frame = lines, framesep = 2.5mm, baselinestretch = 1.2, bgcolor=LG!40]{haskell}
(&*) :: (TIndex k, Prod v v') => Tensor n k v -> Tensor m k v' ->
                                 TProd (Tensor n k v) (Tensor m k v')
(&*) (Scalar x) (Scalar y) = Scalar $ prod x y
(&*) (Scalar x) t2 = fmap (prod x) t2
(&*) (Tensor m) t2 = Tensor $ mapTMap (&* t2) m
(&*) t1 ZeroTensor = ZeroTensor
(&*) ZeroTensor t2 = ZeroTensor
\end{minted}
\caption{Tensor Product Function.}\label{TensorProd}
\end{listing}
Here \mintinline{haskell}{mapTMap} maps a function over the values of the key value pairs \mintinline{haskell}{(k,v)} stored in the \mintinline{haskell}{TMap k v}.
Note that not only the rank of the resulting tensor depends on the ranks of the two input tensors --- The resulting rank is precisely given as the sum of the two input ranks. This is exactly where the grading of the tensor algebra is reflected on the type level --- also the types that are stored in it as values clearly depend on the respective types of the input tensors. This is precisely where the second type-class \mintinline{haskell}{TProd v v'} comes to use. 

When symmetrizing a given tensor, we need to be able to swap the position of some of the tensor's indices. This is done by the following function (\ref{TensorTrans}).
\begin{listing}[hbt!]
\begin{minted}[frame = lines, framesep = 2.5mm, baselinestretch = 1.2, bgcolor=LG!40]{haskell}
tensorTrans :: (TIndex k, TAdd v) => (Int,Int) -> Tensor n k v ->
                                     Tensor n k v
tensorTrans (0, j) t = fromListT l
                where
                    l = map (\(x,y) -> (swapHead j x, y)) $ toListT t
tensorTrans (i, j) (Tensor m) = Tensor $
                                mapTMap (tensorTrans (i-1, j-1)) m
tensorTrans (i ,j) ZeroTensor = ZeroTensor
\end{minted}
\caption{Transposition of Tensors in two Indices.}\label{TensorTrans}
\end{listing}
The two indices w.r.t. which the tensor is transposed are provided by the integer tuple, where the indices are labeled w.r.t. their position starting from 0. The function first traverses the tensor until the level\footnote{We will call the distance of a given node to the root node of the appropriate tree the level of the node.} specified by the first, i.e., the smaller integer is reached. Then all corresponding sub tensors are completely flattened to lists of indices value pairs. This is precisely what \mintinline{haskell}{toListT} does. Now in each of the thus obtained indices value lists of the various sub tensors by using the function \mintinline{haskell}{swapHead} the first and the $j$th entry of the indices are swapped. Finally, by using  \mintinline{haskell}{fromListT}, the indices value lists are again transformed into a tensor.  Applying this function to the previously provided rank $2$ example tensor with the integer pair given as \mintinline{haskell}{(0,1)} we obtain the following transposed rank $2$ tensors that are displayed in figure \ref{ExampleTensTrans}.

\begin{figure}[hbt!]
\centering
\begin{tikzpicture}[roundnode/.style={rectangle, draw=black, fill=LG!40, very thick, minimum size=7mm},]
\node[roundnode]  (I1) at (0,0) {\mintinline{haskell}{Ind3 0}};
\node[roundnode]  (I2) at (0,-2) {\mintinline{haskell}{Ind3 1}};
\node[roundnode]  (I3) at (0,-4) {\mintinline{haskell}{Ind3 2}};
\node[roundnode]  (I4) at (0,-6) {\mintinline{haskell}{Ind3 3}};
\node[roundnode]  (J1) at (4,0) {\mintinline{haskell}{Ind3 3}};
\node[roundnode]  (J2) at (4,-1) {\mintinline{haskell}{Ind3 0}};
\node[roundnode]  (J3) at (4,-2.5) {\mintinline{haskell}{Ind3 3}};
\node[roundnode]  (J4) at (4,-3.5) {\mintinline{haskell}{Ind3 0}};
\node[roundnode]  (J5) at (4,-5) {\mintinline{haskell}{Ind3 2}};
\node[roundnode]  (J6) at (4,-6) {\mintinline{haskell}{Ind3 3}};

\node  (K1) at (8,0) {\mintinline{haskell}{Scalar 1%3}};
\node  (K2) at (8,-1) {\mintinline{haskell}{Scalar 1%2}};
\node  (K3) at (8,-2.5) {\mintinline{haskell}{Scalar 1%1}};
\node  (K4) at (8,-3.5) {\mintinline{haskell}{Scalar 3%1}};
\node  (K5) at (8,-5) {\mintinline{haskell}{Scalar -17%1}};
\node  (K6) at (8,-6) {\mintinline{haskell}{Scalar 1%2}};


\draw [-] (I1) -- (J1);
\draw [-] (I2) -- (J2);
\draw [-] (I2) -- (J3);
\draw [-] (I3) -- (J4);
\draw [-] (I3) -- (J5);
\draw [-] (I4) -- (J6);
\draw [-] (J1) -- (K1);
\draw [-] (J2) -- (K2);
\draw [-] (J3) -- (K3);
\draw [-] (J4) -- (K4);
\draw [-] (J5) -- (K5);
\draw [-] (J6) -- (K6);


\end{tikzpicture}
\caption{Forest Structure of Transposed Rank 2 Tensor from Figure \ref{ExampleTens}.}\label{ExampleTensTrans}
\end{figure}
Now the corresponding values are given as:
\begin{align}
\begin{alignedat}{3}
T^{10} &= \frac{1}{2}, \ \  &  \ \ T^{20} &= 3, \ \  & \ \ T^{22} &= -17,\\
T^{03} &= \frac{1}{3}, & T^{13} &= 1, & T^{33} &= \frac{1}{2}.
\end{alignedat}
\end{align}

Similar to the transposition of two indices we can now obviously construct functions that transpose a given tensor in several indices or even completely reorder (\ref{resortTens}) the positions of the various indices. 
\begin{listing}[hbt!]
\begin{minted}[frame = lines, framesep = 2.5mm, baselinestretch = 1.2, bgcolor=LG!40]{haskell}
resortTens :: (KnownNat n, TIndex k, TAdd v) => [Int] -> Tensor n k v ->
                                                Tensor n k v
resortTens perm t = fromListT $
                    map (\(x,y) -> (resortInd perm x, y)) $ toListT t
\end{minted} 
\caption{General Reordering of Tensor Indices.}\label{resortTens}
\end{listing}
The function takes as argument a list of integers where the $i$th elements specifies the position on which the $i$th index in the tensor shall be sorted. For instance the integer list \mintinline{haskell}{[3,0,2,1]} sorts the 0th index of the tensor to position \mintinline{haskell}{3}, the first index on position \mintinline{haskell}{0}, the second index on position \mintinline{haskell}{2} and the third an last index on position \mintinline{haskell}{1}. Note that the length of the provided integer list must be the same as the number of indices in the given tensor.  The resorting is then achieved by flattening the tensor to a list of indices value pairs, resorting the indices as specified and then reconstructing the tensor from the newly obtained indices value list:
\begin{center}
\begin{cminted}{haskell}
resortTens [3,0,2,1] (fromListT' [([0,1,2,3],1)] :: Tensor 4 Ind3 Rational)
= (fromListT' [([1,3,2,0],1)] :: Tensor 4 Ind3 Rational)
\end{cminted}
\end{center}
Knowing how we can transpose and resort the indices of a given tensor we can easily construct arbitrary symmetrization functions, simply by rearranging the indices of a given tensor accordingly and then adding this newly obtained tensor to the previous one. We only provide the example of the standard symmetrization w.r.t. the exchange of too individual indices explicitly (\ref{symTens}).

\begin{listing}[hbt!] 
\begin{minted}[frame = lines, framesep = 2.5mm, baselinestretch = 1.2, bgcolor=LG!40]{haskell}
symTensFac :: (TIndex k, TAdd v, Prod (SField Rational) v) => (Int,Int)
              -> Tensor n k v -> Tensor n k (TProd (SField Rational) v)
symTensFac inds t = (SField $ 1%2) &. symTens inds t
\end{minted} 
\caption{Pair Symmetrization of Tensors.}\label{symTens}
\end{listing}

Any other symmetrization can be constructed in similar ways.

Of course, we can also evaluate a given tensor by inserting a particular value for one of its indices. This can be achieved as displayed in listing (\ref{evalTens}).
\begin{listing}[hbt!] 
\begin{minted}[frame = lines, framesep = 2.5mm, baselinestretch = 1.2, bgcolor=LG!40]{haskell}
evalTens :: (KnownNat (n+1), TIndex k, TAdd v) => Int -> k ->
            Tensor (n+1) k v -> Tensor n k v
evalTens ind indVal (Tensor m)
            | ind > size -1 || ind < 0 = error "wrong index to evaluate"
            | ind == 0 = fromMaybe ZeroTensor $ lookup indVal m
            | otherwise = fromMaybe ZeroTensor $
                          lookup indVal (getTensorMap newTens)
            where
                size = length $ fst $ head $ toListT' (Tensor m)
                l = [1..ind] ++ 0 : [ind+1..size -1]
                newTens = resortTens l (Tensor m)
\end{minted}
\caption{Evaluation Function for Tensors.}\label{evalTens}
\end{listing}
If the tensor is to be evaluated for its $0$th index, we simply look up the corresponding value in the top level of the forest. In any other case, we first shift the corresponding level of the forest that is to be evaluated to the front to then again proceed as described in the prior case. Evaluating a tensor for a particular value of one of its indices then return the appropriate sub tensor. 

Up to now, we only treated tensors with a single type of indices. For instance, in the example of figure \ref{ExampleTens}, the tensor only had contravariant spacetime indices.
Formulated more rigorously, we only considered the tensor algebra over one specific vector space $V$.
We can however easily generalize the above to the case of not only incorporating the dual to $V$, $V^{\ast}$ to obtain the usual notion of contravariant and covariant tensors, but we can even lift the above functionality to the free tensor algebra over finitely many vector spaces over the same field and their duals $\{V_1,...,V_n,V_1^{\ast},...,V_n^{\ast}\}$:
\begin{align}
    \mathfrak{T}(V_1,...,V_n) := \bigoplus_{r_1,s_1,...,r_n,s_n \geq 0}T^{r_1}_{s_1}V_1 \otimes ... \otimes T^{r_n}_{s_n}V_n.
\end{align}
The elements of this algebra are consequently tensors with $n$-different index types --- in the following formula the type of an index is denoted by an superscripted $(i)$ --- with the $i$th type running over $\mathrm{dim}(V_i)$ and an arbitrary number of indices of each of those types appearing both in contravariant and in covariant position:
\begin{align}
    \mathfrak{T}(V_1,...,V_n) \ni T^{A^{(1)}_1 ... A^{(1)}_{r_1} ... A^{(n)}_1 ... A^{(n)}_{r_n}}
    _{B^{(1)}_1 ... B^{(1)}_{s_1} ... B^{(n)}_1 ... B^{(n)}_{s_n}}.
\end{align}
The rank of such a tensor is represented as a $2n$-tuple of natural numbers. The grading of $\mathfrak{T}(V_1,...,V_n)$ is then provided by this n-tuples, more precisely the individual tensors are labeled by their rank and the operation of taking tensor products is reflected by the component wise addition of these labels. Most important, the rank label will be encoded in the type of such tensors.

We start by generalizing the relevant notions to the treatment of covariant indices.
This is simply achieved by additionally appending tensors of the same index type, that thus represent the particular covariant sub tensors, as values, i.e., as leafs
to the first tensor:
\begin{center}
\begin{cminted}{haskell}
type Tensor2 n1 n2 k v = Tensor n1 k (Tensor n2 k v)
\end{cminted}
\end{center}
Thus \mintinline{haskell}{Tensor2 n1 n2 k v} describes as before a contravariant tensor of rank \mintinline{haskell}{n1}, but this time with values being each provided by an additional covariant tensor of rank \mintinline{haskell}{n2}. In particular, the two index types are the same as of course, the new covariant indices run over the same index range as the contravariant counterparts. The pair \mintinline{haskell}{(n1,n2)} is then merely the usual rank of the tensor.

Note that in the definition of most of the above functions, the types that tensors might store as values were constrained to be instances of \mintinline{haskell}{TScalar} and \mintinline{haskell}{TAlgebra}. 
In order to be able also to use these functions for  \mintinline{haskell}{Tensor2}, i.e., for the case where the stored values are itself tensors, it is necessary that the \mintinline{haskell}{Tensor} type itself provides an instance of these two type classes. In other words, we must now define the necessary functions that make the data type
\mintinline{haskell}{Tensor n k v} an instance of \mintinline{haskell}{TAdd} (\ref{TensTAdd}) and \mintinline{haskell}{Prod} (\ref{TensProd}):

\begin{listing}[hbt!] 
\begin{minted}[frame = lines, framesep = 2.5mm, baselinestretch = 1.2, bgcolor=LG!40]{haskell}
instance (TIndex k, TAdd v) => TAdd (Tensor n k v) where
    addS = (&+)
    negateS = negateTens
    scaleZero = \case
                    ZeroTensor -> True
                    _          -> False
\end{minted}
\caption{Addition Type Class Instance of The Tensor Type.}\label{TensTAdd}
\end{listing}

\begin{listing}[hbt!]
\begin{minted}[frame = lines, framesep = 2.5mm, baselinestretch = 1.2, bgcolor=LG!40]{haskell}
instance (TIndex k, Prod v v') => 
    Prod (Tensor n k v) (Tensor n' k v') where
        type TProd (Tensor n k v) (Tensor n' k v') = 
            Tensor (n+n') k (TProd v v')
        prod = (&*)
\end{minted} 
\caption{Product Type Class Instance of the Tensor Type.}\label{TensProd}
\end{listing}

Now given this data type \mintinline{haskell}{Tensor2 n1 n2 k v} that represents a standard tensor with not only contravariant indices but also covariant ones we can supplement the previous tensor algebra functions by implementing a notion of contracting such a tensor in two of its indices.
We specify the two indices that shall be contracted by their position in the list of contravariant and covariant indices respectively. The contraction can then be achieved as displayed in the listing \ref{TensorContr}.

\begin{listing}[hbt!] 
\begin{minted}[frame = lines, framesep = 2.5mm, baselinestretch = 1.2, bgcolor=LG!40]{haskell}
tensorContr :: (TIndex k, TAdd v) => (Int,Int) -> Tensor2 n1 n2 k v
               -> Tensor2 (n1-1) (n2-1) k v
tensorContr (0,j) t = fromListT tensList
    where
        l = map (\(x,y) -> (x, toListT y)) $ toListT t
        l2 = map (\(x,y) -> (tailInd x,mapMaybe (removeContractionInd j
            (headInd x)) y)) l
        l3 = filter (\(_,y) -> not (null y)) l2
        tensList = map (\(x,y) -> (x, fromListT y)) l3
tensorContr (i,j) (Tensor m) = Tensor $ mapTMap (tensorContr (i-1,j)) m
tensorContr inds ZeroTensor = ZeroTensor
tensorContr inds (Scalar s) = error "cannot contract scalar!"
\end{minted} 
\caption{Contraction of a Tensor.}\label{TensorContr}
\end{listing}

The integer pair \mintinline{haskell}{(Int,Int)} labels the two index positions. If the first integer is not zero, we traverse the tensor forest until we reach the appropriate level specified by it. Next, the remaining sub tensors are all flattened to lists of indices value pairs. In each such list, we filter those indices value pairs that admit equal values for the two indices that are to be contracted and hence contribute when during the contraction, the two indices are set equal. Then these two indices are removed such that the pairs with equal values of the two contraction indices now all have the same indices. This is all done by the function \mintinline{haskell}{removeContractionInd}. Finally, we reconstruct the individual sub tensors from the flattened lists ensuring that the values of these pairs with the same indices are summed up and thus yield the correct value of the contracted tensor. 

Note that with this framework, there are no limitations regarding the number of indices that one might possibly use. For instance, we can now easily construct a data type for tensors that are described by two different types of indices each one appearing in contravariant and covariant fashion by appending the two different \mintinline{haskell}{Tensor2} types:
\begin{center}
\begin{cminted}{haskell}
type AbsTensor4 n1 n2 n3 n4 k1 k2 v = AbsTensor2 n1 n2 k1 
                                     (Tensor2 n3 n4 k2 v)
\end{cminted}
\end{center}
where \mintinline{haskell}{AbsTensor2 n1 n2 k v = Tensor2 n1 n2 k v} is simply a type synonym. In the sparse-tensor library, we explicitly provided type synonyms for tensors that take up to 4 different indices each one appearing in contravariant and covariant position. This type is then called:
\begin{center}
\begin{cminted}{haskell}
type AbsTensor8 n1 n2 n3 n4 n5 n6 n7 n8 k1 k2 k3 k4 v
\end{cminted}
\end{center}
Further types can easily be defined once they are needed.
We also employed the convention that functions that are defined for tensors with n different index types, counting both covariant and contravariant indices, are labeled with the respective number of indices n as the last letter in the function name.

It is important to note that as the tensor type itself is an instance of \mintinline{haskell}{TAdd} and \mintinline{haskell}{Prod} the tensor algebra operations \mintinline{haskell}{(&+),(&-),(&.),(&*)} are always the same no matter how many different indices the tensor at hand uses. Furthermore, also the symmetrization, transposition and the contraction still work exactly the same the only thing that one additionally needs to specify now is for which index type the function should be applied.

This can be done by noting that \mintinline{haskell}{fmap} takes a function and applies it to the leaves of a given tensor. Thus applying fmap successively several times we can decent a fixed number of tensor levels in the forest. This can be used to apply functions to tensors that are stored as leaves of other tensors. We provide the following example (\ref{mapTo3}) of a function that precisely descents $3$ tensor levels.
\begin{listing}[hbt!]
\begin{minted}[frame = lines, framesep = 2.5mm, baselinestretch = 1.2, bgcolor=LG!40]{haskell}
mapTo3 :: (v1 -> v2) -> AbsTensor3 n1 n2 n3 k1 k2 v1 -> 
                        AbsTensor3 n1 n2 n3 k1 k2 v2
mapTo3 = fmap . fmap . fmap
\end{minted}
\caption{Descending 3 Tensor Levels.}\label{mapTo3}
\end{listing}
Using this, we can, for instance, symmetrize a given tensor in the covariant indices of the second index type by the function (\ref{ASymDeep}).
\begin{listing}[hbt!]
\begin{minted}[frame = lines, framesep = 2.5mm, baselinestretch = 1.2, bgcolor=LG!40]{haskell}
symATens5 :: (TIndex k1, TIndex k2, TIndex k3, TAdd v) =>
             (Int,Int) ->
             AbsTensor5 n1 n2 n3 n4 n5 k1 k2 k3 v ->
             AbsTensor5 n1 n2 n3 n4 n5 k1 k2 k3 v
symATens5 = mapTo4 . symTens
\end{minted} 
\caption{Anti-Symmetrization of Indices in the Fourth Tensor Level. }\label{ASymDeep}
\end{listing}
Similarly, all other functions that we have encountered so far can be defined for tensors of arbitrarily many different indices as well. In the sparse-tensor library we explicitly provided all involved functions for all possible index sets that are included in the \mintinline{haskell}{AbsTensor8} type. If one wishes even to use more different indices, functions that are defined for a single index type can easily be lifted to act on the various different indices by using \mintinline{haskell}{fmap} the appropriate number of times. 

This finally covers the most basic functionality that is provided by the sparse-tensor Haskell library. It, however, allows for much more tensor functions that can be used to treat a large number of tensor algebra related problems that arise in theoretical mathematical physics. 
Further details can be found in \cite{sparse-tensor} more specifically in the \textit{Tensor} sub module.

\section{Generation of Lorentz Invariant Basis Tensors}\label{LorentzGen}
Amongst the entire functionality provided by the developed Haskell library, the generation of Lorentz invariant basis tensors requires the most careful discussion. We have already seen that any such Lorentz invariant tensor must be constructed solely from the Minkowski metric and the Levi-Civita symbol both possibly occurring with upper and lower index position. For the following discussion, we restrict to the case where the Lorentz invariant tensor only possesses contravariant spacetime indices. Other cases can obviously be treated analogously.

Any general, such Lorentz invariant tensor is thus given by a linear combination of sums of products that are formed solely from $\eta^{ab}$ and $\epsilon^{abcd}$.
We will simply call the individual terms of such a linear combination \textit{\textbf{ansätze}}. Note that each ansatz necessarily features the same symmetry that is required from the Lorentz invariant tensor. The number of factors in the individual products that are included in the various ansätze is obviously solely determined by the required rank of the Lorentz invariant tensor.

Due to the well known identity $\epsilon^{abcd}\epsilon^{efgh} = 24 \eta^{[a\vert e}\eta^{\vert b \vert f}\eta^{\vert c \vert g}\eta^{\vert d] h}$ we only need to treat the two cases that in the individual products either feature  no contribution from $\epsilon^{abcd}$ or have precisely one $\epsilon^{abcd}$ included. Any other case then reduces to one of these as we can use the identity to pairwise eliminate Levi-Civita symbols by means of Minkowski metrics. 

As we do not only want to construct the most general Lorentz invariant tensor possible, by means of including all possible ansätze in the linear combination, but also want the individual ansätze to form a basis for the space of such Lorentz invariant tensors of given rank and symmetry, we obviously need to get rid of linear dependencies amongst the individual ansätze. Clearly, a set of ansätze is linearly dependent if we find a non-trivial linear combination of them that yields zero. 

There is, however, an additional dimension dependent mechanism that generates further linear dependencies between ansätze that are at first sight not as obvious. Such additional linear dependencies might occur for instance if due to the involved required symmetry of the ansätze we can construct linear combinations of ansätze that are not strictly zero but yield an expression that is totally antisymmetric in $5$ or more of its indices. As we are working in $4$ spacetime dimension, such an expression evaluates to zero on all possible index combinations. Thus the ansätze in consideration are in fact linearly dependent, although at first glance they might not seem to be so. 

In order to distinguish such additional linear dependencies that are generated in this fashion, from the former case we introduce the following terminology: We call a set of ansätze \textit{\textbf{algebraically linearly dependent}} if there exists a non-trivial linear combination of these that yields identical zero. In particular, when investigating algebraic linear dependencies, there is no need to evaluate the components of the expressions explicitly. Algebraic linear dependency is therefore completely  independent from the given dimension that we work in. We call it \textbf{\textit{numerically linearly dependent}} if there exists a non-trivial linear combination that vanishes when evaluated on all possible index combinations. This now clearly is a dimension dependent notion as for instance an expression that is totally antisymmetric in $5$ indices evaluates to zero on all possible index combinations when working in $4$ dimensions, it, however, does not so if working in $5$ dimensions or higher. Obviously, every algebraically linearly dependent set of ansätze is also numerically linearly dependent. The converse is, however, clearly not true. 

The above considerations are best seen when working out a particular example: We take the two expressions $\epsilon^{abcd} \eta^{pq}$ and $\epsilon^{abcp} \eta^{dq}$ and symmetrize s.t. the expressions admit the area metric symmetry in $abcd$, i.e., are antisymmetric in $ab$ and $cd$ and additionally obey the block symmetry $(ab) \leftrightarrow (cd)$. Doing so we get the two ansätze: 
\begin{itemize}
\item[(i)] $\epsilon^{abcd} \eta^{pq}$ 
\item[(ii)] $\epsilon^{abcp} \eta^{dq} - \epsilon^{abdp} \eta^{cq} + \epsilon^{cdap} \eta^{bq} - \epsilon^{cdbp} \eta^{aq}$.
\end{itemize}
Clearly, these two ansätze are not algebraically linearly dependent. Subtracting the second ansatz form the first one we nevertheless find that the result is totally antisymmetric in the indices $abcdp$ and thus evaluated for any possible index combination yields zero. Thus the two expressions are numerically linearly dependent.

Summing up, if the goal not only consists of computing the most general Lorentz invariant tensor of given symmetry but if the individual building blocks the ansätze are further required to be linearly independent it does not suffice to take into account algebraic linear dependencies, but we also have to ensure that the ansätze are numerically linearly independent.  

In order to achieve performance, it is essential to pick suitable data structures for representing a linear combination of ansätze. As usual, the data structures must be tailored towards the specific needs that are in this case, the symmetrization of such linear combinations but also their explicit evaluation at specific index values.  
The individual tensors $\eta^{ab}$ and $\epsilon^{abcd}$ can simply be represented as displayed in (\ref{EtaType}).
\begin{listing}[hbt!] 
\begin{minted}[frame = lines, framesep = 2.5mm, baselinestretch =
1.2, bgcolor=LG!40]{haskell}
data Epsilon = Epsilon {-# UNPACK #-} !Int {-# UNPACK #-} !Int
               {-# UNPACK #-} !Int {-# UNPACK #-} !Int
               deriving (Show, Read, Eq, Ord, Generic, Serialize, NFData)

data Eta = Eta {-# UNPACK #-} !Int {-# UNPACK #-} !Int 
           deriving (Show, Read, Eq, Ord, Generic, Serialize, NFData)

data Var = Var {-# UNPACK #-} !Int {-# UNPACK #-} !Int 
           deriving (Show, Read, Eq, Ord, Generic, Serialize, NFData )
\end{minted} 
\caption{Data Types for Minkowski Metric, Levi-Civita Symbol, and Variables.}\label{EtaType}
\end{listing}
Note that the integer values of the data types \mintinline{haskell}{Eta} and \mintinline{haskell}{Epsilon} are used to label the abstract spacetime indices, i.e., the first index will simply be labels by $1$ the second index by $2$, etc. in particular they do not refer to values that these indices might admit.
Furthermore, we also defined a basic variable data type to encode the parameters that will later multiply the individual ansätze to form a linear combination. The \mintinline{haskell}{Var}
data type simply takes two integer values where the first one refers to an integer factor that multiplies the variable\footnote{When using factor less symmetrization (for details see the discussion following (\ref{ansatzExample})) it actually turns out that it suffices to use integers for representing the factors in front of the different variables.} and the second one provides an identifier that labels the different variables. 

Using these data types, we can encode a general linear combination of ansätze (\ref{AnsForest}).
\begin{listing}[hbt!] 
\begin{minted}[frame = lines, framesep = 2.5mm, baselinestretch =
1.2, bgcolor=LG!40]{haskell}
data AnsatzForestEta = ForestEta (M.Map Eta AnsatzForestEta)| Leaf !Var
                       | EmptyForest 
                       deriving (Show, Read, Eq, Generic, Serialize)

type AnsatzForestEpsilon = M.Map Epsilon AnsatzForestEta
\end{minted} 
\caption{Data Type representing Linear Combinations of Ansätze.}\label{AnsForest}
\end{listing}
Here \mintinline{haskell}{AnsatzForestEta} is the datatype of a linear combination of ansätze that each solely involve $\eta^{ab}$ whereas the individual ansätze encoded by \mintinline{haskell}{AnsatzForestEpsilon} all involve exactly one $\epsilon^{abcd}$ in each of their individual product.

Clearly, there exist no algebraic linear dependencies between ansätze that involve an $\epsilon^{abcd}$ and those that do not. 
It is important to note that there also cannot exist numeric linear dependencies that mix the two types of ansätze. The reason for this is that for a product of $\eta^{ab}$ to yield non-vanishing contributions when evaluated on a list of index values, i.e., setting each individual index to a value in $\{0,1,2,3 \}$ each value must occur an even number of times as the individual $\eta^{ab}$ factors have only diagonal entries. In contrast to that, evaluating a product of several Minkowski metrics and one $\epsilon^{abcd}$ one additionally needs each possible value for a spacetime index precisely once such that the Levi-Civita symbol yields a non-vanishing contribution. Thus for the types of ansätze that include $\epsilon^{abcd}$ to yield a non zero contribution, each value of the indices must occur an odd number of times.

In total, this shows that whenever an ansatz that features no Levi-Civita symbol evaluates to a non zero contribution all possible ansätze that incorporate an $\epsilon^{abcd}$ evaluate to zero and vice versa. Thus the different types of ansätze are also mutually numerically linearly independent. In particular, we see that the problem of finding a list of ansätze that constitute a basis for the space of Lorentz invariant tensors of given rank and symmetry decouples into the two subproblems of finding those that incorporate an $\epsilon^{abcd}$ and those that do not. The two types of ansätze can hence be treated completely independently. To that end, we also chose to represent them using different data types. 

The data type \mintinline{haskell}{Map k v} that is used in the definition of the two ansatz forest types represents a finite map between keys and values that is internally implemented as binary tree\footnote{Details regarding the implementation can be found in \cite{adams_1993}. The \mintinline{haskell}{Map k v} \cite{HackageMap} and many more datatypes can be found in Haskell's central package archive \cite{Hackage}.}. Thus the \mintinline{haskell}{AnsatzForestEta} data type is a forest with nodes being provided by an \mintinline{haskell}{Eta} value and leafs given by a \mintinline{haskell}{Var} value. Similarly, the \mintinline{haskell}{AnsatzForestEpsilon} type is given by such a forest with the difference that compared to \mintinline{haskell}{AnsatzForestEta} now the first level nodes are occupied by \mintinline{haskell}{Epsilon} values. Note that the individual sibling sets of these forests themselves are not provided by a "linear" data structure s.t. lists but by the binary structure of the \mintinline{haskell}{Map k v}. This enhances performance as some of the ansatz forests that we will treat have up to several thousand elements in a single sibling set, and the binary structure allows for faster insertion and lookup of elements. Such a binary structure, for instance, features $\mathcal{O}(\mathrm{log}(n)$ lookup whereas a single linked list in the worst case only provides $\mathcal{O}(n)$ lookup. 

Furthermore, when constructing an \mintinline{haskell}{AnsatzForestEta}, we will always ensure that parent nodes are smaller than all of their children, where the order relation is obtained by comparing the individual integers that are contained in a value of type \mintinline{haskell}{Eta}, starting with the first. All further functions will always produce such sorted forests when evaluated on sorted forests as input. 

The thus obtained ordered forest structure is not only strikingly performant when inserting and looking up individual ansätze and therefore also when adding two forests it is also particularly suited for evaluating a given linear combination of ansätze on specific index values. Note that only 4 out of 16 possible index value pairs yield a non zero contribution to an $\eta^{ab}$ and only 24 out of 256 4 tuples contribute to an $\epsilon^{abcd}$. Thus when explicitly evaluating an ansatz forest on specific index values, a large number of nodes will actually evaluate to zero. Further, note that the forest structure roughly speaking represents a fully factored product. Thus whenever a given node in the forest evaluates to zero, we do not have to evaluate the subforest that is attached to it as the whole expression is then multiplied by zero anyway. This observation allows for a maximally efficient and therefore, rapid evaluation of ansatz forest when implemented by using such tree-based data structure. 

To provide an example of the data types that are used to encode such expressions, we consider the following linear combination of ansätze:
\begin{multline}\label{AnsatzExprEx}
3a_1 \cdot \left (\eta^{ab}\eta^{cd}\eta^{ef} + \eta^{ab}\eta^{ce}\eta^{df} + \eta^{ab}\eta^{cf}\eta^{de} \right ) + a_2 \cdot \left ( \eta^{ac} \eta^{bd} \eta^{ef} + \eta^{ac} \eta^{be} \eta^{df} -2 \eta^{ad} \eta^{be} \eta^{cf} \right ) \\
+ a_3 \cdot \left ( \eta^{ad} \eta^{bc} \eta^{ef} - \eta^{ad} \eta^{bf} \eta^{ce} \right ) + a_4 \cdot \left ( \epsilon^{abcd} \eta^{ef} + \epsilon^{abce} \eta^{df}  \right )   .
\end{multline}
The corresponding representation using the \mintinline{haskell}{AnsatzForestEta} and \mintinline{haskell}{AnsatzForestEta} data types can be seen in figure \ref{AnsatzExprExForest}.
\begin{figure}
\centering
\begin{tikzpicture}[roundnode/.style={rectangle, draw=black, fill=LG!40, very thick, minimum size=7mm},]
\node  (I0) at (0,2) {\large{\mintinline{haskell}{AnsatzForestEta}}};
\node[roundnode]  (I1) at (0,-2) {\mintinline{haskell}{Eta 1 2}};
\node[roundnode]  (I2) at (0,-5) {\mintinline{haskell}{Eta 1 3}};
\node[roundnode]  (I3) at (0,-8) {\mintinline{haskell}{Eta 1 4}};

\node  (I02) at (0,-13) {\large{\mintinline{haskell}{AnsatzForestEpsilon}}};
\node[roundnode]  (I4) at (1,-14.5) {\mintinline{haskell}{Epsilon 1 2 3 4}};
\node[roundnode]  (I5) at (1,-16.5) {\mintinline{haskell}{Epsilon 1 2 3 5}};


\node[roundnode]  (J1) at (4,2) {\mintinline{haskell}{Eta 3 4}};
\node[roundnode]  (J2) at (4,0) {\mintinline{haskell}{Eta 3 5}};
\node[roundnode]  (J3) at (4,-2) {\mintinline{haskell}{Eta 3 6}};
\node[roundnode]  (J4) at (4,-4) {\mintinline{haskell}{Eta 2 4}};
\node[roundnode]  (J5) at (4,-6) {\mintinline{haskell}{Eta 2 5}};
\node[roundnode]  (J6) at (4,-8) {\mintinline{haskell}{Eta 2 3}};
\node[roundnode]  (J7) at (4,-10) {\mintinline{haskell}{Eta 2 5}};
\node[roundnode]  (J8) at (4,-12) {\mintinline{haskell}{Eta 2 6}};
\node[roundnode]  (J9) at (8,-14.5) {\mintinline{haskell}{Eta 5 6}};
\node[roundnode]  (J10) at (8,-16.5) {\mintinline{haskell}{Eta 4 6}};


\node[roundnode]  (K1) at (8,2) {\mintinline{haskell}{Eta 5 6}};
\node[roundnode]  (K2) at (8,0) {\mintinline{haskell}{Eta 4 6}};
\node[roundnode]  (K3) at (8,-2) {\mintinline{haskell}{Eta 4 5}};
\node[roundnode]  (K4) at (8,-4) {\mintinline{haskell}{Eta 5 6}};
\node[roundnode]  (K5) at (8,-6) {\mintinline{haskell}{Eta 4 6}};
\node[roundnode]  (K6) at (8,-8) {\mintinline{haskell}{Eta 5 6}};
\node[roundnode]  (K7) at (8,-10) {\mintinline{haskell}{Eta 3 6}};
\node[roundnode]  (K8) at (8,-12) {\mintinline{haskell}{Eta 3 5}};


\node  (L1) at (12,2) {\mintinline{haskell}{Leaf $ Var 3 1}};
\node  (L2) at (12,0) {\mintinline{haskell}{Leaf $ Var 3 1}};
\node  (L3) at (12,-2) {\mintinline{haskell}{Leaf $ Var 3 1}};
\node  (L4) at (12,-4) {\mintinline{haskell}{Leaf $ Var 1 2}};
\node  (L5) at (12,-6) {\mintinline{haskell}{Leaf $ Var 1 2}};
\node  (L6) at (12,-8) {\mintinline{haskell}{Leaf $ Var 1 3}};
\node  (L7) at (12,-10) {\mintinline{haskell}{Leaf $ Var -2 2}};
\node  (L8) at (12,-12) {\mintinline{haskell}{Leaf $ Var -1 3}};
\node  (L9) at (12,-14.5) {\mintinline{haskell}{Leaf $ Var 1 4}};
\node  (L10) at (12,-16.5) {\mintinline{haskell}{Leaf $ Var 1 4}};




\draw [-] (I1) -- (J1);
\draw [-] (I1) -- (J2);
\draw [-] (I1) -- (J3);
\draw [-] (I2) -- (J4);
\draw [-] (I2) -- (J5);
\draw [-] (I3) -- (J6);
\draw [-] (I3) -- (J7);
\draw [-] (I3) -- (J8);
\draw [-] (I4) -- (J9);
\draw [-] (I5) -- (J10);




\draw [-] (J1) -- (K1);
\draw [-] (J2) -- (K2);
\draw [-] (J2) -- (K2);
\draw [-] (J3) -- (K3);
\draw [-] (J4) -- (K4);
\draw [-] (J5) -- (K5);
\draw [-] (J6) -- (K6);
\draw [-] (J7) -- (K7);
\draw [-] (J8) -- (K8);



\draw [-] (K1) -- (L1);
\draw [-] (K2) -- (L2);
\draw [-] (K3) -- (L3);
\draw [-] (K4) -- (L4);
\draw [-] (K5) -- (L5);
\draw [-] (K6) -- (L6);
\draw [-] (K7) -- (L7);
\draw [-] (K8) -- (L8);
\draw [-] (J9) -- (L9);
\draw [-] (J10) -- (L10);




\end{tikzpicture}
%make Maps at each level visible??
\caption{Forest Structure of Ansatz Linear Combination (\ref{AnsatzExprEx}). }
\label{AnsatzExprExForest}
\end{figure}

Due to the forest data structure, the addition of two such expressions can be implemented quite performant. One simply uses the \mintinline{haskell}{unionWith} function provided by the Data.Map.Strict package \cite{HackageMap}. The function combines two Maps calling a specified combiner function if a key is present in both. The addition of two ansatz forests (\ref{ForestAdd}) can now be achieved by defining the addition of the leaf values in an obvious way and further defining the addition of two ansatz forests that themselves contain sub forests by combining the corresponding Maps with combiner function being again the addition of ansatz forests. Doing so, one eventually recurses over the whole structure. 
\begin{listing}[hbt!] 
\begin{minted}[frame = lines, framesep = 2.5mm, baselinestretch =
1.2, bgcolor=LG!40]{haskell}
addForests :: AnsatzForestEta -> AnsatzForestEta -> AnsatzForestEta
addForests ans EmptyForest = ans
addForests EmptyForest ans = ans
addForests (Leaf var1) (Leaf var2)
        | isZeroVar newLeafVal = EmptyForest
        | otherwise = Leaf newLeafVal
        where
            newLeafVal = addVars var1 var2
addForests (ForestEta m1) (ForestEta m2)
        | M.null newMap = EmptyForest
        | otherwise = ForestEta newMap
         where
            newMap = M.filter (/= EmptyForest) $
                     M.unionWith addForests m1 m2

addForestsEpsilon :: AnsatzForestEpsilon -> AnsatzForestEpsilon ->
                     AnsatzForestEpsilon
addForestsEpsilon m1 m2 = M.filter (/= EmptyForest) $ M.unionWith
                          addForests m1 m2
\end{minted} 
\caption{Addition of Ansatz Forests.}\label{ForestAdd}
\end{listing}\\

Symmetrization (\ref{PairSymFor}) of such an expression can be achieved by merely swapping or permuting individual index identifiers of the various \mintinline{haskell}{Eta} and \mintinline{haskell}{Epsilon} values in the ansatz forest and then adding the result to the former ansatz forest.
\begin{listing}[hbt!] 
\begin{minted}[frame = lines, framesep = 2.5mm, baselinestretch =
1.2, bgcolor=LG!40]{haskell}
pairSymForestEta :: (Int,Int) -> AnsatzForestEta -> AnsatzForestEta
pairSymForestEta inds ans = addForests ans $ swapLabelFEta inds ans

pairSymForestEps :: (Int,Int) -> AnsatzForestEpsilon ->
                    AnsatzForestEpsilon
pairSymForestEps inds ans = addForestsEpsilon ans $ 
                            swapLabelFEps inds ans
\end{minted} 
\caption{Pair Symmetrization of Ansatz Forests.}\label{PairSymFor}
\end{listing}
As swapping labels of forest nodes (\ref{SwapF}) will, in general, destroy the sorting of the forest, we have to reinforce the correct ordering afterward. This is simply done by flattening the forest to a list that contains the individual branches as pairs with the first entry being given by a list containing the individual nodes of the given branch and the second entry being the corresponding value. One then sorts the node list in each such pair according to the required ordering and finally reconstructs the forest from the list of branches.
Haskells lazy evaluation then ensures that this straight forward implementation is surprisingly fast.
\begin{listing}[hbt!] 
\begin{minted}[frame = lines, framesep = 2.5mm, baselinestretch =
1.2, bgcolor=LG!40]{haskell}
swapLabelFEta :: (Int,Int) -> AnsatzForestEta -> AnsatzForestEta
swapLabelFEta inds ans = sortForest.canonicalizeAnsatzEta $ swapAnsatz
        where
            f = swapLabelEta inds
            swapAnsatz = mapNodes f ans

swapLabelFEps :: (Int,Int) -> AnsatzForestEpsilon -> AnsatzForestEpsilon
swapLabelFEps inds ans = sortForestEpsilon.canonicalizeAnsatzEpsilon $ 
                        swapAnsatz
        where
            f = swapLabelEpsilon inds
            swapAnsatz = mapNodesEpsilon f $ M.map 
                                            (swapLabelFEta inds) ans
\end{minted}
\caption{Swap Function for Ansatz Forests.}\label{SwapF}
\end{listing}

Similarly, one can easily define symmetrization functions for arbitrary other kinds of symmetry. We define a \mintinline{haskell}{Symmetry} data type that collects the most used such, i.e., (pair symmetries, pair anti-symmetries, block symmetries, cyclic symmetries, cyclic block symmetries):
\begin{center}
\begin{cminted}{haskell}
type Symmetry = ([(Int,Int)],[(Int,Int)],[([Int],[Int])],[[Int]],[[[Int]]])
\end{cminted}
\end{center}
The various integers refer to the indices that are to be symmetrized. We now can collect the individual symmetrizer functions in one overall function as displayed in (\ref{SymAns}).
\begin{listing}[hbt!] 
\begin{minted}[frame = lines, framesep = 2.5mm, baselinestretch =
1.2, bgcolor=LG!40]{haskell}
symAnsatzForestEta ::Symmetry -> AnsatzForestEta -> AnsatzForestEta
symAnsatzForestEta (sym,asym,blocksym,cyclicsym,cyclicblocksym) ans =
    foldr cyclicBlockSymForestEta (
        foldr cyclicSymForestEta (
            foldr pairBlockSymForestEta (
                foldr pairASymForestEta (
                    foldr pairSymForestEta ans sym
                ) asym
            ) blockSymMap
        ) cyclicsym
    ) cyclicblocksym
    where
        blockSymMap = map swapBlockLabelMap blocksym

symAnsatzForestEps :: Symmetry -> AnsatzForestEpsilon ->
                      AnsatzForestEpsilon
symAnsatzForestEps (sym,asym,blocksym,cyclicsym,cyclicblocksym) ans =
      foldr cyclicBlockSymForestEps (
          foldr cyclicSymForestEps (
              foldr pairBlockSymForestEps (
                  foldr pairASymForestEps (
                      foldr pairSymForestEps ans sym
                  ) asym
              ) blockSymMap
          ) cyclicsym
      ) cyclicblocksym
      where
        blockSymMap = map swapBlockLabelMap blocksym
\end{minted} 
\caption{General Ansatz Forest Symmetrizer Function.}\label{SymAns}
\end{listing}

This summarizes the data structures that we use to represent the tensorial expression that we are going to construct. Further details and additional functions that were omitted here can be found in \cite{sparse-tensor}.  

With this choice of data structures at hand, we now proceed with the construction of Lorentz invariant basis tensors.  
This process can, in general, be divided into three steps:
\begin{itemize}
    \item[(i)] Generating all possible products built either solely from $\eta^{ab}$ or including exactly one $\epsilon^{abcd}$ that feature the required index structure.
    \item[(ii)] Symmetrizing the individual expressions according to the demanded symmetry to obtain all possible ansätze.
    \item[(iii)] Reducing the resulting expressions w.r.t. algebraic and numeric linear dependencies.
\end{itemize}

We start with the first step, computing all possible products. Labeling the indices by integers \mintinline{haskell}{[1,...,n]} one readily obtains all possible such individual products be exhausting all possible ways the indices can be arranged, taking into account the symmetries of $\eta^{ab}$, $\epsilon^{abcd}$ if present and the symmetries that are induced by the product structure. We get the list of all possible index orders for the products that only involve $\eta^{ab}$ by the function \mintinline{haskell}{getEtaInds}.
Similar, the list of all possible indices for the expressions that involve one $\epsilon^{abcd}$ can be computed by \mintinline{haskell}{getEpsilonInds}. The two functions are displayed in the listing \ref{AllInds}.
\begin{listing}[hbt!]
\begin{minted}[frame = lines, framesep = 2.5mm, baselinestretch =
1.2, bgcolor=LG!40]{haskell}
getEtaInds :: [Int] -> [[Int]]
getEtaInds [a,b] = [[a,b]]
getEtaInds (x:xs) = concatMap res firstEta
        where
            firstEta = map (\y -> ([x,y],delete y xs)) xs
            res (a,b) = (++) a <$> getEtaInds b 

getEpsilonInds :: [Int] -> [[Int]]
getEpsilonInds inds = allInds
        where
            i = length inds 
            epsInds = [ [a,b,c,d] | a <- [1..i-3], b <- [a+1..i-2],
                      c <- [b+1..i-1], d <- [c+1..i] ] 
            allInds = map (\x -> map (x ++) $
                      getEtaInds (inds \\ x) ) epsInds 
\end{minted} 
\caption{Computation of All Possible Index Orders.}\label{AllInds}
\end{listing}

Note that in \cite{sparse-tensor} we further reduced these index lists by filtering already here some of the linear dependencies that will later arise due to the symmetrization. This is not necessary as we will treat linear dependencies later anyway. It does, however, increase performance as the lists that are consumed by the algorithm are then smaller from the very beginning.  

We could now proceed with the next step, express each individual product by the appropriate chosen data structure, symmetrize all products individually and then at the very end, reduce occurring linear dependencies. Following these lines, however, bears the problem that intermediate results, i.e., the entirety of the thus constructed ansätze occupies an unreasonable amount of memory. Only at the very end when linearly dependent ansätze are removed this memory can be freed again. In fact, in most practical cases, the memory that would be needed for this approach by far exceeds resources that are typically available.

It is thus best to combine the steps (ii) and (iii) that are outlined above. The algorithm then consumes the list of individual products step by step. For each product a distinction is made: if the product is already present in the ansatz forest it is simply rejected if it is not it is symmetrized to obtain the corresponding ansatz and then added to the ansatz forest. Doing so the intermediate memory swelling is avoided as now from the very beginning, only linearly independent ansätze are added to the forest.

For this approach, it is incredibly important to note that two individual products when symmetrized to the respective ansätze are either identically, or completely disjoint in the sense that they do not share a single common summand. This can be seen from the fact that the permutations of indices that are involved in the symmetrization procedure constitute a subgroup of the permutation group of appropriate size $S_n$ that acts on the products via permutation of their indices. The individual products that contribute to a given symmetrized ansatz then precisely constitute the corresponding orbit under the action of the symmetry subgroup. As orbits of any group action are disjoint, so are the ansätze. 

Thus at a given step of the algorithm, when deciding if a new product should be added to the ansatz forest or discarded, we do not need to symmetrize the individual product. If it is already present in the forest and thus linearly dependent on it, already the non-symmetrized individual product will be present if the individual product is not present also all further expressions that are obtained from symmetrizing it will not be present. Hence the decision whether or not an individual product should be added to the forest can be made without ever having to symmetrize the individual product. Obviously, when such a product is added to the forest, we still have to symmetrize it, but all products that are rejected can now be rejected without requiring an explicit symmetrization.

The above allows for a decisive criterion that given a list of possible products only incorporates those into the ansatz forest that do not possess algebraic linear dependencies amongst each other. When further reducing the ansatz forest w.r.t. numeric linear dependencies there are in principle two approaches how this can be achieved. We can either first use the above technique to construct an ansatz forest that is algebraically linearly independent, then evaluate this ansatz forest for enough index combinations to retrieve all independent components, write these in a matrix where the columns label the individual variables that occur and then use standard linear algebra to compute a basis of these column vectors. Variables that are not present in this basis can then be removed from the ansatz forest as these multiply exactly those ansätze that are linearly dependent. 

Alternatively when constructing the ansatz forest from a list of individual products we can not only check for algebraic linear dependencies when deciding whether or not a new product should be added but if a given product is algebraically  independent from the ansatz forest then also test for numeric linear dependencies. Doing so we would construct from the very beginning an ansatz forest that consists of not only algebraically but also numerically linearly independent ansätze.
It turns out that the first approach results in slightly faster computation times\footnote{At least for the construction of ansätze that exceed a certain size, i.e., a certain number of individual expressions that are involved in them.
For ansätze below that size, the two computation methods take almost the same time.}, whereas the second approach is superior in memory usage, thus we have implemented both algorithms. 
We start provided details regarding the first approach. 

\subsection*{Reducing linear dependencies I:  The fast way}

The idea is straight forward. Representing the individual products involving only $\eta^{ab}$ or also including one $\epsilon^{abcd}$ as lists with the individual list elements representing the factors of the product, we need a function that decides whether or not such a product is already present in a given ansatz forest (\ref{ForestElem}). As we are always dealing with ordered forests and also sorted products we can already conclude that a given product is missing in the forest if any given node is missing in the appropriate level of the forest. Thus only when a product is actually present in the forest, we really need to decent up to the leaf values of the forest.
\begin{listing}[hbt!]
\begin{minted}[frame = lines, framesep = 2.5mm, baselinestretch =
1.2, bgcolor=LG!40]{haskell}
isElem :: [Eta] -> AnsatzForestEta -> Bool
isElem [] (Leaf x) = True
isElem x EmptyForest = False
isElem  (x:xs) (ForestEta m) = case mForest of
                                Just forest -> xs `isElem` forest
                                _           -> False
            where
                mForest = M.lookup x m

isElemEpsilon :: (Epsilon, [Eta]) -> AnsatzForestEpsilon -> Bool
isElemEpsilon (eps,l) m = case mForest of
                            Just forest -> l `isElem` forest
                            _           -> False
            where
                mForest = M.lookup eps m  
\end{minted} 
\caption{Lookup Function for Ansatz Forests.}\label{ForestElem}
\end{listing}

Using these two functions, we simply define two functions that, when a product is missing, symmetrize it to obtain the corresponding ansatz and then add it to the forest, when the product is already present, it is dismissed. These functions are then folded over the list of products taking an empty forest as start value.  Doing this we obtain functions that take a list of products and a value of type \mintinline{haskell}{Symmetry} and constructs from it the two ansatz forests with ansätze being obtained by symmetrizing the various products that are provided by the list. Further, these two functions now ensure that all ansätze that are included in the forests are algebraically linearly independent. 

These two functions can then again be used to define the final functions that solely take the required symmetry and the rank of the to be constructed Lorentz invariant basis tensors as argument and then compute from it an ansatz forest of all possible algebraically linearly independent ansätze that feature the correct number of indices and symmetries. We simply use the priorly defined functions \mintinline{haskell}{getEtaInds} and \mintinline{haskell}{getEpsInds} to compute lists of all possible indices of products of either solely Minkowski metrics or also including one Levi-Civita symbol, all featuring the required number of indices. These two lists are then transformed to lists of the appropriate types \mintinline{haskell}{Eta} and \mintinline{haskell}{Epsilon}, and each individual product is further combined with a different variable. The two lists are reduced as described above, by invoking the two functions \mintinline{haskell}{reduceAnsatzEta'} and \mintinline{haskell}{reduceAnsatzEpsilon'} (\ref{redFast}).
\begin{listing}[hbt!]
\begin{minted}[frame = lines, framesep = 2.5mm, baselinestretch =
1.2, bgcolor=LG!40]{haskell}
reduceAnsatzEta' :: Symmetry -> [([Eta],Var)] -> AnsatzForestEta
reduceAnsatzEta' sym = foldl' addOrRem' EmptyForest
        where
            addOrRem' f ans = if isElem (fst ans) f then f else
                              addForests f (symAnsatzForestEta sym 
                              $ mkForestFromAscList ans)

reduceAnsatzEpsilon' :: Symmetry -> [(Epsilon, [Eta], Var)] ->
                        AnsatzForestEpsilon
reduceAnsatzEpsilon' sym = foldl' addOrRem' M.empty
        where
            addOrRem' f (x,y,z) = if isElemEpsilon (x,y) f then f else
                                  addForestsEpsilon f 
                                  (symAnsatzForestEps sym 
                                  $ mkForestFromAscListEpsilon (x,y,z))  
\end{minted} 
\caption{Reduction of Ansatz Forests.}\label{redFast}
\end{listing}
Finally, the variables in the two ansatz forests that are thus constructed are relabeled. The two functions that construct the two ansatz forests with algebraic linear dependencies removed are displayed in listing \ref{ConstrFast}.
\begin{listing}[hbt!]
\begin{minted}[frame = lines, framesep = 2.5mm, baselinestretch =
1.2, bgcolor=LG!40]{haskell}
getEtaForestFast :: Int -> Symmetry -> AnsatzForestEta
getEtaForestFast ord syms = relabelAnsatzForest 1 $ reduceAnsatzEta' syms 
                            allForests
            where
                allInds = getEtaInds [1..ord] syms
                allVars = mkAllVars
                allForests = zipWith mkEtaList' allVars allInds

getEpsForestFast :: Int -> Symmetry -> AnsatzForestEpsilon
getEpsForestFast ord syms = if ord < 4 then M.empty else
relabelAnsatzForestEpsilon 1 $ reduceAnsatzEpsilon' syms allForests
            where
                allInds = getEpsilonInds [1..ord] syms
                allVars = mkAllVars
                allForests = zipWith mkEpsilonList' allVars allInd
\end{minted} 
\caption{Construct Ansatz Forests: the "Fast" Way.}\label{ConstrFast}
\end{listing}

The remaining work consists of reducing the two constructed ansatz forests w.r.t. numeric linear dependencies. This is best achieved when explicitly retrieving the values that a given ansatz forest admits for all possible values that one might insert for its indices. It is important to note that we do not need to evaluate a given ansatz forest for all possible index values as due to symmetries that might be present we already know that evaluating the ansatz forest on two different lists of index values that are connected via the symmetry we will obtain the same value. Thus we can avoid much work by only evaluating the ansatz forest on one representative out of the equivalence classes that are generated by identifying index lists that are connected via a symmetry. 

We can further reduce the number of necessary evaluations by using the priorly noted fact that any product of etas and therefore, in particular, any \mintinline{haskell}{AnsatzForestEta} can only yield non zero values when evaluated on a list of index values with each value occurring an even number of times. In contrast to that, when evaluating an \mintinline{haskell}{AnsatzForestEpsilon}, we can safely restrict to those evaluation lists that have each index value appearing an odd number of times.

Finally, it is worth noting that the value that we retrieve when evaluating an ansatz forest on a specific index value list at most changes by a sign under arbitrary relabeling of the coordinate axes. Thus we might further reduce the number of necessary evaluations by selecting one index value list out of every set of index value lists that are mutually related by index relabeling. The reason why all ansätze admit this exceptional property, i.e., their values at most changes by a sign under arbitrary coordinate relabeling becomes obvious by recalling that all possible expressions that are built solely from $\eta^{ab}$ and $\epsilon^{abcd}$ are Lorentz invariant. Up to a sign the relabeling of coordinate axes clearly defines a Lorentz transformation, as obviously all those relabeling that keep $x_0$ fixed also keep $\eta_{ab}$ invariant, those that interchange $x_0$ and $x_{\alpha}$ keep eta invariant up to a sign where the sign depends on the precise number of swaps between $x_0$ and spatial coordinates. Hence the fact that the ansatz forests are up to a sign invariant under such coordinate relabeling is no surprise. 

Summing up all of the above considerations can be used to reduce the number of evaluations that are necessary to remove numerical linear dependencies. Clearly employing such techniques is not required as we could also simply evaluate the ansatz forest for all possible index value list. Using the above, however, severely reduces the computation time of the computer program. 
Storing the information regarding which index is set to which value in a \mintinline{haskell}{IntMap Int} \cite{HackageIntMap} (for details regarding the implementation see also \cite{Okasaki98fastmergeable}), i.e., a special of a map type that is tailored towards integer keys, the evaluation of a given forest can be obtained as described in (\ref{EvalOne}).
\begin{listing}[hbt!]
\begin{minted}[frame = lines, framesep = 2.5mm, baselinestretch =
1.2, bgcolor=LG!40]{haskell}
evalAnsatzForestEta :: I.IntMap Int -> AnsatzForestEta -> I.IntMap Int
evalAnsatzForestEta evalM (Leaf (Var x y)) = I.singleton y x
evalAnsatzForestEta evalM (ForestEta m) = M.foldlWithKey' foldF I.empty m
    where
        foldF b k a = let nodeVal = evalNodeEta evalM k
                      in if isNothing nodeVal then b
                         else I.unionWith (+)
                              (I.map (fromJust nodeVal *)
                              (evalAnsatzForestEta evalM a)) b
evalAnsatzForestEta evalM EmptyForest = I.empty

evalAnsatzForestEpsilon :: I.IntMap Int -> AnsatzForestEpsilon ->
                           I.IntMap Int
evalAnsatzForestEpsilon evalM = M.foldlWithKey' foldF I.empty
    where
        foldF b k a = let nodeVal = evalNodeEpsilon evalM k
                      in if isNothing nodeVal then b
                         else I.unionWith (+) 
                              (I.map (fromJust nodeVal *)
                              (evalAnsatzForestEta evalM a)) b  
\end{minted} 
\caption{Evaluation of Ansatz Forests.}\label{EvalOne}
\end{listing}

The \mintinline{haskell}{IntMap Int} that is returned as a result in the above functions encodes the linear combinations of variables that are obtained when evaluating an ansatz forest, i.e., the integer keys of the int map represent the variable labels and the values the corresponding factors in the linear combination. 
When evaluating a given ansatz forest for multiple index value lists, we can gain further performance improvements by noting that the list of all evaluation int maps can, in fact, be divided in chunks that then can be processed parallel. This is achieved by the functions \mintinline{haskell}{evalAllEta} and \mintinline{haskell}{evalAllEpsilon}. Details can be seen in \cite{sparse-tensor}

Once we have evaluated an ansatz forest for all necessary index value lists, we can store the retrieved values in the form of a matrix as described before with the columns labeling the individual variables and the rows labeling the index combination values that we evaluated the ansatz forest on. The ansatz forest can now be reduced w.r.t. numerical linear dependencies by removing linearly dependent column vectors from this matrix. To that end, we use Haskell bindings \cite{HackageEigen} to the C++ linear algebra library Eigen \cite{eigenweb}. Using Eigen subroutines, one can readily reduce linear dependencies amongst the column vectors a given matrix, for instance, by means of an LU decomposition. Finally, once the matrix is reduced, we can proceed by removing all branches with leaf variables that correspond to columns that have been removed from the matrix. This then also removes the numerically linearly dependent ansätze from the ansatz forest (\ref{RedNumLinFast}).
\begin{listing}[hbt!]
\begin{minted}[frame = lines, framesep = 2.5mm, baselinestretch =
1.2, bgcolor=LG!40]{haskell}
reduceLinDepsFastEta :: [I.IntMap Int] -> Symmetry ->
                        AnsatzForestEta -> AnsatzForestEta
reduceLinDepsFastEta evalM symL ansEta = newEtaAns
        where
            etaL = evalAllEta evalM ansEta
            etaVars = getPivots etaL
            allEtaVars = getForestLabels ansEta
            remVarsEta =  allEtaVars \\ etaVars
            newEtaAns = relabelAnsatzForest 1 $
                        removeVarsEta remVarsEta ansEta

reduceLinDepsFastEps :: [I.IntMap Int] -> Symmetry ->
                        AnsatzForestEpsilon -> AnsatzForestEpsilon
reduceLinDepsFastEps evalM symL ansEps = newEpsAns
        where
            epsL = evalAllEpsilon evalM ansEps
            epsVars = getPivots epsL
            allEpsVars = getForestLabelsEpsilon ansEps
            remVarsEps =  allEpsVars \\ epsVars
            newEpsAns = relabelAnsatzForestEpsilon 1 $
                        removeVarsEps remVarsEps ansEps 
\end{minted} 
\caption{Reduction of Numeric Linear Dependencies: the "Fast" Way.}\label{RedNumLinFast}
\end{listing}

This finishes the first, computation time optimized method of constructing a basis of the space of Lorentz invariant tensors of given rank and symmetry that we have implemented in the sparse-tensor library. In order to use the thus computed result, the basis of Lorentz invariant tensors together with the previously presented functionality of the sparse-tensor package we also provide functions that do not only construct the ansatz forests but also return the computed Lorentz invariant basis tensor in the form of a compatible tensor data type. We represent the occurring variables that then necessarily also occur in the components of the tensors as int maps:
\begin{center}
\begin{cminted}{haskell}
newtype AnsVar a = AnsVar (I.IntMap a) deriving (Show, Generic,Serialize)

type AnsVarR = AnsVar (SField Rational)
\end{cminted}
\end{center}
Further, we introduce type synonyms for the typical used tensors types, the usual spacetime tensors that feature contravariant and covariant indices ranging between $0$ and $3$:
\begin{center}
\begin{cminted}{haskell}
type STTens n1 n2 v = AbsTensor2 n1 n2 Ind3 v
\end{cminted}
\end{center}
Moreover, also a more general tensor type that we heavily used for the area metric computations. This tensor type features 3 different index types each appearing in contravariant and covariant position. The first index type ranges over the area metric degrees of freedom and thus ranges from $0$ to $20$, the second index type labels second derivative pairs, and hence runs form $0$ to $9$, and the last type is again the usual spacetime index:
\begin{center}
\begin{cminted}{haskell}
type ATens n1 n2 n3 n4 n5 n6 v = 
     AbsTensor6 n1 n2 n3 n4 n5 n6 Ind20 Ind9 Ind3 v
\end{cminted}
\end{center}
The final functions (\ref{mkAnsatzFast1}), (\ref{mkAnsatzFast2}), (\ref{mkAnsatzFast3}) that use the computation time optimized method and return the result as a triplet of the two ansatz forest and a tensor that collects all their values are.

\begin{listing}[hbt!]
\begin{minted}[frame = lines, framesep = 2.5mm, baselinestretch =
1.2, bgcolor=LG!40]{haskell}
--with explicit symmetrization in tens
mkAnsatzTensorFastSym :: forall (n :: Nat). SingI n => Int -> Symmetry ->
                         [[Int]]-> (AnsatzForestEta, AnsatzForestEpsilon,
                         STTens n 0 (AnsVarR))
mkAnsatzTensorFastSym ord symmetries evalL = (ansEta, ansEps, tens)
    where
        (evalMEtaRed, evalMEpsRed, evalMEtaInds, evalMEpsInds) =
            mkAllEvalMaps symmetries evalL 
        (ansEta, ansEps) =
            mkAnsatzFast ord symmetries evalMEtaRed evalMEpsRed
        tens =
            evalToTensSym symmetries evalMEtaInds evalMEpsInds
                          ansEta ansEps
\end{minted} 
\caption{Ansatz Construction 1.1: with Explicit Symmetrization.}\label{mkAnsatzFast1}
\end{listing}

\begin{listing}[hbt!]
\begin{minted}[frame = lines, framesep = 2.5mm, baselinestretch =
1.2, bgcolor=LG!40]{haskell}
--without explicit symmetriization in tens
mkAnsatzTensorFast :: forall (n :: Nat). SingI n => Int -> Symmetry ->
                      [[Int]]-> (AnsatzForestEta, AnsatzForestEpsilon,
                      STTens n 0 (AnsVarR))
mkAnsatzTensorFast ord symmetries evalL = (ansEta, ansEps, tens)
    where
        (evalMEtaRed, evalMEpsRed, evalMEtaInds, evalMEpsInds) =
            mkAllEvalMaps symmetries evalL 
        (ansEta, ansEps) =
            mkAnsatzFast ord symmetries evalMEtaRed evalMEpsRed
        tens = evalToTens evalMEtaInds evalMEpsInds ansEta ansEps
\end{minted} 
\caption{Ansatz Construction 1.2: without Explicit Symmetrization.}\label{mkAnsatzFast2}
\end{listing}

\begin{listing}[hbt!]
\begin{minted}[frame = lines, framesep = 2.5mm, baselinestretch =
1.2, bgcolor=LG!40]{haskell}
--evaluation to a tensor that uses multiple index types
mkAnsatzTensorFastAbs :: Int -> Symmetry ->
                         [([Int], Int, [IndTupleAbs n1 0 n2 0 n3 0])] ->
                         (AnsatzForestEta, AnsatzForestEpsilon,
                         ATens n1 0 n2 0 n3 0 (AnsVarR))
mkAnsatzTensorFastAbs ord symmetries evalL = (ansEta, ansEps, tens)
    where
        (evalMEtaRed, evalMEpsRed, evalMEtaInds, evalMEpsInds) =
            mkAllEvalMapsAbs symmetries evalL 
        (ansEta, ansEps) =
            mkAnsatzFast ord symmetries evalMEtaRed evalMEpsRed
        tens = evalToTensAbs evalMEtaInds evalMEpsInds ansEta ansEps
\end{minted} 
\caption{Ansatz Construction 1.3: Evaluation to Custom Indices.}\label{mkAnsatzFast3}
\end{listing}

The first function takes as arguments the order of the to be constructed Lorentz invariant basis, i. the number of indices, the symmetry and also a list of all index value lists that are necessary for the evaluation. The evaluation list must contain precisely one representative out of every set of index value lists that are the same under the present symmetry. Further reduction, as described above, must not be incorporated manually as this is achieved by the function \mintinline{haskell}{mkAllEvalMaps}.
The first function returns the Lorentz invariant basis as explicitly symmetrized spacetime tensors.

Note that for higher-ranked tensors and more complicated symmetries the explicit symmetrization of the tensors might be expensive. Furthermore often it suffices to only store one representative of each symmetry equivalence class in the tensor, for instance, if the tensor is contracted against symmetric objects that thus enforce the symmetries in later computations. To that end, we also provided a function that returns the tensors without explicit symmetrization.

Finally the last function does not require a list of index value lists as input but needs a list of triples each consisting of the index value list, the multiplicity of the given index value list that is computed as product of the individual multiplicities of the abstract indices used (for the definition of the multiplicity $\sigma$ see the discussion following definition \ref{interDef}), and also a list of all abstract index tuples that correspond to the present index value list. How these individual ingredients can be computed in detail will be discussed in the following section when we consider an example. The last function then computes a triple consisting of the two ansatz forest and an abstract tensor that uses multiple indices to store the corresponding values. 

Additionally, we also provide the first two of the above three functions in a form that does not require to specify the evaluation list explicitly. Paying the price of a slightly more expensive computation in the following two functions (\ref{mkAnsatzFast'1}) and (\ref{mkAnsatzFast'2}) the evaluation list is constructed fully automatically from the symmetries and the rank of the given tensor.

\begin{listing}[hbt!]
\begin{minted}[frame = lines, framesep = 2.5mm, baselinestretch =
1.2, bgcolor=LG!40]{haskell}
--with explicit symmetrization in tens
mkAnsatzTensorFastSym' :: forall (n :: Nat). SingI n => Int ->
                          Symmetry -> (AnsatzForestEta,
                          AnsatzForestEpsilon,
                          STTens n 0 (AnsVarR))
mkAnsatzTensorFastSym' ord symmetries = mkAnsatzTensorFastSym
                                            ord symmetries evalL
    where
        evalL = filter (`filterAllSym` symmetries) 
                        $ allList ord symmetries
\end{minted} 
\caption{Ansatz Construction 1.4: with Explicit Symmetrization, no Evaluation List Required.}\label{mkAnsatzFast'1}
\end{listing}

\begin{listing}[hbt!]
\begin{minted}[frame = lines, framesep = 2.5mm, baselinestretch =
1.2, bgcolor=LG!40]{haskell}
--without explicit symmetrization in tens
mkAnsatzTensorFast' :: forall (n :: Nat). SingI n => Int -> Symmetry ->
                       (AnsatzForestEta, AnsatzForestEpsilon,
                       STTens n 0 (AnsVarR))
mkAnsatzTensorFast' ord symmetries = mkAnsatzTensorFast
                                        ord symmetries evalL
    where
        evalL = filter (`filterAllSym` symmetries) 
                        $ allList ord symmetries
\end{minted} 
\caption{Ansatz construction 1.5: without Explicit Symmetrization,  no Evaluation List Required.}\label{mkAnsatzFast'2}
\end{listing}

\subsection*{Reducing linear dependencies II:  The memory optimized way}

The main idea of this second way of reducing the ansatz forest w.r.t linear dependencies is to construct an ansatz forest that is not only algebraically but also numerically linearly independent from the very beginning. This can be achieved as follows: As before the algorithm consumes a list of all possible individual products and checks in any given step if the product at hand is missing in the ansatz forest and thus represents an ansatz that is algebraically linearly independent or is already present and can hence be rejected. In contrast to the previous approach now every time a product is missing, we immediately evaluate the corresponding ansatz for all necessary index value lists and check if the ansatz is numerically linearly dependent or not. Only if the new ansatz is also numerically linearly  independent from the present ansatz forest, we add it to the forest.

It is thus crucial that in each step we do not only provide input data that consists of the present ansatz forest which is needed for deciding whether or not the new ansatz is algebraically linearly independent but also include its fully evaluated matrix --- this time with rows labeling the individual variables that are present in the forest and columns labeling the several index value list.
Doing so when deciding whether or not the newly evaluated ansatz is numerically  independent from the ansatz forest, we do not have to evaluate the forest each time. Any time a  new ansatz is added to the forest, it is also added to the matrix that contains all independent components of the forest.

This total procedure is accomplished by the two functions displayed in (\ref{AddorDisc}).
\begin{listing}[hbt!]
\begin{minted}[frame = lines, framesep = 2.5mm, baselinestretch =
1.2, bgcolor=LG!40]{haskell}
addOrDiscardEtaEig :: Symmetry -> [I.IntMap Int] ->
                      (AnsatzForestEta, RankDataEig) -> 
                      [Eta] -> (AnsatzForestEta, RankDataEig)
addOrDiscardEtaEig symList evalM (ans,rDat) etaL
            | isElem etaL ans = (ans,rDat)
            | otherwise = case newRDat of
                               Nothing          -> (ans,rDat)
                               Just newRDat'    -> (sumAns,newRDat')
             where
                newAns = getNewAns symList etaL rDat
                newRDat = getNewRDat evalM newAns rDat
                sumAns = addForests ans newAns

addOrDiscardEpsilonEig :: Symmetry -> [I.IntMap Int] ->
                          (AnsatzForestEpsilon, RankDataEig) ->
                          (Epsilon,[Eta]) ->
                          (AnsatzForestEpsilon, RankDataEig)
addOrDiscardEpsilonEig symList evalM (ans,rDat) (epsL,etaL)
            | isElemEpsilon (epsL,etaL) ans = (ans,rDat)
            | otherwise = case newRDat of
                               Nothing          -> (ans,rDat)
                               Just newRDat'    -> (sumAns,newRDat')
             where
                newAns = getNewAnsEps symList epsL etaL rDat
                newRDat = getNewRDatEps evalM newAns rDat
                sumAns = addForestsEpsilon ans newAns
\end{minted} 
\caption{Add or Discard a new Ansatz.}\label{AddorDisc}
\end{listing}
The type alias \mintinline{haskell}{RankDataEig} represents the ansatz forest data that is necessary for deciding whether or not a new ansatz is numerically linearly dependent on the current ansatz forest and is given by a tuple that consists of one Eigen matrix and one sparse Eigen Matrix:
\begin{center}
\begin{cminted}{haskell}
type RankDataEig = (Mat.MatrixXd, Sparse.SparseMatrixXd)
\end{cminted}
\end{center}
Note that usually, the number of necessary evaluation index value lists by far exceeds the number of variables that are present in a given ansatz forest. Thus the chosen way of evaluating a given forest to a matrix yields a matrix with a large number of columns compared to a small number of rows. In the following, we will refer to this matrix as $M$. The second matrix that is included in \mintinline{haskell}{RankDataEig} is precisely this matrix $M$. To save memory, this matrix is stored in a sparse format. The first matrix in \mintinline{haskell}{RankDataEig} is given by $M M^t$. This matrix is used to achieve performance when computing whether or not a new ansatz is numerically linearly dependent on the current ansatz forest or not. To that end, note that $\mathrm{rank}(MM^t) = \mathrm{rank}(M)$ (see for instance "Bemerkung 2.57" in \cite{LAKnab}). Thus once we have evaluated a new ansatz to a row vector $v$, we can compute the rank of the new ansatz forest that would be obtained by adding the given ansatz to the current forest by:
\begin{align}\label{RankDatmat}
    \mathrm{rank}\left (\begin{bmatrix}
        M \\
        \cmidrule(lr){1-1} 
        v
    \end{bmatrix} \right )
    = \mathrm{rank} \left ( \begin{bmatrix}
        M \\
        \cmidrule(lr){1-1}
        v
    \end{bmatrix} \cdot \begin{bmatrix}
        M^t \ \vert \  v^t 
    \end{bmatrix} \right ) = \mathrm{rank} \left (\begin{bmatrix}
        MM^t & \vline & M v^t \\
        \cmidrule(lr){1-3}
        (Mv^t)^t & \vline & vv^t 
    \end{bmatrix}  \right ).
\end{align}
If the ansatz is numerically linearly independent from the forest appending its evaluated row vector to the current ansatz matrix will result in an increased rank.

Note that in each step of the algorithm once we have evaluated the new ansatz to the row vector $v$ we only have to compute its transpose $v^t$ the matrix-vector product $Mv^t$ and the corresponding transpose $(Mv^t)^t$ and the vector-vector product $v v^t$. In particular, the upper left block in the last matrix in (\ref{RankDatmat}) is already provided from the previous step and thus only needs to be computed newly once an additional ansatz is added to the ansatz forest. Hence we achieve the desired result with a minimum of expensive matrix-matrix operations involved. The rank computation itself is then as before carried out by relying on efficient Eigen subroutines.

The following function (\ref{checkNumLinDep}) takes as arguments the current rank data and the newly evaluated ansatz row vector and computes from it the new rank data, that is it returns the rank data provided by the first matrix in (\ref{RankDatmat}) if the new ansatz is numerically linearly independent from the ansatz forest in the form of a \mintinline{haskell}{Just} value and it returns \mintinline{haskell}{Nothing} if the new ansatz is numerically linearly dependent. 

\begin{listing}[hbt!]
\begin{minted}[frame = lines, framesep = 2.5mm, baselinestretch =
1.2, bgcolor=LG!40]{haskell}
checkNumericLinDepEig :: RankDataEig -> Maybe Sparse.SparseMatrixXd ->
                         Maybe RankDataEig
checkNumericLinDepEig (lastMat, lastFullMat) (Just newVec)
    | eigenRank < maxRank = Nothing
    | otherwise = Just (newMat, newAnsatzMat)
     where
        newVecTrans = Sparse.transpose newVec
        scalar = Sparse.toMatrix $ Sparse.mul newVec newVecTrans
        prodBlock = Sparse.toMatrix $ Sparse.mul lastFullMat newVecTrans
        prodBlockTrans = Mat.transpose prodBlock
        newMat = concatBlockMat lastMat prodBlock prodBlockTrans scalar
        eigenRank = Sol.rank Sol.FullPivLU newMat
        maxRank = min (Mat.cols newMat) (Mat.rows newMat)
        newAnsatzMat = Sparse.fromRows $ 
                       Sparse.getRows lastFullMat ++ [newVec]
checkNumericLinDepEig (lastMat, lastFullMat) Nothing = Nothing
\end{minted} 
\caption{Check Numeric Linear Dependencies.}\label{checkNumLinDep}
\end{listing}

With these functions defined, we can now construct functions that construct algebraically and numerically linearly independent ansatz forests out of a given list of individual products, the given symmetry and the evaluation list (\ref{redMem}).
\begin{listing}[hbt!]
\begin{minted}[frame = lines, framesep = 2.5mm, baselinestretch =
1.2, bgcolor=LG!40]{haskell}
reduceAnsatzEtaEig :: Symmetry -> [[Eta]] -> [I.IntMap Int] ->
                      (AnsatzForestEta,Sparse.SparseMatrixXd)
reduceAnsatzEtaEig symL etaL evalM
    | null evalM = (EmptyForest, Sparse.fromList 0 0 [])
    | null etaL = (EmptyForest, Sparse.fromList 0 0 [])
    | otherwise = (finalForest, finalMat)
    where
        (ans1,rDat1,restEtaL) = mk1stRankDataEtaEig symL etaL evalM
        (finalForest, (_,finalMat)) = foldl' 
                                      (addOrDiscardEtaEig symL evalM)
                                      (ans1,rDat1) restEtaL

reduceAnsatzEpsilonEig :: Symmetry -> [(Epsilon,[Eta])] ->
                          [I.IntMap Int] ->
                          (AnsatzForestEpsilon,Sparse.SparseMatrixXd)
reduceAnsatzEpsilonEig symL epsL evalM
    | null evalM = (M.empty, Sparse.fromList 0 0 [])
    | null epsL = (M.empty, Sparse.fromList 0 0 [])
    | otherwise = (finalForest, finalMat)
    where
        (ans1,rDat1,restEpsL) = mk1stRankDataEpsilonEig symL epsL evalM
        (finalForest, (_,finalMat)) = foldl'
                                      (addOrDiscardEpsilonEig symL evalM)
                                      (ans1,rDat1) restEpsL
\end{minted} 
\caption{Reduce Linear Dependencies: the "Efficient" Way.}\label{redMem}
\end{listing}
Finally, now we can also provide functions (\ref{mkAnsatzEig1}), (\ref{mkAnsatzEig2}) and (\ref{mkAnsatzEig3}) that return a triplet consisting of two ansatz forests corresponding to a Lorentz invariant basis of given rank and symmetry and one tensor that includes all their values, this time employing the memory-optimized algorithm. 
\begin{listing}[hbt!]
\begin{minted}[frame = lines, framesep = 2.5mm, baselinestretch =
1.2, bgcolor=LG!40]{haskell}
--with explicit symmetrization in tens
mkAnsatzTensorEigSym :: forall (n :: Nat). SingI n => Int -> Symmetry ->
                        [[Int]] -> (AnsatzForestEta, AnsatzForestEpsilon,
                        STTens n 0 (AnsVarR))
mkAnsatzTensorEigSym ord symmetries evalL = (ansEta, ansEps, tens)
        where
            (evalMEtaRed, evalMEpsRed, evalMEtaInds, evalMEpsInds) =
                mkAllEvalMaps symmetries evalL 
            (ansEta, ansEps, _, _) =
                getFullForestEig ord symmetries evalMEtaRed evalMEpsRed
            tens = evalToTensSym
                   symmetries evalMEtaInds evalMEpsInds ansEta ansEps
\end{minted} 
\caption{Ansatz Construction 2.1: with Explicit Symmetrization.}\label{mkAnsatzEig1}
\end{listing}
\begin{listing}[hbt!]
\begin{minted}[frame = lines, framesep = 2.5mm, baselinestretch =
1.2, bgcolor=LG!40]{haskell}
--without explicit symmetrization in tens
mkAnsatzTensorEig :: forall (n :: Nat). SingI n => Int -> Symmetry ->
                     [[Int]] -> (AnsatzForestEta, AnsatzForestEpsilon,
                     STTens n 0 (AnsVarR))
mkAnsatzTensorEig ord symmetries evalL = (ansEta, ansEps, tens)
        where
            (evalMEtaRed, evalMEpsRed, evalMEtaInds, evalMEpsInds) =
                mkAllEvalMaps symmetries evalL 
            (ansEta, ansEps, _, _) =
                getFullForestEig ord symmetries evalMEtaRed evalMEpsRed
            tens = evalToTens evalMEtaInds evalMEpsInds ansEta ansEps
\end{minted} 
\caption{Ansatz Construction 2.2: without Explicit Symmetrization.}\label{mkAnsatzEig2}
\end{listing}
\begin{listing}[hbt!]
\begin{minted}[frame = lines, framesep = 2.5mm, baselinestretch =
1.2, bgcolor=LG!40]{haskell}
--evaluation to a tensor that uses multiple abstract index types
mkAnsatzTensorEigAbs :: Int -> Symmetry ->
                        [([Int], Int, [IndTupleAbs n1 0 n2 0 n3 0])] ->
                        (AnsatzForestEta, AnsatzForestEpsilon,
                        ATens n1 0 n2 0 n3 0 (AnsVarR))
mkAnsatzTensorEigAbs ord symmetries evalL = (ansEta, ansEps, tens)
    where
        (evalMEtaRed, evalMEpsRed, evalMEtaInds, evalMEpsInds) =
            mkAllEvalMapsAbs symmetries evalL 
        (ansEta, ansEps, _, _) =
            getFullForestEig ord symmetries evalMEtaRed evalMEpsRed
        tens = evalToTensAbs evalMEtaInds evalMEpsInds ansEta ansEps
\end{minted} 
\caption{Ansatz Construction 2.3: Evaluation to Custom Indices.}\label{mkAnsatzEig3}
\end{listing}
And as before we also provide versions of the first two functions (\ref{mkAnsatzEig1'}) and (\ref{mkAnsatzEig2'}) that automatically construct the evaluation list from the rank and the symmetries specified.
\begin{listing}[hbt!]
\begin{minted}[frame = lines, framesep = 2.5mm, baselinestretch =
1.2, bgcolor=LG!40]{haskell}
--with explicit symmetrization in tens
mkAnsatzTensorEigSym' :: forall (n :: Nat). SingI n =>  Int ->
                         Symmetry ->
                         (AnsatzForestEta, AnsatzForestEpsilon,
                         STTens n 0 (AnsVarR))
mkAnsatzTensorEigSym' ord symmetries = mkAnsatzTensorEigSym
                                       ord symmetries evalL
    where
        evalL =
            filter (`filterAllSym` symmetries) $ allList ord symmetries
\end{minted} 
\caption{Ansatz Construction 2.4: with Explicit Symmetrization,  no Evaluation List Required}\label{mkAnsatzEig1'}
\end{listing}
\begin{listing}[hbt!]
\begin{minted}[frame = lines, framesep = 2.5mm, baselinestretch =
1.2, bgcolor=LG!40]{haskell}
--without explicit symmetrization in tens
mkAnsatzTensorEig' :: forall (n :: Nat). SingI n =>  Int ->
                      Symmetry ->
                      (AnsatzForestEta, AnsatzForestEpsilon,
                      STTens n 0 (AnsVarR))
mkAnsatzTensorEig' ord symmetries = mkAnsatzTensorEig
                                    ord symmetries evalL
    where
        evalL =
            filter (`filterAllSym` symmetries) $ allList ord symmetries
\end{minted} 
\caption{Ansatz Construction 2.5: without Explicit Symmetrization, no Evaluation List Required.}\label{mkAnsatzEig2'}
\end{listing}

The developed sparse-tensor library also provides further functions that can be useful when working with the ansatz forest data types. Details can again be found in \cite{sparse-tensor}, specifically in the \textit{LorentzGenerator} submodule. 

\section{A Step-By-Step Example}
In this section, we are going to consider an example that shall nicely illustrate how the sparse-tensor package can be used to deal with the kind of problems that arise in the perturbative approach to Constructive Gravity. To that end, we are going to consider the two linear-order perturbative equivariance equations (\ref{order1}) in the context of area metric gravity. 

The first step is determining the different indices and thus, the abstract tensor type that is needed for the treatment of perturbative area metric gravity. 
We need one index type that runs over area metric degrees of freedom, one index type for second-order spacetime derivatives and one index type for treating spacetime indices. The corresponding tensor type was already introduced before:

\begin{center}
\begin{cminted}{haskell}
type ATens n1 n2 n3 n4 n5 n6 v = 
     AbsTensor6 n1 n2 n3 n4 n5 n6 Ind20 Ind9 Ind3 v
\end{cminted}
\end{center}

Here the different index types are all defined in the same fashion. As an illustrative example, we discuss \mintinline{haskell}{Ind3} in more detail:
\begin{center}
\begin{cminted}{haskell}
newtype Ind3 =  Ind3 {indVal3 :: Int}
    deriving (Ord, Eq, Show, Read, Generic, NFData, Serialize)
\end{cminted}
\end{center}
We choose to use the \mintinline{haskell}{newtype} declaration in order to ensure that functions that are defined for one particular index type cannot be applied to any other index type and thus preventing possible errors that might occur when mixing up index types.
Note that in order for the tensor algebra functions in sparse-tensor to work for the \mintinline{haskell}{ATens} type, the index types must all be instances of \mintinline{haskell}{TIndex}. This can be simply achieved by defining the function \mintinline{haskell}{toEnum}, \mintinline{haskell}{fromEnum} making them instances of the typeclass \mintinline{haskell}{Enum}: 
\begin{center}
\begin{cminted}{haskell}
instance Enum Ind3 where
    toEnum = Ind3
    fromEnum = indVal3
\end{cminted} 
\end{center}
and then simply making them also an instance of \mintinline{haskell}{TIndex}:

\begin{center}
\begin{cminted}{haskell}
instance TIndex Ind3 where
\end{cminted} 
\end{center}

When taking a closer look at the equations (\ref{order1}), we see that albeit of the three Lorentz invariant basis tensors $a^{A}$, $a^{AI}$ and $a_0$ we further need the two area metric intertwiners $I^I_{abcd}$, $J_I^{abcd}$ with components being displayed in (\ref{AreaI}) and (\ref{AreaJ}), the J-intertwiner of a symmetric index pair whose components can be found in (\ref{interJMet})  , the flat background area metric $N_A$ and the 4 dimensional Kronecker delta $\delta^a_b$. 
The Kronecker delta can readily be obtained (\ref{Kronecker}).
\begin{listing}[hbt!]
\begin{minted}[frame = lines, framesep = 2.5mm, baselinestretch =
1.2, bgcolor=LG!40]{haskell}
delta3 :: ATens 0 0 0 0 1 1 (SField Rational)
delta3 = fromListT6 $ zip
         [(Empty, Empty, Empty, Empty,
         singletonInd (Ind3 i), singletonInd (Ind3 i)) | 
         i <- [0..3]] (repeat $ SField 1)
\end{minted} 
\caption{Construction of Kronecker Delta.}\label{Kronecker}
\end{listing}
The three intertwiners can be constructed according to (\ref{ConstrInterJ2}), (\ref{ConstrINterIArea}) and (\ref{ConstrInterJArea}).
\begin{listing}[hbt!]
\begin{minted}[frame = lines, framesep = 2.5mm, baselinestretch =
1.2, bgcolor=LG!40]{haskell}
interJ2 :: ATens 0 0 0 1 2 0 (SField Rational)
interJ2 = fromListT6 $ fmap (fmap SField) $ filter (\(i,k) -> k /= 0) $
          map (\x -> (x,f x)) inds
    where
        trian2 = trianMap2
        inds = [ (Empty, Empty, Empty, singletonInd $ Ind9 a,
        Append (Ind3 b) $ singletonInd $ Ind3 c, Empty) |
        a <- [0..9], b <- [0..3], c <- [0..3]]
        f (_, _, _, ind1, ind2, _)
            | ind1 == (M.!) trian2 (sortInd ind2) = jMult2 ind2
            | otherwise = 0
\end{minted} 
\caption{Construction of Symmetric Index Pair I Intertwiner.}\label{ConstrInterJ2}
\end{listing}
\begin{listing}[hbt!]
\begin{minted}[frame = lines, framesep = 2.5mm, baselinestretch =
1.2, bgcolor=LG!40]{haskell}
interIArea :: ATens 1 0 0 0 0 4  (SField Rational)
interIArea = fromListT6 $ fmap (fmap SField) $ filter (\(i,k) -> k /= 0) $
             map (\x -> (x,f x)) inds
    where
        trianArea = trianMapArea
        inds = [(singletonInd (Ind20 a), Empty, Empty, Empty, Empty,
               Append (Ind3 b) $ Append (Ind3 c) $ Append (Ind3 d) 
               $ singletonInd $ Ind3 e) |
               a <- [0..20], b <- [0..3], c <- [0..3], d <- [0..3],
               e <- [0..3], not (b == c || d == e)]
        f (ind1, _, _, _, _, ind2)
            | ind1 == (M.!) trianArea indArea = s
            | otherwise = 0
            where
                (indArea, s) = canonicalizeArea ind2
\end{minted} 
\caption{Construction of Area Metric I Intertwiner. }\label{ConstrINterIArea}
\end{listing}
\begin{listing}[hbt!]
\begin{minted}[frame = lines, framesep = 2.5mm, baselinestretch =
1.2, bgcolor=LG!40]{haskell}
interJArea :: ATens 0 1 0 0 4 0 (SField Rational)
interJArea = fromListT6 $ fmap (fmap SField) $ filter (\(i,k) -> k /= 0) $
             map (\x -> (x,f x)) inds
    where
        trianArea = trianMapArea
        inds = [(Empty, singletonInd $ Ind20 a, Empty, Empty,
               Append (Ind3 b) $ Append (Ind3 c) $ Append (Ind3 d)
               $ singletonInd $ Ind3 e, Empty) |
               a <- [0..20], b <- [0..3], c <- [0..3], d <- [0..3],
               e <- [0..3], not (b == c || d == e)]
        f (_, ind1, _, _, ind2, _)
            | ind1 == (M.!) trianArea indArea = s * jMultArea indArea
            | otherwise = 0
            where
                (indArea, s) = canonicalizeArea ind2
\end{minted} 
\caption{Construction of Area Metric J Intertwiner.}\label{ConstrInterJArea}
\end{listing}
Here \mintinline{haskell}{trianMap2} is a Map that relates the indices of a symmetric pair of spacetime indices to the corresponding abstract index of type \mintinline{haskell}{Ind9}, \mintinline{haskell}{trianMapArea} is a Map that relates the $4$ spacetime indices of a given area metric to the corresponding abstract index, \mintinline{haskell}{canonicalizeArea} is a function that brings a set of $4$ spacetime indices obeying the area metric symmetries to canonical order taking into account possible signs that might occur during the resorting, \mintinline{haskell}{jMult2} is a function that computes the $\sigma$ multiplicity of the symmetric index pair and \mintinline{haskell}{jMultArea} computes the $\sigma$ multiplicity (see discussion following definition \ref{interDef} for the precise definition of these multiplicities $\sigma$) of a given set of $4$ spacetime indices that correspond to one area metric.

From the two area metric intertwiners, we can compute the constant tensor $C^{Am}_{Bn}$ according to (\ref{areaGotayMInter}):

\begin{center}
\begin{cminted}{haskell}
interArea :: ATens 1 1 0 0 1 1 (SField Rational)
interArea = (-4) &. contrATens3 (1,1) (contrATens3 (2,2) $
            contrATens3 (3,3) $ interIArea &* interJArea)
\end{cminted}
\end{center}
The flat background area metric can be constructed as (\ref{ConstrFlatArea}).
\begin{listing}[hbt!]
\begin{minted}[frame = lines, framesep = 2.5mm, baselinestretch =
1.2, bgcolor=LG!40]{haskell}
flatArea :: ATens 0 1 0 0 0 0 (SField Rational)
flatArea = fromListT6 $ map (\(i,v) -> ( (Empty, singletonInd $ Ind20 i,
           Empty, Empty, Empty, Empty), SField v))
           [(0,-1),(5,-1),(6,-1),(9,1),(11,-1),
           (12,-1),(15,1),(18,1),(20,1)]
\end{minted} 
\caption{Construction of Flat Area Metric.}\label{ConstrFlatArea}
\end{listing}

The next step is the computation of the three bases of Lorentz invariant tensors. Clearly, $a_0$ is simply a single constant. As such, it can be incorporated in our tensor framework by:
\begin{center}
\begin{cminted}{haskell}
let ans0 = fromListT6' [(([],[],[],[],[],[]), AnsVar $ 
           I.fromList [(1,1)] )] :: ATens 0 0 0 0 0 0 (AnsVarR)
\end{cminted}
\end{center}
The second expansion coefficient is given by $a^A = I^A_{abcd} a^{abcd}$, where the indices $(abcd)$ feature the area metric symmetries. In order to use our functions for constructing the appropriate Lorentz invariant basis tensors we need to provide the symmetry at hand as a value of type \mintinline{haskell}{Symmetry} and also need to provide a list of index value lists that are necessary for the evaluation. Labeling the indices $(abcd)$ by the integers $(1,2,3,4)$ the symmetry is given as:
\begin{center}
\begin{cminted}{haskell}
symList4 :: Symmetry
symList4 = ([], [(1,2),(3,4)], [([1,2],[3,4])], [], [])
\end{cminted}
\end{center}
This simply corresponds to pair anti symmetries in the indices $(1,2)$ and $(3,4)$ and a block symmetry w.r.t. exchange of the two index pairs $(1,2)$ and $(3,4)$. The evaluation list can be constructed as (\ref{ConstrAreaList4}).
\begin{listing}[hbt!]
\begin{minted}[frame = lines, framesep = 2.5mm, baselinestretch =
1.2, bgcolor=LG!40]{haskell}
areaList4 :: [([Int], Int, [IndTupleAbs 1 0 0 0 0 0])]
areaList4 = list
      where
          trianArea = trianMapArea
          list = [ let a' = (I.!) trianArea a in (a', areaMult a',
                 [(singletonInd (Ind20 $ a-1), Empty, Empty, Empty,
                 Empty, Empty)]) | a <- [1..21] ]
\end{minted} 
\caption{Construction of Area Metric Evaluation List 1.}\label{ConstrAreaList4}
\end{listing}
As before \mintinline{haskell}{trianMapArea} is a Map that relates a block of 4 spacetime indices with area metric symmetries to an abstract area metric index and \mintinline{haskell}{areaMult} is a function that computes the multiplicity of a given set of 4 spacetime indices with area metric symmetry. Note that we do not only need to provide the index value list that is necessary for the evaluation of the ansatz forests, but also need to include the multiplicities of the index values and the corresponding abstract indices as we want to make use of the function \mintinline{haskell}{mkAnsatzTensorEigAbs} to evaluate the ansatz forests to the chosen abstract tensor type \mintinline{haskell}{ATens}. 

We can now construct the basis for the expansion coefficient $a^{A}$ by invoking:
\begin{center}
\begin{cminted}{haskell}
let (eta4,eps4,ans4) = mkAnsatzTensorEigAbs 4 symList4 areaList4 :: 
                         (AnsatzForestEta, AnsatzForestEpsilon,
                         ATens 1 0 0 0 0 0 (AnsVarR))
\end{cminted}
\end{center}
If we wish to check the constructed ansatz forest quickly we can do so by using the functions \mintinline{haskell}{drawAnsatzEta} and \mintinline{haskell}{drawAnsatzEpsilon} that return a simple ASCII drawing of the forest structure. Doing so we get:
\begin{center}
\begin{BVerbatim}
(1,3)
|
`---- (2,4) * (4) * x[1]

(1,4)
|
`---- (2,3) * (-4) * x[1]

(1,2,3,4) * (8) * x[2]
\end{BVerbatim}
\end{center}
Proceeding along the same lines for the ansatz $a^{AI} = I^A _{abcd} I^I_{pq} a^{abcdpq}$ we first provide the symmetry:
\begin{center}
\begin{cminted}{haskell}
symList6 :: Symmetry
symList6 = ([(5,6)], [(1,2),(3,4)], [([1,2],[3,4])], [], [])
\end{cminted}
\end{center}
This simply corresponds to an additional pair symmetry in the derivative indices $(5,6)$. The evaluation list can be obtained as (\ref{ConstrAreaList6}).
\begin{listing}[hbt!]
\begin{minted}[frame = lines, framesep = 2.5mm, baselinestretch =
1.2, bgcolor=LG!40]{haskell}
areaList6 :: [([Int], Int, [IndTupleAbs 1 0 1 0 0 0])]
areaList6 = list
      where
          trian2 = trianMap2
          trianArea = trianMapArea
          list = [ let (a',i') = ((I.!) trianArea a, (I.!) trian2 i) in
                 (a' ++ i', areaMult a' * iMult2 i', 
                 [(singletonInd (Ind20 $ a-1), Empty,
                 singletonInd (Ind9 $ i-1),
                 Empty, Empty, Empty)]) | a <- [1..21], i <- [1..10]]
\end{minted} 
\caption{Construction of Area Metric Evaluation List 2.}\label{ConstrAreaList6}
\end{listing}

We now can again construct the ansatz forests:
\begin{center}
\begin{cminted}{haskell}
let (eta6,eps6,ans6) = mkAnsatzTensoreigAbs 6 symList6 areaList6 :: 
                       (AnsatzForestEta, AnsatzForestEpsilon,
                       ATens 1 0 1 0 0 0 (AnsVarR))
\end{cminted}
\end{center}
Note that when constructed, the variables that are included in any pair of ansatz forests start from one. When one uses several ansatz forests in one equation, it is thus necessary to relabel certain variables. The number of variables in a given pair of ansatz forest can, for instance, be obtained by applying the function \mintinline{haskell}{tensorRank6'} on the corresponding tensor. This function arranges the independent components of the tensor in a matrix with the variables labeling the columns and then invokes Eigen subroutines to compute the rank of the matrix. for instance applying it on the ansatz tensor \mintinline{haskell}{ans4} we get \mintinline{haskell}{tensorRank6' ans4 = 2}. Hence we can use it to shift the labels of the variables that are included in the three tensors appropriately:
\begin{center}
\begin{cminted}{haskell}
let ans0' = shiftLabels6 5 ans0
let ans4' = shiftLabels6 3 ans4 
\end{cminted}
\end{center}

Now the first 3 variables are those contained in \mintinline{haskell}{ans6} the next 2 are those in \mintinline{haskell}{ans4} and the last variable is those from \mintinline{haskell}{ans0}.
We can now finally construct the two equations in (\ref{order1}). The source code is displayed in (\ref{ConstrEqn}).
\begin{listing}[hbt!]
\begin{minted}[frame = lines, framesep = 2.5mm, baselinestretch =
1.2, bgcolor=LG!40]{haskell}
eqn1 :: ATens 0 0 0 0 0 0 (AnsVarR) ->
        ATens 1 0 0 0 0 0 (AnsVarR) ->
        ATens 0 0 0 0 1 1 (AnsVarR)
eqn1 ans0 ans4 = contrATens1 (0,0) (ans4 &* flatInter) &+
                 (ans0 &* delta3)
                 
eqn2 :: ATens 1 0 1 0 0 0 (AnsVarR) ->
        ATens 0 0 0 0 3 1 (AnsVarR)
eqn2 ans6 = contrATens2 (0,0) $ contrATens1 (0,0) $ ans6 &*
            contrATens1 (0,1) (interEqn5 &* flatArea)
    where 
        interEqn5 = cyclicSymATens5 [0,1,2] $
                    interJ2 &* interMetric
\end{minted} 
\caption{Construction of Area Metric Perturbative Equivariance  Equations.}\label{ConstrEqn}
\end{listing}

We can now, for instance, check the rank of the equations. This can be achieved by first computing the two equations:
\begin{center}
\begin{cminted}{haskell}
let eqn1Area = eqn1 ans0' ans4' 
let eqn2Area = eqn2 ans6  
\end{cminted}
\end{center}
Then we collect the two equations in a list:
\begin{center}
\begin{cminted}{haskell}
let tList = eqn2Area &.&> (singletonTList6 eqn1Area ::
            TensList6 Ind20 Ind9 Ind3 (AnsVarR)) 
\end{cminted}
\end{center}
Then we compute the rank of the collective tensor equations that are included in such a list to find:
\begin{center}
\begin{cminted}{haskell}
tensorRank6 tList = 2 
\end{cminted}
\end{center}
We can also extract the information given by the two tensor equations in the form of a matrix with as usual columns labeling the 6 variables and rows labeling independent equations. The sparse-tensor function \mintinline{haskell}{toMatList6} for instance returns this matrix in terms of a standard sparse matrix format \mintinline{haskell}{[((Int,Int),(SField Rational))]}, where the pair of integers label row and column and the rational number is the corresponding value. Applying this function we find:
\begin{center}
\begin{cminted}{haskell}
toMatList6 tList = [((1,1),1152 % 1),((1,2),576 % 1),((1,3),(-2304) % 1),
                   ((2,4),(-96) % 1),((2,5),192 % 1),((2,6),1 % 1)]
\end{cminted}
\end{center}
Which in matrix corresponds to:
\begin{align}
    \begin{bmatrix}
    1152 & 576 & -2304 & 0 & 0 & 0 \\
    0 & 0 & 0 & -96 & 192 & 1
    \end{bmatrix}.
\end{align}
Note that the relatively large values are a result of the used factor less symmetrization method. The thus generated output can now easily be used as input for standard computer algebra systems such as Maple or Mathematica that then can be used to solve the perturbative equivariance equations.
The further required tensors that were used to derive the second-order solution displayed in Appendix \ref{AppArea}
are also included in the sparse-tensor sub modules \textit{Gravity} and \textit{Gravity.DiffeoSymEqns} which can also be found in \cite{sparse-tensor}. 


\chapter{Conclusions}
Summing up in this thesis, we have in-depth surveyed the central question of Constructive Gravity, i.e., we have provided a reasoned method for the construction of new theories of gravity. We have translated the central requirements of diffeomorphism invariant dynamics and causal compatibility between matter and gravitational EOM into rigorous mathematics. Diffeomorphism invariance was encoded as an equivalent system of linear, first-order partial differential equations for the gravitational Lagrangian. The causal compatibility requirement was further included by enforcing equality of the vanishing sets of the matter and gravitational principal polynomials.
This finally allowed us to arrive at the first main result that we presented in this thesis, a construction recipe that takes an arbitrary matter theory which is supported by a tensorial background geometry as input data and then provides a step by step manual for the construction of the most general diffeomorphism invariant, compatible theory of gravity Algorithm \ref{Algo1}. 

Also, this achievement is unarguably of great importance it suffers from the fact that for nearly all relevant examples it is strikingly hard to obtain general solutions to the 
PDE system that encodes diffeomorphism invariance. 
This fact forced us to take a slight detour into the field of formal PDE theory, to collect information and tools how one might nevertheless obtain solutions to such a PDE system. 
With the detection of gravitational waves as a possible test for alternative theories of gravity in mind, we decided to take the route perturbative solution techniques\footnote{As remarked in the central part of the thesis it would certainly also be possible to apply symmetry reduction techniques for obtaining a solution. This would then yield a cosmological approximation of alternative gravity theory. Without a doubt, this constitutes an exciting area of future research that might build upon the foundation laid by this thesis.} to such PDEs through power series expansions of the Lagrangian. To justify the perturbative approach, we consider the proof of involution of the equivariance equations
as vital. Only by knowing this, we can really ensure that all information that is contained in the equivariance equations is also included in the perturbative approach. 
Finally, we also cast the requirement of causal compatibility into an equivalent perturbative form, using well-known formulae for the expansion of determinants. The perturbative approach to Constructive Gravity culminated in the formulation of yet another construction manual Algorithm \ref{Algo2} which represents the perturbative version of Algorithm \ref{Algo1}. Providing a given matter theory with tensorial background geometry, an expansion point for this geometry and a chosen expansion order as input data, in Algorithm \ref{Algo2} we outline in detail which steps one has to undertake in order to derive the most general, compatible, diffeomorphism invariant, perturbative theory of gravity. The all-important observation now lies in the fact that this construction algorithm almost only involves techniques from linear algebra. Although the resulting problems represent a technical challenge as once considering real generalizations to GR the dimensions of the resulting linear systems might become quite large, using modern computer hardware they are certainly solvable for many concrete examples. 

We further developed a computer algebra library written in the programming language Haskell that is explicitly designed to deal with problems that arise in this context, but further may also suit as a general-purpose tensor algebra computer program. 
In this thesis, we explained the main ideas that underlie the incorporated data structures and implemented algorithms. Further, we also provided a short step by step user guide. 

Finally, in this thesis, we also considered two real-world applications of the developed perturbative framework. In particular, the two examples, usual metric gravity and the more general area metric gravity are not to be confused with simple toy theories but should rather be seen as real candidates for the description of gravitational interactions. In both examples, we explicitly followed the steps required by Algorithm \ref{Algo2} in constructing a perturbative expansion of the relevant Lagrangians up to third order. While in the metric case by interpreting the two arising constants, as usual, gravitational and cosmological constant, we successfully recovered the appropriate third-order expansion of the Einstein-Hilbert-Lagrangian in the case of area metric gravity there obviously does not exist any established reference theory to which we can compare the computed result. In this case, we obtained a theory that contained $52$ undetermined gravitational parameters. 
With such a third-order expansion of the area metric gravity Lagrangian, it is now possible to not only calculate the propagation but most importantly also the emission of gravitational waves in the context of this theory of gravity. Thus in principle, it is straight forward how one can, for the first time, compare the area metric against real observations. This indeed is precisely where I personally hope that future research will build upon. 
As we did not include any ad hoc assumptions in the construction of the are metric gravity Lagrangian any failed comparison to experiment will not only rule out a particular, specific candidate theory but a whole family of such. In other words, comparing the constructed area metric Lagrangian against observations does not only test a particular modified theory of gravity but the whole idea of an area metric as describing gravitational interactions. 
On the other hand, observations that do not entirely contradict the calculations that one can compute in the context of the developed theory of area metric gravity will yield essential knowledge regarding possible values or boundaries for the $52$ yet undetermined parameters that are contained therein. 
Directly from specific calculations, one might further see which parameters contribute to which regime. 

In precisely the same lines, I also hope that the framework that was presented in this thesis will help in testing further alternative theories of gravity. At least in the perturbative setting, with the developed computer algebra program it is mostly a matter of days to obtain gravitational Lagrangians for any given tensorial geometry. The real challenge thus is to find worthy candidates of spacetime geometries. 

Finally, I also hope that the Haskell library that was developed in the context of this thesis will be further supplemented and improved. This is a subject that I will not entirely leave open for future research. Also, in the future, I will actually actively contribute further improvements to it. One particular thing that we already have in mind is providing bindings to a symbolic simplifier such as \cite{SymPy}, thus also allowing for symbolic tensors. This would then really complete sparse-tensor to a universal tensor algebra library.

With this, I would like to end my thesis. I sincerely hope that the presented results support the problems that we deal with in the context of modifying gravity with a reasoned structure, eliminated at least some of the included guesswork and thus in the sense of Rovelli advance science at least by a small step.


\appendix

\chapter{Perturbative Lagrangians}
\dictum{
Here we display the additional results from the two exemplary applications of the developed framework for constructing perturbative diffeomorphism invariant theories of gravity causal compatible with a given matter theory. 
All provided results were obtained by utilizing the developed computer programm \cite{sparse-tensor}.
More suspicious readers are kindly invited to check the displayed results. The equations that are necessary for this are included in the \textit{Gravity.DiffeoSymEqns} submodule in \cite{sparse-tensor}.
}
\section{2nd-Order Metric Gravity}\label{AppGR}
In the following, the various expressions for the linearly independent Lorentz invariant expansion coefficients with the given index structure and symmetry that we obtained by means of the developed computer program are displayed. In order to allow for a concise notation we display these in terms of the inverse Minkowski metric $\eta^{ab}$, and the contravariant Levi-Civita symbol $\epsilon^{abcd}$ and furthermore do not denote their symmetries explicitly, as these can easily be read off from the way they appear in the expansion (\ref{LGR}). It is, however, essential to keep in mind that their symmetrization yields further factors that therefore are present when we insert these in the perturbative equivariance equations. We find from the developed computer program:\\

\begin{longtable}{LLL} \toprule
\begin{aligned}
&\textbf{coefficient}\\
&(\text{\small abtract indices})
\end{aligned} &
\begin{aligned}
&\textbf{coefficient}\\
&(\text{\small spacetime indices})
\end{aligned} &
\begin{aligned}
&\textbf{ansatz}\\
&\hspace{1cm}
\end{aligned}\\
\addlinespace
\midrule
\addlinespace
\boldsymbol{a_0}  & \boldsymbol{a_0} & \boldsymbol{\mu_1} \\
\addlinespace
\midrule
\addlinespace
\boldsymbol{a^{A}}  & \boldsymbol{a^{ab}} & \boldsymbol{\mu_2} \cdot \eta^{ab} \\
\addlinespace
\midrule
\addlinespace
\boldsymbol{a^{AI}} & \boldsymbol{a^{abpq}}  &  \boldsymbol{\nu_1} \cdot \eta^{ab}\eta^{pq} + \boldsymbol{\nu_2} \cdot \eta^{pa} \eta^{bq} \\
\addlinespace
\midrule
\addlinespace
\boldsymbol{a^{AB}} &  \boldsymbol{a^{abcd}}  &  \boldsymbol{\mu_3} \cdot \eta^{ab}\eta^{cd} + \boldsymbol{\mu_4} \cdot  \eta^{ca} \eta^{bd}\\
\addlinespace
\midrule
\addlinespace
\boldsymbol{a^{ApBq}} & \boldsymbol{a^{abpcdq}}  & 
\begin{aligned}
&\hphantom{+ \ }\boldsymbol{\nu_3} \cdot \eta^{ab}\eta^{pc}\eta^{dq}
+\boldsymbol{\nu_4} \cdot \eta^{ab}\eta^{pq}\eta^{cd}
+\boldsymbol{\nu_5} \cdot \eta^{ap}\eta^{bc}\eta^{dq}
\\
&+\boldsymbol{\nu_6} \cdot \eta^{ac}\eta^{bd}\eta^{pq}
+\boldsymbol{\nu_7} \cdot \eta^{ac}\eta^{bq}\eta^{pd} +
\boldsymbol{\nu_8} \cdot \epsilon^{apcq}\eta^{bd}    
\end{aligned}
\\
\addlinespace
\midrule
\addlinespace
\boldsymbol{a^{ABI}} & \boldsymbol{a^{abcdpq}}  & 
\begin{aligned}
&\hphantom{+ \ }\boldsymbol{\nu_9} \cdot \eta^{ab}\eta^{cd}\eta^{pq}
+\boldsymbol{\nu_{10}} \cdot \eta^{ab}\eta^{cp}\eta^{dq}
+\boldsymbol{\nu_{11}} \cdot \eta^{ac}\eta^{bd}\eta^{pq}
\\
&+\boldsymbol{\nu_{12}} \cdot \eta^{ac}\eta^{bp}\eta^{dq}
+\boldsymbol{\nu_{13}} \cdot \eta^{ap}\eta^{bq}\eta^{cd}
\end{aligned}
\\
\addlinespace
\midrule
\addlinespace
\boldsymbol{a^{ABC}} & \boldsymbol{a^{abcdef}}  & 
\boldsymbol{\mu_5}\cdot \eta^{ab}\eta^{cd}\eta^{ef}
+\boldsymbol{\mu_6}\cdot \eta^{ab}\eta^{ce}\eta^{df}
+\boldsymbol{\mu_7}\cdot \eta^{ac}\eta^{be}\eta^{df}
\\
\addlinespace
\midrule
\addlinespace
\boldsymbol{a^{ABpCq}}  &  \boldsymbol{a^{abcdpefq}}  &
\begin{aligned}
&\hphantom{+ \ }\boldsymbol{\nu_{14}}\cdot\eta^{ab}\eta^{cd}\eta^{pe}\eta^{fq}
+\boldsymbol{\nu_{15}}\cdot\eta^{ab}\eta^{cd}\eta^{pq}\eta^{ef}\\
&+\boldsymbol{\nu_{16}}\cdot\eta^{ab}\eta^{cp}\eta^{de}\eta^{fq}
+\boldsymbol{\nu_{17}}\cdot\eta^{ab}\eta^{ce}\eta^{df}\eta^{pq}\\
&+\boldsymbol{\nu_{18}}\cdot\eta^{ab}\eta^{ce}\eta^{dq}\eta^{pf}
+\boldsymbol{\nu_{19}}\cdot\eta^{ac}\eta^{bd}\eta^{pe}\eta^{fq}\\
&+\boldsymbol{\nu_{20}}\cdot\eta^{ac}\eta^{bd}\eta^{pq}\eta^{ef}
+\boldsymbol{\nu_{21}}\cdot\eta^{ac}\eta^{bp}\eta^{de}\eta^{fq}\\
&+\boldsymbol{\nu_{22}}\cdot\eta^{ac}\eta^{bp}\eta^{dq}\eta^{ef}
+\boldsymbol{\nu_{23}}\cdot\eta^{ac}\eta^{be}\eta^{dp}\eta^{fq}\\
&+\boldsymbol{\nu_{24}}\cdot\eta^{ac}\eta^{be}\eta^{df}\eta^{pq}
+\boldsymbol{\nu_{25}}\cdot\eta^{ac}\eta^{be}\eta^{dq}\eta^{pf}\\
&+\boldsymbol{\nu_{26}}\cdot\eta^{ac}\eta^{bq}\eta^{dp}\eta^{ef}
+\boldsymbol{\nu_{27}}\cdot\eta^{ac}\eta^{bq}\eta^{de}\eta^{pf}\\
&+\boldsymbol{\nu_{28}}\cdot\eta^{ap}\eta^{bq}\eta^{cd}\eta^{ef}
+\boldsymbol{\nu_{29}}\cdot\eta^{ap}\eta^{bq}\eta^{ce}\eta^{df}\\
&+\boldsymbol{\nu_{30}}\cdot\epsilon^{acpe}\eta^{bd}\eta^{fq}
+\boldsymbol{\nu_{31}}\cdot\epsilon^{acpe}\eta^{bf}\eta^{dq}\\
&+\boldsymbol{\nu_{32}}\cdot\epsilon^{acpe}\eta^{bq}\eta^{df}
+\boldsymbol{\nu_{33}}\cdot\epsilon^{acpq}\eta^{bd}\eta^{ef}\\ 
&+\boldsymbol{\nu_{34}}\cdot\epsilon^{acpq}\eta^{be}\eta^{df} 
\end{aligned}
\\
\addlinespace
\midrule
\addlinespace
\boldsymbol{a^{ABCI}}  &  \boldsymbol{a^{abcdefpq}}  &
\begin{aligned}
&\hphantom{+ \ }\boldsymbol{\nu_{35}}\cdot\eta^{ab}\eta^{cd}\eta^{ef}\eta^{pq}
+\boldsymbol{\nu_{36}}\cdot\eta^{ab}\eta^{cd}\eta^{ep}\eta^{fq}\\
&+\boldsymbol{\nu_{37}}\cdot\eta^{ab}\eta^{ce}\eta^{df}\eta^{pq}
+\boldsymbol{\nu_{38}}\cdot\eta^{ab}\eta^{ce}\eta^{dp}\eta^{fq}\\
&+\boldsymbol{\nu_{39}}\cdot\eta^{ab}\eta^{cp}\eta^{dq}\eta^{ef}
+\boldsymbol{\nu_{40}}\cdot\eta^{ac}\eta^{bd}\eta^{ef}\eta^{pq}\\
&+\boldsymbol{\nu_{41}}\cdot\eta^{ac}\eta^{bd}\eta^{ep}\eta^{fq}
+\boldsymbol{\nu_{42}}\cdot\eta^{ac}\eta^{be}\eta^{df}\eta^{pq}\\
&+\boldsymbol{\nu_{43}}\cdot\eta^{ac}\eta^{be}\eta^{dp}\eta^{fq}
+\boldsymbol{\nu_{44}}\cdot\eta^{ac}\eta^{bp}\eta^{dq}\eta^{ef}\\
&+\boldsymbol{\nu_{45}}\cdot\eta^{ae}\eta^{bf}\eta^{cp}\eta^{dq}
+\boldsymbol{\nu_{46}}\cdot\eta^{ae}\eta^{bp}\eta^{cf}\eta^{dq}\\
&+\boldsymbol{\nu_{47}}\cdot\epsilon^{acep}\eta^{bf}\eta^{dq}
\end{aligned}
\\
\addlinespace
\bottomrule
\caption{Lorentz Invariant Expansion Coefficients for Metric Gravity Lagrangian (\ref{LGR}).}\label{LorentzGR1}
\end{longtable}

\vspace{1cm}

From inserting these expansion coefficients into the perturbative equivariance equations, we get the following relation between the parameters.\\

\begin{longtable}{RLRLRL}\toprule
\textbf{parameter} & \textbf{solution \ \ \ \ } &
\textbf{parameter} & \textbf{solution \ \ \ \ } &
\textbf{parameter} & \textbf{solution \ \ \ \ } \\
\addlinespace
\midrule
\addlinespace
\boldsymbol{\mu_{2}} &  1/4\cdot \boldsymbol{\mu_1} &
\boldsymbol{\mu_{3}} & 1/32\cdot \boldsymbol{\mu_1} &
\boldsymbol{\mu_{4}} & -1/16\cdot \boldsymbol{\mu_1}\\
\addlinespace
\boldsymbol{\mu_{5}} & 1/384\cdot \boldsymbol{\mu_1} &
\boldsymbol{\mu_{6}} & -1/64\cdot \boldsymbol{\mu_1} &
\boldsymbol{\mu_{7}} & 1/48\cdot \boldsymbol{\mu_1}\\
\addlinespace
\boldsymbol{\nu_{2}} & -\boldsymbol{\nu_1} &
\boldsymbol{\nu_{3}} & -\boldsymbol{\nu_1} &
\boldsymbol{\nu_{4}} & 1/4\cdot \boldsymbol{\nu_1}\\
\addlinespace
\boldsymbol{\nu_{5}} & \boldsymbol{\nu_1} &
\boldsymbol{\nu_{6}} & -3/4 \cdot  \boldsymbol{\nu_1} &
\boldsymbol{\nu_{7}} & 1/2\cdot \boldsymbol{\nu_1}\\
\addlinespace
\boldsymbol{\nu_{8}} &  0 &
\boldsymbol{\nu_{9}} & 1/4\cdot \boldsymbol{\nu_1}  &
\boldsymbol{\nu_{10}}  & -1/4 \cdot  \boldsymbol{\nu_1}\\
\addlinespace
\boldsymbol{\nu_{11}} & -1/2\cdot \boldsymbol{\nu_1}  &
\boldsymbol{\nu_{12}} & \boldsymbol{\nu_1}  &
\boldsymbol{\nu_{13}}  & -1/2\cdot \boldsymbol{\nu_1}\\
\addlinespace
\boldsymbol{\nu_{14}} & -1/4\cdot \boldsymbol{\nu_1} &
\boldsymbol{\nu_{15}} & 1/16\cdot \boldsymbol{\nu_1} &
\boldsymbol{\nu_{16}} & 1/4\cdot \boldsymbol{\nu_1} \\
\addlinespace
\boldsymbol{\nu_{17}} & -3/16\cdot \boldsymbol{\nu_1} &
\boldsymbol{\nu_{18}} & 1/8\cdot \boldsymbol{\nu_1} &
\boldsymbol{\nu_{19}} & 1/2\cdot \boldsymbol{\nu_1} \\
\addlinespace
\boldsymbol{\nu_{20}} & -1/4\cdot \boldsymbol{\nu_1} &
\boldsymbol{\nu_{21}}  & - \boldsymbol{\nu_1} &
\boldsymbol{\nu_{22}}  & 1/2\cdot \boldsymbol{\nu_1}\\
\addlinespace
\boldsymbol{\nu_{23}} & -1/2\cdot \boldsymbol{\nu_1} &
\boldsymbol{\nu_{24}} & 3/4\cdot \boldsymbol{\nu_1} &
\boldsymbol{\nu_{25}} & -1/4\cdot \boldsymbol{\nu_1} \\
\addlinespace
\boldsymbol{\nu_{26}} & 1/2\cdot \boldsymbol{\nu_1} &
\boldsymbol{\nu_{27}} & -1/2\cdot \boldsymbol{\nu_1}&
\boldsymbol{\nu_{28}} & -1/8\cdot \boldsymbol{\nu_1}\\
\addlinespace
\boldsymbol{\nu_{29}} & 3/8\cdot \boldsymbol{\nu_1} &
\boldsymbol{\nu_{30}} &  0 &
\boldsymbol{\nu_{31}} & 0 \\
\addlinespace
\boldsymbol{\nu_{32}} & 0  &
\boldsymbol{\nu_{33}} & 0 &
\boldsymbol{\nu_{34}} &  0 \\
\addlinespace
\boldsymbol{\nu_{35}} & 1/32\cdot \boldsymbol{\nu_1} &
\boldsymbol{\nu_{36}} & -1/32\cdot \boldsymbol{\nu_1} &
\boldsymbol{\nu_{37}} & -1/8\cdot \boldsymbol{\nu_1}\\
\addlinespace
\boldsymbol{\nu_{38}} & 1/4\cdot \boldsymbol{\nu_1}  &
\boldsymbol{\nu_{39}} & -1/8\cdot \boldsymbol{\nu_1} &
\boldsymbol{\nu_{40}} & -1/16\cdot \boldsymbol{\nu_1}\\
\addlinespace
\boldsymbol{\nu_{41}} & 1/16\cdot \boldsymbol{\nu_1} &
\boldsymbol{\nu_{42}} & 1/4\cdot \boldsymbol{\nu_1}  &
\boldsymbol{\nu_{43}} & -1/2\cdot \boldsymbol{\nu_1}\\
\addlinespace
\boldsymbol{\nu_{44}} & 1/4\cdot \boldsymbol{\nu_1} &
\boldsymbol{\nu_{45}} & 1/4\cdot \boldsymbol{\nu_1} &
\boldsymbol{\nu_{46}} & -1/4\cdot \boldsymbol{\nu_1}\\
\addlinespace
\boldsymbol{\nu_{47}} & 0  &
&   &    &  \\
\addlinespace
\bottomrule
\caption{Solution of Metric Gravity Pertubrative Equivariance Equations. }\label{GRSol}
\end{longtable}

\vspace{1cm}

\section{2nd-Order Area Metric Gravity}\label{AppArea}
Following along the lines prescribed in (\ref{defI}) and (\ref{defJ}) we get the following non-vanishing components for the $\boldsymbol{I^A_{abcd}}$ and $\boldsymbol{J_A^{abcd}}$ intertwiners on $F_{Area}$:\\

\newpage 

\begin{longtable}{RLRLRL}\toprule
\textbf{component} & \textbf{value \ \ \ \ } &
\textbf{component} & \textbf{value \ \ \ \ } &
\textbf{component} & \textbf{value \ \ \ \ } \\
\addlinespace
\midrule 
\addlinespace
\boldsymbol{I^{0}_{0101}} & 1 &
\boldsymbol{I^{0}_{0110}} & -1 &
\boldsymbol{I^{0}_{1001}} & -1 \\
\addlinespace
\boldsymbol{I^{0}_{1010}} & 1 &
\boldsymbol{I^{1}_{0102}} & 1 & 
\boldsymbol{I^{1}_{0120}} & -1 \\
\addlinespace
\boldsymbol{I^{1}_{0201}} & 1 &  
\boldsymbol{I^{1}_{0210}} & -1 &  
\boldsymbol{I^{1}_{1002}} & -1 \\
\addlinespace
\boldsymbol{I^{1}_{1020}} & 1 &
\boldsymbol{I^{1}_{2001}} & -1 &  
\boldsymbol{I^{1}_{2010}} & 1 \\
\addlinespace
\boldsymbol{I^{2}_{0103}} & 1   &  
\boldsymbol{I^{2}_{0130}} & -1   &  
\boldsymbol{I^{2}_{0301}} & 1   \\
\addlinespace
\boldsymbol{I^{2}_{0310}} & -1   &  
\boldsymbol{I^{2}_{1003}} & -1   &  
\boldsymbol{I^{2}_{1030}} & 1       \\
\addlinespace
\boldsymbol{I^{2}_{3001}} & -1   &  
\boldsymbol{I^{2}_{3010}} & 1  &  
\boldsymbol{I^{3}_{0112}} & 1       \\
\addlinespace
\boldsymbol{I^{3}_{0121}} & -1   &  
\boldsymbol{I^{3}_{1012}} & -1   &  
\boldsymbol{I^{3}_{1021}} & 1       \\
\addlinespace
\boldsymbol{I^{3}_{1201}} & 1   &  
\boldsymbol{I^{3}_{1210}} & -1   &  
\boldsymbol{I^{3}_{2101}} & -1       \\
\addlinespace
\boldsymbol{I^{3}_{2110}} & 1   &  
\boldsymbol{I^{4}_{0113}} & 1   &  
\boldsymbol{I^{4}_{0131}} & -1       \\
\addlinespace
\boldsymbol{I^{4}_{1013}} & -1   &  
\boldsymbol{I^{4}_{1031}} & 1   &  
\boldsymbol{I^{4}_{1301}} & 1       \\
\addlinespace
\boldsymbol{I^{4}_{1310}} & -1   &  
\boldsymbol{I^{4}_{3101}} & -1   &  
\boldsymbol{I^{4}_{3110}} & 1       \\
\addlinespace
\boldsymbol{I^{5}_{0123}} & 1   &  
\boldsymbol{I^{5}_{0132}} & -1   &  
\boldsymbol{I^{5}_{1023}} & -1       \\
\addlinespace
\boldsymbol{I^{5}_{1032}} & 1   &  
\boldsymbol{I^{5}_{2301}} & 1   &  
\boldsymbol{I^{5}_{2310}} & -1       \\
\addlinespace
\boldsymbol{I^{5}_{3201}} & -1   &  
\boldsymbol{I^{5}_{3210}} & 1   &  
\boldsymbol{I^{6}_{0202}} & 1       \\
\addlinespace
\boldsymbol{I^{6}_{0220}} & -1   &  
\boldsymbol{I^{6}_{2002}} & -1   &  
\boldsymbol{I^{6}_{2020}} & 1       \\
\addlinespace
\boldsymbol{I^{7}_{0203}} & 1   &  
\boldsymbol{I^{7}_{0230}} & -1   &  
\boldsymbol{I^{7}_{0302}} & 1       \\
\addlinespace
\boldsymbol{I^{7}_{0320}} & -1   &  
\boldsymbol{I^{7}_{2003}} & -1   &  
\boldsymbol{I^{7}_{2030}} & 1       \\
\addlinespace
\boldsymbol{I^{7}_{3002}} & -1   &  
\boldsymbol{I^{7}_{3020}} & 1   &  
\boldsymbol{I^{8}_{0212}} & 1       \\
\addlinespace
\boldsymbol{I^{8}_{0221}} & -1   &  
\boldsymbol{I^{8}_{1202}} & 1   &  
\boldsymbol{I^{8}_{1220}} & -1       \\
\addlinespace
\boldsymbol{I^{8}_{2012}} & -1   &  
\boldsymbol{I^{8}_{2021}} & 1   &  
\boldsymbol{I^{8}_{2102}} & -1       \\
\addlinespace
\boldsymbol{I^{8}_{2120}} & 1   &  
\boldsymbol{I^{9}_{0213}} & 1   &  
\boldsymbol{I^{9}_{0231}} & -1       \\
\addlinespace
\boldsymbol{I^{9}_{1302}} & 1   &  
\boldsymbol{I^{9}_{1320}} & -1   &  
\boldsymbol{I^{9}_{2013}} & -1       \\
\addlinespace
\boldsymbol{I^{9}_{2031}} & 1   &  
\boldsymbol{I^{9}_{3102}} & -1   &  
\boldsymbol{I^{9}_{3120}} & 1       \\
\addlinespace
\boldsymbol{I^{10}_{0223}} & 1   &  
\boldsymbol{I^{10}_{0232}} & -1   &  
\boldsymbol{I^{10}_{2023}} & -1       \\
\addlinespace
\boldsymbol{I^{10}_{2032}} & 1   &  
\boldsymbol{I^{10}_{2302}} & 1   &  
\boldsymbol{I^{10}_{2320}} & -1       \\
\addlinespace
\boldsymbol{I^{10}_{3202}} & -1   &  
\boldsymbol{I^{10}_{3220}} & 1   &  
\boldsymbol{I^{11}_{0303}} & 1       \\
\addlinespace
\boldsymbol{I^{11}_{0330}} & -1   &  
\boldsymbol{I^{11}_{3003}} & -1   &  
\boldsymbol{I^{11}_{3030}} & 1       \\
\addlinespace
\boldsymbol{I^{12}_{0312}} & 1   &  
\boldsymbol{I^{12}_{0321}} & -1   &  
\boldsymbol{I^{12}_{1203}} & 1       \\
\addlinespace
\boldsymbol{I^{12}_{1230}} & -1   &  
\boldsymbol{I^{12}_{2103}} & -1   &  
\boldsymbol{I^{12}_{2130}} & 1       \\
\addlinespace
\boldsymbol{I^{12}_{3012}} & -1   &  
\boldsymbol{I^{12}_{3021}} & 1   &  
\boldsymbol{I^{13}_{0313}} & 1       \\
\addlinespace
\boldsymbol{I^{13}_{0331}} & -1   &  
\boldsymbol{I^{13}_{1303}} & 1   &  
\boldsymbol{I^{13}_{1330}} & -1       \\
\addlinespace
\boldsymbol{I^{13}_{3013}} & -1   &  
\boldsymbol{I^{13}_{3031}} & 1   &  
\boldsymbol{I^{13}_{3103}} & -1       \\
\addlinespace
\boldsymbol{I^{13}_{3130}} & 1   &  
\boldsymbol{I^{14}_{0323}} & 1   &  
\boldsymbol{I^{14}_{0332}} & -1       \\
\addlinespace
\boldsymbol{I^{14}_{2303}} & 1   &  
\boldsymbol{I^{14}_{2330}} & -1   &  
\boldsymbol{I^{14}_{3023}} & -1       \\
\addlinespace
\boldsymbol{I^{14}_{3032}} & 1   &  
\boldsymbol{I^{14}_{3203}} & -1   &  
\boldsymbol{I^{14}_{3230}} & 1       \\
\addlinespace
\boldsymbol{I^{15}_{1212}} & 1   &  
\boldsymbol{I^{15}_{1221}} & -1   &  
\boldsymbol{I^{15}_{2112}} & -1       \\
\addlinespace
\boldsymbol{I^{15}_{2121}} & 1   &  
\boldsymbol{I^{16}_{1213}} & 1   &  
\boldsymbol{I^{16}_{1231}} & -1       \\
\addlinespace
\boldsymbol{I^{16}_{1312}} & 1   &  
\boldsymbol{I^{16}_{1321}} & -1   &  
\boldsymbol{I^{16}_{2113}} & -1       \\
\addlinespace
\boldsymbol{I^{16}_{2131}} & 1   &  
\boldsymbol{I^{16}_{3112}} & -1   &  
\boldsymbol{I^{16}_{3121}} & 1       \\
\addlinespace
\boldsymbol{I^{17}_{1223}} & 1   &  
\boldsymbol{I^{17}_{1232}} & -1   &  
\boldsymbol{I^{17}_{2123}} & -1       \\
\addlinespace
\boldsymbol{I^{17}_{2132}} & 1   &  
\boldsymbol{I^{17}_{2312}} & 1   &  
\boldsymbol{I^{17}_{2321}} & -1       \\
\addlinespace
\boldsymbol{I^{17}_{3212}} & -1   &  
\boldsymbol{I^{17}_{3221}} & 1   &  
\boldsymbol{I^{18}_{1313}} & 1       \\
\addlinespace
\boldsymbol{I^{18}_{1331}} & -1   &  
\boldsymbol{I^{18}_{3113}} & -1   &  
\boldsymbol{I^{18}_{3131}} & 1       \\
\addlinespace
\boldsymbol{I^{19}_{1323}} & 1   &  
\boldsymbol{I^{19}_{1332}} & -1   &  
\boldsymbol{I^{19}_{2313}} & 1       \\
\addlinespace
\boldsymbol{I^{19}_{2331}} & -1   &  
\boldsymbol{I^{19}_{3123}} & -1   &  
\boldsymbol{I^{19}_{3132}} & 1       \\
\addlinespace
\boldsymbol{I^{19}_{3213}} & -1   &  
\boldsymbol{I^{19}_{3231}} & 1   &  
\boldsymbol{I^{20}_{2323}} & 1       \\
\addlinespace
\boldsymbol{I^{20}_{2332}} & -1   &  
\boldsymbol{I^{20}_{3223}} & -1   &  
\boldsymbol{I^{20}_{3232}} & 1   \\
\addlinespace
\bottomrule
\caption{Components: Area Metric I Intertwiner.}\label{AreaI}
\end{longtable}



\begin{longtable}{RLRLRL}\toprule
\textbf{component} & \textbf{value \ \ \ \ } &
\textbf{component} & \textbf{value \ \ \ \ } &
\textbf{component} & \textbf{value \ \ \ \ } \\
\addlinespace
\midrule 
\addlinespace
\boldsymbol{J_{0}^{0101}} & 1/4 & 
\boldsymbol{J_{0}^{0110}} & -1/4 & 
\boldsymbol{J_{0}^{1001}} & -1/4 \\
\addlinespace
\boldsymbol{J_{0}^{1010}} & 1/4 & 
\boldsymbol{J_{1}^{0102}} & 1/8 & 
\boldsymbol{J_{1}^{0120}} & -1/8 \\
\addlinespace
\boldsymbol{J_{1}^{0201}} & 1/8 & 
\boldsymbol{J_{1}^{0210}} & -1/8 & 
\boldsymbol{J_{1}^{1002}} & -1/8 \\
\addlinespace
\boldsymbol{J_{1}^{1020}} & 1/8 & 
\boldsymbol{J_{1}^{2001}} & -1/8 & 
\boldsymbol{J_{1}^{2010}} & 1/8 \\
\addlinespace
\boldsymbol{J_{2}^{0103}} & 1/8 & 
\boldsymbol{J_{2}^{0130}} & -1/8 & 
\boldsymbol{J_{2}^{0301}} & 1/8 \\
\addlinespace
\boldsymbol{J_{2}^{0310}} & -1/8 & 
\boldsymbol{J_{2}^{1003}} & -1/8 & 
\boldsymbol{J_{2}^{1030}} & 1/8 \\
\addlinespace
\boldsymbol{J_{2}^{3001}} & -1/8 & 
\boldsymbol{J_{2}^{3010}} & 1/8 & 
\boldsymbol{J_{3}^{0112}} & 1/8 \\
\addlinespace
\boldsymbol{J_{3}^{0121}} & -1/8 & 
\boldsymbol{J_{3}^{1012}} & -1/8 & 
\boldsymbol{J_{3}^{1021}} & 1/8 \\
\addlinespace
\boldsymbol{J_{3}^{1201}} & 1/8 &
\boldsymbol{J_{3}^{1210}} & -1/8 & 
\boldsymbol{J_{3}^{2101}} & -1/8\\
\addlinespace
\boldsymbol{J_{3}^{2110}} & 1/8 & 
\boldsymbol{J_{4}^{0113}} & 1/8 & 
\boldsymbol{J_{4}^{0131}} & -1/8 \\ 
\addlinespace
\boldsymbol{J_{4}^{1013}} & -1/8 & 
\boldsymbol{J_{4}^{1031}} & 1/8 & 
\boldsymbol{J_{4}^{1301}} & 1/8 \\
\addlinespace
\boldsymbol{J_{4}^{1310}} & -1/8 & 
\boldsymbol{J_{4}^{3101}} & -1/8  &
\boldsymbol{J_{4}^{3110}} & 1/8 \\
\addlinespace
\boldsymbol{J_{5}^{0123}} & 1/8 & 
\boldsymbol{J_{5}^{0132}} & -1/8 & 
\boldsymbol{J_{5}^{1023}} & -1/8 \\
\addlinespace
\boldsymbol{J_{5}^{1032}} & 1/8 & 
\boldsymbol{J_{5}^{2301}} & 1/8 & 
\boldsymbol{J_{5}^{2310}} & -1/8 \\
\addlinespace
\boldsymbol{J_{5}^{3201}} & -1/8 & 
\boldsymbol{J_{5}^{3210}} & 1/8 & 
\boldsymbol{J_{6}^{0202}} & 1/4 \\
\addlinespace
\boldsymbol{J_{6}^{0220}} & -1/4 & 
\boldsymbol{J_{6}^{2002}} & -1/4 & 
\boldsymbol{J_{6}^{2020}} & 1/4 \\
\addlinespace
\boldsymbol{J_{7}^{0203}} & 1/8 & 
\boldsymbol{J_{7}^{0230}} & -1/8 & 
\boldsymbol{J_{7}^{0302}} & 1/8 \\
\addlinespace
\boldsymbol{J_{7}^{0320}} & -1/8 & 
\boldsymbol{J_{7}^{2003}} & -1/8 & 
\boldsymbol{J_{7}^{2030}} & 1/8 \\
\addlinespace
\boldsymbol{J_{7}^{3002}} & -1/8 & 
\boldsymbol{J_{7}^{3020}} & 1/8 & 
\boldsymbol{J_{8}^{0212}} & 1/8 \\
\addlinespace
\boldsymbol{J_{8}^{0221}} & -1/8 & 
\boldsymbol{J_{8}^{1202}} & 1/8 & 
\boldsymbol{J_{8}^{1220}} & -1/8 \\
\addlinespace
\boldsymbol{J_{8}^{2012}} & -1/8 & 
\boldsymbol{J_{8}^{2021}} & 1/8 & 
\boldsymbol{J_{8}^{2102}} & -1/8 \\
\addlinespace
\boldsymbol{J_{8}^{2120}} & 1/8 & 
\boldsymbol{J_{9}^{0213}} & 1/8 &
\boldsymbol{J_{9}^{0231}} & -1/8 \\
\addlinespace
\boldsymbol{J_{9}^{1302}} & 1/8 & 
\boldsymbol{J_{9}^{1320}} & -1/8 & 
\boldsymbol{J_{9}^{2013}} & -1/8 \\
\addlinespace
\boldsymbol{J_{9}^{2031}} & 1/8 & 
\boldsymbol{J_{9}^{3102}} & -1/8 & 
\boldsymbol{J_{9}^{3120}} & 1/8 \\
\addlinespace
\boldsymbol{J_{10}^{0223}} & 1/8 & 
\boldsymbol{J_{10}^{0232}} & -1/8 &
\boldsymbol{J_{10}^{2023}} & -1/8 \\
\addlinespace
\boldsymbol{J_{10}^{2032}} & 1/8 & 
\boldsymbol{J_{10}^{2302}} & 1/8 & 
\boldsymbol{J_{10}^{2320}} & -1/8 \\
\addlinespace
\boldsymbol{J_{10}^{3202}} & -1/8 & 
\boldsymbol{J_{10}^{3220}} & 1/8 & 
\boldsymbol{J_{11}^{0303}} & 1/4 \\
\addlinespace
\boldsymbol{J_{11}^{0330}} & -1/4 & 
\boldsymbol{J_{11}^{3003}} & -1/4 & 
\boldsymbol{J_{11}^{3030}} & 1/4 \\
\addlinespace
\boldsymbol{J_{12}^{0312}} & 1/8 & 
\boldsymbol{J_{12}^{0321}} & -1/8 & 
\boldsymbol{J_{12}^{1203}} & 1/8 \\
\addlinespace
\boldsymbol{J_{12}^{1230}} & -1/8 & 
\boldsymbol{J_{12}^{2103}} & -1/8 & 
\boldsymbol{J_{12}^{2130}} & 1/8 \\ 
\addlinespace
\boldsymbol{J_{12}^{3012}} & -1/8 & 
\boldsymbol{J_{12}^{3021}} & 1/8 & 
\boldsymbol{J_{13}^{0313}} & 1/8 \\
\addlinespace
\boldsymbol{J_{13}^{0331}} & -1/8 & 
\boldsymbol{J_{13}^{1303}} & 1/8 & 
\boldsymbol{J_{13}^{1330}} & -1/8 \\
\addlinespace
\boldsymbol{J_{13}^{3013}} & -1/8 & 
\boldsymbol{J_{13}^{3031}} & 1/8 & 
\boldsymbol{J_{13}^{3103}} & -1/8 \\
\addlinespace
\boldsymbol{J_{13}^{3130}} & 1/8 & 
\boldsymbol{J_{14}^{0323}} & 1/8 & 
\boldsymbol{J_{14}^{0332}} & -1/8 \\
\addlinespace
\boldsymbol{J_{14}^{2303}} & 1/8 & 
\boldsymbol{J_{14}^{2330}} & -1/8 & 
\boldsymbol{J_{14}^{3023}} & -1/8 \\
\addlinespace
\boldsymbol{J_{14}^{3032}} & 1/8 & 
\boldsymbol{J_{14}^{3203}} & -1/8 & 
\boldsymbol{J_{14}^{3230}} & 1/8 \\
\addlinespace
\boldsymbol{J_{15}^{1212}} & 1/4 & 
\boldsymbol{J_{15}^{1221}} & -1/4 & 
\boldsymbol{J_{15}^{2112}} & -1/4 \\
\addlinespace
\boldsymbol{J_{15}^{2121}} & 1/4 & 
\boldsymbol{J_{16}^{1213}} & 1/8 & 
\boldsymbol{J_{16}^{1231}} & -1/8 \\
\addlinespace
\boldsymbol{J_{16}^{1312}} & 1/8 & 
\boldsymbol{J_{16}^{1321}} & -1/8 & 
\boldsymbol{J_{16}^{2113}} & -1/8 \\
\addlinespace
\boldsymbol{J_{16}^{2131}} & 1/8 & 
\boldsymbol{J_{16}^{3112}} & -1/8 & 
\boldsymbol{J_{16}^{3121}} & 1/8 \\
\addlinespace
\boldsymbol{J_{17}^{1223}} & 1/8 & 
\boldsymbol{J_{17}^{1232}} & -1/8 & 
\boldsymbol{J_{17}^{2123}} & -1/8 \\
\addlinespace
\boldsymbol{J_{17}^{2132}} & 1/8 & 
\boldsymbol{J_{17}^{2312}} & 1/8 & 
\boldsymbol{J_{17}^{2321}} & -1/8 \\
\addlinespace
\boldsymbol{J_{17}^{3212}} & -1/8 & 
\boldsymbol{J_{17}^{3221}} & 1/8 & 
\boldsymbol{J_{18}^{1313}} & 1/4 \\
\addlinespace
\boldsymbol{J_{18}^{1331}} & -1/4 & 
\boldsymbol{J_{18}^{3113}} & -1/4 & 
\boldsymbol{J_{18}^{3131}} & 1/4 \\
\addlinespace
\boldsymbol{J_{19}^{1323}} & 1/8 & 
\boldsymbol{J_{19}^{1332}} & -1/8 & 
\boldsymbol{J_{19}^{2313}} & 1/8 \\
\addlinespace
\boldsymbol{J_{19}^{2331}} & -1/8 & 
\boldsymbol{J_{19}^{3123}} & -1/8 & 
\boldsymbol{J_{19}^{3132}} & 1/8 \\
\addlinespace
\boldsymbol{J_{19}^{3213}} & -1/8 & 
\boldsymbol{J_{19}^{3231}} & 1/8 & 
\boldsymbol{J_{20}^{2323}} & 1/4 \\
\addlinespace
\boldsymbol{J_{20}^{2332}} & -1/4 & 
\boldsymbol{J_{20}^{3223}} & -1/4 & 
\boldsymbol{J_{20}^{3232}} & 1/4 \\
\addlinespace
\bottomrule
\caption{Components: Area Metric J Intertwiner.}\label{AreaJ}
\end{longtable}

\vspace{1cm}

When computing the Lorentz invariant expansion coefficients that occur in (\ref{LArea}) using efficient compute algebra is now essential. In principle, they are constructed precisely the same way as in the metric case. As a result of the increased fiber dimension of $F_{Area}$ compared with the metric field bundle $F_{GR}$ however, the general expressions for these Lorentz invariant coefficients are strikingly more complicated. We again do not display the factor-less symmetrizations explicitly to allow for more concise expressions.\\


\begin{longtable}{LLL} \toprule
\begin{aligned}
&\textbf{coefficient}\\
&(\text{\small abtract indices})
\end{aligned} &
\begin{aligned}
&\textbf{coefficient}\\
&(\text{\small spacetime indices})
\end{aligned} &
\begin{aligned}
&\textbf{ansatz}\\
&\hspace{1cm}
\end{aligned}\\
\addlinespace
\midrule
\addlinespace 
\boldsymbol{a_0} & \boldsymbol{a_0}  & \boldsymbol{\mu_1} \\
\addlinespace
\midrule
\addlinespace
\boldsymbol{a^{A}} &  \boldsymbol{a^{abcd}}  & \boldsymbol{\mu_2} \cdot \eta^{ac} \eta^{bd} + \boldsymbol{\mu_{3}} \cdot \epsilon^{abcd} \\
\addlinespace
\midrule
\addlinespace
\boldsymbol{a^{AI}} & \boldsymbol{a^{abcdpq}}  & 
\begin{aligned}
&\hphantom{+ \ }\boldsymbol{\nu_{1}}\cdot\eta^{ac}\eta^{bd}\eta^{pq}
+\boldsymbol{\nu_{2}}\cdot\eta^{ac}\eta^{bp}\eta^{dq}\\
&+\boldsymbol{\nu_{3}}\cdot\epsilon^{abcd}\eta^{pq}
\end{aligned}
\\
\addlinespace
\midrule
\addlinespace
\boldsymbol{a^{AB}} &  \boldsymbol{a^{abcdefgh}} &
\begin{aligned}
&\hphantom{+ \ }\boldsymbol{\mu_{4}}\cdot\eta^{ac}\eta^{bd}\eta^{eg}\eta^{fh}
+\boldsymbol{\mu_{5}}\cdot\eta^{ac}\eta^{be}\eta^{dg}\eta^{fh}\\
&+\boldsymbol{\mu_{6}}\cdot\eta^{ae}\eta^{bf}\eta^{cg}\eta^{dh} 
+\boldsymbol{\mu_{7}}\cdot\eta^{ae}\eta^{bg}\eta^{cf}\eta^{dh}\\
&+\boldsymbol{\mu_{8}}\cdot\epsilon^{abcd}\eta^{eg}\eta^{fh}
+\boldsymbol{\mu_{9}}\cdot\epsilon^{abef}\eta^{cg}\eta^{dh}
\end{aligned}
\\
\addlinespace
\midrule
\addlinespace
\boldsymbol{a^{ApBq}} & \boldsymbol{a^{abcdpefghq}}  &
\begin{aligned}
&\hphantom{+ \ }\boldsymbol{\nu_{4}}\cdot\eta^{ac}\eta^{bd}\eta^{pe}\eta^{fg}\eta^{hq}
+\boldsymbol{\nu_{5}}\cdot\eta^{ac}\eta^{bd}\eta^{pq}\eta^{eg}\eta^{fh}\\
&+\boldsymbol{\nu_{6}}\cdot\eta^{ac}\eta^{bp}\eta^{de}\eta^{fg}\eta^{hq}
+\boldsymbol{\nu_{7}}\cdot\eta^{ac}\eta^{be}\eta^{dg}\eta^{pf}\eta^{hq}\\
&+\boldsymbol{\nu_{8}}\cdot\eta^{ac}\eta^{be}\eta^{dg}\eta^{pq}\eta^{fh}
+\boldsymbol{\nu_{9}}\cdot\eta^{ac}\eta^{be}\eta^{dq}\eta^{pg}\eta^{fh}\\
&+\boldsymbol{\nu_{10}}\cdot\eta^{ap}\eta^{be}\eta^{cf}\eta^{dg}\eta^{hq}
+\boldsymbol{\nu_{11}}\cdot\eta^{ap}\eta^{be}\eta^{cg}\eta^{dh}\eta^{fq}\\
&+\boldsymbol{\nu_{12}}\cdot\eta^{ae}\eta^{bf}\eta^{cg}\eta^{dh}\eta^{pq}
+\boldsymbol{\nu_{13}}\cdot\epsilon^{abcd}\eta^{pe}\eta^{fg}\eta^{hq}\\
&+\boldsymbol{\nu_{14}}\cdot\epsilon^{abcd}\eta^{pq}\eta^{eg}\eta^{fh}
+\boldsymbol{\nu_{15}}\cdot\epsilon^{abpe}\eta^{cf}\eta^{dg}\eta^{hq}\\
&+\boldsymbol{\nu_{16}}\cdot\epsilon^{abpe}\eta^{cg}\eta^{dq}\eta^{fh}
+\boldsymbol{\nu_{17}}\cdot\epsilon^{abef}\eta^{cp}\eta^{dg}\eta^{hq}\\
&+\boldsymbol{\nu_{18}}\cdot\epsilon^{abef}\eta^{cg}\eta^{dh}\eta^{pq}
\end{aligned}
\\
\addlinespace
\midrule
\addlinespace
\boldsymbol{a^{ABI}} & \boldsymbol{a^{abcdefghpq}} &
\begin{aligned}
&\hphantom{+ \ }\boldsymbol{\nu_{19}}\cdot\eta^{ac}\eta^{bd}\eta^{eg}\eta^{fh}\eta^{pq}
+\boldsymbol{\nu_{20}}\cdot\eta^{ac}\eta^{bd}\eta^{eg}\eta^{fp}\eta^{hq}\\
&+\boldsymbol{\nu_{21}}\cdot\eta^{ac}\eta^{be}\eta^{dg}\eta^{fh}\eta^{pq}
+\boldsymbol{\nu_{22}}\cdot\eta^{ac}\eta^{be}\eta^{dg}\eta^{fp}\eta^{hq}\\
&+\boldsymbol{\nu_{23}}\cdot\eta^{ac}\eta^{be}\eta^{dp}\eta^{fg}\eta^{hq}
+\boldsymbol{\nu_{24}}\cdot\eta^{ac}\eta^{bp}\eta^{dq}\eta^{eg}\eta^{fh}\\
&+\boldsymbol{\nu_{25}}\cdot\eta^{ae}\eta^{bf}\eta^{cg}\eta^{dh}\eta^{pq}
+\boldsymbol{\nu_{26}}\cdot\eta^{ae}\eta^{bf}\eta^{cg}\eta^{dp}\eta^{hq}\\
&+\boldsymbol{\nu_{27}}\cdot\eta^{ae}\eta^{bg}\eta^{cf}\eta^{dh}\eta^{pq}
+\boldsymbol{\nu_{28}}\cdot\epsilon^{abcd}\eta^{eg}\eta^{fh}\eta^{pq}\\
&+\boldsymbol{\nu_{29}}\cdot\epsilon^{abcd}\eta^{eg}\eta^{fp}\eta^{hq}
+\boldsymbol{\nu_{30}}\cdot\epsilon^{abef}\eta^{cg}\eta^{dh}\eta^{pq}\\
&+\boldsymbol{\nu_{31}}\cdot\epsilon^{abef}\eta^{cg}\eta^{dp}\eta^{hq}
+\boldsymbol{\nu_{32}}\cdot\epsilon^{abep}\eta^{cf}\eta^{dg}\eta^{hq}\\
&+\boldsymbol{\nu_{33}}\cdot\epsilon^{abep}\eta^{cg}\eta^{dh}\eta^{fq}
+\boldsymbol{\nu_{34}}\cdot\epsilon^{efgh}\eta^{ac}\eta^{bd}\eta^{pq}
\end{aligned}
\\
\addlinespace
\midrule
\addlinespace
\boldsymbol{a^{ABC}} & \boldsymbol{a^{abcdefghijkl}}  &
\begin{aligned}
&\hphantom{+ \ }\boldsymbol{\nu_{10}}\cdot\eta^{ac}\eta^{bd}\eta^{eg}\eta^{fh}\eta^{ik}\eta^{jl}
+\boldsymbol{\nu_{11}}\cdot\eta^{ac}\eta^{bd}\eta^{eg}\eta^{fi}\eta^{hk}\eta^{jl}\\
&+\boldsymbol{\nu_{12}}\cdot\eta^{ac}\eta^{bd}\eta^{ei}\eta^{fj}\eta^{gk}\eta^{hl}
+\boldsymbol{\nu_{13}}\cdot\eta^{ac}\eta^{bd}\eta^{ei}\eta^{fk}\eta^{gj}\eta^{hl}\\
&+\boldsymbol{\nu_{14}}\cdot\eta^{ac}\eta^{be}\eta^{dg}\eta^{fi}\eta^{hk}\eta^{jl}
+\boldsymbol{\nu_{15}}\cdot\eta^{ac}\eta^{be}\eta^{di}\eta^{fg}\eta^{hk}\eta^{jl}\\
&+\boldsymbol{\nu_{16}}\cdot\eta^{ae}\eta^{bf}\eta^{ci}\eta^{dj}\eta^{gk}\eta^{hl}+\boldsymbol{\nu_{17}}\cdot\eta^{ae}\eta^{bf}\eta^{ci}\eta^{dk}\eta^{gj}\eta^{hl}
\\
&+\boldsymbol{\nu_{18}}\cdot\epsilon^{abcd}\eta^{eg}\eta^{fh}\eta^{ik}\eta^{jl}
+\boldsymbol{\nu_{19}}\cdot\epsilon^{abcd}\eta^{eg}\eta^{fi}\eta^{hk}\eta^{jl}\\
&+\boldsymbol{\nu_{20}}\cdot\epsilon^{abcd}\eta^{ei}\eta^{fj}\eta^{gk}\eta^{hl}
+\boldsymbol{\nu_{21}}\cdot\epsilon^{abcd}\eta^{ei}\eta^{fk}\eta^{gj}\eta^{hl}\\
&+\boldsymbol{\nu_{22}}\cdot\epsilon^{abef}\eta^{cg}\eta^{dh}\eta^{ik}\eta^{jl}
+\boldsymbol{\nu_{23}}\cdot\epsilon^{abef}\eta^{cg}\eta^{di}\eta^{hk}\eta^{jl}\\
&+\boldsymbol{\nu_{24}}\cdot\epsilon^{abef}\eta^{ci}\eta^{dj}\eta^{gk}\eta^{hl}
\end{aligned} 
\\
\addlinespace
\midrule
\addlinespace
\boldsymbol{a^{ABpCq}} & \boldsymbol{a^{abcdefghpijklq}}  &
\text{displayed seperately in table \ref{lorentzAreaExtra1}} 

\\
\addlinespace
\midrule
\addlinespace
\boldsymbol{a^{ABCI}} & \boldsymbol{a^{abcdefhijklpq}}  & 
\text{displayed seperately in table \ref{lorentzAreaExtra2}.}
\\
\addlinespace
\bottomrule
\caption{Lorentz Invariant Ansätze for Perturbative Area Metric Gravity Lagrangian (\ref{LArea}).}\label{LorentzArea}
\end{longtable}

\begin{longtable}{L}\toprule
\boldsymbol{a^{abcdefghpijklq}} \\
\addlinespace
\midrule
\addlinespace
\hphantom{+ }\boldsymbol{\nu_{35}}\cdot\eta^{ac}\eta^{bd}\eta^{eg}\eta^{fh}\eta^{pi}\eta^{jk}\eta^{lq}
+\boldsymbol{\nu_{36}}\cdot\eta^{ac}\eta^{bd}\eta^{eg}\eta^{fh}\eta^{pq}\eta^{ik}\eta^{jl}
+\boldsymbol{\nu_{37}}\cdot\eta^{ac}\eta^{bd}\eta^{eg}\eta^{fp}\eta^{hi}\eta^{jk}\eta^{lq}\\
\addlinespace
+\boldsymbol{\nu_{38}}\cdot\eta^{ac}\eta^{bd}\eta^{eg}\eta^{fi}\eta^{hk}\eta^{pj}\eta^{lq}
+\boldsymbol{\nu_{39}}\cdot\eta^{ac}\eta^{bd}\eta^{eg}\eta^{fi}\eta^{hk}\eta^{pq}\eta^{jl}
+\boldsymbol{\nu_{40}}\cdot\eta^{ac}\eta^{bd}\eta^{eg}\eta^{fi}\eta^{hq}\eta^{pk}\eta^{jl}\\
\addlinespace
+\boldsymbol{\nu_{41}}\cdot\eta^{ac}\eta^{bd}\eta^{ep}\eta^{fi}\eta^{gj}\eta^{hk}\eta^{lq}
+\boldsymbol{\nu_{42}}\cdot\eta^{ac}\eta^{bd}\eta^{ep}\eta^{fi}\eta^{gk}\eta^{hl}\eta^{jq}
+\boldsymbol{\nu_{43}}\cdot\eta^{ac}\eta^{bd}\eta^{ei}\eta^{fj}\eta^{gk}\eta^{hl}\eta^{pq}\\
\addlinespace
+\boldsymbol{\nu_{44}}\cdot\eta^{ac}\eta^{be}\eta^{dg}\eta^{fh}\eta^{pi}\eta^{jk}\eta^{lq}
+\boldsymbol{\nu_{45}}\cdot\eta^{ac}\eta^{be}\eta^{dg}\eta^{fh}\eta^{pq}\eta^{ik}\eta^{jl}
+\boldsymbol{\nu_{46}}\cdot\eta^{ac}\eta^{be}\eta^{dg}\eta^{fp}\eta^{hi}\eta^{jk}\eta^{lq}\\
\addlinespace
+\boldsymbol{\nu_{47}}\cdot\eta^{ac}\eta^{be}\eta^{dg}\eta^{fp}\eta^{hq}\eta^{ik}\eta^{jl}
+\boldsymbol{\nu_{48}}\cdot\eta^{ac}\eta^{be}\eta^{dg}\eta^{fi}\eta^{hk}\eta^{pj}\eta^{lq}
+\boldsymbol{\nu_{49}}\cdot\eta^{ac}\eta^{be}\eta^{dg}\eta^{fi}\eta^{hk}\eta^{pq}\eta^{jl}\\
\addlinespace
+\boldsymbol{\nu_{50}}\cdot\eta^{ac}\eta^{be}\eta^{dg}\eta^{fi}\eta^{hq}\eta^{pk}\eta^{jl}
+\boldsymbol{\nu_{51}}\cdot\eta^{ac}\eta^{be}\eta^{dp}\eta^{fg}\eta^{hi}\eta^{jk}\eta^{lq}
+\boldsymbol{\nu_{52}}\cdot\eta^{ac}\eta^{be}\eta^{dp}\eta^{fg}\eta^{hq}\eta^{ik}\eta^{jl}\\
\addlinespace
+\boldsymbol{\nu_{53}}\cdot\eta^{ac}\eta^{be}\eta^{dp}\eta^{fi}\eta^{gj}\eta^{hk}\eta^{lq}
+\boldsymbol{\nu_{54}}\cdot\eta^{ac}\eta^{be}\eta^{dp}\eta^{fi}\eta^{gk}\eta^{hl}\eta^{jq}
+\boldsymbol{\nu_{55}}\cdot\eta^{ac}\eta^{be}\eta^{dp}\eta^{fi}\eta^{gk}\eta^{hq}\eta^{jl}\\
\addlinespace
+\boldsymbol{\nu_{56}}\cdot\eta^{ac}\eta^{be}\eta^{di}\eta^{fg}\eta^{hp}\eta^{jk}\eta^{lq}
+\boldsymbol{\nu_{57}}\cdot\eta^{ac}\eta^{be}\eta^{di}\eta^{fg}\eta^{hj}\eta^{pk}\eta^{lq}
+\boldsymbol{\nu_{58}}\cdot\eta^{ac}\eta^{be}\eta^{di}\eta^{fg}\eta^{hk}\eta^{pj}\eta^{lq}\\
\addlinespace
+\boldsymbol{\nu_{59}}\cdot\eta^{ac}\eta^{be}\eta^{di}\eta^{fg}\eta^{hk}\eta^{pl}\eta^{jq}
+\boldsymbol{\nu_{60}}\cdot\eta^{ac}\eta^{be}\eta^{di}\eta^{fg}\eta^{hk}\eta^{pq}\eta^{jl}
+\boldsymbol{\nu_{61}}\cdot\eta^{ac}\eta^{be}\eta^{di}\eta^{fg}\eta^{hq}\eta^{pk}\eta^{jl}\\
\addlinespace
+\boldsymbol{\nu_{62}}\cdot\eta^{ac}\eta^{be}\eta^{di}\eta^{fp}\eta^{gj}\eta^{hk}\eta^{lq}
+\boldsymbol{\nu_{63}}\cdot\eta^{ac}\eta^{be}\eta^{di}\eta^{fp}\eta^{gk}\eta^{hl}\eta^{jq}
+\boldsymbol{\nu_{64}}\cdot\eta^{ac}\eta^{be}\eta^{di}\eta^{fj}\eta^{gp}\eta^{hk}\eta^{lq}\\
\addlinespace
+\boldsymbol{\nu_{65}}\cdot\eta^{ac}\eta^{be}\eta^{di}\eta^{fk}\eta^{gj}\eta^{hq}\eta^{pl}
+\boldsymbol{\nu_{66}}\cdot\eta^{ac}\eta^{be}\eta^{dq}\eta^{fg}\eta^{hp}\eta^{ik}\eta^{jl}
+\boldsymbol{\nu_{67}}\cdot\eta^{ac}\eta^{be}\eta^{dq}\eta^{fg}\eta^{hi}\eta^{pk}\eta^{jl}\\
\addlinespace
+\boldsymbol{\nu_{68}}\cdot\eta^{ac}\eta^{bp}\eta^{dq}\eta^{eg}\eta^{fh}\eta^{ik}\eta^{jl}
+\boldsymbol{\nu_{69}}\cdot\eta^{ae}\eta^{bf}\eta^{cg}\eta^{dh}\eta^{pi}\eta^{jk}\eta^{lq}
+\boldsymbol{\nu_{70}}\cdot\eta^{ae}\eta^{bf}\eta^{cg}\eta^{dh}\eta^{pq}\eta^{ik}\eta^{jl}\\
\addlinespace
+\boldsymbol{\nu_{71}}\cdot\eta^{ae}\eta^{bf}\eta^{cg}\eta^{dp}\eta^{hi}\eta^{jk}\eta^{lq}
+\boldsymbol{\nu_{72}}\cdot\eta^{ae}\eta^{bf}\eta^{cg}\eta^{dp}\eta^{hq}\eta^{ik}\eta^{jl}
+\boldsymbol{\nu_{73}}\cdot\eta^{ae}\eta^{bf}\eta^{cg}\eta^{di}\eta^{hp}\eta^{jk}\eta^{lq}\\
\addlinespace
+\boldsymbol{\nu_{74}}\cdot\eta^{ae}\eta^{bf}\eta^{cg}\eta^{di}\eta^{hj}\eta^{pk}\eta^{lq}
+\boldsymbol{\nu_{75}}\cdot\eta^{ae}\eta^{bf}\eta^{cg}\eta^{di}\eta^{hk}\eta^{pj}\eta^{lq}
+\boldsymbol{\nu_{76}}\cdot\eta^{ae}\eta^{bf}\eta^{cg}\eta^{di}\eta^{hk}\eta^{pl}\eta^{jq}\\
\addlinespace
+\boldsymbol{\nu_{77}}\cdot\eta^{ae}\eta^{bf}\eta^{cg}\eta^{di}\eta^{hk}\eta^{pq}\eta^{jl}
+\boldsymbol{\nu_{78}}\cdot\eta^{ae}\eta^{bf}\eta^{cg}\eta^{di}\eta^{hq}\eta^{pk}\eta^{jl}
+\boldsymbol{\nu_{79}}\cdot\eta^{ae}\eta^{bf}\eta^{cg}\eta^{dq}\eta^{hp}\eta^{ik}\eta^{jl}\\
\addlinespace
+\boldsymbol{\nu_{80}}\cdot\eta^{ae}\eta^{bf}\eta^{cg}\eta^{dq}\eta^{hi}\eta^{pk}\eta^{jl}
+\boldsymbol{\nu_{81}}\cdot\eta^{ae}\eta^{bf}\eta^{cp}\eta^{di}\eta^{gj}\eta^{hk}\eta^{lq}
+\boldsymbol{\nu_{82}}\cdot\eta^{ae}\eta^{bf}\eta^{cp}\eta^{di}\eta^{gk}\eta^{hl}\eta^{jq}\\
\addlinespace
+\boldsymbol{\nu_{83}}\cdot\eta^{ae}\eta^{bf}\eta^{cp}\eta^{di}\eta^{gk}\eta^{hq}\eta^{jl}
+\boldsymbol{\nu_{84}}\cdot\eta^{ae}\eta^{bf}\eta^{ci}\eta^{dj}\eta^{gp}\eta^{hk}\eta^{lq}
+\boldsymbol{\nu_{85}}\cdot\eta^{ae}\eta^{bf}\eta^{ci}\eta^{dj}\eta^{gk}\eta^{hl}\eta^{pq}\\
\addlinespace
+\boldsymbol{\nu_{86}}\cdot\eta^{ae}\eta^{bf}\eta^{ci}\eta^{dk}\eta^{gp}\eta^{hj}\eta^{lq}
+\boldsymbol{\nu_{87}}\cdot\eta^{ae}\eta^{bg}\eta^{cf}\eta^{dh}\eta^{pi}\eta^{jk}\eta^{lq}
+\boldsymbol{\nu_{88}}\cdot\eta^{ae}\eta^{bg}\eta^{cf}\eta^{dh}\eta^{pq}\eta^{ik}\eta^{jl}\\
\addlinespace
+\boldsymbol{\nu_{89}}\cdot\eta^{ae}\eta^{bp}\eta^{cf}\eta^{di}\eta^{gj}\eta^{hk}\eta^{lq}
+\boldsymbol{\nu_{90}}\cdot\eta^{ae}\eta^{bp}\eta^{cf}\eta^{di}\eta^{gk}\eta^{hq}\eta^{jl}
+\boldsymbol{\nu_{91}}\cdot\eta^{ae}\eta^{bi}\eta^{cf}\eta^{dj}\eta^{gp}\eta^{hk}\eta^{lq}\\
\addlinespace
+\boldsymbol{\nu_{92}}\cdot\eta^{ae}\eta^{bi}\eta^{cf}\eta^{dj}\eta^{gk}\eta^{hl}\eta^{pq}
+\boldsymbol{\nu_{93}}\cdot\epsilon^{abcd}\eta^{eg}\eta^{fh}\eta^{pi}\eta^{jk}\eta^{lq}
+\boldsymbol{\nu_{94}}\cdot\epsilon^{abcd}\eta^{eg}\eta^{fh}\eta^{pq}\eta^{ik}\eta^{jl}\\
\addlinespace
+\boldsymbol{\nu_{95}}\cdot\epsilon^{abcd}\eta^{eg}\eta^{fp}\eta^{hi}\eta^{jk}\eta^{lq}
+\boldsymbol{\nu_{96}}\cdot\epsilon^{abcd}\eta^{eg}\eta^{fi}\eta^{hk}\eta^{pj}\eta^{lq}
+\boldsymbol{\nu_{97}}\cdot\epsilon^{abcd}\eta^{eg}\eta^{fi}\eta^{hk}\eta^{pq}\eta^{jl}\\
\addlinespace
+\boldsymbol{\nu_{98}}\cdot\epsilon^{abcd}\eta^{eg}\eta^{fi}\eta^{hq}\eta^{pk}\eta^{jl}
+\boldsymbol{\nu_{99}}\cdot\epsilon^{abcd}\eta^{ep}\eta^{fi}\eta^{gj}\eta^{hk}\eta^{lq}
+\boldsymbol{\nu_{100}}\cdot\epsilon^{abcd}\eta^{ep}\eta^{fi}\eta^{gk}\eta^{hl}\eta^{jq}\\
\addlinespace
+\boldsymbol{\nu_{101}}\cdot\epsilon^{abcd}\eta^{ei}\eta^{fj}\eta^{gk}\eta^{hl}\eta^{pq}
+\boldsymbol{\nu_{102}}\cdot\epsilon^{abef}\eta^{cg}\eta^{dh}\eta^{pi}\eta^{jk}\eta^{lq}
+\boldsymbol{\nu_{103}}\cdot\epsilon^{abef}\eta^{cg}\eta^{dh}\eta^{pq}\eta^{ik}\eta^{jl}\\
\addlinespace
+\boldsymbol{\nu_{104}}\cdot\epsilon^{abef}\eta^{cg}\eta^{dp}\eta^{hi}\eta^{jk}\eta^{lq}
+\boldsymbol{\nu_{105}}\cdot\epsilon^{abef}\eta^{cg}\eta^{dp}\eta^{hq}\eta^{ik}\eta^{jl}
+\boldsymbol{\nu_{106}}\cdot\epsilon^{abef}\eta^{cg}\eta^{di}\eta^{hp}\eta^{jk}\eta^{lq}\\
\addlinespace
+\boldsymbol{\nu_{107}}\cdot\epsilon^{abef}\eta^{cg}\eta^{di}\eta^{hj}\eta^{pk}\eta^{lq}
+\boldsymbol{\nu_{108}}\cdot\epsilon^{abef}\eta^{cg}\eta^{di}\eta^{hk}\eta^{pj}\eta^{lq}
+\boldsymbol{\nu_{109}}\cdot\epsilon^{abef}\eta^{cg}\eta^{di}\eta^{hk}\eta^{pl}\eta^{jq}\\
\addlinespace
+\boldsymbol{\nu_{110}}\cdot\epsilon^{abef}\eta^{cg}\eta^{di}\eta^{hk}\eta^{pq}\eta^{jl}
+\boldsymbol{\nu_{111}}\cdot\epsilon^{abef}\eta^{cg}\eta^{di}\eta^{hq}\eta^{pk}\eta^{jl}
+\boldsymbol{\nu_{112}}\cdot\epsilon^{abef}\eta^{cg}\eta^{dq}\eta^{hp}\eta^{ik}\eta^{jl}\\
\addlinespace
+\boldsymbol{\nu_{113}}\cdot\epsilon^{abef}\eta^{cg}\eta^{dq}\eta^{hi}\eta^{pk}\eta^{jl}
+\boldsymbol{\nu_{114}}\cdot\epsilon^{abef}\eta^{cp}\eta^{di}\eta^{gj}\eta^{hk}\eta^{lq}
+\boldsymbol{\nu_{115}}\cdot\epsilon^{abef}\eta^{cp}\eta^{di}\eta^{gk}\eta^{hl}\eta^{jq}\\
\addlinespace
+\boldsymbol{\nu_{116}}\cdot\epsilon^{abef}\eta^{cp}\eta^{di}\eta^{gk}\eta^{hq}\eta^{jl}
+\boldsymbol{\nu_{117}}\cdot\epsilon^{abef}\eta^{ci}\eta^{dj}\eta^{gp}\eta^{hk}\eta^{lq}
+\boldsymbol{\nu_{118}}\cdot\epsilon^{abef}\eta^{ci}\eta^{dj}\eta^{gk}\eta^{hl}\eta^{pq}\\
\addlinespace
+\boldsymbol{\nu_{119}}\cdot\epsilon^{abef}\eta^{ci}\eta^{dk}\eta^{gp}\eta^{hj}\eta^{lq}
+\boldsymbol{\nu_{120}}\cdot\epsilon^{abep}\eta^{cf}\eta^{dg}\eta^{hi}\eta^{jk}\eta^{lq}
+\boldsymbol{\nu_{121}}\cdot\epsilon^{abep}\eta^{cf}\eta^{dg}\eta^{hq}\eta^{ik}\eta^{jl}\\
\addlinespace
+\boldsymbol{\nu_{122}}\cdot\epsilon^{abep}\eta^{cf}\eta^{di}\eta^{gj}\eta^{hk}\eta^{lq}
+\boldsymbol{\nu_{123}}\cdot\epsilon^{abep}\eta^{cf}\eta^{di}\eta^{gk}\eta^{hl}\eta^{jq}
+\boldsymbol{\nu_{124}}\cdot\epsilon^{abep}\eta^{cf}\eta^{di}\eta^{gk}\eta^{hq}\eta^{jl}\\
\addlinespace
+\boldsymbol{\nu_{125}}\cdot\epsilon^{abep}\eta^{cg}\eta^{dh}\eta^{fi}\eta^{jk}\eta^{lq}
+\boldsymbol{\nu_{126}}\cdot\epsilon^{abep}\eta^{cg}\eta^{dh}\eta^{fq}\eta^{ik}\eta^{jl}
+\boldsymbol{\nu_{127}}\cdot\epsilon^{abep}\eta^{cg}\eta^{di}\eta^{fj}\eta^{hk}\eta^{lq}\\
\addlinespace
+\boldsymbol{\nu_{128}}\cdot\epsilon^{abep}\eta^{cg}\eta^{di}\eta^{fk}\eta^{hq}\eta^{jl}
+\boldsymbol{\nu_{129}}\cdot\epsilon^{abei}\eta^{cf}\eta^{dg}\eta^{hp}\eta^{jk}\eta^{lq}
+\boldsymbol{\nu_{130}}\cdot\epsilon^{abei}\eta^{cf}\eta^{dg}\eta^{hj}\eta^{pk}\eta^{lq}\\
\addlinespace
+\boldsymbol{\nu_{131}}\cdot\epsilon^{abei}\eta^{cf}\eta^{dg}\eta^{hk}\eta^{pj}\eta^{lq}
+\boldsymbol{\nu_{132}}\cdot\epsilon^{abei}\eta^{cf}\eta^{dg}\eta^{hk}\eta^{pl}\eta^{jq}
+\boldsymbol{\nu_{133}}\cdot\epsilon^{abei}\eta^{cf}\eta^{dg}\eta^{hk}\eta^{pq}\eta^{jl}\\
\addlinespace
+\boldsymbol{\nu_{134}}\cdot\epsilon^{abei}\eta^{cf}\eta^{dg}\eta^{hq}\eta^{pk}\eta^{jl}
+\boldsymbol{\nu_{135}}\cdot\epsilon^{abei}\eta^{cf}\eta^{dp}\eta^{gj}\eta^{hk}\eta^{lq}
+\boldsymbol{\nu_{136}}\cdot\epsilon^{abei}\eta^{cf}\eta^{dp}\eta^{gk}\eta^{hl}\eta^{jq}\\
\addlinespace
+\boldsymbol{\nu_{137}}\cdot\epsilon^{abei}\eta^{cf}\eta^{dp}\eta^{gk}\eta^{hq}\eta^{jl}
+\boldsymbol{\nu_{138}}\cdot\epsilon^{abei}\eta^{cf}\eta^{dj}\eta^{gp}\eta^{hk}\eta^{lq}
+\boldsymbol{\nu_{139}}\cdot\epsilon^{abei}\eta^{cf}\eta^{dj}\eta^{gk}\eta^{hl}\eta^{pq}\\
\addlinespace
+\boldsymbol{\nu_{140}}\cdot\epsilon^{abei}\eta^{cf}\eta^{dk}\eta^{gp}\eta^{hj}\eta^{lq}
+\boldsymbol{\nu_{141}}\cdot\epsilon^{abei}\eta^{cf}\eta^{dk}\eta^{gp}\eta^{hl}\eta^{jq}
+\boldsymbol{\nu_{142}}\cdot\epsilon^{abei}\eta^{cf}\eta^{dk}\eta^{gj}\eta^{hq}\eta^{pl}\\
\addlinespace
+\boldsymbol{\nu_{143}}\cdot\epsilon^{abeq}\eta^{cf}\eta^{dg}\eta^{hp}\eta^{ik}\eta^{jl}
+\boldsymbol{\nu_{144}}\cdot\epsilon^{efgh}\eta^{ac}\eta^{bd}\eta^{pi}\eta^{jk}\eta^{lq}\\
\addlinespace
\bottomrule
\caption{Lorentz Invariant Ansatz: $a^{abcdefghpijklq}$.}\label{lorentzAreaExtra1}
\end{longtable}

\vspace{1cm}

\begin{longtable}{L}\toprule
\boldsymbol{a^{abcdefhijklpq}} \\
\addlinespace
\midrule
\addlinespace
\hphantom{+}\boldsymbol{\nu_{145}}\cdot\eta^{ac}\eta^{bd}\eta^{eg}\eta^{fh}\eta^{ik}\eta^{jl}\eta^{pq}
+\boldsymbol{\nu_{146}}\cdot\eta^{ac}\eta^{bd}\eta^{eg}\eta^{fh}\eta^{ik}\eta^{jp}\eta^{lq}
+\boldsymbol{\nu_{147}}\cdot\eta^{ac}\eta^{bd}\eta^{eg}\eta^{fi}\eta^{hk}\eta^{jl}\eta^{pq}\\
\addlinespace
+\boldsymbol{\nu_{148}}\cdot\eta^{ac}\eta^{bd}\eta^{eg}\eta^{fi}\eta^{hk}\eta^{jp}\eta^{lq}
+\boldsymbol{\nu_{149}}\cdot\eta^{ac}\eta^{bd}\eta^{eg}\eta^{fi}\eta^{hp}\eta^{jk}\eta^{lq}
+\boldsymbol{\nu_{150}}\cdot\eta^{ac}\eta^{bd}\eta^{eg}\eta^{fp}\eta^{hq}\eta^{ik}\eta^{jl}\\
\addlinespace
+\boldsymbol{\nu_{151}}\cdot\eta^{ac}\eta^{bd}\eta^{ei}\eta^{fj}\eta^{gk}\eta^{hl}\eta^{pq}
+\boldsymbol{\nu_{152}}\cdot\eta^{ac}\eta^{bd}\eta^{ei}\eta^{fj}\eta^{gk}\eta^{hp}\eta^{lq}
+\boldsymbol{\nu_{153}}\cdot\eta^{ac}\eta^{bd}\eta^{ei}\eta^{fk}\eta^{gj}\eta^{hl}\eta^{pq}\\
\addlinespace
+\boldsymbol{\nu_{154}}\cdot\eta^{ac}\eta^{be}\eta^{dg}\eta^{fh}\eta^{ik}\eta^{jl}\eta^{pq}
+\boldsymbol{\nu_{155}}\cdot\eta^{ac}\eta^{be}\eta^{dg}\eta^{fh}\eta^{ik}\eta^{jp}\eta^{lq}
+\boldsymbol{\nu_{156}}\cdot\eta^{ac}\eta^{be}\eta^{dg}\eta^{fi}\eta^{hk}\eta^{jl}\eta^{pq}\\
\addlinespace
+\boldsymbol{\nu_{157}}\cdot\eta^{ac}\eta^{be}\eta^{dg}\eta^{fi}\eta^{hk}\eta^{jp}\eta^{lq}
+\boldsymbol{\nu_{158}}\cdot\eta^{ac}\eta^{be}\eta^{dg}\eta^{fi}\eta^{hp}\eta^{jk}\eta^{lq}
+\boldsymbol{\nu_{159}}\cdot\eta^{ac}\eta^{be}\eta^{dg}\eta^{fp}\eta^{hq}\eta^{ik}\eta^{jl}\\
\addlinespace
+\boldsymbol{\nu_{160}}\cdot\eta^{ac}\eta^{be}\eta^{di}\eta^{fg}\eta^{hk}\eta^{jl}\eta^{pq}
+\boldsymbol{\nu_{161}}\cdot\eta^{ac}\eta^{be}\eta^{di}\eta^{fg}\eta^{hk}\eta^{jp}\eta^{lq}
+\boldsymbol{\nu_{162}}\cdot\eta^{ac}\eta^{be}\eta^{di}\eta^{fg}\eta^{hp}\eta^{jk}\eta^{lq}\\
\addlinespace
+\boldsymbol{\nu_{163}}\cdot\eta^{ac}\eta^{be}\eta^{di}\eta^{fj}\eta^{gk}\eta^{hl}\eta^{pq}
+\boldsymbol{\nu_{164}}\cdot\eta^{ac}\eta^{be}\eta^{di}\eta^{fj}\eta^{gk}\eta^{hp}\eta^{lq}
+\boldsymbol{\nu_{165}}\cdot\eta^{ac}\eta^{be}\eta^{di}\eta^{fk}\eta^{gj}\eta^{hp}\eta^{lq}\\
\addlinespace
+\boldsymbol{\nu_{166}}\cdot\eta^{ac}\eta^{be}\eta^{di}\eta^{fk}\eta^{gl}\eta^{hp}\eta^{jq}
+\boldsymbol{\nu_{167}}\cdot\eta^{ac}\eta^{be}\eta^{di}\eta^{fp}\eta^{gj}\eta^{hk}\eta^{lq}
+\boldsymbol{\nu_{168}}\cdot\eta^{ac}\eta^{be}\eta^{di}\eta^{fp}\eta^{gk}\eta^{hl}\eta^{jq}\\
\addlinespace
+\boldsymbol{\nu_{169}}\cdot\eta^{ac}\eta^{be}\eta^{di}\eta^{fp}\eta^{gk}\eta^{hq}\eta^{jl}
+\boldsymbol{\nu_{170}}\cdot\eta^{ac}\eta^{be}\eta^{dp}\eta^{fg}\eta^{hq}\eta^{ik}\eta^{jl}
+\boldsymbol{\nu_{171}}\cdot\eta^{ac}\eta^{be}\eta^{dp}\eta^{fi}\eta^{gj}\eta^{hk}\eta^{lq}\\
\addlinespace
+\boldsymbol{\nu_{172}}\cdot\eta^{ac}\eta^{bi}\eta^{dk}\eta^{eg}\eta^{fp}\eta^{hq}\eta^{jl}
+\boldsymbol{\nu_{173}}\cdot\eta^{ac}\eta^{bi}\eta^{dk}\eta^{ej}\eta^{fp}\eta^{gl}\eta^{hq}
+\boldsymbol{\nu_{174}}\cdot\eta^{ae}\eta^{bf}\eta^{cg}\eta^{dh}\eta^{ik}\eta^{jl}\eta^{pq}\\
\addlinespace
+\boldsymbol{\nu_{175}}\cdot\eta^{ae}\eta^{bf}\eta^{cg}\eta^{dh}\eta^{ik}\eta^{jp}\eta^{lq}
+\boldsymbol{\nu_{176}}\cdot\eta^{ae}\eta^{bf}\eta^{ci}\eta^{dj}\eta^{gk}\eta^{hl}\eta^{pq}
+\boldsymbol{\nu_{177}}\cdot\eta^{ae}\eta^{bf}\eta^{ci}\eta^{dj}\eta^{gk}\eta^{hp}\eta^{lq}\\
\addlinespace
+\boldsymbol{\nu_{178}}\cdot\eta^{ae}\eta^{bf}\eta^{ci}\eta^{dk}\eta^{gj}\eta^{hl}\eta^{pq}
+\boldsymbol{\nu_{179}}\cdot\eta^{ae}\eta^{bg}\eta^{cf}\eta^{dh}\eta^{ik}\eta^{jl}\eta^{pq}
+\boldsymbol{\nu_{180}}\cdot\eta^{ae}\eta^{bg}\eta^{cf}\eta^{dh}\eta^{ik}\eta^{jp}\eta^{lq}\\
\addlinespace
+\boldsymbol{\nu_{181}}\cdot\eta^{ae}\eta^{bg}\eta^{ci}\eta^{dj}\eta^{fk}\eta^{hl}\eta^{pq}
+\boldsymbol{\nu_{182}}\cdot\eta^{ae}\eta^{bg}\eta^{ci}\eta^{dj}\eta^{fk}\eta^{hp}\eta^{lq}
+\boldsymbol{\nu_{183}}\cdot\eta^{ae}\eta^{bi}\eta^{cg}\eta^{dk}\eta^{fp}\eta^{hq}\eta^{jl}\\
\addlinespace
+ \boldsymbol{\nu_{184}}\cdot\epsilon^{abcd}\eta^{eg}\eta^{fh}\eta^{ik}\eta^{jl}\eta^{pq}
+\boldsymbol{\nu_{185}}\cdot\epsilon^{abcd}\eta^{eg}\eta^{fh}\eta^{ik}\eta^{jp}\eta^{lq}
+\boldsymbol{\nu_{186}}\cdot\epsilon^{abcd}\eta^{eg}\eta^{fi}\eta^{hk}\eta^{jl}\eta^{pq}\\
\addlinespace
+\boldsymbol{\nu_{187}}\cdot\epsilon^{abcd}\eta^{eg}\eta^{fi}\eta^{hk}\eta^{jp}\eta^{lq}
+\boldsymbol{\nu_{188}}\cdot\epsilon^{abcd}\eta^{eg}\eta^{fi}\eta^{hp}\eta^{jk}\eta^{lq}
+\boldsymbol{\nu_{189}}\cdot\epsilon^{abcd}\eta^{eg}\eta^{fp}\eta^{hq}\eta^{ik}\eta^{jl}\\
\addlinespace
+\boldsymbol{\nu_{190}}\cdot\epsilon^{abcd}\eta^{ei}\eta^{fj}\eta^{gk}\eta^{hl}\eta^{pq}
+\boldsymbol{\nu_{191}}\cdot\epsilon^{abcd}\eta^{ei}\eta^{fj}\eta^{gk}\eta^{hp}\eta^{lq}
+\boldsymbol{\nu_{192}}\cdot\epsilon^{abcd}\eta^{ei}\eta^{fk}\eta^{gj}\eta^{hl}\eta^{pq}\\
\addlinespace
+\boldsymbol{\nu_{193}}\cdot\epsilon^{abef}\eta^{cg}\eta^{dh}\eta^{ik}\eta^{jl}\eta^{pq}
+\boldsymbol{\nu_{194}}\cdot\epsilon^{abef}\eta^{cg}\eta^{dh}\eta^{ik}\eta^{jp}\eta^{lq}
+\boldsymbol{\nu_{195}}\cdot\epsilon^{abef}\eta^{cg}\eta^{di}\eta^{hk}\eta^{jl}\eta^{pq}\\
\addlinespace
+\boldsymbol{\nu_{196}}\cdot\epsilon^{abef}\eta^{cg}\eta^{di}\eta^{hk}\eta^{jp}\eta^{lq}
+\boldsymbol{\nu_{197}}\cdot\epsilon^{abef}\eta^{cg}\eta^{di}\eta^{hp}\eta^{jk}\eta^{lq}
+\boldsymbol{\nu_{198}}\cdot\epsilon^{abef}\eta^{cg}\eta^{dp}\eta^{hq}\eta^{ik}\eta^{jl}\\
\addlinespace
+\boldsymbol{\nu_{199}}\cdot\epsilon^{abef}\eta^{ci}\eta^{dj}\eta^{gk}\eta^{hl}\eta^{pq}
+\boldsymbol{\nu_{200}}\cdot\epsilon^{abef}\eta^{ci}\eta^{dj}\eta^{gk}\eta^{hp}\eta^{lq}
+\boldsymbol{\nu_{201}}\cdot\epsilon^{abef}\eta^{ci}\eta^{dk}\eta^{gj}\eta^{hl}\eta^{pq}\\
\addlinespace
+\boldsymbol{\nu_{202}}\cdot\epsilon^{abei}\eta^{cf}\eta^{dj}\eta^{gk}\eta^{hl}\eta^{pq}
+\boldsymbol{\nu_{203}}\cdot\epsilon^{abei}\eta^{cf}\eta^{dj}\eta^{gk}\eta^{hp}\eta^{lq}
+\boldsymbol{\nu_{204}}\cdot\epsilon^{abei}\eta^{cf}\eta^{dk}\eta^{gj}\eta^{hl}\eta^{pq}\\
\addlinespace
+\boldsymbol{\nu_{205}}\cdot\epsilon^{abei}\eta^{cf}\eta^{dk}\eta^{gj}\eta^{hp}\eta^{lq}
+\boldsymbol{\nu_{206}}\cdot\epsilon^{abei}\eta^{cf}\eta^{dk}\eta^{gl}\eta^{hp}\eta^{jq}
+\boldsymbol{\nu_{207}}\cdot\epsilon^{abei}\eta^{cf}\eta^{dp}\eta^{gj}\eta^{hk}\eta^{lq}\\
\addlinespace
+\boldsymbol{\nu_{208}}\cdot\epsilon^{abei}\eta^{cf}\eta^{dp}\eta^{gk}\eta^{hl}\eta^{jq}
+\boldsymbol{\nu_{209}}\cdot\epsilon^{abei}\eta^{cf}\eta^{dp}\eta^{gk}\eta^{hq}\eta^{jl}
+\boldsymbol{\nu_{210}}\cdot\epsilon^{abep}\eta^{cf}\eta^{di}\eta^{gj}\eta^{hk}\eta^{lq}\\
\addlinespace
+\boldsymbol{\nu_{211}}\cdot\epsilon^{abij}\eta^{ce}\eta^{df}\eta^{gk}\eta^{hl}\eta^{pq}
+\boldsymbol{\nu_{212}}\cdot\epsilon^{abij}\eta^{ce}\eta^{df}\eta^{gk}\eta^{hp}\eta^{lq}
+\boldsymbol{\nu_{213}}\cdot\epsilon^{abij}\eta^{ce}\eta^{dk}\eta^{fp}\eta^{gl}\eta^{hq}\\
\addlinespace
+\boldsymbol{\nu_{214}}\cdot\epsilon^{abip}\eta^{ce}\eta^{df}\eta^{gj}\eta^{hk}\eta^{lq}
+\boldsymbol{\nu_{215}}\cdot\epsilon^{abip}\eta^{ce}\eta^{df}\eta^{gk}\eta^{hl}\eta^{jq}
+\boldsymbol{\nu_{216}}\cdot\epsilon^{ijkl}\eta^{ac}\eta^{bd}\eta^{eg}\eta^{fh}\eta^{pq}\\
\addlinespace
\bottomrule
\caption{Lorentz Invariant Ansatz: $a^{abcdefhijklpq}$.}\label{lorentzAreaExtra2}
\end{longtable}

\vspace{1cm}

Inserting these expressions in the perturbative equivariance equations (\ref{order1}), (\ref{order2}) and (\ref{order3}) ans solving these with respect to the $52$ parameters:\\

\begin{multline}\label{AreaParas}
\bigl\{ \ 
\boldsymbol{\mu_{3}},\boldsymbol{\mu_{5}},\boldsymbol{\mu_{6}},\boldsymbol{\mu_{7}},\boldsymbol{\mu_{11}},\boldsymbol{\mu_{13}},\boldsymbol{\mu_{14}},\boldsymbol{\mu_{17}},\boldsymbol{\mu_{19}},\boldsymbol{\mu_{23}},\boldsymbol{\nu_{7}},\boldsymbol{\nu_{9}},\boldsymbol{\nu_{11}},\boldsymbol{\nu_{20}},\boldsymbol{\nu_{41}},\boldsymbol{\nu_{42}},\boldsymbol{\nu_{48}},\boldsymbol{\nu_{51}},\boldsymbol{\nu_{55}},\boldsymbol{\nu_{58}},
\\
\boldsymbol{\nu_{59}},\boldsymbol{\nu_{69}},\boldsymbol{\nu_{77}},\boldsymbol{\nu_{79}},\boldsymbol{\nu_{80}},
\boldsymbol{\nu_{81}},\boldsymbol{\nu_{83}},\boldsymbol{\nu_{90}},\boldsymbol{\nu_{108}},\boldsymbol{\nu_{109}},\boldsymbol{\nu_{119}},\boldsymbol{\nu_{124}},\boldsymbol{\nu_{127}},\boldsymbol{\nu_{131}},\boldsymbol{\nu_{132}},\boldsymbol{\nu_{135}},\boldsymbol{\nu_{137}},
\\
\boldsymbol{\nu_{139}},\boldsymbol{\nu_{149}},\boldsymbol{\nu_{151}},\boldsymbol{\nu_{153}},\boldsymbol{\nu_{155}},\boldsymbol{\nu_{160}},\boldsymbol{\nu_{164}},\boldsymbol{\nu_{165}},\boldsymbol{\nu_{175}},\boldsymbol{\nu_{178}},\boldsymbol{\nu_{181}},\boldsymbol{\nu_{194}},\boldsymbol{\nu_{202}},\boldsymbol{\nu_{204}},\boldsymbol{\nu_{206}}
\ \bigr\}
\\
\end{multline}

We get the following expressions for the remaining $188$ parameters in terms of these:\\

\begingroup
\renewcommand{\arraystretch}{2.0}
\begin{longtable}{LL}\toprule
\addlinespace
\textbf{parameter} & \textbf{solution} \\
\addlinespace
\midrule
\addlinespace
\boldsymbol{\mu_{1}} & -1536\,\boldsymbol{\mu_{5}}-3072\,\boldsymbol{\mu_{6}}-1536\,\boldsymbol{\mu_{7}}-192\,\boldsymbol{\mu_{3}}
\\
\addlinespace
\midrule
\addlinespace
\boldsymbol{\mu_{2}} & -16\,\boldsymbol{\mu_{5}}-32\,\boldsymbol{\mu_{6}}-16\,\boldsymbol{\mu_{7}}
\\
\addlinespace
\midrule
\addlinespace
\boldsymbol{\mu_{4}} & -1/4\,\boldsymbol{\mu_{5}}-1/3\,\boldsymbol{\mu_{6}}+1/12\,\boldsymbol{\mu_{7}}
\\
\addlinespace
\midrule
\addlinespace
\boldsymbol{\mu_{8}} & 8\,\boldsymbol{\mu_{19}}+1/6\,\boldsymbol{\mu_{7}}-1/6\,\boldsymbol{\mu_{6}}+8\,\boldsymbol{\mu_{23}}-1/24\,\boldsymbol{\mu_{5}}-4\,\boldsymbol{\mu_{11}}-14/3\,\boldsymbol{\mu_{14}}
\\
\addlinespace
\midrule
\addlinespace
\boldsymbol{\mu_{9}} & -24\,\boldsymbol{\mu_{23}}+1/8\,\boldsymbol{\mu_{5}}+12\,\boldsymbol{\mu_{11}}+14\,\boldsymbol{\mu_{14}}-24\,\boldsymbol{\mu_{19}}
\\
\addlinespace
\midrule
\addlinespace
\boldsymbol{\mu_{10}} & -{\frac {5\,\boldsymbol{\mu_{11}}}{36}}-{\frac {\boldsymbol{\mu_{6}}}{864}}+{\frac {\boldsymbol{\mu_{7}}}{864}}+1/12\,\boldsymbol{\mu_{13}}-{\frac{\boldsymbol{\mu_{14}}}{216}}+1/36\,\boldsymbol{\mu_{17}}+{\frac {\boldsymbol{\mu_{5}}}{1728}}
\\
\addlinespace
\midrule
\addlinespace
\boldsymbol{\mu_{12}} & 1/48\,\boldsymbol{\mu_{6}}-{\frac {\boldsymbol{\mu_{7}}}{192}}-1/2\,\boldsymbol{\mu_{11}}-1/2\,\boldsymbol{\mu_{13}}-1/2\,\boldsymbol{\mu_{14}}
\\
\addlinespace
\midrule
\addlinespace
\boldsymbol{\mu_{15}} & -\boldsymbol{\mu_{14}}
\\
\addlinespace
\midrule
\addlinespace
\boldsymbol{\mu_{16}} & -{\frac {\boldsymbol{\mu_{5}}}{96}}-1/6\,\boldsymbol{\mu_{14}}-1/24\,\boldsymbol{\mu_{6}}-1/48\,\boldsymbol{\mu_{7}}-1/2\,\boldsymbol{\mu_{17}}
\\
\addlinespace
\midrule
\addlinespace
\boldsymbol{\mu_{18}} & -{\frac {\boldsymbol{\mu_{6}}}{288}}-{\frac {\boldsymbol{\mu_{7}}}{2304}}+1/8\,\boldsymbol{\mu_{13}}-{\frac{\boldsymbol{\mu_{14}}}{72}}-1/12\,\boldsymbol{\mu_{19}}-{\frac {\boldsymbol{\mu_{5}}}{1152}}
\\
\addlinespace
\midrule
\addlinespace
\boldsymbol{\mu_{20}} & 
\begin{aligned}
&-{\frac {17\,\boldsymbol{\mu_{7}}}{3456}}-1/12\,\boldsymbol{\mu_{13}}-1/36\,\boldsymbol{\mu_{17}}+{\frac {7\,\boldsymbol{\mu_{6}}}{1728}}-{\frac {7\,\boldsymbol{\mu_{19}}}{12}}-{\frac{7\,\boldsymbol{\mu_{23}}}{36}}\\
&+{\frac{7\,\boldsymbol{\mu_{5}}}{6912}}+{\frac {7\,\boldsymbol{\mu_{11}}}{72}}+{\frac {49\,\boldsymbol{\mu_{14}}}{432}}
\end{aligned}
\\
\addlinespace
\midrule
\addlinespace
\boldsymbol{\mu_{21}} & 
\begin{aligned}
&\hphantom{+}{\frac {\boldsymbol{\mu_{6}}}{432}}-{\frac{\boldsymbol{\mu_{7}}}{1728}}-1/3\,\boldsymbol{\mu_{19}}+1/6\,\boldsymbol{\mu_{13}}+1/18\,\boldsymbol{\mu_{17}}-1/9\,\boldsymbol{\mu_{23}}\\
&+{\frac {\boldsymbol{\mu_{5}}}{1728}}+1/18\,\boldsymbol{\mu_{11}}+{\frac {7\,\boldsymbol{\mu_{14}}}{108}}
\end{aligned}
\\
\addlinespace
\midrule
\addlinespace
\boldsymbol{\mu_{22}} & {\frac {\boldsymbol{\mu_{5}}}{192}}+1/4\,\boldsymbol{\mu_{11}}+1/3\,\boldsymbol{\mu_{14}}+{\frac {\boldsymbol{\mu_{6}}}{96}}+{\frac{\boldsymbol{\mu_{7}}}{192}}-1/4\,\boldsymbol{\mu_{23}}-1/2\,\boldsymbol{\mu_{19}}+1/8\,\boldsymbol{\mu_{17}}
\\
\addlinespace
\midrule
\addlinespace
\boldsymbol{\mu_{24}} & -{\frac {\boldsymbol{\mu_{5}}}{128}}-3/4\,\boldsymbol{\mu_{11}}-{\frac {7\,\boldsymbol{\mu_{14}}}{8}}+3/2\,\boldsymbol{\mu_{19}}+1/2\,\boldsymbol{\mu_{23}}
\\
\addlinespace
\midrule
\addlinespace
\boldsymbol{\nu_{1}} & -64\,\boldsymbol{\nu_{83}}-32\,\boldsymbol{\nu_{90}}-48\,\boldsymbol{\nu_{59}}+2\,\boldsymbol{\nu_{9}}-48\,\boldsymbol{\nu_{48}}-16\,\boldsymbol{\nu_{55}}-48\,\boldsymbol{\nu_{58}}-1/2\,\boldsymbol{\nu_{7}}
\\
\addlinespace
\midrule
\addlinespace
\boldsymbol{\nu_{2}} & 4\,\boldsymbol{\nu_{7}}-16\,\boldsymbol{\nu_{9}}
\\
\addlinespace
\midrule
\addlinespace
\boldsymbol{\nu_{3}} & 3/4\,\boldsymbol{\nu_{7}}-3\,\boldsymbol{\nu_{9}}-32\,\boldsymbol{\nu_{83}}-16\,\boldsymbol{\nu_{90}}-24\,\boldsymbol{\nu_{59}}-24\,\boldsymbol{\nu_{48}}-8\,\boldsymbol{\nu_{55}}-24\,\boldsymbol{\nu_{58}}
\\
\addlinespace
\midrule
\addlinespace
\boldsymbol{\nu_{4}} & -8\,\boldsymbol{\nu_{83}}-4\,\boldsymbol{\nu_{90}}-6\,\boldsymbol{\nu_{59}}+1/4\,\boldsymbol{\nu_{9}}-6\,\boldsymbol{\nu_{48}}-2\,\boldsymbol{\nu_{55}}-6\,\boldsymbol{\nu_{58}}+{\frac {7\,\boldsymbol{\nu_{7}}}{16}}
\\
\addlinespace
\midrule
\addlinespace
\boldsymbol{\nu_{5}} & 
\begin{aligned}
&\hphantom{-}{\frac {\boldsymbol{\nu_{7}}}{96}}-8/3\,\boldsymbol{\nu_{48}}-4/3\,\boldsymbol{\nu_{55}}-8/3\,\boldsymbol{\nu_{59}}+1/8\,\boldsymbol{\nu_{11}}-4\,\boldsymbol{\nu_{83}}-2\,\boldsymbol{\nu_{90}}\\
&-8/
3\,\boldsymbol{\nu_{58}}+{\frac {5\,\boldsymbol{\nu_{9}}}{24}}
\end{aligned}
\\
\addlinespace
\midrule
\addlinespace
\boldsymbol{\nu_{6}} & -3/4\,\boldsymbol{\nu_{7}}
\\
\addlinespace
\midrule
\addlinespace
\boldsymbol{\nu_{8}} & 16\,\boldsymbol{\nu_{83}}+8\,\boldsymbol{\nu_{90}}+8\,\boldsymbol{\nu_{59}}+8\,\boldsymbol{\nu_{48}}+8\,\boldsymbol{\nu_{55}}+8\,\boldsymbol{\nu_{58}}-1/2\,\boldsymbol{\nu_{7}}
\\
\addlinespace
\midrule
\addlinespace
\boldsymbol{\nu_{10}} & -2\,\boldsymbol{\nu_{7}}+4\,\boldsymbol{\nu_{9}}+2\,\boldsymbol{\nu_{11}}
\\
\addlinespace
\midrule
\addlinespace
\boldsymbol{\nu_{12}} & 1/8\,\boldsymbol{\nu_{7}}+2\,\boldsymbol{\nu_{59}}+2\,\boldsymbol{\nu_{48}}-2\,\boldsymbol{\nu_{55}}+2\,\boldsymbol{\nu_{58}}
\\
\addlinespace
\midrule
\addlinespace
\boldsymbol{\nu_{13}} & 
\begin{aligned}
&\hphantom{-}
{\frac {224\,\boldsymbol{\nu_{108}}}{61}}+{\frac {3\,\boldsymbol{\nu_{7}}}{32}}-{\frac {16\,\boldsymbol{\nu_{124}}}{61}}-{\frac {48\,\boldsymbol{\nu_{131}}}{61}}-{\frac{48\,\boldsymbol{\nu_{132}}}{61}}+{\frac {80\,\boldsymbol{\nu_{127}}}{61}}-3\,\boldsymbol{\nu_{48}}\\
&-\boldsymbol{\nu_{55}}-3\,\boldsymbol{\nu_{59}}-{\frac {436\,\boldsymbol{\nu_{83}}}{61}}+{\frac {70\,\boldsymbol{\nu_{90}}}{61}}+{\frac {224\,\boldsymbol{\nu_{109}}}{61}}-{\frac {1264\,\boldsymbol{\nu_{137}}}{61}}-3\,\boldsymbol{\nu_{58}}\\
&-3/8\,\boldsymbol{\nu_{9}}
\end{aligned}
\\
\addlinespace
\midrule
\addlinespace
\boldsymbol{\nu_{14}} & 
\begin{aligned} 
&\hphantom{-}{\frac {152\,\boldsymbol{\nu_{108}}}{61}}-{\frac {5\,\boldsymbol{\nu_{7}}}{48}}+{\frac {24\,\boldsymbol{\nu_{124}}}{61}}-{\frac {28\,\boldsymbol{\nu_{131}}}{183}}-{\frac{28\,\boldsymbol{\nu_{132}}}{183}}+{\frac {128\,\boldsymbol{\nu_{127}}}{183}}\\
&-1/3\,\boldsymbol{\nu_{48}}+1/3\,\boldsymbol{\nu_{55}}-1/3\,\boldsymbol{\nu_{59}}+1/8\,\boldsymbol{\nu_{11}}-{\frac {356\,\boldsymbol{\nu_{83}}}{183}}+{\frac {356\,\boldsymbol{\nu_{90}}}{183}}\\
&+{\frac {152\,\boldsymbol{\nu_{109}}}{61}}-{\frac{2120\,\boldsymbol{\nu_{137}}}{183}}-1/3\,\boldsymbol{\nu_{58}}+1/6\,\boldsymbol{\nu_{9}}
\end{aligned}
\\
\addlinespace
\midrule
\addlinespace
\boldsymbol{\nu_{15}} & 
\begin{aligned} 
&-{\frac {48\,\boldsymbol{\nu_{124}}}{61}}+{\frac {672\,\boldsymbol{\nu_{109}}}{61}}-{\frac{3792\,\boldsymbol{\nu_{137}}}{61}}+{\frac {240\,\boldsymbol{\nu_{127}}}{61}}+{\frac {672\,
\boldsymbol{\nu_{108}}}{61}}-{\frac {144\,\boldsymbol{\nu_{132}}}{61}}\\
&-{\frac {144\,\boldsymbol{\nu_{131}}}{61}}-{\frac {576\,\boldsymbol{\nu_{83}}}{61}}+{\frac {576\,\boldsymbol{\nu_{90}}}{61}}
\end{aligned}
\\
\addlinespace
\midrule
\addlinespace
\boldsymbol{\nu_{16}} & 
\begin{aligned} 
-&{\frac {16\,\boldsymbol{\nu_{124}}}{61}}+{\frac {224\,\boldsymbol{\nu_{109}}}{61}}-{\frac {1264\,\boldsymbol{\nu_{137}}}{61}}+{\frac {80\,\boldsymbol{\nu_{127}}}{61}}+{
\frac {224\,\boldsymbol{\nu_{108}}}{61}}-{\frac {48\,\boldsymbol{\nu_{132}}}{61}}\\
&-{\frac {48\,\boldsymbol{\nu_{131}}}{61}}-{\frac {192\,\boldsymbol{\nu_{83}}}{61}}+{\frac {192\,\boldsymbol{\nu_{90}}}{61}}
\end{aligned}
\\
\addlinespace
\midrule
\addlinespace
\boldsymbol{\nu_{17}} & 
\begin{aligned}
&\hphantom{+}{\frac {8\,\boldsymbol{\nu_{124}}}{61}}-{\frac {112\,\boldsymbol{\nu_{109}}}{61}}+{\frac {632\,\boldsymbol{\nu_{137}}}{61}}-{\frac {40\,\boldsymbol{\nu_{127}}}{61}}-{\frac {112\,\boldsymbol{\nu_{108}}}{61}}+{\frac {24\,\boldsymbol{\nu_{132}}}{61}}\\
&+{\frac {24\,\boldsymbol{\nu_{131}}}{61}} +{\frac {96\,\boldsymbol{\nu_{83}}}{61}}-{\frac {96\,
\boldsymbol{\nu_{90}}}{61}}
\end{aligned}
\\
\addlinespace
\midrule
\addlinespace
\boldsymbol{\nu_{18}} & 
\begin{aligned}
&-{\frac {232\,\boldsymbol{\nu_{108}}}{61}}-{\frac {88\,\boldsymbol{\nu_{124}}}{61}}-{\frac {20\,\boldsymbol{\nu_{131}}}{61}}-{\frac {20\,\boldsymbol{\nu_{132}}}{61}}-{\frac {48\,\boldsymbol{\nu_{127}}}{61}}+{\frac {164\,\boldsymbol{\nu_{83}}}{61}}\\
&-{\frac {164\,\boldsymbol{\nu_{90}}}{61}}-{\frac {232\,\boldsymbol{\nu_{109}}}{61}}+{\frac{856\,\boldsymbol{\nu_{137}}}{61}}
\end{aligned}
\\
\addlinespace
\midrule
\addlinespace
\boldsymbol{\nu_{19}} & 
\begin{aligned} 
&-1/2\,\boldsymbol{\nu_{20}}-4\,\boldsymbol{\nu_{69}}-4/3\,\boldsymbol{\nu_{80}}-4\,\boldsymbol{\nu_{83}}+4/3\,\boldsymbol{\nu_{51}}-4\,\boldsymbol{\nu_{90}}+5/6\,\boldsymbol{\nu_{59}}\\
&-{\frac{5\,\boldsymbol{\nu_{9}}}{16}}
-5/6\,\boldsymbol{\nu_{48}}+1/6\,\boldsymbol{\nu_{55}}+5/6\,\boldsymbol{\nu_{58}}-{\frac {17\,\boldsymbol{\nu_{7}}}{192}}
\end{aligned}
\\
\addlinespace
\midrule
\addlinespace
\boldsymbol{\nu_{21}} &
\begin{aligned}
&\hphantom{+}4\,\boldsymbol{\nu_{80}}+32\,\boldsymbol{\nu_{83}}-4\,\boldsymbol{\nu_{51}}+16\,\boldsymbol{\nu_{90}}+8\,\boldsymbol{\nu_{59}}+5/4\,\boldsymbol{\nu_{9}}+4\,\boldsymbol{\nu_{48}}+8\,\boldsymbol{\nu_{58}}\\
&+3/8\, \boldsymbol{\nu{7}}
\end{aligned}
\\
\addlinespace
\midrule
\addlinespace
\boldsymbol{\nu_{22}} & -32\,\boldsymbol{\nu_{83}}-16\,\boldsymbol{\nu_{90}}-8\,\boldsymbol{\nu_{59}}-\boldsymbol{\nu_{9}}-8\,\boldsymbol{\nu_{58}}-1/2\,\boldsymbol{\nu_{7}}
\\
\addlinespace
\midrule
\addlinespace
\boldsymbol{\nu_{23}} & 32\,\boldsymbol{\nu_{83}}+16\,\boldsymbol{\nu_{90}}+8\,\boldsymbol{\nu_{59}}+8\,\boldsymbol{\nu_{58}}+3/4\,\boldsymbol{\nu_{7}}
\\
\addlinespace
\midrule
\addlinespace
\boldsymbol{\nu_{24}} & 4\,\boldsymbol{\nu_{83}}+2\,\boldsymbol{\nu_{90}}+3\,\boldsymbol{\nu_{59}}-1/8\,\boldsymbol{\nu_{9}}+3\,\boldsymbol{\nu_{48}}+\boldsymbol{\nu_{55}}+3\,\boldsymbol{\nu_{58}}+1/32\,\boldsymbol{\nu_{7}}
\\
\addlinespace
\midrule
\addlinespace
\boldsymbol{\nu_{25}} & -4\,\boldsymbol{\nu_{83}}+2\,\boldsymbol{\nu_{90}}-4\,\boldsymbol{\nu_{59}}-1/4\,\boldsymbol{\nu_{9}}-4\,\boldsymbol{\nu_{58}}+1/8\,\boldsymbol{\nu_{7}}+8\,\boldsymbol{\nu_{69}}
\\
\addlinespace
\midrule
\addlinespace
\boldsymbol{\nu_{26}} & 32\,\boldsymbol{\nu_{83}}+16\,\boldsymbol{\nu_{90}}+8\,\boldsymbol{\nu_{59}}+2\,\boldsymbol{\nu_{9}}+8\,\boldsymbol{\nu_{58}}+1/4\,\boldsymbol{\nu_{7}}
\\
\addlinespace
\midrule
\addlinespace
\boldsymbol{\nu_{27}} & 
\begin{aligned} 
&-16\,\boldsymbol{\nu_{69}}-16\,\boldsymbol{\nu_{83}}-16\,\boldsymbol{\nu_{90}}+6\,\boldsymbol{\nu_{59}}-\boldsymbol{\nu_{9}}+6\,\boldsymbol{\nu_{58}}-{\frac{9\,\boldsymbol{\nu_{7}}}{16}}-4\,\boldsymbol{\nu_{80}}+4\,\boldsymbol{\nu_{51}}\\
&+2\,\boldsymbol{\nu_{48}}+2\,\boldsymbol{\nu_{55}}
\end{aligned}
\\
\addlinespace
\midrule
\addlinespace
\boldsymbol{\nu_{28}} & 
\begin{aligned} 
&-{\frac {29\,\boldsymbol{\nu_{108}}}{61}}-2\,\boldsymbol{\nu_{69}}-{\frac {11\,\boldsymbol{\nu_{7}}}{768}}-{\frac {127\,\boldsymbol{\nu_{124}}}{366}}+{\frac{28\,\boldsymbol{\nu_{131}}}{61}}+{\frac {51\,\boldsymbol{\nu_{132}}}{244}}-{\frac {11\,\boldsymbol{\nu_{127}}}{732}}\\
&+\boldsymbol{\nu_{79}}+1/2\,\boldsymbol{\nu_{51}}+{\frac {19\,\boldsymbol{\nu_{48}}}{24}}
+{\frac {13\,\boldsymbol{\nu_{55}}}{24}}+{\frac {43\,\boldsymbol{\nu_{59}}}{24}}-1/6\,\boldsymbol{\nu_{80}}+{\frac {917\,\boldsymbol{\nu_{83}}}{732}}\\
&-{\frac {92\,\boldsymbol{\nu_{90}}}{183}}+{\frac {3\,\boldsymbol{\nu_{109}}}{122}}+{\frac {153\,\boldsymbol{\nu_{137}}}{122}}-1/4\,\boldsymbol{\nu_{20}}+{\frac {43\,\boldsymbol{\nu_{58}}}{24}}-{\frac {7\,\boldsymbol{\nu_{9}}}{64}},
\end{aligned}
\\
\addlinespace
\midrule
\addlinespace
\boldsymbol{\nu_{29}} & 
\begin{aligned} 
&\hphantom{-}1/2\,\boldsymbol{\nu_{20}}+{\frac {16\,\boldsymbol{\nu_{124}}}{61}}+{\frac {304\,\boldsymbol{\nu_{109}}}{183}}-{\frac {444\,\boldsymbol{\nu_{137}}}{61}}+{\frac {4\,\boldsymbol{\nu_{127}}}{183}}+{\frac {304\,\boldsymbol{\nu_{108}}}{183}}\\
&-{\frac {100\,\boldsymbol{\nu_{132}}}{183}}-{\frac {100\,\boldsymbol{\nu_{131}}}{183}}-{\frac {1132\,\boldsymbol{\nu_{83}}}{183}}-{\frac {332\,\boldsymbol{\nu_{90}}}{183}}-4/3\,\boldsymbol{\nu_{59}}-4/3\,\boldsymbol{\nu_{58}}\\
&-1/8\,\boldsymbol{\nu_{7}}
\end{aligned}
\\
\addlinespace
\midrule
\addlinespace
\boldsymbol{\nu_{30}} & 
\begin{aligned} 
&\hphantom{-}{\frac {174\,\boldsymbol{\nu_{108}}}{61}}+{\frac {5\,\boldsymbol{\nu_{7}}}{128}}+{\frac {127\,\boldsymbol{\nu_{124}}}{61}}-{\frac {168\,\boldsymbol{\nu_{131}}}{61}}-{\frac{153\,\boldsymbol{\nu_{132}}}{122}}+{\frac {11\,\boldsymbol{\nu_{127}}}{122}}-6\,\boldsymbol{\nu_{79}}\\
&-\boldsymbol{\nu_{51}}-9/4\,\boldsymbol{\nu_{48}}-7/4\,\boldsymbol{\nu_{55}}-5/4\,\boldsymbol{\nu_{59}}-\boldsymbol{\nu_{80}}+{\frac {59\,\boldsymbol{\nu_{83}}}{122}}+{\frac {62\,\boldsymbol{\nu_{90}}}{61}}-{\frac{9\,\boldsymbol{\nu_{109}}}{61}}\\
&-{\frac {459\,\boldsymbol{\nu_{137}}}{61}}-5/4\,\boldsymbol{\nu_{58}}+{\frac {7\,\boldsymbol{\nu_{9}}}{32}}
\end{aligned}
\\
\addlinespace
\midrule
\addlinespace
\boldsymbol{\nu_{31}} & 
\begin{aligned}
&-{\frac {48\,\boldsymbol{\nu_{124}}}{61}}-{\frac {304\,\boldsymbol{\nu_{109}}}{61}}+{\frac {1332\,\boldsymbol{\nu_{137}}}{61}}-{\frac {4\,\boldsymbol{\nu_{127}}}{61}}-{\frac{304\,\boldsymbol{\nu_{108}}}{61}}+{\frac {100\,\boldsymbol{\nu_{132}}}{61}}\\
&+{\frac {100\,\boldsymbol{\nu_{131}}}{61}}+{\frac {156\,\boldsymbol{\nu_{83}}}{61}}-{\frac {156\,\boldsymbol{\nu_{90}}}{61}}
\end{aligned}
\\
\addlinespace
\midrule
\addlinespace
\boldsymbol{\nu_{32}} & 3/16\,\boldsymbol{\nu_{7}}-3/4\,\boldsymbol{\nu_{9}}-8\,\boldsymbol{\nu_{83}}-4\,\boldsymbol{\nu_{90}}-6\,\boldsymbol{\nu_{59}}-6\,\boldsymbol{\nu_{48}}-2\,\boldsymbol{\nu_{55}}-6\,\boldsymbol{\nu_{58}}
\\
\addlinespace
\midrule
\addlinespace
\boldsymbol{\nu_{33}} & 
\begin{aligned}
&-{\frac {48\,
\boldsymbol{\nu_{124}}}{61}}-{\frac {304\,\boldsymbol{\nu_{109}}}{61}}+{\frac {1332\,\boldsymbol{\nu_{137}}}{61}}-{\frac {4\,\boldsymbol{\nu_{127}}}{61}}-{\frac {304\,\boldsymbol{\nu_{108}}}{61}}+{\frac {100\,\boldsymbol{\nu_{132}}}{61}}\\
&+{\frac {100\,\boldsymbol{\nu_{131}}}{61}}-{\frac {88\,\boldsymbol{\nu_{83}}}{61}}-{\frac {278\,\boldsymbol{\nu_{90}}}{61}}+{\frac {3\,\boldsymbol{\nu_{7}}}{32}}-3/8\,\boldsymbol{\nu_{9}}-3\,\boldsymbol{\nu_{59}}-3\,\boldsymbol{\nu_{48}}\\
&-\boldsymbol{\nu_{55}}-3\,\boldsymbol{\nu_{58}}
\end{aligned}
\\
\addlinespace
\midrule
\addlinespace
\boldsymbol{\nu_{34}} & 
\begin{aligned}
&-{\frac {11\,\boldsymbol{\nu_{108}}}{183}}-2\,\boldsymbol{\nu_{69}}-{\frac {65\,\boldsymbol{\nu_{7}}}{768}}-{\frac {103\,\boldsymbol{\nu_{124}}}{366}}+{\frac {59\,\boldsymbol{\nu_{131}}}{183}}+{\frac {53\,\boldsymbol{\nu_{132}}}{732}}-{\frac {7\,\boldsymbol{\nu_{127}}}{732}}\\
&+\boldsymbol{\nu_{79}}+1/2\,\boldsymbol{\nu_{51}}+{\frac {25\,\boldsymbol{\nu_{48}}}{24}}+5/8\,\boldsymbol{\nu_{55}}+{\frac {11\,\boldsymbol{\nu_{59}}}{8}}-1/6\,\boldsymbol{\nu_{80}}-{\frac {947\,\boldsymbol{\nu_{83}}}{732}}\\
&-{\frac {533\,\boldsymbol{\nu_{90}}}{366}}+{\frac {161\,\boldsymbol{\nu_{109}}}{366}}
-{\frac {69\,\boldsymbol{\nu_{137}}}{122}}\hphantom{=}+{\frac {11\,\boldsymbol{\nu_{58}}}{8}}-{\frac {5\,\boldsymbol{\nu_{9}}}{64}}
\end{aligned}
\\
\addlinespace
\midrule
\addlinespace
\boldsymbol{\nu_{35}} & 
\begin{aligned} 
&-1/2\,\boldsymbol{\nu_{69}}+{\frac {31\,\boldsymbol{\nu_{7}}}{1536}}+1/6\,\boldsymbol{\nu_{51}}+{\frac {67\,\boldsymbol{\nu_{48}}}{48}}+1/48\,\boldsymbol{\nu_{55}}+{\frac{65\,\boldsymbol{\nu_{59}}}{48}}-1/6\,\boldsymbol{\nu_{80}}\\
&-1/4\,\boldsymbol{\nu_{90}}-3/16\,\boldsymbol{\nu_{20}}-1/2\,\boldsymbol{\nu_{41}}+\boldsymbol{\nu_{42}}+{\frac {65\,\boldsymbol{\nu_{58}}}{48}}-{\frac {21\,\boldsymbol{\nu_{9}}}{128}}
\end{aligned}
\\
\addlinespace
\midrule
\addlinespace
\boldsymbol{\nu_{36}} & 
\begin{aligned}
&-{\frac {3\,\boldsymbol{\nu_{20}}}{64}}-{\frac {305\,\boldsymbol{\nu_{7}}}{55296}}-{\frac {2011\,\boldsymbol{\nu_{9}}}{13824}}+{\frac{7\,\boldsymbol{\nu_{11}}}{2304}}+1/4\,\boldsymbol{\nu_{42}}+{\frac {1231\,\boldsymbol{\nu_{48}}}{1728}}+{\frac {31\,\boldsymbol{\nu_{51}}}{144}}+\\
&{\frac {593\,\boldsymbol{\nu_{55}}}{1728}}+{\frac {841\,\boldsymbol{\nu_{58}}}{1728}}+{\frac {841\,\boldsymbol{\nu_{59}}}{1728}}
-1/4\,\boldsymbol{\nu_{69}}+{\frac {5\,\boldsymbol{\nu_{77}}}{24}}+1/6\,\boldsymbol{\nu_{79}}-{\frac {
23\,\boldsymbol{\nu_{80}}}{144}}\\
&+1/32\,\boldsymbol{\nu_{81}}-{\frac {50597\,\boldsymbol{\nu_{83}}}{52704}}-{\frac {61859\,\boldsymbol{\nu_{90}}}{105408}}-{\frac{143\,\boldsymbol{\nu_{108}}}{4392}}-{\frac {103\,\boldsymbol{\nu_{109}}}{8784}}\\
&+{\frac {\boldsymbol{\nu_{119}}}{72}}-{\frac {53\,\boldsymbol{\nu_{124}}}{1098}}-{\frac {1249\,\boldsymbol{\nu_{127}}}{52704}}+{\frac {2101\,\boldsymbol{\nu_{131}}}{26352}}+{\frac{359\,\boldsymbol{\nu_{132}}}{52704}}\\
&+1/8\,\boldsymbol{\nu_{135}}-{\frac {295\,\boldsymbol{\nu_{137}}}{6588}}+1/12\,\boldsymbol{\nu_{139}}
\end{aligned}
\\
\addlinespace
\midrule
\addlinespace
\boldsymbol{\nu_{37}} & 
\begin{aligned} 
&-3/4\,\boldsymbol{\nu_{48}}-1/2\,\boldsymbol{\nu_{59}}+1/4\,\boldsymbol{\nu_{83}}+1/8\,\boldsymbol{\nu_{90}}+3/16\,\boldsymbol{\nu_{20}}+3/4\,\boldsymbol{\nu_{41}}\\
&-3/2\,\boldsymbol{\nu_{42}}-1/2\,\boldsymbol{\nu_{58}}+{
\frac {3\,\boldsymbol{\nu_{9}}}{32}}
\end{aligned}
\\
\addlinespace
\midrule
\addlinespace
\boldsymbol{\nu_{38}} &
\begin{aligned}
&\hphantom{+}1/16\,\boldsymbol{\nu_{7}}+\boldsymbol{\nu_{48}}+\boldsymbol{\nu_{59}}+\boldsymbol{\nu_{83}}+1/2\,\boldsymbol{\nu_{90}}-1/4\,\boldsymbol{\nu_{20}}-\boldsymbol{\nu_{41}}+2\,
\boldsymbol{\nu_{42}}\\
&+\boldsymbol{\nu_{58}}-1/8\,\boldsymbol{\nu_{9}}
\end{aligned}
\\
\addlinespace
\midrule
\addlinespace
\boldsymbol{\nu_{39}} &
\begin{aligned} 
&\hphantom{-}{\frac {5\,\boldsymbol{\nu_{20}}}{32}}-
{\frac {193\,\boldsymbol{\nu_{7}}}{1536}}+{\frac {293\,\boldsymbol{\nu_{9}}}{384}}+{\frac {7
\,\boldsymbol{\nu_{11}}}{64}}-1/4\,\boldsymbol{\nu_{41}}-\boldsymbol{\nu_{42}}-{\frac {43\,\boldsymbol{\nu_{48}}}{48}}-{\frac {7\,\boldsymbol{\nu_{51}}}{12}}\\
&-{\frac {7\,\boldsymbol{\nu_{55}}}{16}}-3/16\,\boldsymbol{\nu_{58}}-3/16\,\boldsymbol{\nu_{59}}+\boldsymbol{\nu_{69}}-1/2\,\boldsymbol{\nu_{77}}+{\frac {7\,\boldsymbol{\nu_{80}}}{12}}-1/8\,\boldsymbol{\nu_{81}}\\
&+{\frac {6679\,\boldsymbol{\nu_{83}}}{1464}}+{\frac {
8419\,\boldsymbol{\nu_{90}}}{2928}}+{\frac {53\,\boldsymbol{\nu_{108}}}{122}}+{\frac {45\,
\boldsymbol{\nu_{109}}}{244}}
-1/2\,\boldsymbol{\nu_{119}}+{\frac {79\,\boldsymbol{\nu_{124}}}{122}}\\
&+{
\frac {323\,\boldsymbol{\nu_{127}}}{1464}}-{\frac {347\,\boldsymbol{\nu_{131}}}{732}}-{
\frac {877\,\boldsymbol{\nu_{132}}}{1464}}-1/2\,\boldsymbol{\nu_{135}}-{\frac {65\,\boldsymbol{\nu_{137}}}{366}}-\boldsymbol{\nu_{139}}
\end{aligned}
\\
\addlinespace
\midrule
\addlinespace
\boldsymbol{\nu_{40}} & 
\begin{aligned}
&\hphantom{+}1/32\,\boldsymbol{\nu_{7}}+5/4\,\boldsymbol{\nu_{48}}+\boldsymbol{\nu_{59}}+1/4\,\boldsymbol{\nu_{83}}+1/8\,\boldsymbol{\nu_{90}}-1/8\,\boldsymbol{\nu_{20}}-1/4\,\boldsymbol{\nu_{41}}\\
&+1/2\,\boldsymbol{\nu_{42}}+\boldsymbol{\nu_{58}}-{\frac {3\,\boldsymbol{\nu_{9}}}{32}}
\end{aligned}
\\
\addlinespace
\midrule
\addlinespace
\boldsymbol{\nu_{43}} & 
\begin{aligned}
&-{\frac {53\,\boldsymbol{\nu_{108}}}{244}}+{\frac {227\,\boldsymbol{\nu_{7}}}{3072}}-{\frac {79\,\boldsymbol{\nu_{124}}}{244}}+1/4\,\boldsymbol{\nu_{135}}+{\frac {347\,\boldsymbol{\nu_{131}}}{1464}}+{\frac {877\,\boldsymbol{\nu_{132}}}{2928}}\\
&-{\frac {323\,\boldsymbol{\nu_{127}}}{2928}}-{\frac {19\,\boldsymbol{\nu_{48}}}{96}}-{\frac {5\,\boldsymbol{\nu_{55}}}{96}}-{\frac {43\,\boldsymbol{\nu_{59}}}{96}}+1/16\,\boldsymbol{\nu_{81}}-{\frac {7\,\boldsymbol{\nu_{11}}}{128}}\\
&-{\frac 
{2653\,\boldsymbol{\nu_{83}}}{2928}}-{\frac {2929\,\boldsymbol{\nu_{90}}}{5856}}-{\frac {45\,\boldsymbol{\nu_{109}}}{488}}+{\frac {65\,\boldsymbol{\nu_{137}}}{732}}+1/4\,\boldsymbol{\nu_{41}}+1/
4\,\boldsymbol{\nu_{42}}\\
&-{\frac {43\,\boldsymbol{\nu_{58}}}{96}}-{\frac {167\,\boldsymbol{\nu_{9}}}{768}}+1/4\,\boldsymbol{\nu_{119}}+1/2\,\boldsymbol{\nu_{139}}
\end{aligned}
\\
\addlinespace
\midrule
\addlinespace
\boldsymbol{\nu_{44}} &
\begin{aligned}
&\hphantom{+}1/2\,\boldsymbol{\nu_{80}}+4\,\boldsymbol{\nu_{83}}-1/2\,\boldsymbol{\nu_{51}}+2\,\boldsymbol{\nu_{90}}-1/2\,\boldsymbol{\nu_{59}}+{\frac {9\,\boldsymbol{\nu_{9}}}{32}}-1/2\,\boldsymbol{\nu_{48}}\\
&-1/2\,\boldsymbol{\nu_{58}}+{\frac {3\,\boldsymbol{\nu_{7}}}{64}}
\end{aligned}
\\
\addlinespace
\midrule
\addlinespace
\boldsymbol{\nu_{45}} & 
\begin{aligned} 
&-{\frac {4\,\boldsymbol{\nu_{108}}}{183}}+{\frac {299\,\boldsymbol{\nu_{7}}}{2304}}-{\frac {25\,\boldsymbol{\nu_{124}}}{732}}-1/2\,\boldsymbol{\nu_{135}}-{\frac {265\,\boldsymbol{\nu_{131}}}{1098}}+{\frac {142\,\boldsymbol{\nu_{132}}}{549}}\\
&+{\frac {35\,\boldsymbol{\nu_{127}}}{1098}}-\boldsymbol{\nu_{79}}-5/6\,\boldsymbol{\nu_{51}}-{\frac {169\,\boldsymbol{\nu_{48}}}{72}}-{
\frac {107\,\boldsymbol{\nu_{55}}}{72}}-{\frac {85\,\boldsymbol{\nu_{59}}}{72}}+1/2\,\boldsymbol{\nu_{80}}\\
&-{\frac {7\,\boldsymbol{\nu_{11}}}{96}}-5/6\,\boldsymbol{\nu_{77}}+{\frac {5101\,\boldsymbol{\nu_{83}}}{1098}}+{\frac {2585\,\boldsymbol{\nu_{90}}}{1098}}-{\frac {4\,\boldsymbol{\nu_{109}}}{
183}}\\
&+{\frac {785\,\boldsymbol{\nu_{137}}}{2196}}-{\frac {85\,\boldsymbol{\nu_{58}}}{72}}+{
\frac {175\,\boldsymbol{\nu_{9}}}{576}}+1/6\,\boldsymbol{\nu_{119}}\end{aligned}
\\
\addlinespace
\midrule
\addlinespace
\boldsymbol{\nu_{46}} & -4\,\boldsymbol{\nu_{83}}-2\,\boldsymbol{\nu_{90}}-\boldsymbol{\nu_{59}}-\boldsymbol{\nu_{48}}-\boldsymbol{\nu_{58}}-1/8\,\boldsymbol{\nu_{7}}
\\
\addlinespace
\midrule
\addlinespace
\boldsymbol{\nu_{47}} & -2\,\boldsymbol{\nu_{83}}-\boldsymbol{\nu_{90}}-1/8\,\boldsymbol{\nu_{9}}+1/2\,\boldsymbol{\nu_{48}}
\\
\addlinespace
\midrule
\addlinespace
\boldsymbol{\nu_{49}} & 
\begin{aligned} 
&\hphantom{+}\boldsymbol{\nu_{77}}-\boldsymbol{\nu_{80}}-9\,\boldsymbol{\nu_{83}}+\boldsymbol{\nu_{51}}-9/2\,\boldsymbol{\nu_{90}}-1/2
\,\boldsymbol{\nu_{59}}-5/8\,\boldsymbol{\nu_{9}}+\boldsymbol{\nu_{48}}+\boldsymbol{\nu_{55}}\\
&-1/2\,\boldsymbol{\nu_{58}}-1/16
\,\boldsymbol{\nu_{7}}
\end{aligned}
\\
\addlinespace
\midrule
\addlinespace
\boldsymbol{\nu_{50}} & 
4\,\boldsymbol{\nu_{83}}+2\,\boldsymbol{\nu_{90}}+1/4\,\boldsymbol{\nu_{9}}-\boldsymbol{\nu_{48}}
\\
\addlinespace
\midrule
\addlinespace
\boldsymbol{\nu_{52}} & -\boldsymbol{\nu_{79}}+\boldsymbol{\nu_{83}}+1/2\,\boldsymbol{\nu_{90}}-1/2\,\boldsymbol{\nu_{59}}+1/8\,\boldsymbol{\nu_{9}}-\boldsymbol{\nu_{48}}-1/2\,\boldsymbol{\nu_{55}}-1/2\,\boldsymbol{\nu_{58}}
\\
\addlinespace
\midrule
\addlinespace
\boldsymbol{\nu_{53}} & -1/8\,\boldsymbol{\nu_{11}}-1/2\,\boldsymbol{\nu_{59}}-1/2\,\boldsymbol{\nu_{9}}-1/2\,\boldsymbol{\nu_{48}}+1/2
\,\boldsymbol{\nu_{55}}-1/2\,\boldsymbol{\nu_{58}}+{\frac {7\,\boldsymbol{\nu_{7}}}{32}}
\\
\addlinespace
\midrule
\addlinespace
\boldsymbol{\nu_{54}} & 
\begin{aligned}
&-1/
16\,\boldsymbol{\nu_{11}}-4\,\boldsymbol{\nu_{83}}-2\,\boldsymbol{\nu_{90}}-3/4\,\boldsymbol{\nu_{59}}-3/8\,\boldsymbol{\nu_{9}}+3/4\,\boldsymbol{\nu_{48}}-1/4\,\boldsymbol{\nu_{55}}\\
&-3/4\,\boldsymbol{\nu_{58}}+{\frac {3\,\boldsymbol{\nu_{7}}}{64}}
\end{aligned}
\\
\addlinespace
\midrule
\addlinespace
\boldsymbol{\nu_{56}} & 2\,\boldsymbol{\nu_{83}}-\boldsymbol{\nu_{51}}+\boldsymbol{\nu_{90}}-1/2\,\boldsymbol{\nu_{55}}+{\frac {7\,\boldsymbol{\nu_{7}}}{64}}-1/2\,\boldsymbol{\nu_{59}}-1/2\,\boldsymbol{\nu_{48}}-1/2\,\boldsymbol{\nu_{58}}
\\
\addlinespace
\midrule
\addlinespace
\boldsymbol{\nu_{57}} &  2\,\boldsymbol{\nu_{51}}+2\,\boldsymbol{\nu_{59}}+\boldsymbol{\nu_{48}}+\boldsymbol{\nu_{55}}+
\boldsymbol{\nu_{58}}-1/16\,\boldsymbol{\nu_{7}}
\\
\addlinespace
\midrule
\addlinespace
\boldsymbol{\nu_{60}} & 
\begin{aligned}
&-\boldsymbol{\nu_{77}}+\boldsymbol{\nu_{80}}+8\,\boldsymbol{\nu_{83}}-\boldsymbol{\nu_{51}}+4\,\boldsymbol{\nu_{90}}-1/2\,\boldsymbol{\nu_{59}}+5/8\,\boldsymbol{\nu_{9}}-2\,\boldsymbol{\nu_{48}}\\
&-\boldsymbol{\nu_{55}}-1/2\,\boldsymbol{\nu_{58}}+1/16\,\boldsymbol{\nu_{7}}
\end{aligned}
\\
\addlinespace
\midrule
\addlinespace
\boldsymbol{\nu_{61}} & 
\begin{aligned}
&-\boldsymbol{\nu_{80}}
-4\,\boldsymbol{\nu_{83}}-2\,\boldsymbol{\nu_{90}}-1/2\,\boldsymbol{\nu_{59}}-{\frac {5\,\boldsymbol{\nu_{9}}}{16}
}+1/2\,\boldsymbol{\nu_{48}}+1/2\,\boldsymbol{\nu_{55}}\\
&-1/2\,\boldsymbol{\nu_{58}}-1/32\,\boldsymbol{\nu_{7}}
\end{aligned}
\\
\addlinespace
\midrule
\addlinespace
\nu_
{{62}} & \boldsymbol{\nu_{59}}+1/4\,\boldsymbol{\nu_{9}}+\boldsymbol{\nu_{48}}-\boldsymbol{\nu_{55}}+\boldsymbol{\nu_{58}}-1/16
\,\boldsymbol{\nu_{7}}
\\
\addlinespace
\midrule
\addlinespace
\boldsymbol{\nu_{63}} & 4\,\boldsymbol{\nu_{83}}+2\,\boldsymbol{\nu_{90}}+5/4\,\boldsymbol{\nu_{59}}+3/8
\,\boldsymbol{\nu_{9}}-1/4\,\boldsymbol{\nu_{48}}-1/4\,\boldsymbol{\nu_{55}}+5/4\,\boldsymbol{\nu_{58}}-{\frac {
\boldsymbol{\nu_{7}}}{64}}
\\
\addlinespace
\midrule
\addlinespace
\boldsymbol{\nu_{64}} & 2\,\boldsymbol{\nu_{83}}+\boldsymbol{\nu_{90}}+1/2\,\boldsymbol{\nu_{59}}-1/
8\,\boldsymbol{\nu_{9}}+1/2\,\boldsymbol{\nu_{48}}+1/2\,\boldsymbol{\nu_{55}}+1/2\,\boldsymbol{\nu_{58}}+{\frac {3
\,\boldsymbol{\nu_{7}}}{32}}
\\
\addlinespace
\midrule
\addlinespace
\boldsymbol{\nu_{65}} & -1/2\,\boldsymbol{\nu_{59}}-1/2\,\boldsymbol{\nu_{48}}+1/2\,\boldsymbol{\nu_{55}}-1/2\,\boldsymbol{\nu_{58}}-1/32\,\boldsymbol{\nu_{7}}
\\
\addlinespace
\midrule
\addlinespace
\boldsymbol{\nu_{66}} &
\begin{aligned}
&\hphantom{+}\boldsymbol{\nu_{79}}-1/2\,\boldsymbol{\nu_{83}}-1/4\,\boldsymbol{\nu_{90}}-5/8\,\boldsymbol{\nu_{59}}-{\frac {\boldsymbol{\nu_{9}}}{64}}-5/8\,
\boldsymbol{\nu_{48}}+1/8\,\boldsymbol{\nu_{55}}\\
&-5/8\,\boldsymbol{\nu_{58}}+{\frac {\boldsymbol{\nu_{7}}}{256}}
\end{aligned}
\\
\addlinespace
\midrule
\addlinespace
\boldsymbol{\nu_{67}} & \boldsymbol{\nu_{80}}+2\,\boldsymbol{\nu_{83}}+\boldsymbol{\nu_{90}}+2\,\boldsymbol{\nu_{59}}+1/8\,\boldsymbol{\nu_{9}}+2\,\boldsymbol{\nu_{48}}+2\,\boldsymbol{\nu_{58}}
\\
\addlinespace
\midrule
\addlinespace
\boldsymbol{\nu_{68}} & {\frac {11\,\boldsymbol{\nu_{7}}}{1536}}-{\frac {5\,\boldsymbol{\nu_{48}}}{24}}-1/24\,\boldsymbol{\nu_{55}}-{\frac {5\,\boldsymbol{\nu_{59}}}{24}}-{\frac {\boldsymbol{\nu_{11}}}{128}}-1/4\,\boldsymbol{\nu_{83}}-1/8\,\boldsymbol{\nu_{90}}-{
\frac {5\,\boldsymbol{\nu_{58}}}{24}}-{\frac {5\,\boldsymbol{\nu_{9}}}{384}}
\\
\addlinespace
\midrule
\addlinespace
\boldsymbol{\nu_{70}} & 
\begin{aligned} 
&\hphantom{+}1/3
\,\boldsymbol{\nu_{135}}-{\frac {113\,\boldsymbol{\nu_{124}}}{366}}-{\frac {65\,\boldsymbol{\nu_{109}}
}{366}}+{\frac {169\,\boldsymbol{\nu_{137}}}{244}}-{\frac {15\,\boldsymbol{\nu_{127}}}{122}}
-{\frac {21\,\boldsymbol{\nu_{108}}}{61}}\\
&+{\frac {9\,\boldsymbol{\nu_{132}}}{122}}+{\frac {
44\,\boldsymbol{\nu_{131}}}{183}}+1/2\,\boldsymbol{\nu_{79}}-{\frac {7\,\boldsymbol{\nu_{77}}}{36}}-{
\frac {\boldsymbol{\nu_{11}}}{96}}+1/18\,\boldsymbol{\nu_{80}}+{\frac {259\,\boldsymbol{\nu_{83}}}{4392
}}\\
&+1/9\,\boldsymbol{\nu_{51}}-{\frac {3629\,\boldsymbol{\nu_{90}}}{8784}}+{\frac {73\,\boldsymbol{\nu_{59}}}{288}}-{\frac {91\,\boldsymbol{\nu_{9}}}{2304}}+{\frac {97\,\boldsymbol{\nu_{48}}}{288}
}+{\frac {59\,\boldsymbol{\nu_{55}}}{288}}+{\frac {73\,\boldsymbol{\nu_{58}}}{288}}\\
&+{\frac {
11\,\boldsymbol{\nu_{7}}}{9216}}
\end{aligned}
\\
\addlinespace
\midrule
\addlinespace
\boldsymbol{\nu_{71}} & \boldsymbol{\nu_{83}}+\boldsymbol{\nu_{51}}+1/2\,\boldsymbol{\nu_{90}}
+\boldsymbol{\nu_{55}}-1/32\,\boldsymbol{\nu_{7}}
\\
\addlinespace
\midrule
\addlinespace
\boldsymbol{\nu_{72}} &
\begin{aligned}
&-\boldsymbol{\nu_{79}}+2\,\boldsymbol{\nu_{83}}+\boldsymbol{\nu_{90}}+1/8\,\boldsymbol{\nu_{59}}+{\frac {9\,\boldsymbol{\nu_{9}}}{64}}-3/8\,\boldsymbol{\nu_{48}}-1/8\,
\boldsymbol{\nu_{55}}+1/8\,\boldsymbol{\nu_{58}}\\
&-{\frac {\boldsymbol{\nu_{7}}}{256}}
\end{aligned}
\\
\addlinespace
\midrule
\addlinespace
\boldsymbol{\nu_{73}} & 2\,\boldsymbol{\nu_{83}}-\boldsymbol{\nu_{51}}+\boldsymbol{\nu_{90}}-\boldsymbol{\nu_{55}}+1/8\,\boldsymbol{\nu_{7}}
\\
\addlinespace
\midrule
\addlinespace
\boldsymbol{\nu_{74}} & 2\,
\boldsymbol{\nu_{51}}+\boldsymbol{\nu_{59}}+2\,\boldsymbol{\nu_{55}}-1/8\,\boldsymbol{\nu_{7}}
\\
\addlinespace
\midrule
\addlinespace
\boldsymbol{\nu_{75}} & \boldsymbol{\nu_{58}}
\\
\addlinespace
\midrule
\addlinespace
\boldsymbol{\nu_{76}} & \boldsymbol{\nu_{83}}+1/2\,\boldsymbol{\nu_{90}}+\boldsymbol{\nu_{59}}
\\
\addlinespace
\midrule
\addlinespace
\boldsymbol{\nu_{78}} & -\boldsymbol{\nu_{80}}-5\,\boldsymbol{\nu_{83}}-5/2\,\boldsymbol{\nu_{90}}-\boldsymbol{\nu_{59}}-3/8\,\boldsymbol{\nu_{9}}-\nu_{{58}
}
\\
\addlinespace
\midrule
\addlinespace
\boldsymbol{\nu_{82}}&=1/2\,\boldsymbol{\nu_{81}}+5/2\,\boldsymbol{\nu_{83}}+2\,\boldsymbol{\nu_{90}}+\boldsymbol{\nu_{59}}+
1/4\,\boldsymbol{\nu_{9}}+\boldsymbol{\nu_{58}}
\\
\addlinespace
\midrule
\addlinespace
\boldsymbol{\nu_{84}} & 1/2\,\boldsymbol{\nu_{81}}-5/2\,\boldsymbol{\nu_{83}}-1
/4\,\boldsymbol{\nu_{90}}-1/2\,\boldsymbol{\nu_{59}}-1/8\,\boldsymbol{\nu_{9}}-1/2\,\boldsymbol{\nu_{58}}
\\
\addlinespace
\midrule
\addlinespace
\nu_{{85
}} & 
\begin{aligned}
&\hphantom{+}{\frac {227\,\boldsymbol{\nu_{108}}}{366}}+{\frac {179\,\boldsymbol{\nu_{7}}}{4608}}+{
\frac {229\,\boldsymbol{\nu_{124}}}{732}}-{\frac {13\,\boldsymbol{\nu_{131}}}{2196}}+{\frac 
{523\,\boldsymbol{\nu_{132}}}{4392}}+{\frac {511\,\boldsymbol{\nu_{127}}}{4392}}\\
&-1/2\,\boldsymbol{\nu_{79}}-1/4\,\boldsymbol{\nu_{51}}-{\frac {103\,\boldsymbol{\nu_{48}}}{144}}-{\frac {65\,\boldsymbol{\nu_{55}}}{144}}-{\frac {55\,\boldsymbol{\nu_{59}}}{144}}+1/12\,\boldsymbol{\nu_{80}}\\
&-1/48\,\boldsymbol{\nu_{11}}-1/3\,\boldsymbol{\nu_{77}}+{\frac {3251\,\boldsymbol{\nu_{83}}}{4392}}
+{\frac {1943\,
\boldsymbol{\nu_{90}}}{2196}}+{\frac {271\,\boldsymbol{\nu_{109}}}{732}}-{\frac {2479\,\boldsymbol{\nu_{137}}}{1098}}\\
&-{\frac {55\,\boldsymbol{\nu_{58}}}{144}}+{\frac {73\,\boldsymbol{\nu_{9}}}{
1152}}+1/3\,\boldsymbol{\nu_{119}}+\boldsymbol{\nu_{139}}
\end{aligned}
\\
\addlinespace
\midrule
\addlinespace
\boldsymbol{\nu_{86}} & -\boldsymbol{\nu_{81}}+4\,\boldsymbol{\nu_{83}}+1/2\,\boldsymbol{\nu_{90}}+\boldsymbol{\nu_{59}}+1/2\,\boldsymbol{\nu_{9}}+\boldsymbol{\nu_{58}}-1/8\,\boldsymbol{\nu_{7}}
\\
\addlinespace
\midrule
\addlinespace
\boldsymbol{\nu_{87}} & 
\begin{aligned}
&-2\,\boldsymbol{\nu_{69}}-{\frac {5\,\boldsymbol{\nu_{7}}}{128}}+1/2\,\boldsymbol{\nu_{51}}
+1/4\,\boldsymbol{\nu_{48}}+1/4\,\boldsymbol{\nu_{55}}+1/4\,\boldsymbol{\nu_{59}}-1/2\,\boldsymbol{\nu_{80}}\\
&-4\,
\boldsymbol{\nu_{83}}-2\,\boldsymbol{\nu_{90}}+1/4\,\boldsymbol{\nu_{58}}-{\frac {5\,\boldsymbol{\nu_{9}}}{16}}
\end{aligned}
\\
\addlinespace
\midrule
\addlinespace
\boldsymbol{\nu_{88}} & 
\begin{aligned}
&-1/6\,\boldsymbol{\nu_{119}}-1/6\,\boldsymbol{\nu_{135}}+{\frac {159\,\boldsymbol{\nu_{124}}
}{244}}+{\frac {23\,\boldsymbol{\nu_{109}}}{61}}-{\frac {3827\,\boldsymbol{\nu_{137}}}{2196}
}+{\frac {235\,\boldsymbol{\nu_{127}}}{1098}}\\
&+{\frac {130\,\boldsymbol{\nu_{108}}}{183}}-{
\frac {223\,\boldsymbol{\nu_{132}}}{549}}-{\frac {263\,\boldsymbol{\nu_{131}}}{1098}}+2/9\,
\boldsymbol{\nu_{77}}+1/16\,\boldsymbol{\nu_{11}}-1/9\,\boldsymbol{\nu_{80}}\\
&-{\frac {925\,\boldsymbol{\nu_{83}}}{
732}}+1/9\,\boldsymbol{\nu_{51}}+{\frac {325\,\boldsymbol{\nu_{90}}}{1464}}+{\frac {13\,\boldsymbol{\nu_{59}}}{144}}+{\frac {41\,\boldsymbol{\nu_{9}}}{1152}}+{\frac {49\,\boldsymbol{\nu_{48}}}{
144}}\\
&+{\frac {59\,\boldsymbol{\nu_{55}}}{144}}+{\frac {13\,\boldsymbol{\nu_{58}}}{144}}-{
\frac {37\,\boldsymbol{\nu_{7}}}{512}},
\end{aligned}
\\
\addlinespace
\midrule
\addlinespace
\boldsymbol{\nu_{89}} & -8\,\boldsymbol{\nu_{83}}-\boldsymbol{\nu_{90}}-2\,
\boldsymbol{\nu_{59}}-1/2\,\boldsymbol{\nu_{9}}-2\,\boldsymbol{\nu_{58}}
\\
\addlinespace
\midrule
\addlinespace
\boldsymbol{\nu_{91}} & -1/2\,\boldsymbol{\nu_{90}}+1/
8\,\boldsymbol{\nu_{9}}-1/16\,\boldsymbol{\nu_{7}}
\\
\addlinespace
\midrule
\addlinespace
\boldsymbol{\nu_{92}} & 
\begin{aligned}
&-{\frac {227\,\boldsymbol{\nu_{108}}}{183
}}-{\frac {107\,\boldsymbol{\nu_{7}}}{2304}}-{\frac {229\,\boldsymbol{\nu_{124}}}{366}}+{
\frac {13\,\boldsymbol{\nu_{131}}}{1098}}-{\frac {523\,\boldsymbol{\nu_{132}}}{2196}}-{
\frac {511\,\boldsymbol{\nu_{127}}}{2196}}\\
&+\boldsymbol{\nu_{79}}+{\frac {31\,\boldsymbol{\nu_{48}}}{72}
}+{\frac {29\,\boldsymbol{\nu_{55}}}{72}}+{\frac {37\,\boldsymbol{\nu_{59}}}{72}}+1/3\,\boldsymbol{\nu_{80}}+1/24\,\boldsymbol{\nu_{11}}\\
&-1/3\,\boldsymbol{\nu_{77}}+{\frac {5533\,\boldsymbol{\nu_{83}}}{2196}
}+{\frac {253\,\boldsymbol{\nu_{90}}}{1098}}-{\frac {271\,\boldsymbol{\nu_{109}}}{366}}+{
\frac {2479\,\boldsymbol{\nu_{137}}}{549}}+{\frac {37\,\boldsymbol{\nu_{58}}}{72}}\\
&+{\frac {
107\,\boldsymbol{\nu_{9}}}{576}}-2/3\,\boldsymbol{\nu_{119}}-2\,\boldsymbol{\nu_{139}}
\end{aligned}
\\
\addlinespace
\midrule
\addlinespace
\boldsymbol{\nu_{93}} & 
\begin{aligned}
&-{
\frac {583\,\boldsymbol{\nu_{108}}}{1464}}-1/4\,\boldsymbol{\nu_{69}}+{\frac {85\,\boldsymbol{\nu_{7}}
}{6144}}-{\frac {91\,\boldsymbol{\nu_{124}}}{2928}}+{\frac {199\,\boldsymbol{\nu_{131}}}{
1464}}+{\frac {613\,\boldsymbol{\nu_{132}}}{5856}}\\
&-{\frac {205\,\boldsymbol{\nu_{127}}}{1952
}}+1/8\,\boldsymbol{\nu_{79}}+1/16\,\boldsymbol{\nu_{51}}+{\frac {67\,\boldsymbol{\nu_{48}}}{192}}+{
\frac {13\,\boldsymbol{\nu_{55}}}{192}}+{\frac {25\,\boldsymbol{\nu_{59}}}{64}}-1/48\,\boldsymbol{\nu_{80}}\\
&+{\frac {3001\,\boldsymbol{\nu_{83}}}{5856}}-{\frac {8\,\boldsymbol{\nu_{90}}}{183}}-{
\frac {983\,\boldsymbol{\nu_{109}}}{2928}}+{\frac {979\,\boldsymbol{\nu_{137}}}{976}}-{
\frac {3\,\boldsymbol{\nu_{20}}}{32}}-1/4\,\boldsymbol{\nu_{41}}+1/2\,\boldsymbol{\nu_{42}}\\
&+{\frac {25
\,\boldsymbol{\nu_{58}}}{64}}-{\frac {23\,\boldsymbol{\nu_{9}}}{512}}
\end{aligned}
\\
\addlinespace
\midrule
\addlinespace
\boldsymbol{\nu_{94}} & 
\begin{aligned}
&-{\frac {3
\,\boldsymbol{\nu_{20}}}{128}}+{\frac {1301\,\boldsymbol{\nu_{7}}}{110592}}-{\frac {1001\,
\boldsymbol{\nu_{9}}}{27648}}-{\frac {31\,\boldsymbol{\nu_{11}}}{4608}}+1/8\,\boldsymbol{\nu_{42}}+{
\frac {125\,\boldsymbol{\nu_{48}}}{3456}}+{\frac {5\,\boldsymbol{\nu_{51}}}{288}}\\
&+{\frac {67
\,\boldsymbol{\nu_{55}}}{3456}}+{\frac {191\,\boldsymbol{\nu_{58}}}{3456}}+{\frac {191\,\boldsymbol{\nu_{59}}}{3456}}-1/8\,\boldsymbol{\nu_{69}}-1/48\,\boldsymbol{\nu_{77}}+1/24\,\boldsymbol{\nu_{79}}-{
\frac {\boldsymbol{\nu_{80}}}{288}}\\
&+{\frac {\boldsymbol{\nu_{81}}}{192}}+{\frac {7289\,\boldsymbol{\nu_{83}}}{105408}}-{\frac {16225\,\boldsymbol{\nu_{90}}}{210816}}-{\frac {1045\,\boldsymbol{\nu_{108}}}{8784}}-{\frac {1541\,\boldsymbol{\nu_{109}}}{17568}}+{\frac {5\,\nu_{{
119}}}{144}}\\
&-{\frac {409\,\boldsymbol{\nu_{124}}}{8784}}-{\frac {3215\,\nu_{{127}
}}{105408}}+{\frac {3191\,\boldsymbol{\nu_{131}}}{52704}}+{\frac {5833\,\boldsymbol{\nu_{132}}}{105408}}+1/48\,\boldsymbol{\nu_{135}}+{\frac {823\,\boldsymbol{\nu_{137}}}{3294}}\\
&+1/24\,
\boldsymbol{\nu_{139}}
\end{aligned}
\\
\addlinespace
\midrule
\addlinespace
\boldsymbol{\nu_{95}} & 
\begin{aligned}
&\hphantom{+}{\frac {35\,\boldsymbol{\nu_{108}}}{61}}-{\frac {3\,\boldsymbol{\nu_{7}}}{128}}-{\frac {5\,\boldsymbol{\nu_{124}}}{122}}-{\frac {15\,\boldsymbol{\nu_{131}}}{122}
}-{\frac {15\,\boldsymbol{\nu_{132}}}{122}}+{\frac {25\,\boldsymbol{\nu_{127}}}{122}}-3/8\,
\boldsymbol{\nu_{48}}\\
&-1/4\,\boldsymbol{\nu_{59}}-{\frac {301\,\boldsymbol{\nu_{83}}}{488}}+{\frac {53\,
\boldsymbol{\nu_{90}}}{976}}+{\frac {35\,\boldsymbol{\nu_{109}}}{61}}-{\frac {106\,\boldsymbol{\nu_{137}}}{61}}+{\frac {3\,\boldsymbol{\nu_{20}}}{32}}+3/8\,\boldsymbol{\nu_{41}}\\
&-3/4\,\boldsymbol{\nu_{42}}-1
/4\,\boldsymbol{\nu_{58}}+{\frac {3\,\boldsymbol{\nu_{9}}}{64}}
\end{aligned}
\\
\addlinespace
\midrule
\addlinespace
\boldsymbol{\nu_{96}} & 
\begin{aligned}
&-{\frac {88\,\boldsymbol{\nu_{108}}}{183}}+1/32\,\boldsymbol{\nu_{7}}+{\frac {5\,\boldsymbol{\nu_{124}}}{61}}+{\frac {29
\,\boldsymbol{\nu_{131}}}{366}}+{\frac {29\,\boldsymbol{\nu_{132}}}{366}}+{\frac {5\,\boldsymbol{\nu_{127}}}{732}}+1/2\,\boldsymbol{\nu_{48}}\\
&+1/3\,\boldsymbol{\nu_{59}}+{\frac {119\,\boldsymbol{\nu_{83}}}{
183}}+{\frac {73\,\boldsymbol{\nu_{90}}}{732}}-{\frac {88\,\boldsymbol{\nu_{109}}}{183}}+{
\frac {119\,\boldsymbol{\nu_{137}}}{122}}-1/8\,\boldsymbol{\nu_{20}}-1/2\,\boldsymbol{\nu_{41}}\\
&+\boldsymbol{\nu_{42}}+1/3\,\boldsymbol{\nu_{58}}-1/16\,\boldsymbol{\nu_{9}}
\end{aligned}
\\
\addlinespace
\midrule
\addlinespace
\boldsymbol{\nu_{97}} & 
\begin{aligned}
&\hphantom{+}{\frac {5\,\boldsymbol{\nu_{20}}
}{64}}-{\frac {839\,\boldsymbol{\nu_{7}}}{9216}}+{\frac {587\,\boldsymbol{\nu_{9}}}{2304}}+{
\frac {25\,\boldsymbol{\nu_{11}}}{384}}-1/8\,\boldsymbol{\nu_{41}}-1/2\,\boldsymbol{\nu_{42}}+{\frac {
43\,\boldsymbol{\nu_{48}}}{288}}\\
&-1/24\,\boldsymbol{\nu_{51}}+{\frac {29\,\boldsymbol{\nu_{55}}}{288}}+{
\frac {37\,\boldsymbol{\nu_{58}}}{288}}+{\frac {37\,\boldsymbol{\nu_{59}}}{288}}+1/2\,\boldsymbol{\nu_{69}}+1/12\,\boldsymbol{\nu_{77}}+1/24\,\boldsymbol{\nu_{80}}\\
&-1/16\,\boldsymbol{\nu_{81}}+{\frac {3409\,
\boldsymbol{\nu_{83}}}{8784}}+{\frac {10201\,\boldsymbol{\nu_{90}}}{17568}}+{\frac {105\,\boldsymbol{\nu_{108}}}{244}}+{\frac {149\,\boldsymbol{\nu_{109}}}{488}}-{\frac {5\,\boldsymbol{\nu_{119}}
}{12}}\\
&+{\frac {21\,\boldsymbol{\nu_{124}}}{61}}+{\frac {1289\,\boldsymbol{\nu_{127}}}{8784}}-{\frac {1625\,\boldsymbol{\nu_{131}}}{4392}}-{\frac {3799\,\boldsymbol{\nu_{132}}}{8784}}-1
/4\,\boldsymbol{\nu_{135}}
-{\frac {577\,\boldsymbol{\nu_{137}}}{1098}}\\
&-1/2\,\boldsymbol{\nu_{139}}
\end{aligned}
\\
\addlinespace
\midrule
\addlinespace
\nu_
{{98}} & 
\begin{aligned}
&-{\frac {59\,\boldsymbol{\nu_{108}}}{183}}+{\frac {\boldsymbol{\nu_{7}}}{64}}-{\frac 
{3\,\boldsymbol{\nu_{124}}}{122}}+{\frac {17\,\boldsymbol{\nu_{131}}}{183}}+{\frac {17\,\boldsymbol{\nu_{132}}}{183}}-{\frac {8\,\boldsymbol{\nu_{127}}}{183}}+1/8\,\boldsymbol{\nu_{48}}\\
&+1/6\,\boldsymbol{\nu_{59}}+{\frac {283\,\boldsymbol{\nu_{83}}}{488}}+{\frac {105\,\boldsymbol{\nu_{90}}}{976}}-{
\frac {59\,\boldsymbol{\nu_{109}}}{183}}+{\frac {129\,\boldsymbol{\nu_{137}}}{122}}-1/16\,
\boldsymbol{\nu_{20}}\\
&-1/8\,\boldsymbol{\nu_{41}}+1/4\,\boldsymbol{\nu_{42}}+1/6\,\boldsymbol{\nu_{58}}-{\frac {\boldsymbol{\nu_{9}}}{64}}
\end{aligned}
\\
\addlinespace
\midrule
\addlinespace
\boldsymbol{\nu_{99}} & 
\begin{aligned}
&\hphantom{-}{\frac {74\,\boldsymbol{\nu_{108}}}{183}}+{\frac {23\,\boldsymbol{\nu_{7}}}{768}}+{\frac {11\,\boldsymbol{\nu_{124}}}{122}}+1/3\,\boldsymbol{\nu_{135}}+{\frac {
19\,\boldsymbol{\nu_{131}}}{183}}-{\frac {23\,\boldsymbol{\nu_{132}}}{366}}+{\frac {79\,\boldsymbol{\nu_{127}}}{366}}\\
&-3/8\,\boldsymbol{\nu_{48}}+{\frac {5\,\boldsymbol{\nu_{55}}}{24}}-{\frac {13
\,\boldsymbol{\nu_{59}}}{24}}+1/6\,\boldsymbol{\nu_{81}}-{\frac {56\,\boldsymbol{\nu_{83}}}{61}}-{
\frac {5\,\boldsymbol{\nu_{90}}}{61}}+{\frac {13\,\boldsymbol{\nu_{109}}}{183}}\\
&-{\frac {443
\,\boldsymbol{\nu_{137}}}{366}}+1/2\,\boldsymbol{\nu_{41}}-{\frac {13\,\boldsymbol{\nu_{58}}}{24}}-{
\frac {9\,\boldsymbol{\nu_{9}}}{64}}
+1/3\,\boldsymbol{\nu_{119}}
\end{aligned}
\\
\addlinespace
\midrule
\addlinespace
\boldsymbol{\nu_{100}} & 
\begin{aligned}
&\hphantom{+}{\frac {10\,\boldsymbol{\nu_{108}}}{183}}+{\frac {23\,\boldsymbol{\nu_{7}}}{1536}}+{\frac {31\,\boldsymbol{\nu_{124}}}{244}}+1/6\,\boldsymbol{\nu_{135}}+{\frac {35\,\boldsymbol{\nu_{131}}}{732}}-{\frac {13\,
\boldsymbol{\nu_{132}}}{366}}+{\frac {107\,\boldsymbol{\nu_{127}}}{1464}}\\
&+1/16\,\boldsymbol{\nu_{48}}+{\frac {5\,\boldsymbol{\nu_{55}}}{48}}-{\frac {13\,\boldsymbol{\nu_{59}}}{48}}+1/12\,\boldsymbol{\nu_{81}}-{\frac {148\,\boldsymbol{\nu_{83}}}{183}}-{\frac {463\,\boldsymbol{\nu_{90}}}{1464}}-{
\frac {41\,\boldsymbol{\nu_{109}}}{366}}\\
&-{\frac {139\,\boldsymbol{\nu_{137}}}{366}}+1/2\,\boldsymbol{\nu_{42}}-{\frac {13\,\boldsymbol{\nu_{58}}}{48}}-{\frac {13\,\boldsymbol{\nu_{9}}}{128}}+1/6
\,\boldsymbol{\nu_{119}}
\end{aligned}
\\
\addlinespace
\midrule
\addlinespace
\boldsymbol{\nu_{101}} & 
\begin{aligned} 
&\hphantom{+}{\frac {209\,\boldsymbol{\nu_{7}}}{4608}}-{\frac {61\,
\boldsymbol{\nu_{9}}}{576}}-{\frac {25\,\boldsymbol{\nu_{11}}}{768}}+1/8\,\boldsymbol{\nu_{41}}+1/8\,
\boldsymbol{\nu_{42}}-{\frac {17\,\boldsymbol{\nu_{48}}}{72}}-1/24\,\boldsymbol{\nu_{51}}\\
&-{\frac {17\,
\boldsymbol{\nu_{55}}}{144}}-{\frac {89\,\boldsymbol{\nu_{58}}}{288}}-{\frac {89\,\boldsymbol{\nu_{59}}}{288}}-1/24\,\boldsymbol{\nu_{77}}-1/8\,\boldsymbol{\nu_{79}}+1/32\,\boldsymbol{\nu_{81}}-{\frac {385
\,\boldsymbol{\nu_{83}}}{1098}}\\
&-{\frac {6895\,\boldsymbol{\nu_{90}}}{35136}}-{\frac {53\,\boldsymbol{\nu_{108}}}{732}}-{\frac {53\,\boldsymbol{\nu_{109}}}{732}}+{\frac {5\,\boldsymbol{\nu_{119}}
}{24}}-{\frac {377\,\boldsymbol{\nu_{124}}}{2928}}-{\frac {445\,\boldsymbol{\nu_{127}}}{8784
}}\\
&+{\frac {469\,\boldsymbol{\nu_{131}}}{4392}}+{\frac {1487\,\boldsymbol{\nu_{132}}}{8784}}+
1/8\,\boldsymbol{\nu_{135}}-{\frac {179\,\boldsymbol{\nu_{137}}}{2196}}+1/4\,\boldsymbol{\nu_{139}}
\end{aligned}
\\
\addlinespace
\midrule
\addlinespace
\boldsymbol{\nu_{102}} & 
\begin{aligned}
&\hphantom{+}{\frac {155\,\boldsymbol{\nu_{124}}}{488}}+{\frac {331\,\boldsymbol{\nu_{109}}}{488
}}-{\frac {809\,\boldsymbol{\nu_{137}}}{488}}+{\frac {219\,\boldsymbol{\nu_{127}}}{976}}+{
\frac {257\,\boldsymbol{\nu_{108}}}{244}}-{\frac {473\,\boldsymbol{\nu_{132}}}{976}}\\
&-{\frac 
{41\,\boldsymbol{\nu_{131}}}{61}}-3/4\,\boldsymbol{\nu_{79}}-1/8\,\boldsymbol{\nu_{80}}-{\frac {\boldsymbol{\nu_{83}}}{976}}-1/8\,\boldsymbol{\nu_{51}}+{\frac {23\,\boldsymbol{\nu_{90}}}{122}}-{\frac {5\,
\boldsymbol{\nu_{59}}}{32}}\\
&+{\frac {7\,\boldsymbol{\nu_{9}}}{256}}-{\frac {9\,\boldsymbol{\nu_{48}}}{32
}}-{\frac {7\,\boldsymbol{\nu_{55}}}{32}}-{\frac {5\,\boldsymbol{\nu_{58}}}{32}}+{\frac {5\,
\boldsymbol{\nu_{7}}}{1024}}
\end{aligned}
\\
\addlinespace
\midrule
\addlinespace
\boldsymbol{\nu_{103}} & 
\begin{aligned}
&\hphantom{+}{\frac {89\,\boldsymbol{\nu_{108}}}{183}}-{\frac {
67\,\boldsymbol{\nu_{7}}}{9216}}+{\frac {53\,\boldsymbol{\nu_{124}}}{732}}-{\frac {140\,\boldsymbol{\nu_{131}}}{549}}-{\frac {1691\,\boldsymbol{\nu_{132}}}{8784}}+{\frac {907\,\boldsymbol{\nu_{127}}}{8784}}\\
&-1/4\,\boldsymbol{\nu_{79}}+{\frac {11\,\boldsymbol{\nu_{48}}}{288}}+{\frac {
\boldsymbol{\nu_{55}}}{288}}+{\frac {5\,\boldsymbol{\nu_{59}}}{288}}-1/12\,\boldsymbol{\nu_{80}}+1/16\,\boldsymbol{\nu_{81}}+{\frac {\boldsymbol{\nu_{11}}}{192}}\\
&+1/12\,\boldsymbol{\nu_{77}}-{\frac {3409\,
\boldsymbol{\nu_{83}}}{8784}}+{\frac {779\,\boldsymbol{\nu_{90}}}{17568}}+{\frac {529\,\boldsymbol{\nu_{109}}}{1464}}-{\frac {4463\,\boldsymbol{\nu_{137}}}{4392}}+{\frac {5\,\boldsymbol{\nu_{58}}
}{288}}\\
&-{\frac {5\,\boldsymbol{\nu_{9}}}{2304}}-1/12\,\boldsymbol{\nu_{119}}
\end{aligned}
\\
\addlinespace
\midrule
\addlinespace
\boldsymbol{\nu_{104}} & 
\begin{aligned}
&\hphantom{+}{\frac {175\,\boldsymbol{\nu_{124}}}{244}}-{\frac {437\,\boldsymbol{\nu_{109}}}{244}}+{\frac 
{649\,\boldsymbol{\nu_{137}}}{244}}-{\frac {225\,\boldsymbol{\nu_{127}}}{488}}-{\frac {127\,
\boldsymbol{\nu_{108}}}{122}}+{\frac {135\,\boldsymbol{\nu_{132}}}{488}}\\
&-{\frac {6\,\boldsymbol{\nu_{131}}}{61}}-3/2\,\boldsymbol{\nu_{79}}-1/4\,\boldsymbol{\nu_{80}}+{\frac {235\,\boldsymbol{\nu_{83}}}{
488}}-1/4\,\boldsymbol{\nu_{51}}-{\frac {13\,\boldsymbol{\nu_{90}}}{122}}-{\frac {5\,\boldsymbol{\nu_{59}}}{16}}\\
&+{\frac {7\,\boldsymbol{\nu_{9}}}{128}}-{\frac {9\,\boldsymbol{\nu_{48}}}{16}}-{
\frac {7\,\boldsymbol{\nu_{55}}}{16}}-{\frac {5\,\boldsymbol{\nu_{58}}}{16}}+{\frac {5\,\boldsymbol{\nu_{7}}}{512}}
\end{aligned}
\\
\addlinespace
\midrule
\addlinespace
\boldsymbol{\nu_{105}} & 
\begin{aligned}
&\hphantom{+}{\frac {227\,\boldsymbol{\nu_{124}}}{1464}}-{\frac {1165
\,\boldsymbol{\nu_{109}}}{1464}}+{\frac {3049\,\boldsymbol{\nu_{137}}}{1464}}-{\frac {167\,
\boldsymbol{\nu_{127}}}{976}}-{\frac {491\,\boldsymbol{\nu_{108}}}{732}}+{\frac {149\,\boldsymbol{\nu_{132}}}{976}}\\
&+{\frac {11\,\boldsymbol{\nu_{131}}}{122}}-1/4\,\boldsymbol{\nu_{79}}-1/24\,\boldsymbol{\nu_{80}}+{\frac {751\,\boldsymbol{\nu_{83}}}{2928}}-1/12\,\boldsymbol{\nu_{51}}-{\frac {71\,
\boldsymbol{\nu_{90}}}{366}}\\
&-{\frac {3\,\boldsymbol{\nu_{59}}}{32}}+{\frac {7\,\boldsymbol{\nu_{9}}}{
768}}-{\frac {3\,\boldsymbol{\nu_{48}}}{32}}-{\frac {11\,\boldsymbol{\nu_{55}}}{96}}-1/32\,
\boldsymbol{\nu_{58}}+{\frac {13\,\boldsymbol{\nu_{7}}}{3072}}
\end{aligned}
\\
\addlinespace
\midrule
\addlinespace
\boldsymbol{\nu_{106}} & 
\begin{aligned} 
&-{\frac {75\,\boldsymbol{\nu_{124}}}{244}}+{\frac {13\,\boldsymbol{\nu_{109}}}{244}}+{\frac {297\,\boldsymbol{\nu_{137}}
}{244}}-{\frac {43\,\boldsymbol{\nu_{127}}}{488}}-{\frac {85\,\boldsymbol{\nu_{108}}}{122}}-{\frac {23\,\boldsymbol{\nu_{132}}}{488}}\\
&+{\frac {20\,\boldsymbol{\nu_{131}}}{61}}+3/2\,\boldsymbol{\nu_{79}}+1/4\,\boldsymbol{\nu_{80}}-{\frac {31\,\boldsymbol{\nu_{83}}}{488}}+1/4\,\boldsymbol{\nu_{51}}-{
\frac {19\,\boldsymbol{\nu_{90}}}{61}}+{\frac {5\,\boldsymbol{\nu_{59}}}{16}}\\
&-{\frac {7\,\boldsymbol{\nu_{9}}}{128}}+{\frac {9\,\boldsymbol{\nu_{48}}}{16}}+{\frac {7\,\boldsymbol{\nu_{55}}}{16}}+
{\frac {5\,\boldsymbol{\nu_{58}}}{16}}-{\frac {5\,\boldsymbol{\nu_{7}}}{512}}
\end{aligned}
\\
\addlinespace
\midrule
\addlinespace
\boldsymbol{\nu_{107}} & 
\begin{aligned}
&\hphantom{+}{\frac {67\,\boldsymbol{\nu_{124}}}{122}}-{\frac {145\,\boldsymbol{\nu_{109}}}{122}}+{\frac {
176\,\boldsymbol{\nu_{137}}}{61}}-{\frac {121\,\boldsymbol{\nu_{127}}}{244}}-{\frac {42\,\boldsymbol{\nu_{108}}}{61}}+{\frac {97\,\boldsymbol{\nu_{132}}}{244}}\\
&-{\frac {43\,\boldsymbol{\nu_{131}}
}{122}}-3\,\boldsymbol{\nu_{79}}-1/2\,\boldsymbol{\nu_{80}}+{\frac {205\,\boldsymbol{\nu_{83}}}{244}}-1
/2\,\boldsymbol{\nu_{51}}-{\frac {11\,\boldsymbol{\nu_{90}}}{122}}-5/8\,\boldsymbol{\nu_{59}}\\
&+{\frac {7
\,\boldsymbol{\nu_{9}}}{64}}-{\frac {9\,\boldsymbol{\nu_{48}}}{8}}-{\frac {7\,\boldsymbol{\nu_{55}}}{8}
}-5/8\,\boldsymbol{\nu_{58}}+{\frac {5\,\boldsymbol{\nu_{7}}}{256}}
\end{aligned}
\\
\addlinespace
\midrule
\addlinespace
\boldsymbol{\nu_{110}} & 
\begin{aligned}
&\hphantom{+}1/2\,\boldsymbol{\nu_{119}}+{\frac {107\,\boldsymbol{\nu_{124}}}{244}}-{\frac {39\,\boldsymbol{\nu_{109}}}{61}}+{
\frac {593\,\boldsymbol{\nu_{137}}}{732}}-{\frac {20\,\boldsymbol{\nu_{127}}}{183}}-{\frac {
39\,\boldsymbol{\nu_{108}}}{61}}\\
&+{\frac {73\,\boldsymbol{\nu_{132}}}{183}}+{\frac {73\,\boldsymbol{\nu_{131}}}{183}}-1/32\,\boldsymbol{\nu_{11}}-{\frac {209\,\boldsymbol{\nu_{83}}}{366}}-{\frac {
131\,\boldsymbol{\nu_{90}}}{732}}-{\frac {7\,\boldsymbol{\nu_{59}}}{24}}-{\frac {17\,\boldsymbol{\nu_{9}}}{192}}\\
&-{\frac {7\,\boldsymbol{\nu_{48}}}{24}}-{\frac {5\,\boldsymbol{\nu_{55}}}{24}}-{
\frac {7\,\boldsymbol{\nu_{58}}}{24}}+{\frac {29\,\boldsymbol{\nu_{7}}}{768}}
\end{aligned}
\\
\addlinespace
\midrule
\addlinespace
\boldsymbol{\nu_{111}} & 
\begin{aligned}
&\hphantom{+}{\frac {21\,\boldsymbol{\nu_{124}}}{61}}+{\frac {72\,\boldsymbol{\nu_{109}}}{61}}-{\frac {232
\,\boldsymbol{\nu_{137}}}{61}}+{\frac {17\,\boldsymbol{\nu_{127}}}{61}}+{\frac {72\,\boldsymbol{\nu_{108}}}{61}}-{\frac {57\,\boldsymbol{\nu_{132}}}{122}}\\
&-{\frac {57\,\boldsymbol{\nu_{131}}}{
122}}-{\frac {45\,\boldsymbol{\nu_{83}}}{122}}+{\frac {45\,\boldsymbol{\nu_{90}}}{122}}
\end{aligned}
\\
\addlinespace
\midrule
\addlinespace
\boldsymbol{\nu_{112}} & 
\begin{aligned}
&\hphantom{+}{\frac {75\,\boldsymbol{\nu_{127}}}{976}}-{\frac {53\,\boldsymbol{\nu_{124}}}{1464}}
+{\frac {193\,\boldsymbol{\nu_{109}}}{1464}}-{\frac {283\,\boldsymbol{\nu_{137}}}{1464}}+{
\frac {5\,\boldsymbol{\nu_{108}}}{732}}-{\frac {45\,\boldsymbol{\nu_{132}}}{976}}+{\frac {
\boldsymbol{\nu_{131}}}{61}}\\
&+1/4\,\boldsymbol{\nu_{79}}+1/24\,\boldsymbol{\nu_{80}}-{\frac {235\,\boldsymbol{\nu_{83}}}{2928}}+1/12\,\boldsymbol{\nu_{51}}+{\frac {13\,\boldsymbol{\nu_{90}}}{732}}+{\frac {3
\,\boldsymbol{\nu_{59}}}{32}}-{\frac {7\,\boldsymbol{\nu_{9}}}{768}}\\
&+{\frac {3\,\boldsymbol{\nu_{48}}}{32}}+{\frac {11\,\boldsymbol{\nu_{55}}}{96}}+1/32\,\boldsymbol{\nu_{58}}-{\frac {13\,\boldsymbol{\nu_{7}}}{3072}}
\end{aligned}
\\
\addlinespace
\midrule
\addlinespace
\boldsymbol{\nu_{113}} & 
\begin{aligned}
&-{\frac {17\,\boldsymbol{\nu_{124}}}{61}}-{\frac {6\,\boldsymbol{\nu_{109}}}{61}}+{\frac {60\,\boldsymbol{\nu_{137}}}{61}}-{\frac {13\,\boldsymbol{\nu_{127}}}{
122}}-{\frac {6\,\boldsymbol{\nu_{108}}}{61}}+{\frac {10\,\boldsymbol{\nu_{132}}}{61}}\\
&+{
\frac {10\,\boldsymbol{\nu_{131}}}{61}}+{\frac {19\,\boldsymbol{\nu_{83}}}{122}}-{\frac {19
\,\boldsymbol{\nu_{90}}}{122}}
\end{aligned}
\\
\addlinespace
\midrule
\addlinespace
\boldsymbol{\nu_{114}} & 
\begin{aligned}
&-\boldsymbol{\nu_{119}}+{\frac {6\,\boldsymbol{\nu_{124}}}{
61}}+{\frac {15\,\boldsymbol{\nu_{109}}}{122}}+{\frac {47\,\boldsymbol{\nu_{137}}}{61}}-{
\frac {59\,\boldsymbol{\nu_{127}}}{244}}-{\frac {23\,\boldsymbol{\nu_{108}}}{61}}-{\frac {
111\,\boldsymbol{\nu_{132}}}{244}}\\
&-{\frac {43\,\boldsymbol{\nu_{131}}}{61}}+{\frac {105\,\boldsymbol{\nu_{83}}}{244}}+{\frac {39\,\boldsymbol{\nu_{90}}}{122}}+3/8\,\boldsymbol{\nu_{59}}+{\frac {3
\,\boldsymbol{\nu_{9}}}{64}}+3/8\,\boldsymbol{\nu_{48}}+1/8\,\boldsymbol{\nu_{55}}\\
&+3/8\,\boldsymbol{\nu_{58}}-{
\frac {3\,\boldsymbol{\nu_{7}}}{256}}
\end{aligned}
\\
\addlinespace
\midrule
\addlinespace
\boldsymbol{\nu_{115}} & 
\begin{aligned}
&-1/2\,\boldsymbol{\nu_{119}}-{\frac {59\,
\boldsymbol{\nu_{124}}}{244}}+{\frac {33\,\boldsymbol{\nu_{109}}}{244}}+{\frac {341\,\boldsymbol{\nu_{137}}}{244}}+{\frac {41\,\boldsymbol{\nu_{127}}}{488}}-{\frac {7\,\boldsymbol{\nu_{108}}}{61
}}-{\frac {49\,\boldsymbol{\nu_{132}}}{488}}\\
&-{\frac {55\,\boldsymbol{\nu_{131}}}{244}}+{
\frac {231\,\boldsymbol{\nu_{83}}}{488}}-{\frac {6\,\boldsymbol{\nu_{90}}}{61}}+3/16\,\boldsymbol{\nu_{59}}+{\frac {3\,\boldsymbol{\nu_{9}}}{128}}+3/16\,\boldsymbol{\nu_{48}}+1/16\,\boldsymbol{\nu_{55}}\\
&+3/
16\,\boldsymbol{\nu_{58}}-{\frac {3\,\boldsymbol{\nu_{7}}}{512}}
\end{aligned}
\\
\addlinespace
\midrule
\addlinespace
\boldsymbol{\nu_{116}} & 
\begin{aligned}
&\hphantom{+}{\frac {43\,\boldsymbol{\nu_{124}}}{122}}-{\frac {57\,\boldsymbol{\nu_{109}}}{61}}+{\frac {347\,\boldsymbol{\nu_{137}}
}{122}}-{\frac {16\,\boldsymbol{\nu_{127}}}{61}}-{\frac {57\,\boldsymbol{\nu_{108}}}{61}}+{
\frac {7\,\boldsymbol{\nu_{132}}}{122}}\\
&+{\frac {7\,\boldsymbol{\nu_{131}}}{122}}+{\frac {14
\,\boldsymbol{\nu_{83}}}{61}}-{\frac {14\,\boldsymbol{\nu_{90}}}{61}}
\end{aligned}
\\
\addlinespace
\midrule
\addlinespace
\boldsymbol{\nu_{117}} & 
\begin{aligned}
&-1/2\,\nu_
{{119}}-{\frac {13\,\boldsymbol{\nu_{124}}}{244}}+{\frac {91\,\boldsymbol{\nu_{109}}}{122}}-
{\frac {783\,\boldsymbol{\nu_{137}}}{244}}+{\frac {65\,\boldsymbol{\nu_{127}}}{244}}+{\frac 
{91\,\boldsymbol{\nu_{108}}}{122}}\\
&-{\frac {39\,\boldsymbol{\nu_{132}}}{244}}-{\frac {39\,\boldsymbol{\nu_{131}}}{244}}-{\frac {95\,\boldsymbol{\nu_{83}}}{244}}+{\frac {95\,\boldsymbol{\nu_{90}}}{
244}}
\end{aligned}
\\
\addlinespace
\midrule
\addlinespace
\boldsymbol{\nu_{118}} & 
\begin{aligned}
&-1/2\,\boldsymbol{\nu_{139}}-1/2\,\boldsymbol{\nu_{119}}-{\frac {119\,\boldsymbol{\nu_{124}}}{488}}+{\frac {19\,\boldsymbol{\nu_{109}}}{488}}+{\frac {437\,\boldsymbol{\nu_{137}}
}{366}}-{\frac {151\,\boldsymbol{\nu_{127}}}{2928}}\\
&-{\frac {41\,\boldsymbol{\nu_{108}}}{122}
}-{\frac {373\,\boldsymbol{\nu_{132}}}{2928}}+{\frac {11\,\boldsymbol{\nu_{131}}}{183}}+3/4
\,\boldsymbol{\nu_{79}}+1/32\,\boldsymbol{\nu_{11}}+1/8\,\boldsymbol{\nu_{80}}\\
&+{\frac {1375\,\boldsymbol{\nu_{83}}
}{2928}}+1/8\,\boldsymbol{\nu_{51}}+{\frac {17\,\boldsymbol{\nu_{90}}}{183}}+{\frac {43\,\boldsymbol{\nu_{59}}}{96}}+{\frac {47\,\boldsymbol{\nu_{9}}}{768}}+{\frac {55\,\boldsymbol{\nu_{48}}}{96}
}+{\frac {41\,\boldsymbol{\nu_{55}}}{96}}\\
&+{\frac {43\,\boldsymbol{\nu_{58}}}{96}}-{\frac {
131\,\boldsymbol{\nu_{7}}}{3072}}
\end{aligned}
\\
\addlinespace
\midrule
\addlinespace
\boldsymbol{\nu_{120}} & 
\begin{aligned}
&-{\frac {101\,\boldsymbol{\nu_{124}}}{122}}+{
\frac {11\,\boldsymbol{\nu_{109}}}{122}}+{\frac {195\,\boldsymbol{\nu_{137}}}{122}}-{\frac {
149\,\boldsymbol{\nu_{127}}}{244}}-{\frac {25\,\boldsymbol{\nu_{108}}}{61}}-{\frac {301\,\boldsymbol{\nu_{132}}}{244}}\\
&+{\frac {\boldsymbol{\nu_{131}}}{61}}+3\,\boldsymbol{\nu_{79}}+1/2\,\boldsymbol{\nu_{80}}-{\frac {45\,\boldsymbol{\nu_{83}}}{244}}+1/2\,\boldsymbol{\nu_{51}}-{\frac {69\,\boldsymbol{\nu_{90}}
}{122}}+5/8\,\boldsymbol{\nu_{59}}-{\frac {7\,\boldsymbol{\nu_{9}}}{64}}\\
&+{\frac {9\,\boldsymbol{\nu_{48}}}{8}}+{\frac {7\,\boldsymbol{\nu_{55}}}{8}}+5/8\,\boldsymbol{\nu_{58}}-{\frac {5\,\boldsymbol{\nu_{7}}}{256}}
\end{aligned}
\\
\addlinespace
\midrule
\addlinespace
\boldsymbol{\nu_{121}} & 
\begin{aligned}
&-{\frac {257\,\boldsymbol{\nu_{124}}}{732}}+{\frac {487\,\boldsymbol{\nu_{109}}}{732}}-{\frac {749\,\boldsymbol{\nu_{137}}}{244}}-{\frac {53\,\boldsymbol{\nu_{127}}}{1464}}+{\frac {335\,\boldsymbol{\nu_{108}}}{366}}-{\frac {749\,\boldsymbol{\nu_{132}}}{
1464}}\\
&-{\frac {25\,\boldsymbol{\nu_{131}}}{183}}+1/2\,\boldsymbol{\nu_{79}}+1/32\,\boldsymbol{\nu_{11}}+1/12\,\boldsymbol{\nu_{80}}+{\frac {115\,\boldsymbol{\nu_{83}}}{1464}}+1/6\,\boldsymbol{\nu_{51}}\\
&+{
\frac {100\,\boldsymbol{\nu_{90}}}{183}}+{\frac {23\,\boldsymbol{\nu_{59}}}{48}}+{\frac {9\,
\boldsymbol{\nu_{9}}}{128}}+{\frac {23\,\boldsymbol{\nu_{48}}}{48}}+{\frac {7\,\boldsymbol{\nu_{55}}}{
16}}+{\frac {17\,\boldsymbol{\nu_{58}}}{48}}-{\frac {71\,\boldsymbol{\nu_{7}}}{1536}}
\end{aligned}
\\
\addlinespace
\midrule
\addlinespace
\boldsymbol{\nu_{122}} & 
\begin{aligned}
&-\boldsymbol{\nu_{135}}-{\frac {11\,\boldsymbol{\nu_{124}}}{61}}-{\frac {29\,\boldsymbol{\nu_{109}}}{61}}+{\frac {229\,\boldsymbol{\nu_{137}}}{61}}-{\frac {73\,\boldsymbol{\nu_{127}}}{122}}
-{\frac {90\,\boldsymbol{\nu_{108}}}{61}}+{\frac {117\,\boldsymbol{\nu_{132}}}{122}}\\
&+{\frac 
{28\,\boldsymbol{\nu_{131}}}{61}}-{\frac {81\,\boldsymbol{\nu_{83}}}{122}}-{\frac {51\,\boldsymbol{\nu_{90}}}{61}}-3/4\,\boldsymbol{\nu_{59}}-{\frac {3\,\boldsymbol{\nu_{9}}}{32}}-3/4\,\boldsymbol{\nu_{48}}
-1/4\,\boldsymbol{\nu_{55}}\\
&-3/4\,\boldsymbol{\nu_{58}}+{\frac {3\,\boldsymbol{\nu_{7}}}{128}}
\end{aligned}
\\
\addlinespace
\midrule
\addlinespace
\boldsymbol{\nu_{123}} & 
\begin{aligned}
&-1/2\,\boldsymbol{\nu_{135}}-{\frac {25\,\boldsymbol{\nu_{124}}}{61}}-{\frac {93\,\boldsymbol{\nu_{109}}}{122}}+{\frac {282\,\boldsymbol{\nu_{137}}}{61}}-{\frac {49\,\boldsymbol{\nu_{127}}}{
244}}-{\frac {77\,\boldsymbol{\nu_{108}}}{61}}\\
&+{\frac {127\,\boldsymbol{\nu_{132}}}{244}}+{
\frac {33\,\boldsymbol{\nu_{131}}}{122}}-{\frac {41\,\boldsymbol{\nu_{83}}}{244}}-{\frac {71
\,\boldsymbol{\nu_{90}}}{122}}-3/8\,\boldsymbol{\nu_{59}}-{\frac {3\,\boldsymbol{\nu_{9}}}{64}}-3/8\,
\boldsymbol{\nu_{48}}\\
&-1/8\,\boldsymbol{\nu_{55}}-3/8\,\boldsymbol{\nu_{58}}+{\frac {3\,\boldsymbol{\nu_{7}}}{256}}
\end{aligned}
\\
\addlinespace
\midrule
\addlinespace
\boldsymbol{\nu_{125}} & 
\begin{aligned}
&-{\frac {13\,\boldsymbol{\nu_{124}}}{122}}-{\frac {123\,\boldsymbol{\nu_{109}}}{122}}+{\frac {188\,\boldsymbol{\nu_{137}}}{61}}-{\frac {53\,\boldsymbol{\nu_{127}}}{244}}-
{\frac {31\,\boldsymbol{\nu_{108}}}{61}}-{\frac {17\,\boldsymbol{\nu_{132}}}{244}}\\
&+{\frac {
11\,\boldsymbol{\nu_{131}}}{61}}+{\frac {115\,\boldsymbol{\nu_{83}}}{244}}-{\frac {115\,\boldsymbol{\nu_{90}}}{244}}
\end{aligned}
\\
\addlinespace
\midrule
\addlinespace
\boldsymbol{\nu_{126}} & 
\begin{aligned}
&\hphantom{+}{\frac {13\,\boldsymbol{\nu_{124}}}{244}}+{\frac {\boldsymbol{\nu_{109}}}{244}}-{\frac {457\,\boldsymbol{\nu_{137}}}{732}}-{\frac {85\,\boldsymbol{\nu_{127}}
}{1464}}+{\frac {31\,\boldsymbol{\nu_{108}}}{122}}-{\frac {193\,\boldsymbol{\nu_{132}}}{1464
}}-{\frac {5\,\boldsymbol{\nu_{131}}}{732}}\\
&+{\frac {\boldsymbol{\nu_{11}}}{64}}+{\frac {265
\,\boldsymbol{\nu_{83}}}{1464}}+{\frac {71\,\boldsymbol{\nu_{90}}}{366}}+{\frac {7\,\boldsymbol{\nu_{59}}}{48}}+{\frac {17\,\boldsymbol{\nu_{9}}}{384}}+{\frac {7\,\boldsymbol{\nu_{48}}}{48}}+{
\frac {5\,\boldsymbol{\nu_{55}}}{48}}+{\frac {7\,\boldsymbol{\nu_{58}}}{48}}\\
&-{\frac {29\,\boldsymbol{\nu_{7}}}{1536}}
\end{aligned}
\\
\addlinespace
\midrule
\addlinespace
\boldsymbol{\nu_{128}} & 
\begin{aligned}
&-{\frac {38\,\boldsymbol{\nu_{124}}}{61}}+{\frac {44\,
\boldsymbol{\nu_{109}}}{61}}-{\frac {196\,\boldsymbol{\nu_{137}}}{61}}+{\frac {7\,\boldsymbol{\nu_{127}}}{61}}+{\frac {44\,\boldsymbol{\nu_{108}}}{61}}+{\frac {8\,\boldsymbol{\nu_{132}}}{61}}+{
\frac {8\,\boldsymbol{\nu_{131}}}{61}}\\
&-{\frac {29\,\boldsymbol{\nu_{83}}}{61}}+{\frac {29\,
\boldsymbol{\nu_{90}}}{61}}
\end{aligned}
\\
\addlinespace
\midrule
\addlinespace
\boldsymbol{\nu_{129}} & 
\begin{aligned}
&\hphantom{+}{\frac {71\,\boldsymbol{\nu_{124}}}{122}}+{\frac {43
\,\boldsymbol{\nu_{109}}}{122}}-{\frac {613\,\boldsymbol{\nu_{137}}}{122}}+{\frac {327\,\boldsymbol{\nu_{127}}}{244}}+{\frac {113\,\boldsymbol{\nu_{108}}}{61}}-{\frac {\boldsymbol{\nu_{132}}}{244
}}-{\frac {46\,\boldsymbol{\nu_{131}}}{61}}\\
&-3\,\boldsymbol{\nu_{79}}-1/2\,\boldsymbol{\nu_{80}}-{\frac 
{65\,\boldsymbol{\nu_{83}}}{244}}-1/2\,\boldsymbol{\nu_{51}}+{\frac {62\,\boldsymbol{\nu_{90}}}{61}}-5/
8\,\boldsymbol{\nu_{59}}+{\frac {7\,\boldsymbol{\nu_{9}}}{64}}\\
&-{\frac {9\,\boldsymbol{\nu_{48}}}{8}}-{
\frac {7\,\boldsymbol{\nu_{55}}}{8}}-5/8\,\boldsymbol{\nu_{58}}+{\frac {5\,\boldsymbol{\nu_{7}}}{256}}
\end{aligned}
\\
\addlinespace
\midrule
\addlinespace
\boldsymbol{\nu_{130}} & 
\begin{aligned}
&-{\frac {97\,\boldsymbol{\nu_{124}}}{61}}+{\frac {77\,\boldsymbol{\nu_{109}}}{61
}}+{\frac {84\,\boldsymbol{\nu_{137}}}{61}}-{\frac {67\,\boldsymbol{\nu_{127}}}{122}}-{
\frac {106\,\boldsymbol{\nu_{108}}}{61}}-{\frac {33\,\boldsymbol{\nu_{132}}}{122}}\\
&+{\frac {
14\,\boldsymbol{\nu_{131}}}{61}}+6\,\boldsymbol{\nu_{79}}+\boldsymbol{\nu_{80}}
-{\frac {71\,\boldsymbol{\nu_{83}}
}{122}}+\boldsymbol{\nu_{51}}-{\frac {56\,\boldsymbol{\nu_{90}}}{61}}+5/4\,\boldsymbol{\nu_{59}}-{
\frac {7\,\boldsymbol{\nu_{9}}}{32}}\\
&+9/4\,\boldsymbol{\nu_{48}}+7/4\,\boldsymbol{\nu_{55}}+5/4\,\boldsymbol{\nu_{58}}-{\frac {5\,\boldsymbol{\nu_{7}}}{128}}
\end{aligned}
\\
\addlinespace
\midrule
\addlinespace
\boldsymbol{\nu_{133}} & 
\begin{aligned}
&-\boldsymbol{\nu_{119}}+{\frac {93\,
\boldsymbol{\nu_{124}}}{122}}+{\frac {20\,\boldsymbol{\nu_{109}}}{61}}-{\frac {41\,\boldsymbol{\nu_{137}}}{366}}+{\frac {4\,\boldsymbol{\nu_{127}}}{183}}+{\frac {20\,\boldsymbol{\nu_{108}}}{61}}-
{\frac {161\,\boldsymbol{\nu_{132}}}{183}}\\
&-{\frac {161\,\boldsymbol{\nu_{131}}}{183}}+1/16\,
\boldsymbol{\nu_{11}}+{\frac {149\,\boldsymbol{\nu_{83}}}{183}}+{\frac {251\,\boldsymbol{\nu_{90}}}{366
}}+{\frac {7\,\boldsymbol{\nu_{59}}}{12}}+{\frac {17\,\boldsymbol{\nu_{9}}}{96}}+{\frac {7\,
\boldsymbol{\nu_{48}}}{12}}\\
&+{\frac {5\,\boldsymbol{\nu_{55}}}{12}}+{\frac {7\,\boldsymbol{\nu_{58}}}{12
}}-{\frac {29\,\boldsymbol{\nu_{7}}}{384}}
\end{aligned}
\\
\addlinespace
\midrule
\addlinespace
\boldsymbol{\nu_{134}} & -{\frac {26\,\boldsymbol{\nu_{124}}}{
61}}-{\frac {2\,\boldsymbol{\nu_{109}}}{61}}+{\frac {20\,\boldsymbol{\nu_{137}}}{61}}+{
\frac {8\,\boldsymbol{\nu_{127}}}{61}}-{\frac {2\,\boldsymbol{\nu_{108}}}{61}}+{\frac {44\,
\boldsymbol{\nu_{132}}}{61}}+{\frac {44\,\boldsymbol{\nu_{131}}}{61}}-{\frac {7\,\boldsymbol{\nu_{83}}
}{61}}+{\frac {7\,\boldsymbol{\nu_{90}}}{61}}
\\
\addlinespace
\midrule
\addlinespace
\boldsymbol{\nu_{136}} & 
\begin{aligned}
&\hphantom{+}1/2\,\boldsymbol{\nu_{135}}+{
\frac {7\,\boldsymbol{\nu_{124}}}{61}}+{\frac {24\,\boldsymbol{\nu_{109}}}{61}}-{\frac {179
\,\boldsymbol{\nu_{137}}}{61}}-{\frac {9\,\boldsymbol{\nu_{127}}}{122}}+{\frac {24\,\boldsymbol{\nu_{108}}}{61}}-{\frac {19\,\boldsymbol{\nu_{132}}}{122}}\\
&-{\frac {19\,\boldsymbol{\nu_{131}}}{
122}}-{\frac {15\,\boldsymbol{\nu_{83}}}{122}}+{\frac {15\,\boldsymbol{\nu_{90}}}{122}}
\end{aligned}
\\
\addlinespace
\midrule
\addlinespace
\boldsymbol{\nu_{138}} & 
\begin{aligned} 
&-\boldsymbol{\nu_{135}}-{\frac {47\,\boldsymbol{\nu_{124}}}{61}}-{\frac {13\,\boldsymbol{\nu_{109}}}{61}}+{\frac {252\,\boldsymbol{\nu_{137}}}{61}}-{\frac {79\,\boldsymbol{\nu_{127}}}{
122}}-{\frac {74\,\boldsymbol{\nu_{108}}}{61}}+{\frac {23\,\boldsymbol{\nu_{132}}}{122}}\\
&-{
\frac {19\,\boldsymbol{\nu_{131}}}{61}}+{\frac {153\,\boldsymbol{\nu_{83}}}{122}}+{\frac {15
\,\boldsymbol{\nu_{90}}}{61}}+3/4\,\boldsymbol{\nu_{59}}+{\frac {3\,\boldsymbol{\nu_{9}}}{32}}+3/4\,\boldsymbol{\nu_{48}}+1/4\,\boldsymbol{\nu_{55}}\\
&+3/4\,\boldsymbol{\nu_{58}}-{\frac {3\,\boldsymbol{\nu_{7}}}{128}}
\end{aligned}
\\
\addlinespace
\midrule
\addlinespace
\boldsymbol{\nu_{140}} & 
\begin{aligned} 
&\hphantom{+}\boldsymbol{\nu_{135}}+{\frac {44\,\boldsymbol{\nu_{124}}}{61}}+{\frac {55\,\boldsymbol{\nu_{109}}}{61}}-{\frac {428\,\boldsymbol{\nu_{137}}}{61}}+{\frac {109\,\boldsymbol{\nu_{127}}
}{122}}+{\frac {116\,\boldsymbol{\nu_{108}}}{61}}-{\frac {41\,\boldsymbol{\nu_{132}}}{122}}\\
&+
{\frac {10\,\boldsymbol{\nu_{131}}}{61}}-{\frac {225\,\boldsymbol{\nu_{83}}}{122}}+{\frac {
21\,\boldsymbol{\nu_{90}}}{61}}-3/4\,\boldsymbol{\nu_{59}}-{\frac {3\,\boldsymbol{\nu_{9}}}{32}}-3/4\,
\boldsymbol{\nu_{48}}-1/4\,\boldsymbol{\nu_{55}}\\
&-3/4\,\boldsymbol{\nu_{58}}+{\frac {3\,\boldsymbol{\nu_{7}}}{128}}
\end{aligned}
\\
\addlinespace
\midrule
\addlinespace
\boldsymbol{\nu_{141}} & 
\begin{aligned}
&-\boldsymbol{\nu_{135}}-{\frac {45\,\boldsymbol{\nu_{124}}}{61}}-{\frac {41\,
\boldsymbol{\nu_{109}}}{61}}+{\frac {349\,\boldsymbol{\nu_{137}}}{61}}-{\frac {99\,\boldsymbol{\nu_{127}}}{122}}-{\frac {102\,\boldsymbol{\nu_{108}}}{61}}+{\frac {35\,\boldsymbol{\nu_{132}}}{122}
}\\
&-{\frac {13\,\boldsymbol{\nu_{131}}}{61}}+{\frac {201\,\boldsymbol{\nu_{83}}}{122}}-{\frac 
{9\,\boldsymbol{\nu_{90}}}{61}}+3/4\,\boldsymbol{\nu_{59}}+{\frac {3\,\boldsymbol{\nu_{9}}}{32}}+3/4\,
\boldsymbol{\nu_{48}}+1/4\,\boldsymbol{\nu_{55}}\\
&+3/4\,\boldsymbol{\nu_{58}}-{\frac {3\,\boldsymbol{\nu_{7}}}{128}}
\end{aligned}
\\
\addlinespace
\midrule
\addlinespace
\boldsymbol{\nu_{142}} & 
\begin{aligned}
&\hphantom{+}{\frac {22\,\boldsymbol{\nu_{124}}}{61}}+{\frac {58\,\boldsymbol{\nu_{109}}}{61
}}-{\frac {214\,\boldsymbol{\nu_{137}}}{61}}+{\frac {12\,\boldsymbol{\nu_{127}}}{61}}+{
\frac {58\,\boldsymbol{\nu_{108}}}{61}}+{\frac {5\,\boldsymbol{\nu_{132}}}{61}}+{\frac {5\,
\boldsymbol{\nu_{131}}}{61}}\\
&-{\frac {41\,\boldsymbol{\nu_{83}}}{61}}+{\frac {41\,\boldsymbol{\nu_{90}}
}{61}}
\end{aligned}
\\
\addlinespace
\midrule
\addlinespace
\boldsymbol{\nu_{143}} & 
\begin{aligned}
&\hphantom{+}{\frac {53\,\boldsymbol{\nu_{124}}}{732}}-{\frac {193\,\boldsymbol{\nu_{109}}}{732}}+{\frac {283\,\boldsymbol{\nu_{137}}}{732}}+{\frac {169\,\boldsymbol{\nu_{127}}
}{488}}-{\frac {5\,\boldsymbol{\nu_{108}}}{366}}+{\frac {45\,\boldsymbol{\nu_{132}}}{488}}-{
\frac {2\,\boldsymbol{\nu_{131}}}{61}}\\
&-1/2\,\boldsymbol{\nu_{79}}-1/12\,\boldsymbol{\nu_{80}}+{\frac {
235\,\boldsymbol{\nu_{83}}}{1464}}-1/6\,\boldsymbol{\nu_{51}}-{\frac {13\,\boldsymbol{\nu_{90}}}{366}}-
3/16\,\boldsymbol{\nu_{59}}+{\frac {7\,\boldsymbol{\nu_{9}}}{384}}\\
&-3/16\,\boldsymbol{\nu_{48}}-{\frac {
11\,\boldsymbol{\nu_{55}}}{48}}-1/16\,\boldsymbol{\nu_{58}}+{\frac {13\,\boldsymbol{\nu_{7}}}{1536}}
\end{aligned}
\\
\addlinespace
\midrule
\addlinespace
\boldsymbol{\nu_{144}} & 
\begin{aligned}
&-{\frac {233\,\boldsymbol{\nu_{108}}}{1464}}-1/4\,\boldsymbol{\nu_{69}}-{\frac {
83\,\boldsymbol{\nu_{7}}}{6144}}-{\frac {385\,\boldsymbol{\nu_{124}}}{2928}}+{\frac {185\,
\boldsymbol{\nu_{131}}}{1464}}+{\frac {923\,\boldsymbol{\nu_{132}}}{5856}}\\
&-{\frac {481\,\boldsymbol{\nu_{127}}}{5856}}+1/8\,\boldsymbol{\nu_{79}}+1/16\,\boldsymbol{\nu_{51}}+{\frac {43\,\boldsymbol{\nu_{48}}}{192}}+{\frac {7\,\boldsymbol{\nu_{55}}}{64}}+{\frac {17\,\boldsymbol{\nu_{59}}}{64}}-1/48
\,\boldsymbol{\nu_{80}}\\
&-{\frac {29\,\boldsymbol{\nu_{83}}}{5856}}-{\frac {443\,\boldsymbol{\nu_{90}}}{
2928}}+{\frac {83\,\boldsymbol{\nu_{109}}}{2928}}+{\frac {435\,\boldsymbol{\nu_{137}}}{976}}
+{\frac {17\,\boldsymbol{\nu_{58}}}{64}}+{\frac {\boldsymbol{\nu_{9}}}{512}}
\end{aligned}
\\
\addlinespace
\midrule
\addlinespace
\boldsymbol{\nu_{145}} & 
\begin{aligned}
&-1/48\,\boldsymbol{\nu_{149}}+1/4\,\boldsymbol{\nu_{151}}+1/4\,\boldsymbol{\nu_{153}}+1/6\,\boldsymbol{\nu_{155}}-{
\frac {13\,\boldsymbol{\nu_{160}}}{48}}+2/9\,\boldsymbol{\nu_{164}}\\
&+{\frac {35\,\boldsymbol{\nu_{165}}
}{144}}+1/4\,\boldsymbol{\nu_{175}}+1/24\,\boldsymbol{\nu_{178}}+1/16\,\boldsymbol{\nu_{181}}+1/12\,\boldsymbol{\nu_{202}}-1/24\,\boldsymbol{\nu_{204}}\\
&+1/24\,\boldsymbol{\nu_{206}}-{\frac {\boldsymbol{\nu_{20}}}{96}}+
{\frac {419\,\boldsymbol{\nu_{7}}}{18432}}+{\frac {575\,\boldsymbol{\nu_{9}}}{4608}}-{\frac 
{49\,\boldsymbol{\nu_{48}}}{1152}}-{\frac {29\,\boldsymbol{\nu_{51}}}{192}}-{\frac {23\,\boldsymbol{\nu_{55}}}{384}}+\\
&{\frac {161\,\boldsymbol{\nu_{58}}}{288}}+{\frac {161\,\boldsymbol{\nu_{59}}}{
288}}-{\frac {13\,\boldsymbol{\nu_{69}}}{48}}+1/24\,\boldsymbol{\nu_{79}}+{\frac {95\,\boldsymbol{\nu_{80}}}{576}}+{\frac {1727\,\boldsymbol{\nu_{83}}}{732}}\\
&+{\frac {12323\,\boldsymbol{\nu_{90}}
}{11712}}-{\frac {11\,\boldsymbol{\nu_{108}}}{4392}}+{\frac {161\,\boldsymbol{\nu_{109}}}{
8784}}-{\frac {103\,\boldsymbol{\nu_{124}}}{8784}}-{\frac {7\,\boldsymbol{\nu_{127}}}{17568}
}+{\frac {59\,\boldsymbol{\nu_{131}}}{4392}}\\
&+{\frac {53\,\boldsymbol{\nu_{132}}}{17568}}-{
\frac {23\,\boldsymbol{\nu_{137}}}{976}}
\end{aligned}
\\
\addlinespace
\midrule
\addlinespace
\boldsymbol{\nu_{146}} & 
\begin{aligned}
&-1/2\,\boldsymbol{\nu_{175}}-{\frac {11
\,\boldsymbol{\nu_{164}}}{72}}-1/3\,\boldsymbol{\nu_{155}}-1/6\,\boldsymbol{\nu_{69}}-{\frac {31\,\boldsymbol{\nu_{7}}}{1152}}+{\frac {7\,\boldsymbol{\nu_{51}}}{72}}+{\frac {7\,\boldsymbol{\nu_{48}}}{144}}\\
&+
{\frac {7\,\boldsymbol{\nu_{55}}}{144}}-{\frac {5\,\boldsymbol{\nu_{59}}}{12}}-{\frac {7\,
\boldsymbol{\nu_{80}}}{72}}+1/3\,\boldsymbol{\nu_{149}}-{\frac {83\,\boldsymbol{\nu_{83}}}{36}}-{\frac 
{89\,\boldsymbol{\nu_{90}}}{72}}-{\frac {\boldsymbol{\nu_{20}}}{64}}-{\frac {5\,\boldsymbol{\nu_{58}}}{
12}}\\
&-{\frac {7\,\boldsymbol{\nu_{165}}}{36}}-{\frac {77\,\boldsymbol{\nu_{9}}}{576}}
\end{aligned}
\\
\addlinespace
\midrule
\addlinespace
\boldsymbol{\nu_{147}} & 
\begin{aligned}
&\hphantom{+}1/2\,\boldsymbol{\nu_{69}}-{\frac {67\,\boldsymbol{\nu_{7}}}{1536}}-2\,\boldsymbol{\nu_{151}}-\boldsymbol{\nu_{153}}+{\frac {5\,\boldsymbol{\nu_{51}}}{24}}+{\frac {5\,\boldsymbol{\nu_{48}}}{48}}+{\frac {5\,\boldsymbol{\nu_{55}}}{48}}-{\frac {47\,\boldsymbol{\nu_{59}}}{48}}\\
&-{\frac {5\,\nu
_{{80}}}{24}}+\boldsymbol{\nu_{160}}-4\,\boldsymbol{\nu_{83}}-7/4\,\boldsymbol{\nu_{90}}+1/16\,\boldsymbol{\nu_{20}}-{\frac {47\,\boldsymbol{\nu_{58}}}{48}}-3/16\,\boldsymbol{\nu_{9}}
\end{aligned}
\\
\addlinespace
\midrule
\addlinespace
\boldsymbol{\nu_{148}} & 
\begin{aligned}
&-{\frac {
\boldsymbol{\nu_{7}}}{256}}+1/4\,\boldsymbol{\nu_{51}}+1/8\,\boldsymbol{\nu_{48}}+1/8\,\boldsymbol{\nu_{55}}+5/8\,
\boldsymbol{\nu_{59}}-1/4\,\boldsymbol{\nu_{80}}-\boldsymbol{\nu_{149}}\\
&+3/2\,\boldsymbol{\nu_{83}}+3/4\,\boldsymbol{\nu_{90}}
+1/16\,\boldsymbol{\nu_{20}}+5/8\,\boldsymbol{\nu_{58}}+1/8\,\boldsymbol{\nu_{9}}
\end{aligned}
\\
\addlinespace
\midrule
\addlinespace
\boldsymbol{\nu_{150}} & 
\begin{aligned}
&\hphantom{+}1/4\,\boldsymbol{\nu_{69}}-{\frac {5\,\boldsymbol{\nu_{7}}}{1536}}+1/24\,\boldsymbol{\nu_{51}}+{\frac {5\,\boldsymbol{\nu_{48}}}{24}}+1/12\,\boldsymbol{\nu_{55}}+{\frac {11\,\boldsymbol{\nu_{59}}}{48}}-1/24\,\boldsymbol{\nu_{80}}\\
&+5/8\,\boldsymbol{\nu_{83}}+{\frac {7\,\boldsymbol{\nu_{90}}}{16}}+1/32\,\boldsymbol{\nu_{20}}+{
\frac {11\,\boldsymbol{\nu_{58}}}{48}}+1/16\,\boldsymbol{\nu_{9}}
\end{aligned}
\\
\addlinespace
\midrule
\addlinespace
\boldsymbol{\nu_{152}} & 
\begin{aligned}
&\hphantom{+}\boldsymbol{\nu_{149}}-
\boldsymbol{\nu_{165}}-\boldsymbol{\nu_{164}}-1/8\,\boldsymbol{\nu_{20}}-5\,\boldsymbol{\nu_{83}}-5/2\,\boldsymbol{\nu_{90}}-
5/4\,\boldsymbol{\nu_{59}}-3/8\,\boldsymbol{\nu_{9}}\\
&-5/4\,\boldsymbol{\nu_{58}}-{\frac {3\,\boldsymbol{\nu_{7}}}{
128}}
\end{aligned}
\\
\addlinespace
\midrule
\addlinespace
\boldsymbol{\nu_{154}} & 
\begin{aligned}
&-1/2\,\boldsymbol{\nu_{155}}+5/6\,\boldsymbol{\nu_{160}}-{\frac {13\,\boldsymbol{\nu_{164}}}{16}}-3/4\,\boldsymbol{\nu_{165}}-1/4\,\boldsymbol{\nu_{181}}-1/2\,\boldsymbol{\nu_{202}}\\
&+1/4\,
\boldsymbol{\nu_{204}}-1/4\,\boldsymbol{\nu_{206}}-{\frac {221\,\boldsymbol{\nu_{7}}}{6144}}-{\frac {
545\,\boldsymbol{\nu_{9}}}{1536}}-{\frac {19\,\boldsymbol{\nu_{48}}}{192}}+{\frac {11\,\boldsymbol{\nu_{51}}}{48}}+{\frac {\boldsymbol{\nu_{55}}}{64}}\\
&-{\frac {343\,\boldsymbol{\nu_{58}}}{192}}-{
\frac {343\,\boldsymbol{\nu_{59}}}{192}}+\boldsymbol{\nu_{69}}-1/4\,\boldsymbol{\nu_{79}}-{\frac {5\,
\boldsymbol{\nu_{80}}}{16}}-{\frac {4445\,\boldsymbol{\nu_{83}}}{732}}\\
&-{\frac {15131\,\boldsymbol{\nu_{90}}}{5856}}+{\frac {11\,\boldsymbol{\nu_{108}}}{732}}-{\frac {161\,\boldsymbol{\nu_{109}}}{
1464}}+{\frac {103\,\boldsymbol{\nu_{124}}}{1464}}+{\frac {7\,\boldsymbol{\nu_{127}}}{2928}}
-{\frac {59\,\boldsymbol{\nu_{131}}}{732}}\\
&-{\frac {53\,\boldsymbol{\nu_{132}}}{2928}}+{
\frac {69\,\boldsymbol{\nu_{137}}}{488}}
\end{aligned}
\\
\addlinespace
\midrule
\addlinespace
\boldsymbol{\nu_{156}} & {\frac {\boldsymbol{\nu_{7}}}{256}}+3/8
\,\boldsymbol{\nu_{59}}-\boldsymbol{\nu_{160}}+3/2\,\boldsymbol{\nu_{83}}+3/4\,\boldsymbol{\nu_{90}}+3/8\,\boldsymbol{\nu_{58}}+1/8\,\boldsymbol{\nu_{9}}
\\
\addlinespace
\midrule
\addlinespace
\boldsymbol{\nu_{157}} & 
\begin{aligned}
&\hphantom{+}\boldsymbol{\nu_{164}}+{\frac {11\,\boldsymbol{\nu_{7}}}{128}}
-1/2\,\boldsymbol{\nu_{51}}-1/4\,\boldsymbol{\nu_{48}}-1/4\,\boldsymbol{\nu_{55}}+5/4\,\boldsymbol{\nu_{59}}+1/2\,
\boldsymbol{\nu_{80}}+7\,\boldsymbol{\nu_{83}}\\
&+7/2\,\boldsymbol{\nu_{90}}+5/4\,\boldsymbol{\nu_{58}}+\boldsymbol{\nu_{165}}+3
/8\,\boldsymbol{\nu_{9}}
\end{aligned}
\\
\addlinespace
\midrule
\addlinespace
\boldsymbol{\nu_{158}} & 
\begin{aligned}
&-\boldsymbol{\nu_{165}}-\boldsymbol{\nu_{164}}-\boldsymbol{\nu_{80}}-12\,\boldsymbol{\nu_{83}}+\boldsymbol{\nu_{51}}-6\,\boldsymbol{\nu_{90}}-2\,\boldsymbol{\nu_{59}}-5/8\,\boldsymbol{\nu_{9}}+1/2\,\boldsymbol{\nu_{48}}\\
&+1/2\,\boldsymbol{\nu_{55}}-2\,\boldsymbol{\nu_{58}}-{\frac {5\,\boldsymbol{\nu_{7}}}{32}}
 \end{aligned}
\\
\addlinespace
\midrule
\addlinespace
\boldsymbol{\nu_{159}} & 
\begin{aligned}
&-{\frac {11\,\boldsymbol{\nu_{7}}}{512}}+1/4\,\boldsymbol{\nu_{51}}-1/16\,\boldsymbol{\nu_{48}}+1
/16\,\boldsymbol{\nu_{55}}-{\frac {5\,\boldsymbol{\nu_{59}}}{16}}-1/4\,\boldsymbol{\nu_{80}}-7/4\,\boldsymbol{\nu_{83}}\\
&-{\frac {7\,\boldsymbol{\nu_{90}}}{8}}-{\frac {5\,\boldsymbol{\nu_{58}}}{16}}-{\frac {
11\,\boldsymbol{\nu_{9}}}{128}}
\end{aligned}
\\
\addlinespace
\midrule
\addlinespace
\boldsymbol{\nu_{161}} & 
\begin{aligned}
&-\boldsymbol{\nu_{164}}-{\frac {11\,\boldsymbol{\nu_{7}}}{
128}}+1/2\,\boldsymbol{\nu_{51}}+1/4\,\boldsymbol{\nu_{48}}+1/4\,\boldsymbol{\nu_{55}}-3/2\,\boldsymbol{\nu_{59}}-
1/2\,\boldsymbol{\nu_{80}}-8\,\boldsymbol{\nu_{83}}\\
&-4\,\boldsymbol{\nu_{90}}-3/2\,\boldsymbol{\nu_{58}}-\boldsymbol{\nu_{165}}-{\frac {15\,\boldsymbol{\nu_{9}}}{32}}
\end{aligned}
\\
\addlinespace
\midrule
\addlinespace
\boldsymbol{\nu_{162}} & 
\begin{aligned}
&\hphantom{+}\boldsymbol{\nu_{165}}+\boldsymbol{\nu_{164}}+1/2
\,\boldsymbol{\nu_{80}}+12\,\boldsymbol{\nu_{83}}-1/2\,\boldsymbol{\nu_{51}}+6\,\boldsymbol{\nu_{90}}+5/2\,\boldsymbol{\nu_{59}}+{\frac {11\,\boldsymbol{\nu_{9}}}{16}}\\
&-1/4\,\boldsymbol{\nu_{48}}-1/4\,\boldsymbol{\nu_{55}}+5/2\,
\boldsymbol{\nu_{58}}+1/8\,\boldsymbol{\nu_{7}}
\end{aligned}
\\
\addlinespace
\midrule
\addlinespace
\boldsymbol{\nu_{163}} & 
13/2\,\boldsymbol{\nu_{83}}+{\frac {13\,\boldsymbol{\nu_{90}}}{4}}+{\frac {13\,\boldsymbol{\nu_{59}}}{8}}+{\frac {13\,\boldsymbol{\nu_{58}}}{8}}+{
\frac {15\,\boldsymbol{\nu_{7}}}{256}}-\boldsymbol{\nu_{160}}+3/8\,\boldsymbol{\nu_{9}}
\\
\addlinespace
\midrule
\addlinespace
\boldsymbol{\nu_{166}}&-
\boldsymbol{\nu_{165}}-\boldsymbol{\nu_{164}}+4\,\boldsymbol{\nu_{69}}+2\,\boldsymbol{\nu_{83}}+3\,\boldsymbol{\nu_{90}}-\boldsymbol{\nu_{59}}+1/8\,\boldsymbol{\nu_{9}}-\boldsymbol{\nu_{58}}+{\frac {3\,\boldsymbol{\nu_{7}}}{32}}
\\
\addlinespace
\midrule
\addlinespace
\nu_{{167}
}& -\boldsymbol{\nu_{165}}
\\
\addlinespace
\midrule
\addlinespace
\boldsymbol{\nu_{168}} & 2\,\boldsymbol{\nu_{69}}+\boldsymbol{\nu_{83}}+3/2\,\boldsymbol{\nu_{90}}-1/
2\,\boldsymbol{\nu_{59}}+1/16\,\boldsymbol{\nu_{9}}-1/2\,\boldsymbol{\nu_{58}}+{\frac {3\,\boldsymbol{\nu_{7}}}{64
}}
\\
\addlinespace
\midrule
\addlinespace
\boldsymbol{\nu_{169}} &
\begin{aligned}
&\hphantom{+}\boldsymbol{\nu_{80}}+2\,\boldsymbol{\nu_{83}}-\boldsymbol{\nu_{51}}+\boldsymbol{\nu_{90}}-1/2\,\boldsymbol{\nu_{59}}+1/8\,\boldsymbol{\nu_{9}}-1/2\,\boldsymbol{\nu_{48}}-1/2\,\boldsymbol{\nu_{55}}\\
&-1/2\,\boldsymbol{\nu_{58}}
+{\frac {3\,\boldsymbol{\nu_{7}}}{64}}
\end{aligned}
\\
\addlinespace
\midrule
\addlinespace
\boldsymbol{\nu_{170}} & {\frac {7\,\boldsymbol{\nu_{7}}}{1024}}+{
\frac {9\,\boldsymbol{\nu_{48}}}{32}}+{\frac {3\,\boldsymbol{\nu_{55}}}{32}}+{\frac {21\,\boldsymbol{\nu_{59}}}{32}}+{\frac {15\,\boldsymbol{\nu_{83}}}{8}}+{\frac {15\,\boldsymbol{\nu_{90}}}{16}}
+{\frac {21\,\boldsymbol{\nu_{58}}}{32}}+{\frac {29\,\boldsymbol{\nu_{9}}}{256}}
\\
\addlinespace
\midrule
\addlinespace
\boldsymbol{\nu_{171}} & 
\begin{aligned}
&\hphantom{+}\boldsymbol{\nu_{165}}+\boldsymbol{\nu_{164}}+\boldsymbol{\nu_{80}}+2\,\boldsymbol{\nu_{83}}-\boldsymbol{\nu_{51}}+\boldsymbol{\nu_{90}}-1/2\,\boldsymbol{\nu_{59}}\\
&+1/8\,\boldsymbol{\nu_{9}}-1/2\,\boldsymbol{\nu_{48}}-1/2\,\boldsymbol{\nu_{55}}-1/2\,
\boldsymbol{\nu_{58}}+{\frac {3\,\boldsymbol{\nu_{7}}}{64}}
\end{aligned}
\\
\addlinespace
\midrule
\addlinespace
\boldsymbol{\nu_{172}} & 
\begin{aligned}
&\hphantom{+}  {\frac {\boldsymbol{\nu_{7}}}{
64}}+1/4\,\boldsymbol{\nu_{51}}-1/4\,\boldsymbol{\nu_{48}}+1/4\,\boldsymbol{\nu_{59}}-1/4\,\boldsymbol{\nu_{80}}+
\boldsymbol{\nu_{83}}+1/2\,\boldsymbol{\nu_{90}}+1/4\,\boldsymbol{\nu_{58}}\\
&+{\frac {3\,\boldsymbol{\nu_{9}}}{64}}
\end{aligned}
\\
\addlinespace
\midrule
\addlinespace
\boldsymbol{\nu_{173}} & 
\begin{aligned}
&\hphantom{+}1/2\,\boldsymbol{\nu_{80}}-\boldsymbol{\nu_{83}}-1/2\,\boldsymbol{\nu_{51}}-1/2\,\boldsymbol{\nu_{90}}
-3/4\,\boldsymbol{\nu_{59}}-1/16\,\boldsymbol{\nu_{9}}-1/4\,\boldsymbol{\nu_{48}}\\
&-1/4\,\boldsymbol{\nu_{55}}-3/4\,
\boldsymbol{\nu_{58}}+{\frac {\boldsymbol{\nu_{7}}}{128}}
\end{aligned}
\\
\addlinespace
\midrule
\addlinespace
\boldsymbol{\nu_{174}} & 
\begin{aligned}
&\hphantom{+}1/24\,\boldsymbol{\nu_{149}}+1/6
\,\boldsymbol{\nu_{151}}-1/6\,\boldsymbol{\nu_{153}}-{\frac {11\,\boldsymbol{\nu_{160}}}{72}}+{\frac {7
\,\boldsymbol{\nu_{164}}}{72}}+{\frac {\boldsymbol{\nu_{165}}}{72}}-1/2\,\boldsymbol{\nu_{175}}\\
&-1/12\,
\boldsymbol{\nu_{178}}+1/8\,\boldsymbol{\nu_{181}}+1/6\,\boldsymbol{\nu_{202}}-1/12\,\boldsymbol{\nu_{204}}+1/12\,
\boldsymbol{\nu_{206}}-{\frac {\boldsymbol{\nu_{20}}}{192}}-{\frac {3\,\boldsymbol{\nu_{7}}}{2048}}\\
&+{
\frac {95\,\boldsymbol{\nu_{9}}}{2304}}+{\frac {13\,\boldsymbol{\nu_{48}}}{144}}+{\frac {\boldsymbol{\nu_{51}}}{144}}+1/24\,\boldsymbol{\nu_{55}}+{\frac {215\,\boldsymbol{\nu_{58}}}{576}}+{\frac 
{215\,\boldsymbol{\nu_{59}}}{576}}-1/3\,\boldsymbol{\nu_{69}}\\
&+1/12\,\boldsymbol{\nu_{79}}+1/48\,\boldsymbol{\nu_{80}}+{\frac {371\,\boldsymbol{\nu_{83}}}{488}}+{\frac {1343\,\boldsymbol{\nu_{90}}}{5856}}-{
\frac {11\,\boldsymbol{\nu_{108}}}{2196}}+{\frac {161\,\boldsymbol{\nu_{109}}}{4392}}\\
&-{
\frac {103\,\boldsymbol{\nu_{124}}}{4392}}-{\frac {7\,\boldsymbol{\nu_{127}}}{8784}}+{\frac 
{59\,\boldsymbol{\nu_{131}}}{2196}}+{\frac {53\,\boldsymbol{\nu_{132}}}{8784}}-{\frac {23\,
\boldsymbol{\nu_{137}}}{488}}
\end{aligned}
\\
\addlinespace
\midrule
\addlinespace
\boldsymbol{\nu_{176}} & 
\begin{aligned}
&-{\frac {\boldsymbol{\nu_{7}}}{64}}-1/16\,\boldsymbol{\nu_{51}}-1/8\,\boldsymbol{\nu_{48}}-1/16\,\boldsymbol{\nu_{55}}-{\frac {11\,\boldsymbol{\nu_{59}}}{16}}+1/16
\,\boldsymbol{\nu_{80}}+1/2\,\boldsymbol{\nu_{160}}\\
&-9/4\,\boldsymbol{\nu_{83}}-{\frac {9\,\boldsymbol{\nu_{90}}}{8
}}-1/2\,\boldsymbol{\nu_{178}}-1/2\,\boldsymbol{\nu_{181}}-{\frac {11\,\boldsymbol{\nu_{58}}}{16}}-{
\frac {33\,\boldsymbol{\nu_{9}}}{256}}
\end{aligned}
\\
\addlinespace
\midrule
\addlinespace
\boldsymbol{\nu_{177}} &
\begin{aligned}
&-\boldsymbol{\nu_{165}}-3/2\,\boldsymbol{\nu_{164}}+
2\,\boldsymbol{\nu_{69}}-1/2\,\boldsymbol{\nu_{80}}-3\,\boldsymbol{\nu_{83}}+1/2\,\boldsymbol{\nu_{51}}-1/2\,\boldsymbol{\nu_{90}}-\boldsymbol{\nu_{59}}\\
&-3/16\,\boldsymbol{\nu_{9}}+1/4\,\boldsymbol{\nu_{48}}+1/4\,\boldsymbol{\nu_{55}}-\boldsymbol{\nu_{58}}
\end{aligned}
\\
\addlinespace
\midrule
\addlinespace
\boldsymbol{\nu_{179}} &
\begin{aligned}
&-1/12\,\boldsymbol{\nu_{149}}-1/3\,\boldsymbol{\nu_{151}}+1/3\,\boldsymbol{\nu_{153}}
+1/2\,\boldsymbol{\nu_{155}}-1/36\,\boldsymbol{\nu_{160}}+{\frac {89\,\boldsymbol{\nu_{164}}}{144}}\\
&+{
\frac {13\,\boldsymbol{\nu_{165}}}{18}}+\boldsymbol{\nu_{175}}+1/6\,\boldsymbol{\nu_{178}}+1/6\,\boldsymbol{\nu_{202}}-1/12\,\boldsymbol{\nu_{204}}+1/12\,\boldsymbol{\nu_{206}}\\
&+1/24\,\boldsymbol{\nu_{20}}+{\frac {5\,
\boldsymbol{\nu_{7}}}{128}}+{\frac {623\,\boldsymbol{\nu_{9}}}{2304}}-{\frac {13\,\boldsymbol{\nu_{48}}
}{288}}-{\frac {19\,\boldsymbol{\nu_{51}}}{72}}-{\frac {3\,\boldsymbol{\nu_{55}}}{32}}+{
\frac {131\,\boldsymbol{\nu_{58}}}{144}}\\
&+{\frac {131\,\boldsymbol{\nu_{59}}}{144}}-1/12\,\boldsymbol{\nu_{69}}+1/12\,\boldsymbol{\nu_{79}}+{\frac {7\,\boldsymbol{\nu_{80}}}{24}}+{\frac {12779\,
\boldsymbol{\nu_{83}}}{2928}}+{\frac {3157\,\boldsymbol{\nu_{90}}}{1464}}\\
&-{\frac {11\,\boldsymbol{\nu_{108}}}{2196}}+{\frac {161\,\boldsymbol{\nu_{109}}}{4392}}-{\frac {103\,\boldsymbol{\nu_{124}}}{4392}}-{\frac {7\,\boldsymbol{\nu_{127}}}{8784}}+{\frac {59\,\boldsymbol{\nu_{131}}}{2196
}}+{\frac {53\,\boldsymbol{\nu_{132}}}{8784}}\\
&-{\frac {23\,\boldsymbol{\nu_{137}}}{488}}
\end{aligned}
\\
\addlinespace
\midrule
\addlinespace
\boldsymbol{\nu_{180}} & 
\begin{aligned}
&-2\,\boldsymbol{\nu_{175}}-\boldsymbol{\nu_{164}}-\boldsymbol{\nu_{155}}-{\frac {5\,\boldsymbol{\nu_{7}}}{
64}}+1/2\,\boldsymbol{\nu_{51}}+1/4\,\boldsymbol{\nu_{48}}+1/4\,\boldsymbol{\nu_{55}}-{\frac {11\,\boldsymbol{\nu_{59}}}{8}}\\
&-1/2\,\boldsymbol{\nu_{80}}-15/2\,\boldsymbol{\nu_{83}}-{\frac {15\,\boldsymbol{\nu_{90}}}{4}
}-1/16\,\boldsymbol{\nu_{20}}-{\frac {11\,\boldsymbol{\nu_{58}}}{8}}-\boldsymbol{\nu_{165}}-{\frac {29
\,\boldsymbol{\nu_{9}}}{64}}
\end{aligned}
\\
\addlinespace
\midrule
\addlinespace
\boldsymbol{\nu_{182}} & \boldsymbol{\nu_{164}}-4\,\boldsymbol{\nu_{69}}-2\,\boldsymbol{\nu_{83}}-
3\,\boldsymbol{\nu_{90}}+\boldsymbol{\nu_{59}}-1/8\,\boldsymbol{\nu_{9}}+\boldsymbol{\nu_{58}}-{\frac {3\,\boldsymbol{\nu_{7}}}{32}}
\\
\addlinespace
\midrule
\addlinespace
\boldsymbol{\nu_{183}} & -2\,\boldsymbol{\nu_{83}}-\boldsymbol{\nu_{90}}-1/2\,\boldsymbol{\nu_{59}}-1/8\,\boldsymbol{\nu_{9}}-1/2\,\boldsymbol{\nu_{58}}-{\frac {\boldsymbol{\nu_{7}}}{64}}
\\
\addlinespace
\midrule
\addlinespace
\boldsymbol{\nu_{184}} & 
\begin{aligned}
&-1/48\,\nu_
{{149}}+1/12\,\boldsymbol{\nu_{151}}+1/6\,\boldsymbol{\nu_{153}}+1/12\,\boldsymbol{\nu_{155}}+{\frac {5
\,\boldsymbol{\nu_{160}}}{144}}+{\frac {13\,\boldsymbol{\nu_{164}}}{192}}\\
&+{\frac {5\,\boldsymbol{\nu_{165}}}{48}}+1/4\,\boldsymbol{\nu_{175}}-1/48\,\boldsymbol{\nu_{181}}+1/6\,\boldsymbol{\nu_{194}}-1/24\,
\boldsymbol{\nu_{202}}-1/48\,\boldsymbol{\nu_{204}}\\
&+1/48\,\boldsymbol{\nu_{206}}+{\frac {119\,\boldsymbol{\nu_{7}}}{18432}}+{\frac {283\,\boldsymbol{\nu_{9}}}{9216}}-{\frac {\boldsymbol{\nu_{48}}}{1152}}-{
\frac {7\,\boldsymbol{\nu_{51}}}{576}}-{\frac {\boldsymbol{\nu_{55}}}{1152}}+{\frac {71\,\boldsymbol{\nu_{58}}}{576}}\\
&+{\frac {71\,\boldsymbol{\nu_{59}}}{576}}+1/48\,\boldsymbol{\nu_{69}}+1/32\,
\boldsymbol{\nu_{79}}+{\frac {13\,\boldsymbol{\nu_{80}}}{576}}+{\frac {13379\,\boldsymbol{\nu_{83}}}{
23424}}+{\frac {6995\,\boldsymbol{\nu_{90}}}{23424}}\\
&-{\frac {109\,\boldsymbol{\nu_{108}}}{
17568}}+{\frac {331\,\boldsymbol{\nu_{109}}}{35136}}-{\frac {37\,\boldsymbol{\nu_{124}}}{
3904}}-{\frac {25\,\boldsymbol{\nu_{127}}}{70272}}+{\frac {101\,\boldsymbol{\nu_{131}}}{8784
}}+{\frac {259\,\boldsymbol{\nu_{132}}}{70272}}\\
&+{\frac {5\,\boldsymbol{\nu_{137}}}{3904}}
\end{aligned}
\\
\addlinespace
\midrule
\addlinespace
\boldsymbol{\nu_{185}} & 
\begin{aligned}
&\hphantom{+}1/6\,\boldsymbol{\nu_{149}}-1/6\,\boldsymbol{\nu_{155}}-{\frac {89\,\boldsymbol{\nu_{164}}
}{288}}-{\frac {23\,\boldsymbol{\nu_{165}}}{72}}-1/2\,\boldsymbol{\nu_{175}}-1/3\,\boldsymbol{\nu_{194}}\\
&+1/24\,\boldsymbol{\nu_{206}}-{\frac {\boldsymbol{\nu_{20}}}{64}}-{\frac {77\,\boldsymbol{\nu_{7}}}{
4096}}-{\frac {1271\,\boldsymbol{\nu_{9}}}{9216}}-{\frac {23\,\boldsymbol{\nu_{48}}}{1152}}+
{\frac {\boldsymbol{\nu_{51}}}{72}}-{\frac {13\,\boldsymbol{\nu_{55}}}{1152}}-{\frac {217\,
\boldsymbol{\nu_{58}}}{384}}\\
&-{\frac {217\,\boldsymbol{\nu_{59}}}{384}}-1/12\,\boldsymbol{\nu_{79}}-1/24
\,\boldsymbol{\nu_{80}}-{\frac {13375\,\boldsymbol{\nu_{83}}}{5856}}-{\frac {13327\,\boldsymbol{\nu_{90}}}{11712}}+{\frac {17\,\boldsymbol{\nu_{108}}}{549}}\\
&-{\frac {47\,\boldsymbol{\nu_{109}}}{
4392}}+{\frac {121\,\boldsymbol{\nu_{124}}}{4392}}+{\frac {5\,\boldsymbol{\nu_{127}}}{4392}}
-{\frac {311\,\boldsymbol{\nu_{131}}}{8784}}-{\frac {8\,\boldsymbol{\nu_{132}}}{549}}-{
\frac {65\,\boldsymbol{\nu_{137}}}{976}}
\end{aligned}
\\
\addlinespace
\midrule
\addlinespace
\boldsymbol{\nu_{186}} & 
\begin{aligned}
&-\boldsymbol{\nu_{151}}-1/2\,\boldsymbol{\nu_{153}}-1/12\,\boldsymbol{\nu_{160}}+1/4\,\boldsymbol{\nu_{181}}+1/2\,\boldsymbol{\nu_{202}}+1/4\,\boldsymbol{\nu_{204}}\\
&+1/32\,\boldsymbol{\nu_{20}}+{\frac {25\,\boldsymbol{\nu_{7}}}{6144}}+{\frac {41\,\boldsymbol{\nu_{9}}
}{1536}}-{\frac {5\,\boldsymbol{\nu_{48}}}{64}}-{\frac {7\,\boldsymbol{\nu_{51}}}{48}}-{
\frac {17\,\boldsymbol{\nu_{55}}}{192}}-{\frac {41\,\boldsymbol{\nu_{58}}}{192}}-{\frac {41
\,\boldsymbol{\nu_{59}}}{192}}\\
&+1/4\,\boldsymbol{\nu_{69}}-1/8\,\boldsymbol{\nu_{79}}+{\frac {5\,\boldsymbol{\nu_{80}}}{48}}+{\frac {101\,\boldsymbol{\nu_{83}}}{1952}}+{\frac {163\,\boldsymbol{\nu_{90}}}{
976}}+{\frac {29\,\boldsymbol{\nu_{108}}}{488}}-{\frac {3\,\boldsymbol{\nu_{109}}}{976}}\\
&+{
\frac {127\,\boldsymbol{\nu_{124}}}{2928}}+{\frac {11\,\boldsymbol{\nu_{127}}}{5856}}-{
\frac {7\,\boldsymbol{\nu_{131}}}{122}}-{\frac {51\,\boldsymbol{\nu_{132}}}{1952}}-{\frac {
153\,\boldsymbol{\nu_{137}}}{976}}
\end{aligned}
\\
\addlinespace
\midrule
\addlinespace
\boldsymbol{\nu_{187}} & 
\begin{aligned}
&\hphantom{+}{\frac {17\,\boldsymbol{\nu_{164}}}{24}}-{
\frac {29\,\boldsymbol{\nu_{108}}}{244}}-\boldsymbol{\nu_{69}}-{\frac {\boldsymbol{\nu_{7}}}{128}}-{
\frac {127\,\boldsymbol{\nu_{124}}}{1464}}+{\frac {7\,\boldsymbol{\nu_{131}}}{61}}+{\frac {
51\,\boldsymbol{\nu_{132}}}{976}}\\
&-{\frac {11\,\boldsymbol{\nu_{127}}}{2928}}+1/4\,\boldsymbol{\nu_{79}}
+{\frac {5\,\boldsymbol{\nu_{51}}}{24}}+{\frac {5\,\boldsymbol{\nu_{48}}}{24}}+1/6\,\boldsymbol{\nu_{55}}+4/3\,\boldsymbol{\nu_{59}}-1/8\,\boldsymbol{\nu_{80}}\\
&-1/2\,\boldsymbol{\nu_{206}}-1/2\,\boldsymbol{\nu_{149}}+{
\frac {8359\,\boldsymbol{\nu_{83}}}{2928}}+{\frac {2621\,\boldsymbol{\nu_{90}}}{2928}}+{
\frac {3\,\boldsymbol{\nu_{109}}}{488}}+{\frac {153\,\boldsymbol{\nu_{137}}}{488}}\\
&+1/32\,\boldsymbol{\nu_{20}}+4/3\,\boldsymbol{\nu_{58}}+{\frac {7\,\boldsymbol{\nu_{165}}}{12}}+{\frac {35\,\boldsymbol{\nu_{9}}}{192}}
\end{aligned}
\\
\addlinespace
\midrule
\addlinespace
\boldsymbol{\nu_{188}} & 
\begin{aligned}
&-{\frac {17\,\boldsymbol{\nu_{164}}}{24}}-{\frac {65\,\boldsymbol{\nu_{108}}}{732}}+\boldsymbol{\nu_{69}}+{\frac {79\,\boldsymbol{\nu_{124}}}{1464}}-{\frac {17
\,\boldsymbol{\nu_{131}}}{366}}+{\frac {47\,\boldsymbol{\nu_{132}}}{2928}}+{\frac {\boldsymbol{\nu_{127}}}{976}}\\
&-1/4\,\boldsymbol{\nu_{79}}-{\frac {5\,\boldsymbol{\nu_{51}}}{24}}-{\frac {5\,\boldsymbol{\nu_{48}}}{24}}-1/6\,\boldsymbol{\nu_{55}}-{\frac {17\,\boldsymbol{\nu_{59}}}{12}}+1/8\,\boldsymbol{\nu_{80}}+1/2\,\boldsymbol{\nu_{206}}\\
&+1/2\,\boldsymbol{\nu_{149}}-{\frac {9023\,\boldsymbol{\nu_{83}}}{2928}}-
{\frac {3421\,\boldsymbol{\nu_{90}}}{2928}}-{\frac {313\,\boldsymbol{\nu_{109}}}{1464}}+{
\frac {291\,\boldsymbol{\nu_{137}}}{488}}-{\frac {17\,\boldsymbol{\nu_{58}}}{12}}\\
&-{\frac {7
\,\boldsymbol{\nu_{165}}}{12}}-{\frac {35\,\boldsymbol{\nu_{9}}}{192}}
\end{aligned}
\\
\addlinespace
\midrule
\addlinespace
\boldsymbol{\nu_{189}} & 
\begin{aligned}
&-{\frac {
123\,\boldsymbol{\nu_{108}}}{976}}+1/8\,\boldsymbol{\nu_{69}}+{\frac {11\,\boldsymbol{\nu_{7}}}{12288}}
-{\frac {17\,\boldsymbol{\nu_{124}}}{5856}}+{\frac {11\,\boldsymbol{\nu_{131}}}{488}}+{
\frac {149\,\boldsymbol{\nu_{132}}}{3904}}-{\frac {13\,\boldsymbol{\nu_{127}}}{11712}}\\
&-1/16
\,\boldsymbol{\nu_{79}}-1/32\,\boldsymbol{\nu_{51}}-{\frac {19\,\boldsymbol{\nu_{48}}}{384}}-{\frac {13
\,\boldsymbol{\nu_{55}}}{384}}-{\frac {43\,\boldsymbol{\nu_{59}}}{384}}+{\frac {\boldsymbol{\nu_{80}}}{
96}}+{\frac {19\,\boldsymbol{\nu_{83}}}{11712}}\\
&-{\frac {71\,\boldsymbol{\nu_{90}}}{1464}}-{
\frac {307\,\boldsymbol{\nu_{109}}}{1952}}+{\frac {1179\,\boldsymbol{\nu_{137}}}{1952}}+{
\frac {\boldsymbol{\nu_{20}}}{64}}-{\frac {43\,\boldsymbol{\nu_{58}}}{384}}+{\frac {7\,\boldsymbol{\nu_{9}}}{1024}}
\end{aligned}
\\
\addlinespace
\midrule
\addlinespace
\boldsymbol{\nu_{190}} &
\begin{aligned}
&\hphantom{+}{\frac {5\,\boldsymbol{\nu_{164}}}{576}}+1/16\,\boldsymbol{\nu_{69}}+{\frac {25\,\boldsymbol{\nu_{7}}}{18432}}+{\frac {5\,\boldsymbol{\nu_{151}}}{12}}+1/12\,
\boldsymbol{\nu_{153}}+{\frac {11\,\boldsymbol{\nu_{51}}}{288}}-{\frac {17\,\boldsymbol{\nu_{48}}}{1152
}}\\
&+{\frac {\boldsymbol{\nu_{55}}}{128}}+{\frac {5\,\boldsymbol{\nu_{59}}}{576}}-{\frac {11\,
\boldsymbol{\nu_{80}}}{288}}-1/16\,\boldsymbol{\nu_{206}}+{\frac {5\,\boldsymbol{\nu_{160}}}{144}}-1/48
\,\boldsymbol{\nu_{149}}+{\frac {19\,\boldsymbol{\nu_{83}}}{288}}\\
&+{\frac {37\,\boldsymbol{\nu_{90}}}{
576}}-1/24\,\boldsymbol{\nu_{178}}-1/8\,\boldsymbol{\nu_{181}}-{\frac {5\,\boldsymbol{\nu_{202}}}{24}}-
1/16\,\boldsymbol{\nu_{204}}+{\frac {\boldsymbol{\nu_{20}}}{384}}+{\frac {5\,\boldsymbol{\nu_{58}}}{576
}}\\
&-{\frac {\boldsymbol{\nu_{165}}}{144}}+{\frac {47\,\boldsymbol{\nu_{9}}}{9216}}
\end{aligned}
\\
\addlinespace
\midrule
\addlinespace
\boldsymbol{\nu_{191}} &
\begin{aligned}
&-{\frac {17\,\boldsymbol{\nu_{164}}}{24}}+{\frac {11\,\boldsymbol{\nu_{108}}}{732}}+\boldsymbol{\nu_{69}}+{\frac {\boldsymbol{\nu_{7}}}{64}}+{\frac {103\,\boldsymbol{\nu_{124}}}{1464}}-{\frac {
59\,\boldsymbol{\nu_{131}}}{732}}-{\frac {53\,\boldsymbol{\nu_{132}}}{2928}}\\
&+{\frac {7\,\boldsymbol{\nu_{127}}}{2928}}-1/4\,\boldsymbol{\nu_{79}}-{\frac {5\,\boldsymbol{\nu_{51}}}{24}}-{\frac {5
\,\boldsymbol{\nu_{48}}}{24}}-1/6\,\boldsymbol{\nu_{55}}-5/4\,\boldsymbol{\nu_{59}}+1/8\,\boldsymbol{\nu_{80}}\\
&+1/2
\,\boldsymbol{\nu_{206}}+1/2\,\boldsymbol{\nu_{149}}-{\frac {2409\,\boldsymbol{\nu_{83}}}{976}}-{\frac 
{763\,\boldsymbol{\nu_{90}}}{976}}-{\frac {161\,\boldsymbol{\nu_{109}}}{1464}}+{\frac {69\,
\boldsymbol{\nu_{137}}}{488}}\\
&-1/16\,\boldsymbol{\nu_{20}}-5/4\,\boldsymbol{\nu_{58}}-{\frac {7\,\boldsymbol{\nu_{165}}}{12}}-{\frac {35\,\boldsymbol{\nu_{9}}}{192}}
\end{aligned}
\\
\addlinespace
\midrule
\addlinespace
\boldsymbol{\nu_{192}} & 
\begin{aligned}
&-{\frac {5\,\boldsymbol{\nu_{164}}}{288}}-1/8\,\boldsymbol{\nu_{69}}-{\frac {23\,\boldsymbol{\nu_{7}}}{4608}}+1/6\,\boldsymbol{\nu_{151}}+1/3\,\boldsymbol{\nu_{153}}+{\frac {\boldsymbol{\nu_{51}}}{144}}+{\frac {5\,\boldsymbol{\nu_{48}}}{576}}+{\frac {\boldsymbol{\nu_{55}}}{192}}\\
&-1/36\,\boldsymbol{\nu_{59}}-{\frac {\boldsymbol{\nu_{80}}
}{144}}+1/8\,\boldsymbol{\nu_{206}}+{\frac {\boldsymbol{\nu_{160}}}{72}}+1/24\,\boldsymbol{\nu_{149}}-{
\frac {49\,\boldsymbol{\nu_{83}}}{144}}-{\frac {67\,\boldsymbol{\nu_{90}}}{288}}\\
&+1/12\,\boldsymbol{\nu_{178}}-1/12\,\boldsymbol{\nu_{202}}-1/8\,\boldsymbol{\nu_{204}}-{\frac {\boldsymbol{\nu_{20}}}{192}}-1/
36\,\boldsymbol{\nu_{58}}+{\frac {\boldsymbol{\nu_{165}}}{72}}-{\frac {107\,\boldsymbol{\nu_{9}}}{4608}
}
\end{aligned}
\\
\addlinespace
\midrule
\addlinespace
\boldsymbol{\nu_{193}} & 
\begin{aligned}
&-1/16\,\boldsymbol{\nu_{160}}+{\frac {11\,\boldsymbol{\nu_{164}}}{192}}+1/24\,
\boldsymbol{\nu_{165}}+1/8\,\boldsymbol{\nu_{178}}+1/32\,\boldsymbol{\nu_{181}}-1/2\,\boldsymbol{\nu_{194}}\\
&+1/16\,
\boldsymbol{\nu_{202}}+1/32\,\boldsymbol{\nu_{204}}-1/16\,\boldsymbol{\nu_{206}}+{\frac {11\,\boldsymbol{\nu_{7}}
}{16384}}+{\frac {289\,\boldsymbol{\nu_{9}}}{12288}}-{\frac {5\,\boldsymbol{\nu_{48}}}{1536}
}-{\frac {\boldsymbol{\nu_{51}}}{96}}\\
&-{\frac {5\,\boldsymbol{\nu_{55}}}{512}}+{\frac {211\,
\boldsymbol{\nu_{58}}}{1536}}+{\frac {211\,\boldsymbol{\nu_{59}}}{1536}}-1/8\,\boldsymbol{\nu_{69}}-{
\frac {3\,\boldsymbol{\nu_{79}}}{64}}-{\frac {\boldsymbol{\nu_{80}}}{192}}+{\frac {19061\,
\boldsymbol{\nu_{83}}}{46848}}\\
&+{\frac {2273\,\boldsymbol{\nu_{90}}}{23424}}-{\frac {499\,\boldsymbol{\nu_{108}}}{11712}}-{\frac {1547\,\boldsymbol{\nu_{109}}}{23424}}+{\frac {47\,\boldsymbol{\nu_{124}}}{7808}}-{\frac {7\,\boldsymbol{\nu_{127}}}{46848}}-{\frac {\boldsymbol{\nu_{131}}}{
5856}}\\
&+{\frac {541\,\boldsymbol{\nu_{132}}}{46848}}+{\frac {1761\,\boldsymbol{\nu_{137}}}{
7808}}
\end{aligned}
\\
\addlinespace
\midrule
\addlinespace
\boldsymbol{\nu_{195}} & 
\begin{aligned}
&\hphantom{+}{\frac {65\,\boldsymbol{\nu_{108}}}{976}}-{\frac {37\,\boldsymbol{\nu_{7}}}{4096}}-{\frac {79\,\boldsymbol{\nu_{124}}}{1952}}+{\frac {17\,\boldsymbol{\nu_{131}}}{488
}}-{\frac {47\,\boldsymbol{\nu_{132}}}{3904}}-{\frac {3\,\boldsymbol{\nu_{127}}}{3904}}+3/16
\,\boldsymbol{\nu_{79}}\\
&+{\frac {\boldsymbol{\nu_{48}}}{128}}+{\frac {3\,\boldsymbol{\nu_{55}}}{128}}-{
\frac {39\,\boldsymbol{\nu_{59}}}{128}}+1/16\,\boldsymbol{\nu_{80}}+1/4\,\boldsymbol{\nu_{160}}-{\frac 
{4763\,\boldsymbol{\nu_{83}}}{3904}}-{\frac {251\,\boldsymbol{\nu_{90}}}{488}}\\
&+{\frac {313\,
\boldsymbol{\nu_{109}}}{1952}}-{\frac {873\,\boldsymbol{\nu_{137}}}{1952}}-3/8\,\boldsymbol{\nu_{181}}-
3/4\,\boldsymbol{\nu_{202}}-3/8\,\boldsymbol{\nu_{204}}\\
&-{\frac {39\,\boldsymbol{\nu_{58}}}{128}}-{
\frac {73\,\boldsymbol{\nu_{9}}}{1024}}
\end{aligned}
\\
\addlinespace
\midrule
\addlinespace
\boldsymbol{\nu_{196}} & 
\begin{aligned}
&-{\frac {7\,\boldsymbol{\nu_{164}}}{16}}
+{\frac {19\,\boldsymbol{\nu_{108}}}{122}}+3/2\,\boldsymbol{\nu_{69}}+{\frac {63\,\boldsymbol{\nu_{7}}
}{2048}}+{\frac {3\,\boldsymbol{\nu_{124}}}{122}}-{\frac {25\,\boldsymbol{\nu_{131}}}{488}}-
{\frac {25\,\boldsymbol{\nu_{132}}}{488}}\\
&+{\frac {\boldsymbol{\nu_{127}}}{488}}-{\frac {3\,
\boldsymbol{\nu_{48}}}{64}}-{\frac {\boldsymbol{\nu_{55}}}{64}}-{\frac {39\,\boldsymbol{\nu_{59}}}{64}}
+3/4\,\boldsymbol{\nu_{206}}-{\frac {139\,\boldsymbol{\nu_{83}}}{976}}+{\frac {1559\,\boldsymbol{\nu_{90}}}{1952}}\\
&+{\frac {19\,\boldsymbol{\nu_{109}}}{122}}-{\frac {333\,\boldsymbol{\nu_{137}}}{
488}}-{\frac {39\,\boldsymbol{\nu_{58}}}{64}}-1/4\,\boldsymbol{\nu_{165}}-{\frac {3\,\boldsymbol{\nu_{9}}}{512}}
\end{aligned}
\\
\addlinespace
\midrule
\addlinespace
\boldsymbol{\nu_{197}} & 
\begin{aligned}
&\hphantom{+}{\frac {7\,\boldsymbol{\nu_{164}}}{16}}+{\frac {57\,\boldsymbol{\nu_{108}}}{122}}-3/2\,\boldsymbol{\nu_{69}}-{\frac {63\,\boldsymbol{\nu_{7}}}{2048}}+{\frac {9\,
\boldsymbol{\nu_{124}}}{122}}-{\frac {75\,\boldsymbol{\nu_{131}}}{488}}-{\frac {75\,\boldsymbol{\nu_{132}}}{488}}\\
&+{\frac {3\,\boldsymbol{\nu_{127}}}{488}}+{\frac {3\,\boldsymbol{\nu_{48}}}{64}}
+{\frac {\boldsymbol{\nu_{55}}}{64}}+{\frac {39\,\boldsymbol{\nu_{59}}}{64}}-3/4\,\nu_{{206}
}-{\frac {173\,\boldsymbol{\nu_{83}}}{976}}-{\frac {935\,\boldsymbol{\nu_{90}}}{1952}}\\
&+{
\frac {57\,\boldsymbol{\nu_{109}}}{122}}-{\frac {999\,\boldsymbol{\nu_{137}}}{488}}+{\frac {
39\,\boldsymbol{\nu_{58}}}{64}}+1/4\,\boldsymbol{\nu_{165}}+{\frac {3\,\boldsymbol{\nu_{9}}}{512}}
\end{aligned}
\\
\addlinespace
\midrule
\addlinespace
\boldsymbol{\nu_{198}} & 
{\frac {3\,\boldsymbol{\nu_{124}}}{61}}+{\frac {19\,\boldsymbol{\nu_{109}}}{61}}-{
\frac {333\,\boldsymbol{\nu_{137}}}{244}}+{\frac {\boldsymbol{\nu_{127}}}{244}}+{\frac {19\,
\boldsymbol{\nu_{108}}}{61}}-{\frac {25\,\boldsymbol{\nu_{132}}}{244}}-{\frac {25\,\boldsymbol{\nu_{131}}}{244}}-{\frac {39\,\boldsymbol{\nu_{83}}}{244}}+{\frac {39\,\boldsymbol{\nu_{90}}}{244}}
\\
\addlinespace
\midrule
\addlinespace
\boldsymbol{\nu_{199}} &
\begin{aligned}
&-{\frac {11\,\boldsymbol{\nu_{108}}}{1952}}-3/16\,\boldsymbol{\nu_{69}}-{\frac {
33\,\boldsymbol{\nu_{7}}}{8192}}-{\frac {103\,\boldsymbol{\nu_{124}}}{3904}}+{\frac {59\,\boldsymbol{\nu_{131}}}{1952}}+{\frac {53\,\boldsymbol{\nu_{132}}}{7808}}-{\frac {7\,\boldsymbol{\nu_{127}}}{7808}}\\
&+{\frac {3\,\boldsymbol{\nu_{79}}}{32}}+1/16\,\boldsymbol{\nu_{51}}+{\frac {33\,\boldsymbol{\nu_{48}}}{256}}+{\frac {19\,\boldsymbol{\nu_{55}}}{256}}+{\frac {77\,\boldsymbol{\nu_{59}}}{
256}}-1/32\,\boldsymbol{\nu_{80}}-1/8\,\boldsymbol{\nu_{160}}\\
&+{\frac {3445\,\boldsymbol{\nu_{83}}}{7808
}}+{\frac {565\,\boldsymbol{\nu_{90}}}{3904}}+{\frac {161\,\boldsymbol{\nu_{109}}}{3904}}-{
\frac {207\,\boldsymbol{\nu_{137}}}{3904}}+3/16\,\boldsymbol{\nu_{181}}\\
&+1/8\,\boldsymbol{\nu_{202}}+1/
16\,\boldsymbol{\nu_{204}}+{\frac {77\,\boldsymbol{\nu_{58}}}{256}}+{\frac {51\,\boldsymbol{\nu_{9}}}{
2048}}
\end{aligned}
\\
\addlinespace
\midrule
\addlinespace
\boldsymbol{\nu_{200}} & 
\begin{aligned}
&\hphantom{+}{\frac {7\,\boldsymbol{\nu_{164}}}{16}}-{\frac {11\,\boldsymbol{\nu_{108}}}{1464}}-\boldsymbol{\nu_{69}}-{\frac {23\,\boldsymbol{\nu_{7}}}{1536}}-{\frac {103\,\boldsymbol{\nu_{124}}}{2928}}+{\frac {59\,\boldsymbol{\nu_{131}}}{1464}}+{\frac {53\,\boldsymbol{\nu_{132}}
}{5856}}-{\frac {7\,\boldsymbol{\nu_{127}}}{5856}}\\
&+1/8\,\boldsymbol{\nu_{79}}-1/16\,\boldsymbol{\nu_{51}}+1/48\,\boldsymbol{\nu_{48}}+{\frac {7\,\boldsymbol{\nu_{59}}}{16}}+{\frac {5\,\boldsymbol{\nu_{80}}
}{48}}-1/4\,\boldsymbol{\nu_{206}}+{\frac {2347\,\boldsymbol{\nu_{83}}}{5856}}\\
&-{\frac {1615
\,\boldsymbol{\nu_{90}}}{5856}}+{\frac {161\,\boldsymbol{\nu_{109}}}{2928}}-{\frac {69\,\boldsymbol{\nu_{137}}}{976}}+{\frac {7\,\boldsymbol{\nu_{58}}}{16}}+1/4\,\boldsymbol{\nu_{165}}+{\frac {3
\,\boldsymbol{\nu_{9}}}{128}}
\end{aligned}
\\
\addlinespace
\midrule
\addlinespace
\boldsymbol{\nu_{201}} &
\begin{aligned}
&\hphantom{+}{\frac {11\,\boldsymbol{\nu_{108}}}{976}}+3/8\,\boldsymbol{\nu_{69}}+{\frac {65\,\boldsymbol{\nu_{7}}}{4096}}+{\frac {103\,\boldsymbol{\nu_{124}}}{1952}}
-{\frac {59\,\boldsymbol{\nu_{131}}}{976}}-{\frac {53\,\boldsymbol{\nu_{132}}}{3904}}+{
\frac {7\,\boldsymbol{\nu_{127}}}{3904}}\\
&-3/16\,\boldsymbol{\nu_{79}}-{\frac {3\,\boldsymbol{\nu_{51}}}{
32}}-{\frac {25\,\boldsymbol{\nu_{48}}}{128}}-{\frac {15\,\boldsymbol{\nu_{55}}}{128}}-{
\frac {33\,\boldsymbol{\nu_{59}}}{128}}+1/32\,\boldsymbol{\nu_{80}}+{\frac {947\,\boldsymbol{\nu_{83}}
}{3904}}\\
&+{\frac {533\,\boldsymbol{\nu_{90}}}{1952}}-{\frac {161\,\boldsymbol{\nu_{109}}}{
1952}}+{\frac {207\,\boldsymbol{\nu_{137}}}{1952}}-{\frac {33\,\boldsymbol{\nu_{58}}}{128}}+
{\frac {15\,\boldsymbol{\nu_{9}}}{1024}}
\end{aligned}
\\
\addlinespace
\midrule
\addlinespace
\boldsymbol{\nu_{203}} & 
\begin{aligned}
&\hphantom{+}{\frac {11\,\boldsymbol{\nu_{108}}}{732
}}-\boldsymbol{\nu_{69}}-{\frac {97\,\boldsymbol{\nu_{7}}}{3072}}+{\frac {103\,\boldsymbol{\nu_{124}}}{
1464}}-{\frac {59\,\boldsymbol{\nu_{131}}}{732}}-{\frac {53\,\boldsymbol{\nu_{132}}}{2928}}+
{\frac {7\,\boldsymbol{\nu_{127}}}{2928}}\\
&-1/4\,\boldsymbol{\nu_{79}}+1/8\,\boldsymbol{\nu_{51}}+{\frac 
{5\,\boldsymbol{\nu_{48}}}{96}}+1/32\,\boldsymbol{\nu_{55}}+{\frac {11\,\boldsymbol{\nu_{59}}}{32}}-{
\frac {5\,\boldsymbol{\nu_{80}}}{24}}-{\frac {1981\,\boldsymbol{\nu_{83}}}{2928}}\\
&-{\frac {
1297\,\boldsymbol{\nu_{90}}}{1464}}-{\frac {161\,\boldsymbol{\nu_{109}}}{1464}}+{\frac {69\,
\boldsymbol{\nu_{137}}}{488}}+{\frac {11\,\boldsymbol{\nu_{58}}}{32}}-{\frac {9\,\boldsymbol{\nu_{9}}}{
256}}
\end{aligned}
\\
\addlinespace
\midrule
\addlinespace
\boldsymbol{\nu_{205}} &
\begin{aligned}
&-\boldsymbol{\nu_{206}}-{\frac {103\,\boldsymbol{\nu_{124}}}{488}}+{\frac 
{161\,\boldsymbol{\nu_{109}}}{488}}-{\frac {207\,\boldsymbol{\nu_{137}}}{488}}-{\frac {7\,
\boldsymbol{\nu_{127}}}{976}}-{\frac {11\,\boldsymbol{\nu_{108}}}{244}}+{\frac {53\,\boldsymbol{\nu_{132}}}{976}}\\
&+{\frac {59\,\boldsymbol{\nu_{131}}}{244}}+3/4\,\boldsymbol{\nu_{79}}+1/8\,\boldsymbol{\nu_{80}}-{\frac {215\,\boldsymbol{\nu_{83}}}{976}}+1/8\,\boldsymbol{\nu_{51}}+{\frac {2\,\boldsymbol{\nu_{90}}}{61}}+{\frac {5\,\boldsymbol{\nu_{59}}}{32}}\\
&-{\frac {7\,\boldsymbol{\nu_{9}}}{256}}+{
\frac {9\,\boldsymbol{\nu_{48}}}{32}}+{\frac {7\,\boldsymbol{\nu_{55}}}{32}}+{\frac {5\,\boldsymbol{\nu_{58}}}{32}}-{\frac {5\,\boldsymbol{\nu_{7}}}{1024}}
\end{aligned}
\\
\addlinespace
\midrule
\addlinespace
\boldsymbol{\nu_{207}} & 
\begin{aligned}
&\hphantom{+}\boldsymbol{\nu_{206}}-{
\frac {103\,\boldsymbol{\nu_{124}}}{1464}}+{\frac {161\,\boldsymbol{\nu_{109}}}{1464}}+\boldsymbol{\nu_{69}}-{\frac {69\,\boldsymbol{\nu_{137}}}{488}}-{\frac {7\,\boldsymbol{\nu_{127}}}{2928}}-{
\frac {11\,\boldsymbol{\nu_{108}}}{732}}\\
&+{\frac {53\,\boldsymbol{\nu_{132}}}{2928}}+{\frac {
59\,\boldsymbol{\nu_{131}}}{732}}+1/4\,\boldsymbol{\nu_{79}}+{\frac {5\,\boldsymbol{\nu_{80}}}{24}}+{
\frac {1249\,\boldsymbol{\nu_{83}}}{2928}}-1/8\,\boldsymbol{\nu_{51}}+{\frac {557\,\boldsymbol{\nu_{90}}}{732}}\\
&-{\frac {17\,\boldsymbol{\nu_{59}}}{32}}+{\frac {3\,\boldsymbol{\nu_{9}}}{256}}-{
\frac {23\,\boldsymbol{\nu_{48}}}{96}}-{\frac {3\,\boldsymbol{\nu_{55}}}{32}}-{\frac {17\,
\boldsymbol{\nu_{58}}}{32}}+{\frac {115\,\boldsymbol{\nu_{7}}}{3072}}
\end{aligned}
\\
\addlinespace
\midrule
\addlinespace
\boldsymbol{\nu_{208}} & 
\begin{aligned}
&\hphantom{+}{\frac {3
\,\boldsymbol{\nu_{124}}}{61}}+{\frac {19\,\boldsymbol{\nu_{109}}}{61}}-{\frac {333\,\boldsymbol{\nu_{137}}}{244}}+{\frac {\boldsymbol{\nu_{127}}}{244}}+{\frac {19\,\boldsymbol{\nu_{108}}}{61}}-
{\frac {25\,\boldsymbol{\nu_{132}}}{244}}-{\frac {25\,\boldsymbol{\nu_{131}}}{244}}\\
&-{\frac {
139\,\boldsymbol{\nu_{83}}}{488}}+{\frac {95\,\boldsymbol{\nu_{90}}}{976}}-{\frac {3\,\boldsymbol{\nu_{59}}}{32}}-{\frac {3\,\boldsymbol{\nu_{9}}}{256}}-{\frac {3\,\boldsymbol{\nu_{48}}}{32}}-1/
32\,\boldsymbol{\nu_{55}}-{\frac {3\,\boldsymbol{\nu_{58}}}{32}}\\
&+{\frac {3\,\boldsymbol{\nu_{7}}}{1024}
}
\end{aligned}
\\
\addlinespace
\midrule
\addlinespace
\boldsymbol{\nu_{209}} &
\begin{aligned}
&-{\frac {12\,\boldsymbol{\nu_{124}}}{61}}-{\frac {76\,\boldsymbol{\nu_{109}}}{
61}}+{\frac {333\,\boldsymbol{\nu_{137}}}{61}}-{\frac {\boldsymbol{\nu_{127}}}{61}}-{\frac {
76\,\boldsymbol{\nu_{108}}}{61}}+{\frac {25\,\boldsymbol{\nu_{132}}}{61}}+{\frac {25\,\nu_{{
131}}}{61}}\\
&+{\frac {39\,\boldsymbol{\nu_{83}}}{61}}-{\frac {39\,\boldsymbol{\nu_{90}}}{61}}
\end{aligned}
\\
\addlinespace
\midrule
\addlinespace
\boldsymbol{\nu_{210}} &
\begin{aligned}
&-\boldsymbol{\nu_{206}}-{\frac {103\,\boldsymbol{\nu_{124}}}{732}}+{\frac {161\,
\boldsymbol{\nu_{109}}}{732}}-\boldsymbol{\nu_{69}}-{\frac {69\,\boldsymbol{\nu_{137}}}{244}}-{\frac {7
\,\boldsymbol{\nu_{127}}}{1464}}-{\frac {11\,\boldsymbol{\nu_{108}}}{366}}\\
&+{\frac {53\,\boldsymbol{\nu_{132}}}{1464}}+{\frac {59\,\boldsymbol{\nu_{131}}}{366}}+1/2\,\boldsymbol{\nu_{79}}-1/12\,
\boldsymbol{\nu_{80}}-{\frac {1313\,\boldsymbol{\nu_{83}}}{1464}}+1/4\,\boldsymbol{\nu_{51}}\\
&-{\frac {
1249\,\boldsymbol{\nu_{90}}}{1464}}+1/2\,\boldsymbol{\nu_{59}}-1/16\,\boldsymbol{\nu_{9}}+1/3\,\boldsymbol{\nu_{48}}+1/4\,\boldsymbol{\nu_{55}}+1/2\,\boldsymbol{\nu_{58}}-{\frac {7\,\boldsymbol{\nu_{7}}}{192}}
 \end{aligned}
\\
\addlinespace
\midrule
\addlinespace
\boldsymbol{\nu_{211}} &
\begin{aligned}
&-{\frac {239\,\boldsymbol{\nu_{108}}}{1952}}+{\frac {27\,\boldsymbol{\nu_{7}}}{8192}}-
{\frac {175\,\boldsymbol{\nu_{124}}}{3904}}+{\frac {67\,\boldsymbol{\nu_{131}}}{976}}+{
\frac {353\,\boldsymbol{\nu_{132}}}{7808}}-{\frac {19\,\boldsymbol{\nu_{127}}}{7808}}\\
&+{
\frac {3\,\boldsymbol{\nu_{79}}}{32}}+1/32\,\boldsymbol{\nu_{51}}+{\frac {17\,\boldsymbol{\nu_{48}}}{
256}}+{\frac {11\,\boldsymbol{\nu_{55}}}{256}}+{\frac {49\,\boldsymbol{\nu_{59}}}{256}}-1/8
\,\boldsymbol{\nu_{160}}\\
&+{\frac {4645\,\boldsymbol{\nu_{83}}}{7808}}+{\frac {55\,\boldsymbol{\nu_{90}}
}{244}}-{\frac {295\,\boldsymbol{\nu_{109}}}{3904}}+{\frac {1791\,\boldsymbol{\nu_{137}}}{
3904}}+3/16\,\boldsymbol{\nu_{181}}\\
&+1/8\,\boldsymbol{\nu_{202}}+1/16\,\boldsymbol{\nu_{204}}+{\frac {49
\,\boldsymbol{\nu_{58}}}{256}}+{\frac {59\,\boldsymbol{\nu_{9}}}{2048}}
\end{aligned}
\\
\addlinespace
\midrule
\addlinespace
\boldsymbol{\nu_{212}} &
\begin{aligned}
&\hphantom{+}{\frac {
7\,\boldsymbol{\nu_{164}}}{32}}+{\frac {217\,\boldsymbol{\nu_{108}}}{2928}}-1/2\,\boldsymbol{\nu_{69}}-
{\frac {23\,\boldsymbol{\nu_{7}}}{3072}}-{\frac {31\,\boldsymbol{\nu_{124}}}{5856}}-{\frac {
\boldsymbol{\nu_{131}}}{183}}-{\frac {247\,\boldsymbol{\nu_{132}}}{11712}}\\
&+{\frac {5\,\boldsymbol{\nu_{127}}}{11712}}+1/16\,\boldsymbol{\nu_{79}}-1/32\,\boldsymbol{\nu_{51}}+{\frac {\boldsymbol{\nu_{48}}}{
96}}+{\frac {7\,\boldsymbol{\nu_{59}}}{32}}+{\frac {5\,\boldsymbol{\nu_{80}}}{96}}-1/8\,\boldsymbol{\nu_{206}}\\
&+{\frac {1879\,\boldsymbol{\nu_{83}}}{11712}}-{\frac {1147\,\boldsymbol{\nu_{90}}}{
11712}}+{\frac {617\,\boldsymbol{\nu_{109}}}{5856}}-{\frac {735\,\boldsymbol{\nu_{137}}}{
1952}}+{\frac {7\,\boldsymbol{\nu_{58}}}{32}}\\
&+1/8\,\boldsymbol{\nu_{165}}+{\frac {3\,\boldsymbol{\nu_{9}}}{256}}
\end{aligned}
\\
\addlinespace
\midrule
\addlinespace
\boldsymbol{\nu_{213}} & 
\begin{aligned}
&\hphantom{+}{\frac {79\,\boldsymbol{\nu_{124}}}{488}}-{\frac {313\,\boldsymbol{\nu_{109}}}{488}}+{\frac {873\,\boldsymbol{\nu_{137}}}{488}}+{\frac {3\,\boldsymbol{\nu_{127}}
}{976}}-{\frac {65\,\boldsymbol{\nu_{108}}}{244}}+{\frac {47\,\boldsymbol{\nu_{132}}}{976}}\\
&-
{\frac {17\,\boldsymbol{\nu_{131}}}{122}}-3/4\,\boldsymbol{\nu_{79}}-1/8\,\boldsymbol{\nu_{80}}+{\frac 
{371\,\boldsymbol{\nu_{83}}}{976}}-1/8\,\boldsymbol{\nu_{51}}-{\frac {47\,\boldsymbol{\nu_{90}}}{244}}\\
&-
{\frac {5\,\boldsymbol{\nu_{59}}}{32}}+{\frac {7\,\boldsymbol{\nu_{9}}}{256}}-{\frac {9\,\boldsymbol{\nu_{48}}}{32}}-{\frac {7\,\boldsymbol{\nu_{55}}}{32}}-{\frac {5\,\boldsymbol{\nu_{58}}}{32}}+
{\frac {5\,\boldsymbol{\nu_{7}}}{1024}}
\end{aligned}
\\
\addlinespace
\midrule
\addlinespace
\boldsymbol{\nu_{214}} &
\begin{aligned}
&-{\frac {11\,\boldsymbol{\nu_{108}}}{732
}}-1/2\,\boldsymbol{\nu_{69}}-{\frac {37\,\boldsymbol{\nu_{7}}}{1536}}-{\frac {103\,\boldsymbol{\nu_{124}}}{1464}}+{\frac {59\,\boldsymbol{\nu_{131}}}{732}}+{\frac {53\,\boldsymbol{\nu_{132}}}{
2928}}-{\frac {7\,\boldsymbol{\nu_{127}}}{2928}}\\
&+1/4\,\boldsymbol{\nu_{79}}+1/8\,\boldsymbol{\nu_{51}}+
{\frac {17\,\boldsymbol{\nu_{48}}}{48}}+3/16\,\boldsymbol{\nu_{55}}+{\frac {7\,\boldsymbol{\nu_{59}}}{
16}}-1/24\,\boldsymbol{\nu_{80}}-{\frac {581\,\boldsymbol{\nu_{83}}}{2928}}\\
&-{\frac {883\,\boldsymbol{\nu_{90}}}{2928}}+{\frac {161\,\boldsymbol{\nu_{109}}}{1464}}-{\frac {69\,\boldsymbol{\nu_{137}}}{488}}+{\frac {7\,\boldsymbol{\nu_{58}}}{16}}-{\frac {\boldsymbol{\nu_{9}}}{128}}
\end{aligned}
\\
\addlinespace
\midrule
\addlinespace
\boldsymbol{\nu_{215}} & 
\begin{aligned}
&\hphantom{+}{\frac {7\,\boldsymbol{\nu_{164}}}{32}}+{\frac {217\,\boldsymbol{\nu_{108}}}{976}}-3/4
\,\boldsymbol{\nu_{69}}-{\frac {5\,\boldsymbol{\nu_{7}}}{256}}-{\frac {31\,\boldsymbol{\nu_{124}}}{1952
}}-{\frac {\boldsymbol{\nu_{131}}}{61}}-{\frac {247\,\boldsymbol{\nu_{132}}}{3904}}\\
&+{\frac {
5\,\boldsymbol{\nu_{127}}}{3904}}+3/16\,\boldsymbol{\nu_{79}}+1/32\,\boldsymbol{\nu_{51}}+3/16\,\boldsymbol{\nu_{48}}+{\frac {3\,\boldsymbol{\nu_{55}}}{32}}+{\frac {7\,\boldsymbol{\nu_{59}}}{16}}\\
&+1/32\,\boldsymbol{\nu_{80}}-1/8\,\boldsymbol{\nu_{206}}-{\frac {73\,\boldsymbol{\nu_{83}}}{3904}}-{\frac {659\,
\boldsymbol{\nu_{90}}}{3904}}+{\frac {617\,\boldsymbol{\nu_{109}}}{1952}}-{\frac {2205\,\boldsymbol{\nu_{137}}}{1952}}\\
&+{\frac {7\,\boldsymbol{\nu_{58}}}{16}}+1/8\,\boldsymbol{\nu_{165}}+{\frac {
\boldsymbol{\nu_{9}}}{128}}
\end{aligned}
\\
\addlinespace
\midrule
\addlinespace
\boldsymbol{\nu_{216}} & 
\begin{aligned}
&-{\frac {\boldsymbol{\nu_{149}}}{96}}-1/24\,\boldsymbol{\nu_{151}}+1/24\,\boldsymbol{\nu_{153}}+{\frac {\boldsymbol{\nu_{160}}}{96}}+{\frac {5\,\boldsymbol{\nu_{164}}}{384}}+1/48\,\boldsymbol{\nu_{165}}+1/48\,\boldsymbol{\nu_{178}}\\
&+1/48\,\boldsymbol{\nu_{202}}-1/32\,
\boldsymbol{\nu_{204}}+1/32\,\boldsymbol{\nu_{206}}+{\frac {\boldsymbol{\nu_{20}}}{768}}+{\frac {67\,
\boldsymbol{\nu_{7}}}{49152}}+{\frac {9\,\boldsymbol{\nu_{9}}}{4096}}-{\frac {41\,\boldsymbol{\nu_{48}}}{1536}}\\
&-{\frac {\boldsymbol{\nu_{51}}}{64}}-{\frac {9\,\boldsymbol{\nu_{55}}}{512}}-{\frac 
{17\,\boldsymbol{\nu_{58}}}{512}}-{\frac {17\,\boldsymbol{\nu_{59}}}{512}}+1/32\,\boldsymbol{\nu_{69}}-
1/32\,\boldsymbol{\nu_{79}}+{\frac {\boldsymbol{\nu_{80}}}{192}}\\
&+{\frac {199\,\boldsymbol{\nu_{83}}}{
11712}}+{\frac {851\,\boldsymbol{\nu_{90}}}{46848}}+{\frac {11\,\boldsymbol{\nu_{108}}}{5856
}}-{\frac {161\,\boldsymbol{\nu_{109}}}{11712}}+{\frac {103\,\boldsymbol{\nu_{124}}}{11712}}
+{\frac {7\,\boldsymbol{\nu_{127}}}{23424}}\\
&-{\frac {59\,\boldsymbol{\nu_{131}}}{5856}}-{
\frac {53\,\boldsymbol{\nu_{132}}}{23424}}+{\frac {69\,\boldsymbol{\nu_{137}}}{3904}}
\end{aligned}
\\
\addlinespace
\bottomrule
\caption{Solution of Perturbative Area Metric Equivariance Equations.}\label{AreaSol}
\end{longtable}
\endgroup


\printbibliography[
heading=bibintoc,
title={Bibliography}
]

\printunsrtglossaries

\chapter*{Selbständigkeitserklärung}

Hiermit erkläre ich, dass ich diese Arbeit selbständig angefertigt habe. Ich habe nach bestem Gewissen alle verwendeten Hilfsmittel und Quellen angegeben. Des Weiteren wurde diese Arbeit weder einem anderen Prüfungsamt vorgelegt noch veröffentlicht.\\
$\vspace{3cm}$

$\overline{ \text{ Ort, Datum} \hspace{5cm}} \hspace{2cm} \overline{\text{ Unterschrift} \hspace{3cm}}$
\end{document}
